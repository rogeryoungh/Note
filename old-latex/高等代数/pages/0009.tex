\chapter{线性映射}

\section{线性映射及其运算}

\begin{definition}[线性映射]\index{xianxingyingshe@线性映射}
    设域 $F$ 上两个线性空间 $V,V'$,若映射 $\bfA : V \to V'$ 保持加法运算和纯量乘法运算,即
    \[ \bfA(\alpha + \beta) = \bfA(\alpha) + \bfA(\beta), \quad \bfA(k\alpha) = k \bfA(\alpha) \]
    则称 $\bfA$ 是 $V$ 到 $V'$ 的一个线性映射。
\end{definition}

线性空间 $V$ 到自身的线性映射通常称为 $V$ 上的线性变换,域 $F$ 上线性空间 $V$ 到 $F$ 的线性映射称为 $V$ 上的线性函数。\index{xianxingbianhuan@线性变换}

$\bfA(\alpha)$ 也可写成 $\bfA\alpha$。

特殊的,从 $V$ 映射到 $V'$ 的零向量是零映射,记作 $\vzero$;$V$ 上映射到自身的变换叫做恒等变换,记作 $\bfI$;映射 $\alpha$ 到 $k\alpha$ 的变换叫由 $k$ 决定的数乘变换,记作 $\boldsymbol{k}$。

定积分 $\displaystyle \boldsymbol{J}(f(x)) = \int_a^b f(x) \dd x$ 也是 $C[a,b]$ 到 $\RR$ 的一个线性映射。

\paragraph{线性映射的存在性}

\begin{theorem}
    设域 $F$ 上的线性空间 $V,V'$,且 $V$ 是有限维的。从 $V$ 中取一个基 $\ji{\alpha}{n}$,从 $V'$ 任意取定 $n$ 个向量 $\ji{\gamma}{n}$(可以有相同),令
    \[ \bfA : V \to V', \alpha = \sum_{i=1}^n \alpha_i\alpha_i \mapsto \sum_{i=1}^n \alpha_i\gamma_i \]
    则 $\bfA$ 是 $V$ 到 $V'$ 的一个线性映射,且 $\bfA(\alpha_i) = \gamma_i$。
\end{theorem}

设域 $F$ 上的线性空间 $V$ 有两个子空间 $U,W$,且 $V = U \oplus W$。任取 $\alpha \in V$,设 $\alpha = \alpha_1 + \alpha_2, \alpha_1 \in U, \alpha_2\in W$。令
\[ \bfP_U : V \to V, \alpha \mapsto \alpha_1 \]
则 $\bfP_U$ 是 $V$ 上的一个线性变换。称 $\bfP_U$ 是平行于 $W$ 在 $U$ 上的投影,它满足
\[ \bfP_U(\alpha) =
\begin{cases}
    \alpha,& \alpha \in U\\
    0 , & \alpha \in W
\end{cases} \]
可以证明,满足该式的投影变换是唯一的。

类似的,定义 $\bfP_W(\alpha) = \alpha_2$,称它为平行于 $U$ 在 $W$ 的投影。

\begin{definition}[幂等变换]\index{midengibanhuan@幂等变换}
    线性空间 $V$ 上的线性变换 $\bfA$ 如果满足 $\bfA^2 = \bfA$,则称 $\bfA$ 是幂等变换。
\end{definition}

\begin{definition}
    线性空间 $V$ 上的两个线性变换 $\bfA, \bfB$ 如果满足 $\bfA\bfB = \bfB\bfA = 0$,则称 $\bfA$ 与 $\bfB$ 正交。
\end{definition}

任取 $\alpha\in V$,设 $\alpha = \alpha_1 + \alpha_2, \alpha_1 \in U, \alpha_2 \in W$,则
\begin{equation*}
    \begin{aligned}
        \bfP_U^2(\alpha) = \bfP_U(\bfP_U(\alpha)) = \alpha_1 = \bfP_U(\alpha)\\
        \bfP_W^2(\alpha) = \bfP_W(\bfP_W(\alpha)) = \alpha_1 = \bfP_W(\alpha)
    \end{aligned}
\end{equation*}
且
\[ \bfP_U\bfP_W(\alpha) = \bfP_U(\alpha_2) = 0 = \bfP_W(\alpha_1) = \bfP_W\bfP_U(\alpha) \]
由此得出
\[ \bfP_U^2 = \bfP_U, \bfP_W^2 = \bfP_W, \bfP_U \]

\paragraph{线性映射的运算}

设域 $F$ 上的线性空间 $V,V'$,把所有 $V \to V'$ 的线性映射所组成的集合记作 $\Hom(V,V')$,同样有 $\Hom{V,V}$。

设域 $F$ 上的线性空间 $V,U,W$,其中 $\bfA \in \Hom(V,U), \bfB \in \Hom(U,W)$。线性映射作为映射,有映射的乘法 $\bfB\bfA$。

若 $\bfA\in \Hom(V,V')$ 可逆,则 $\bfA$ 是 $V \to V'$ 的一同构映射,从而 $\bfA^{-1}$ 是 $V' \to V$ 的同步映射。于是 $\bfA^{-1} \in \Hom(V',V)$。

设 $\bfA,\bfB \in \Hom (V,V')$,由于陪域 $V'$ 是线性空间,因此可以定义加法和纯量乘法如下
\[ (\bfA + \bfB) \alpha \coloneqq \bfA\alpha + \bfB\alpha, (k\bfA)\alpha \coloneqq k(\bfA \alpha) \]
显然其运算结果都是线性映射,称 $\bfA + \bfB$ 是 $\bfA$ 与 $\bfB$ 的和,$k\bfA$ 是 $k$ 与 $\bfA$ 的纯量乘积。

不难验证,$\Hom(V,V')$ 是域 $F$ 上的线性空间。特别的,$\Hom(V,V)$ 是一个有单位元的环,还可证明其上变换的乘法与纯量加法满足
\[ k(\bfA\bfB) = (k\bfA)\bfB = \bfA(k\bfB) \]

于是有

\begin{definition}[代数]\index{daishu@代数}
    设域 $F$ 上的线性空间 $A$ 对于其上的加法和纯量乘法是一个有单位元的交换环,且
    \[ k(\alpha \beta) = (k\alpha)\beta = \alpha(k\beta), \forall k \in F,\alpha,\beta \in A \]
    那么称 $A$ 是一个(结合)代数,把线性空间 $A$ 的维数称为代数 $A$ 的维数。
\end{definition}

容易看出,$\Hom(V,V)$ 是域 $F$ 上的代数,$M_n(F)$ 也是域 $F$ 上的代数。

因此可以在 $\Hom(V,V)$ 上定义 $\bfA$ 的正整数幂
\[ \bfA^m \coloneqq \bfA^m \cdot \bfA, \bfA^0 = \bfI \]
容易验证
\[ \bfA^m \cdot \bfA^n = \bfA^{m+n}, (\bfA^m)^n = \bfA^{mn} \]
设 $f(x) = a_0 + a_1 x + \cdots + a_mx^m \in F[x]$,用 $\bfA$ 代入得
\[ f(\bfA) = a_0\bfI + a_1\bfA + \cdots + a_m \bfA^m \]
显然 $f(\bfA) \in \Hom(V,V)$,称 $f(\bfA)$ 是 $\bfA$ 的多项式。

把 $\bfA$ 的所有多项式的全体记作 $F[\bfA]$,不难发现其是 $\Hom(V,V)$ 的一个子环,且 $F[\bfA]$ 是交换环。

$F[\bfA]$ 中所有数乘变换组成的集合是 $F[\bfA]$ 是 $F[\bfA]$ 的一个子环,且域 $F$ 到这个子环之间存在双射
\[ \tau : k \mapsto \boldsymbol{k} \]
双射 $\tau$ 保持加法与乘法运算,因此 $F[\bfA]$ 可以看作 $F$ 的一个扩环。于是 $F$ 上一元多项式中的不定元 $x$ 可以用 $F[\bfA]$ 中任意元素代入,从而在 $F[x]$ 中关于加法和乘法的等式在 $F[\bfA]$ 中也成立。

更多的,我们还有镜面反射变换
\[ \boldsymbol{R}_U = \bfI - 2\bfP_W \]
平移变换
\[ \boldsymbol{T_a} : f(x) \mapsto f(x+a) \]
利用 Taylor 公式不难发现
\begin{equation*}
    \begin{aligned}
        f(x+a) &= f(x) + a\bfA + \frac{a^2}{2!}f''(x) + \cdots + \frac{a^{n-1}}{(n-1)!} f^{(n-1)}(x)\\
        &= \left( \bfI + a\boldsymbol{D} + \frac{a^2}{2!}\boldsymbol{D} + \cdots + \frac{a^{n-1}}{(n-1)!} \boldsymbol{D}^{(n-1)} \right)f(x)
    \end{aligned}
\end{equation*}
即平移 $\boldsymbol{T}_a$ 是导数 $\boldsymbol{D}$ 的一个多项式。

\section{线性映射的核与象}

\begin{definition}[核] \index{he@核}
    设域 $F$ 上的线性空间 $V,V'$,其中 $\bfA : V \to V'$,令 $V'$ 的零向量在 $\bfA$ 下的原象集称为 $\bfA$ 的和,记作 $\Ker \bfA$,即
    \[ \Ker \bfA \coloneqq \{ \alpha \in V \mid \bfA\alpha = 0 \} \]
    $\bfA$ 的象(值域)记作 $\Im A$ 或 $\bfA V$。
\end{definition}

\begin{proposition}
    设域 $F$ 上线性空间 $V,V'$ 的线性映射 $\bfA : V \to V'$,则
    
    \num{1} $\bfA$ 是单射当且仅当 $\Ker \bfA = 0$。

    \num{2} $\bfA$ 是满射当且仅当 $\Im \bfA = V'$。
\end{proposition}

\begin{theorem}
    设域 $F$ 上线性空间 $V,V'$ 的线性映射 $\bfA : V \to V'$,则
    \[ V / \Ker \bfA \cong \Im \bfA \]
\end{theorem}

\begin{theorem}
    设域 $F$ 上线性空间 $V,V'$,且 $V$ 是有限维的。设线性映射 $\bfA : V \to V'$,则 $\Ker \bfA$ 和 $\Im \bfA$ 都是有限维的,且
    \[ \dim(\Ker \bfA) + \dim(\Im \bfA) = \dim(V) \]
\end{theorem}

当 $V$ 是有限维时,线性映射 $\bfA : V \to V'$,则 $\bfA$ 的核的维数也称为 $\bfA$ 的测度。$\bfA$ 的象 $\Im \bfA$ 的维数称为 $\bfA$ 的秩,记作 $\rank(\bfA)$。

\section{线性映射和线性变换的矩阵表示}

设域 $F$ 上的有限维线性空间 $V,V'$,其上的线性映射 $\bfA : V \to V'$。设 $\dim V = n,\dim V' = s$。我们知道 $\bfA$ 被其在 $V$ 上的一个基所确定,不妨取 $V$ 上的一个基 $\ji{\alpha}{n}$,$\bfA$ 完全被 $\bfA \alpha_1,\cdots,\bfA \alpha_n$ 决定。由于 $\bfA \alpha_i \in V'$,因此在 $V'$ 中取一个基 $\ji{\eta}{s}$,$\bfA \alpha_i$ 被它在基 $\ji{\eta}{s}$ 下的坐标所决定,形式的记作
\[ 
    \left( \bfA \alpha_1 , \bfA \alpha_2 , \cdots , \bfA \alpha_n \right) = 
    \left( \eta_1 , \eta_2 , \cdots , \eta_n \right)
    \left( \begin{matrix}
        a_{11} & a_{11} & \cdots & a_{1n}\\
        a_{21} & a_{22} & \cdots & a_{2n}\\
        \vdots & \vdots &        & \vdots\\
        a_{s1} & a_{s1} & \cdots & a_{sn}
    \end{matrix} \right)
\]
把 $s\times n$ 矩阵记作 $A$,它的第 $j$ 列就是 $\bfA \alpha_j$ 在 $\ji{\eta}{s}$ 下的坐标。称 $A$ 是线性映射 $\bfA$ 在 $V$ 的基 $\ji{\alpha}{n}$ 和 $V'$ 的基 $\ji{\eta}{s}$ 下的矩阵。于是线性映射 $\bfA$ 有了矩阵表示。

通常把 $(\bfA \alpha_1,\cdots,\bfA \alpha_n)$ 记成 $\bfA(\ji{\alpha}{n})$ 于是上式可以记成
\[ \bfA(\ji{\alpha}{n}) = (\ji{\alpha}{n}) A\]

对于 $V$ 上的线性变换 $\bfA$,由于 $\bfA \alpha_i \in V$,因此 $\bfA \alpha_i$ 可以用 $V$ 的基 $\ji{\alpha}{n}$ 线性表出,于是有
\[ 
    \left( \bfA \alpha_1 , \bfA \alpha_2 , \cdots , \bfA \alpha_n \right) = 
    \left( \alpha_1 , \alpha_2 , \cdots , \alpha_n \right)
    \left( \begin{matrix}
        a_{11} & a_{11} & \cdots & a_{1n}\\
        a_{21} & a_{22} & \cdots & a_{2n}\\
        \vdots & \vdots &        & \vdots\\
        a_{n1} & a_{n1} & \cdots & a_{nn}
    \end{matrix} \right)
\]
把右端的 $n$ 级矩阵记作 $A$,它的第 $j$ 列就是 $\bfA \alpha_j$ 在 $\ji{\alpha}{n}$ 下的坐标。称 $A$ 是线性映射 $\bfA$ 在 $V$ 的基 $\ji{\alpha}{n}$ 下的矩阵。于是线性映射 $\bfA$ 有了矩阵表示。
\[ \bfA(\ji{\alpha}{n}) = (\ji{\alpha}{n}) A\]

设域 $F$ 上 $n$ 维线性空间 $V$ 上的幂等变换,有
\[ V = \Im \bfA \oplus \Ker \bfA \]
且 $\bfA$ 是平行于 $\Ker \bfA$ 在 $\Im \bfA$ 上的投影。在 $\Im \bfA$ 中取一个基 $\ji{\alpha}{r}$;在 $\Ker \bfA$ 中取一个基 $\ji{\beta}{{n-r}}$,则 $\ji{\alpha}{r},\ji{\beta}{{n-r}}$ 是 $V$ 的一个基,由于 $\bfA$ 是平行于 $\Ker \bfA$ 在 $\Im A$ 上的投影,因此
\[ \bfA \alpha_i = \alpha_i, \bfA \beta_j = 0 \]
从而幂等变换 $\bfA$ 在基 $\ji{\alpha}{r},\ji{\beta}{{n-r}}$ 下的矩阵 $\bfA$ 为
\[ A = \left(\begin{matrix}
    E_r & 0\\
    0   & 0
\end{matrix}\right) \]
其中 $r = \rank(A) = \dim(\Im \bfA) = \rank(\bfA)$。

\paragraph{$\Hom(V,V')$ 与 $M_{s \times n}(F)$ 的关系,$\Hom(V,V)$ 与 $M_n(F)$ 的关系}

\begin{theorem}
    设域 $F$ 上的 $n$ 维线性空间 $V$ 和 $s$ 维的线性空间 $V'$,则线性映射 $\bfA : V \to V'$ 与它在 $V$ 的一个基和 $V'$ 的一个基下的矩阵 $A$ 的对应 $\sigma$ 是线性空间 $\Hom(V,V') \to M_{s \times n}(F)$ 的同构映射,从而
    \[ \Hom(V,V') \cong M_{s \times n}(F) \]
    \[ \dim(\Hom(V,V')) = sn = (\dim V)(\dim V') \]
\end{theorem}

特别的有
\[ \Hom(V,V) \cong M_{n}(F) \]
\[ \dim(\Hom(V,V)) = sn = (\dim V)^2 \]

注意到 $\Hom(V,V)$ 与 $M_n(F)$ 都是域 $F$ 上的代数,它们都有加法、纯量乘法、乘法运算,可以证明 $\sigma$ 映射保持乘法运算
\[ \sigma(\bfA \bfB) = AB = \sigma(\bfA)\sigma(\bfB) \]
因此 $\sigma$ 是 $\Hom(V,V) \to M_n(F)$ 的同构映射。

\begin{definition}
    设域 $F$ 上的代数 $F,F'$,如果存在双射 $\sigma : M \to M'$,使得 $\sigma$ 既是线性空间 $M \to M'$ 的同构映射,又是环 $M \to M'$ 的同构映射,那么称代数 $M$ 与 $M'$ 是同构的,并且称 $\sigma$ 是代数 $M \to M'$ 的一个同构映射。
\end{definition}

\begin{theorem}
    设域 $F$ 上 $n$ 维线性空间 $V$ 上的线性变换 $\bfA$,$\bfA$ 与它在 $V$  的一个基下的矩阵 $A$ 的对应是代数 $\Hom(V,V) \to M_n(F)$ 的同构映射,从而代数 $\Hom(V,V)$ 与 $M_n(F)$ 是同构的。
\end{theorem}

\paragraph{向量在线性映射(或线性变换)下的象的坐标}。

设域 $F$ 上的 $n$ 维线性空间和 $s$ 维线性空间,线性映射 $\bfA : V \to V'$ 在 $V$ 的一个基 $\ji{\alpha}{n}$ 和 $V'$ 的一个基 $\ji{\eta}{s}$ 下的矩阵为 $A$。设 $V$ 中向量 $\alpha$ 在基 $\ji{\alpha}{n}$ 下的坐标为 $X$,有
\[ \bfA \alpha = (\ji{\eta}{n})AX \]
则 $\bfA \alpha$ 在基 $\ji{\eta}{s}$ 下的坐标为 $AX$。

特别的,设线性空间 $V$ 中的一组基 $\ji{\alpha}{n}$,其上的线性变换 $\bfA$ 在此基下的矩阵为 $A$。把向量 $\alpha$ 在此基下的坐标记作 $X$,有
\[ \bfA \alpha = (\ji{\alpha}{n})AX \]
即向量 $\bfA$ 在此基下的坐标是 $AX$。

若向量 $\gamma$ 在基 $\ji{\alpha}{n}$ 下的坐标为 $Y$,则
\[ \bfA\alpha = \gamma \Leftrightarrow AX = Y \]

\paragraph{线性变换在不同基下的矩阵之间的关系}

\begin{theorem}
    设域 $F$ 上的 $n$ 维线性空间,线性变换 $\bfA$ 在基 $\ji{\alpha}{n}$ 下的矩阵为 $A$,在基 $\ji{\eta}{s}$ 下的矩阵为 $B$,从基 $\ji{\alpha}{n}$ 到基 $\ji{\eta}{s}$ 的过度矩阵为 $S$,则
    \[ B = S^{-1}AS \]
\end{theorem}

即 同一个线性变换 $\bfA$ 在 $V$ 的不同基下的矩阵是相似的。

由于行列式、秩、迹都是相似关系下的不变量,因此我们把 $\bfA$ 在 $V$ 的基下的矩阵 $A$ 的行列式、秩、迹分别叫做线性变换 $\bfA$ 的行列式、秩、迹,依次记作 $\det(\bfA),\rank(\bfA),\tr(\bfA)$。

\section{线性变换的特征值和特征向量,线性变换可对角化的条件}

\begin{definition}
    设域 $F$ 上的线性空间 $V$ 上的一个线性变换 $\bfA$,若 $V$ 中存在一个非零向量 $\xi$,存在 $\lambda_0 \in F$ 使得
    \[ \bfA \xi = \lambda_0\xi \]
    则称 $\lambda_0$ 是线性变换 $\bfA$ 的一个特征值,称 $\xi$ 是 $\bfA$ 的属于特征值 $\lambda_0$ 的一个特征向量。
\end{definition}

对于几何空间 $V$,那么 $\bfA$ 对 $\xi$ 的作用是把 $\xi$ 拉伸或压缩 $\lambda_0$ 倍。

\begin{theorem}
    设域 $F$ 上 $n$ 为线性空间 $V$ 上的线性变换 $\bfA$ 可对角化当且仅当 $\bfA$ 有 $n$ 个线性无关的特征向量 $\ji{\xi}{n}$,此时 $\bfA$ 在 $\ji{\xi}{n}$ 下的矩阵为
    \[ \diag\{\ji{\lambda}{n}\} \]
    其中 $\lambda_i$ 是 $\xi_i$ 所属的特征值。该矩阵称为 $\bfA$ 的标准型。除了主对角线上元素的排列次序外,$\bfA$ 的标准型是由 $\bfA$ 唯一决定的。
\end{theorem}

设域 $F$ 上线性空间 $V$ 上的线性变换 $\bfA$,$\lambda_0$ 是 $\bfA$ 的一个特征值,令
\[ V_{\lambda_0} \coloneqq \{ \alpha \in V \mid \bfA\alpha = \lambda_0\alpha \} \]
易验证 $V_{\lambda_0}$ 是 $V$ 的一个子空间,称 $V_{\lambda_0}$ 是数域特征值 $\lambda_0$ 的特征子空间。且
\[ V_{\lambda_0} = \Ker(\lambda_0\bfI - \bfA) \]

\begin{proposition}
    域 $F$ 上 $n$ 维线性空间 $V$ 上的线性变换 $\bfA$ 可对角化当且仅当下式成立
    \[ V = V_{\lambda_1} \oplus \cdots \oplus V_{\lambda_s} \]
    其中 $\ji{\lambda}{s}$ 是 $\bfA$ 全部的不同的特征值。
\end{proposition}

由于 $n$ 维线性空间 $V$ 上线性变换 $\bfA$ 在 $V$ 的不同基下的矩阵是相似的,而相似的矩阵有相等的多项式,因此我们把 $\bfA$ 在 $V$ 的一个基下的矩阵 $A$ 的特征多项式称为线性变换 $\bfA$ 的特征多项式。设 $\lambda_i$ 是 $\bfA$ 的一个特征值,把 $\lambda_i$ 作为特征多项式的根的重数叫做 $\lambda_i$ 的代数重数,把 $\bfA$ 的属于特征值 $\lambda_i$ 的特征子空间 $V_{\lambda_i}$ 的维数叫做 $\lambda_i$ 的几何重数。\index{daishuchongshu@代数重数} \index{jihechognshu@几何重数}

因此 $\bfA$ 可标准化当且仅当 $\bfA$ 的标准型为对角矩阵,其主对角线上的元素是 $\bfA$ 的全部特征值,且每个特征值 $\lambda_i$ 出现的次数等于它的几何重数。即当 $\bfA$ 可对角化时,$\bfA$ 的特征多项式为
\[ (\lambda - \lambda_1)^{r_1}\cdots(\lambda - \lambda_s)^{r_s} \]

\begin{theorem}
    域 $F$ 上 $n$ 维线性空间 $V$ 上的线性变换 $\bfA$ 可对角化当且仅当 $\bfA$ 的特征多项式在 $F[\lambda]$ 中可分解成
    \[ (\lambda - \lambda_1)^{l_1} \cdots (\lambda - \lambda_s)^{l_s} \]
    其中 $\ji{\lambda}{s}$ 两两不等,其诶 $\bfA$ 的每个特征值 $\lambda_i$ 的几何重数等于它的代数重数。
\end{theorem}

\section{线性变换的不变子空间}

\begin{definition}[不变子空间] \index{bubianzikongjian@不变子空间}
    设域 $F$ 上的线性空间 $V$ 上的线性变换 $\bfA$,若 $V$ 的子空间 $W$ 如果满足对任意 $\alpha \in W$ 都有 $\bfA \alpha \in W$ 那么称 $W$ 是 $\bfA$ 的不变子空间,简称为 $\bfA$- 子空间。
\end{definition}

设线性空间 $V$ 上的可交换线性变换 $\bfA,\bfB$,容易验证 $\Ker \bfB,\Im \bfB,\bfB$ 的特征子空间都是 $\bfA$- 子空间

\begin{theorem}
    设域 $F$ 上 $n$ 维线性空间 $V$ 上的线性变换 $\bfA$,则 $\bfA$ 在 $V$ 的一个基下的矩阵为分块对角矩阵当且仅当 $V$ 能被分解为 $\bfA$ 的非平凡不变子空间的直和:$V = W_1 \oplus \cdots \oplus W_s$,并且 $A_i$ 是 $\bfA \mid W_i$ 在 $W_i$ 下的矩阵。
\end{theorem}

\begin{theorem}
    设域 $F$ 上的线性空间 $V$ 上的线性变换 $\bfA$,若在 $K[x]$ 中有
    \[ f(x) = f_1(x) \cdots f_s(x) \]
    且 $f_1(x),\cdots,f_s(x)$ 两两互素,则
    \[ \Ker f(\bfA) = \Ker f_1(\bfA) \oplus \cdots \oplus \Ker f_s(\bfA) \]
\end{theorem}

\begin{definition}[零化多项式] \index{linghuaduoxiangshi@零化多项式}
    设域 $F$ 上线性空间 $V$ 上的线性变换 $\bfA$,若 $F$ 上的一元多项式 $f(\bfA) = \vzero$,那么称 $f(x)$ 为 $\bfA$ 的一个零化多项式。
\end{definition}

设 $\dim V = 0$,则 $\dim(\Hom(V,V)) = n^2$,从而
\[ \bfI, \bfA, \bfA^2, \cdots, \bfA^{n^2} \]
一定线性相关,于是存在一组不全为 $0$ 的元素 $k_0,\cdots,k_{n^2}$ 使得
\[ k_0\bfI + k_1\bfA + k_2\bfA^2 + \cdots + k_{n^2}\bfA^{n^2} = \vzero \]
令 $f(x) = k_0 + k_1x + k_2x^2 + \cdots + k_{n^2}x^{n^2}$,则有 $f(\bfA) = \vzero$,于是 $f(x)$ 是 $\bfA$ 的一个零化多项式。

\begin{definition}
    设 $F$ 上的 $n$ 级矩阵 $A$,若 $f(x) \in F[x]$ 使得 $f(A) = 0$,那么称 $f(x)$ 是 $A$ 的一个零化多项式。
\end{definition}

\begin{theorem}[Hamilton - Cayley 定理]
    设域 $F$ 上的 $n$ 级矩阵,则 $A$ 的特征多项式 $f(\lambda)$ 是 $A$ 的一个零化多项式。
\end{theorem}

利用该定理可以把 $V$ 分解为 $\bfA$ 的非平凡不变子空间的直和:设 $\bfA$ 的特征多项式 $f(\lambda)$ 在 $F[\lambda]$ 中分解为
\[ f(\lambda) = p_1^{r_1}(\lambda) \cdots p_s^{r_s}(\lambda) \]
其中 $p_i(\lambda)$ 是 $F$ 上两两不等的首一不可约多项式,则
\[ V = \Ker(p_1^{r_1}(\bfA)) \oplus \cdots \oplus \Ker(p_s^{r_s}(\bfA)) \]

如果 $f(\lambda)$ 可以分解为
\[ f(\lambda) = (\lambda - \lambda_1)^{r_1} \cdots (\lambda - \lambda_s)^{r_s} \]
其中 $\lambda_i$ 是 $F$ 中两两不等的元素,则
\[ V = \Ker((\bfA - \lambda_1\bfI)^{r_1}) \oplus \cdots \oplus \Ker((\bfA - \lambda_s\bfI)^{r_s}) \]
其中 $\Ker((\bfA - \lambda_j\bfI)^{r_j})$ 称为 $\bfA$ 的根子空间。

\section{*线性变换和矩阵的最小多项式}

\begin{definition}[最小多项式] \index{zuixiaoduoxiangshi@最小多项式}
    设域 $F$ 上线性空间 $V$ 的一个线性变换 $\bfA$ 的所有非零的零化多项式中,次数最低的首项系数为 $1$ 的多项式称为 $\bfA$ 的最小多项式。
\end{definition}

\section{幂零变换的 Jordan 标准型}

\begin{definition}
    若 $\eta \in W$,且存在一个正整数 $t$ 使得 $\bfB^{t-1}\eta \ne 0, \bfB^t\eta = 0$,则称子空间 $\langle \bfB^{t-1}\eta,\bfB\eta,\eta \rangle$ 是由 $\eta$ 生成的 $\bfB$- 强循环子空间。 
\end{definition}

\begin{theorem}
    设域 $F$ 上 $r$ 维线性空间 $W$ 上的幂零变换 $\bfB$,其幂零指数为 $l$,则 $W$ 能分解成 $\dim W_0$ 个 $\bfB$- 强循环子空间的直和,其中 $W_0$ 是 $\bfB$ 的属于特征值 $0$ 的特征子空间。
\end{theorem}

\begin{theorem}
    设域 $F$ 上 $r$ 维线性空间 $W$ 上的幂零变换,其幂零指数为 $l$,则 $W$ 中存在一个基使得 $\bfB$ 在此基下的矩阵 $B$ 为一个 Jordan 形矩阵,其中每个 Jordan 块的主对角元都是 $0$,且级数不超过 $l$;Jordan 块的总数等于 $\dim(\Ker \bfB) = r - \rank(\bfB)$;$t$ 级 Jordan 块的个数 $N(t)$ 为
    \[ N(t) = \rank(\bfB^{t+1}) + \rank(\bfB^{t-1}) - 2 \rank(\bfB^t) \]
    把 $B$ 称为 $\bfB$ 的 Jordan 标准型。除了 Jordan 块的排列次序外,$\bfB$ 的 Jordan 标准型是唯一的。
\end{theorem}

\section{线性变换的有理标准型}

\section{线性函数与对偶空间}

\begin{definition}[线性函数]\index{xianxinghanshu@线性函数}
    设域 $F$ 上的线性空间 $V$,映射 $f : V \to F$ 若对任意的 $k \in F, \alpha,\beta \in V$ 满足
    \begin{equation*}
        \begin{aligned}
            f(\alpha+\beta) &= f(\alpha) + f(\beta) \\
            f(k\alpha) &= kf(\alpha)
        \end{aligned}
    \end{equation*}
    那么称 $f$ 是 $V$ 上的线性函数。
\end{definition}

例如,令
\[ \tr : M_n(F) \to F, A = (a_{ij}) \mapsto \sum_{i=1}^{n} a_{ii} \]
容易验证 $\tr$ 是 $M_n(F)$ 上的线性函数,称它为迹函数。

设 $V$ 是域 $F$ 上的线性空间,由于 $V$ 上的线性函数可看成 $V \to F$ 的映射,因此可以把 $V$ 上所有线性函数的全集记作 $\Hom(V,F)$。它是域 $F$ 上的一个线性空间,称它为 $V$ 上的线性函数空间。

容易看出 $\Hom(V,f) \cong V$,任取 $f \in \Hom(V,F)$ 由于 $f$ 完全被它在 $V$ 上的一个基 $\ji{\alpha}{n}$ 决定,因此对应法则
\[ \sigma : \Hom(V,F) \to F^n, \quad f \mapsto (f(\alpha_1),\cdots,f(\alpha_n)) \]
是一个映射,显然 $\sigma$ 是满射、单射,且保持加法和纯量运算,因此 $\sigma$ 是 $\Hom(V,F) \to F^n$ 上的同构映射,从而 $\sigma^{-1}$ 是 $\Hom(V,F) \to F^n$ 的同构映射。在 $F^n$ 中取标准基 $\ji{\bfeps}{n}$ 则 $\sigma^{-1}(\bfeps_1),\cdots,\sigma^{-1}(\bfeps_1)$ 是 $\Hom(V,F)$ 的一个基。

记 $f_i = \sigma^{_1}(\bfeps_i)$,则 $\sigma(f_i) = \bfeps$。于是有 $f_i(\alpha_j) = \delta_{ij}$。

$\Hom(V,F)$ 的这个基 $\ji{f}{n}$ 称为 $V$ 的基 $\ji{\alpha}{n}$ 的对偶基,把 $\Hom(V,F)$ 称为 $V$ 的对偶空间,记作 $V^*$。

\begin{theorem}
    设域 $F$ 上 $n$ 维线性空间 $V$,在 $V$ 中取两个基:$\ji{\alpha}{n}$ 与 $\ji{\beta}{n}$;$V^*$ 中相对应的对偶基分别为 $\ji{f}{n}$ 与 $\ji{g}{n}$。如果 $V$ 中基 $\ji{\alpha}{n}$ 到基 $\ji{\beta}{n}$ 的过渡矩阵是 $A$,那么 $V^*$ 中基 $\ji{f}{n}$ 到基 $\ji{g}{n}$ 的过渡矩阵 $B$ 为
    \[ B = \transpose{(A^{-1})} \]
\end{theorem}

设域 $F$ 上 $n$ 维线性空间 $V$,取 $V$ 一个基 $\ji{\alpha}{n}$,在 $V^*$ 中取相应的对偶基 $\ji{f}{n}$。映射
\[ \sigma : V \to V^*, \quad \alpha = \sum_{i=1}^n x_i\alpha_i \mapsto \sum_{i=1}^n x_if_i \]
是一个同构映射,把 $\alpha$ 在 $\sigma$ 下的像记作 $f_\alpha$ 或 $\alpha^*$。对 $V$ 中任意向量 $\beta = \sum_{i=1}^n y_i\alpha_i$,由于 $f_i(\beta) = y_i$,因此有
\[ f_\alpha(\beta) = \left( \sum_{i=1}^n x_if_i \right) = \sum_{i=1}^n x_if_i(\beta) = \sum_{i=1}^n x_iy_i \]
这表明 $\alpha$ 在 $\sigma$ 下的像 $f_\alpha$ 在 $\beta$ 处的函数值等于 $\alpha$ 与 $\beta$ 的坐标的对应分量乘积之和。

进一步地,我们可以考虑对偶空间 $V^*$ 的对偶空间 $(V^*)^*$,简记为 $V^{**}$,称 $V^{**}$ 是 $V$ 的双重对偶空间,有
\[ V \cong V^{*} \cong V^{**} \]

设 $\ji{\alpha}{n}$ 是 $V$ 的一个基,$V^*$ 中相应的对偶基为 $\ji{f}{n}$。设同构映射 
\[ \sigma : V \to V^*, \quad \alpha \mapsto f_\alpha \]
同理,有同构映射
\[ \tau : V^* \to V^{**}, \quad f_\alpha \mapsto \alpha^{**} \]
任取 $f \in V$,有
\[ f_\alpha = \sum_{i=1}^n x_if_i, \quad f = \sum_{i=1}^n f(\alpha_i) f_i \]
从而
\[ \alpha^{**}(f) = \sum_{i=1}^n x_if(\alpha_i) = f\left( \sum_{i=1}^n x_i\alpha_i \right) = f(\alpha) \]
由于  $(\tau\sigma)\alpha = \tau(\sigma\alpha) = \tau(f_\alpha) = \alpha^{**}$,因此 $\alpha^{**}(f)$ 的值不依赖于 $V$ 中基的选择。我们称这种不依赖于基的选择的同构映射为标准同构或自然同构。即 $\tau\sigma : V \to V^{**}$ 是自然同构。\index{zirantonggou@自然同构}

于是 $V$ 与 $V^{*}$ 互为对偶空间。
