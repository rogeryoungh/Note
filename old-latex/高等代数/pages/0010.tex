\chapter{具有度量的线性空间}

\section{双线性函数}

\begin{definition}[双线性函数]\index{shuangxianxinghanshu@双线性函数}
    设域 $F$ 上的线性空间 $V$,映射 $f : V \times V \to F$ 如果对任意的 $k_1,k_2 \in F$ 和任意的 $\beta_1,\alpha_2,\beta_1,\beta_2,\alpha,\beta \in V$ 有

    \num{1} $f(k_1\alpha_1 + k_2\alpha_2,\beta) = k_1 f(\alpha_1,\beta) + k_2f(\alpha_2,\beta)$

    \num{2} $f(\alpha,k_1\beta_1 + k_2\beta_2) = k_1 f(\alpha,\beta_1) + k_2f(\alpha,\beta_2)$

    那么称 $f$ 是 $V$ 上一个双线性函数,$f$ 也写成 $f(\alpha,\beta)$。
\end{definition}

即当 $\beta$ 固定时,映射 $\alpha \mapsto f(\alpha,\beta)$ 是 $V$ 上的一个线性函数,记作 $\beta_R$。

即当 $\alpha$ 固定时,映射 $\beta \mapsto f(\alpha,\beta)$ 是 $V$ 上的一个线性函数,记作 $\alpha_L$。

设域 $F$ 上 $n$ 维线性空间 $V$ 中取一个基 $\ji{\alpha}{n}$,设 $V$ 中向量 $\alpha,\beta$ 在此基下的坐标为
\[ X = \transpose{(\ji{x}{n})}, \quad Y = \transpose{(\ji{y}{n})} \]
设 $f$ 是 $V$ 上的一个双线性函数,则
\[ f(\alpha,\beta) = f\left( \sum_{i=1}^n x_i\alpha_i,\sum_{j=1}^ny_j\alpha_j \right) = \sum_{i=1}^n\sum_{j=1}^n x_iy_jf(\alpha_i,\alpha_j) \]
令
\[ A = \left(\begin{matrix}
    f(\alpha_1,\alpha_1) & f(\alpha_1,\alpha_2) & \cdots & f(\alpha_1,\alpha_n)\\
    f(\alpha_2,\alpha_1) & f(\alpha_2,\alpha_2) & \cdots & f(\alpha_2,\alpha_n)\\
    \vdots                 & \vdots                 &        & \vdots                \\
    f(\alpha_n,\alpha_1) & f(\alpha_n,\alpha_2) & \cdots & f(\alpha_n,\alpha_n)
\end{matrix}\right) \]

称 $A$ 是双线性函数 $f$ 在基 $\ji{\alpha}{n}$ 下的度量矩阵,它是由 $f$ 及基 $\ji{\alpha}{n}$ 唯一决定的。于是有
\[ f(\alpha,\beta) = \transpose{X}AY \]
反之,任给域 $F$ 上一个 $n$ 级矩阵 $A = (a_{ij})$,定义映射 $f : V \times V \to F$ 如下
\[ f(\alpha,\beta) = \transpose{X}AY = \sum_{i=1}^n\sum_{j=1}^na_{ij}x_iy_j \]
则 $f$ 是 $V$ 上的一个双线性函数,且 $f$ 在基 $\ji{\alpha}{n}$ 下的度量矩阵为 $A$。

因此若 $\transpose{X}AY = \transpose{X}BY$,则 $A = B$。

称表达式 $\transpose{X}AY$ 为 $X$ 与 $Y$ 的双线性形。

\begin{theorem}
    设域 $F$ 上 $n$ 维线性空间 $V$ 上的一个双线性函数,取 $V$ 中两个基 $\ji{\alpha}{n}$ 与 $\ji{\beta}{n}$,设
    \[ (\ji{\beta}{n}) = (\ji{\alpha}{n})P \]
    且 $f$ 在这两个基的度量矩阵分别为 $A,B$,则
    \[ B = \transpose{P}AP \]
\end{theorem}

即 $f$ 在 $V$ 的不同基下的度量矩阵是合同的,他们有相同的秩,于是称度量矩阵的秩为 $f$ 的矩阵秩,记作 $\rank_m f$。

设域 $F$ 上线性空间 $V$ 上双线性函数 $f$,则称 $V^*$ 的子空间
\[ \langle \alpha_L,\beta_R \mid \alpha,\beta \in V \rangle \]
称为 $f$ 的秩空间,把 $f$ 的秩空间的维数称为 $f$ 的秩,记作 $\rank f$。

可以证明,$f$ 的矩阵秩不超过 $f$ 的秩。

\begin{definition}
    设域 $F$ 上线性空间 $V$ 上的双线性函数 $f$,则 $V$ 的子集
    \[ \{ \alpha \in V \mid f(\alpha,\beta) = 0 \} \]
    称为 $f$ 在 $V$ 中的做根,记作 $\rad_{L} V$,$V$ 的另一个子集
    \[ \{ \beta \in V \mid f(\alpha,\beta) = 0 \} \]
    称为 $f$ 在 $V$ 中的右根,记作 $\rad_{R} V$。
\end{definition}

容易验证,$f$ 在 $V$ 中的左根和右根都是 $V$ 的子空间。

\begin{definition}
    如果 $V$ 上双线性函数 $f$ 的做根和右根都是零子空间,那么称 $f$ 是非退化的。
\end{definition}

\begin{theorem}
    域 $F$ 上 $n$ 维线性空间 $V$ 上的双线性函数 $f$ 是非退化的,当且仅当 $f$ 在 $V$ 的一个基下的度量矩阵是满秩矩阵。
\end{theorem}

\begin{definition}
    设域 $F$ 上线性空间 $V$ 上的一个双线性函数,如果
    \[ f(\alpha,\beta) = f(\beta,\alpha) \]
    那么称 $f$ 是对称的,如果
    \[ f(\alpha,\beta) = -f(\beta,\alpha) \]
    那么称 $f$ 是反对称的(斜对称的)。
\end{definition}

\begin{theorem}
    设特征不为 $2$ 的域 $F$ 上 $n$ 维线性空间 $V$ 上的对称双线性函数 $f$,则 $V$ 中存在一个基使得 $f$ 在此基下的度量矩阵为对角矩阵,
\end{theorem}

\begin{theorem}
    设特征不为 $2$ 的域 $F$ 上 $n$ 维线性空间 $V$ 上的反对称双线性函数 $f$,则存在 $V$ 的一个基,把它记成 $\delta_{1},\delta_{-1},\cdots,\delta_{r},\delta_{-r},\eta_1,\cdots,\eta_s$(其中 $0 \leqslant 2r \leqslant n, s = n-2r$),使得 $f$ 在这个基下的度量矩阵具有形式
    \[ \diag\left\{
        \left(\begin{matrix} 0 & 1 \\ -1 & 0 \end{matrix}\right),
        \cdots,
        \left(\begin{matrix} 0 & 1 \\ -1 & 0 \end{matrix}\right),
        0,\cdots,0
    \right\} \]
\end{theorem}

\begin{definition}
    设域 $F$ 上的线性空间 $V$,映射 $q : V \to F$ 称为 $V$ 上的二次函数,如果存在 $V$ 上的一个对称双线性函数 $f$,使得
    \[ q(\alpha) = f(\alpha,\alpha) \]
\end{definition}

显然,对于一个对称双线性函数 $f$ 就有唯一的一个二次函数 $q$。

\begin{theorem}
    设特征不为 $2$ 的域 $F$ 上的线性空间 $V$,$q$ 是 $V$ 上的一个二次函数,则存在 $V$ 上唯一的对称双线性函数 $f$ 使得
    \[ f(\alpha,\alpha) = q(\alpha) \]
\end{theorem}

于是设域 $F$ 上 $n$ 维线性空间 $V$ 上的对称双线性函数 $f$ 和其对应的二次函数 $q$。设 $f$ 在 $V$ 的一个基 $\ji{\alpha}{n}$ 下的度量矩阵 $A = (a_{ij})$,则对于 $\alpha = (\ji{\alpha}{n})X,\beta = (\ji{\alpha}{n})Y$,有
\[ f(\alpha,\beta) = \transpose{X}AY \]
从而有
\[ q(\alpha) = f(\alpha,\alpha) = \transpose{X}AX \]
即 $q$ 在基 $\ji{\alpha}{n}$ 下的表达式是 $n$ 元二次型 $\transpose{X}AX$,称其中的对称矩阵 $A$ 为二次函数 $q$ 在基 $\ji{\alpha}{n}$ 下的矩阵。于是可以用二次型的理论研究双线性函数,也可以用对称双线性函数来研究二次型。

\begin{theorem}[惯性定理]
    实数域上任意一个 $n$ 元二次型都可以经过非退化线性替换化成规范形,并且规范形是唯一的 。
\end{theorem}

\begin{theorem}[Witt 消去律的推广]
    设特征不为 $2$ 的域 $F$ 上的 $n$ 级对称矩阵 $A_1,A_2$,$m$ 级对称矩阵 $B_1,B_2$。如果
    \[ \left(\begin{matrix}
        A_1 & 0   \\
        0   & B_1 \\
    \end{matrix}\right) \simeq \left(\begin{matrix}
        A_2 & 0   \\
        0   & B_2 \\
    \end{matrix}\right) \]
    且 $A_1 \simeq A_2$,那么 $B_1 \simeq B_2$。
\end{theorem}

\paragraph{双线性函数空间}

\begin{definition}
    设域 $F$ 上的线性空间 $V$,我们把 $V$ 上所有双线性函数组成的集合记作 $T_2(V)$,容易验证 $T_2(V)$ 对域函数的加法和纯量乘法成为域 $F$ 上的一个线性空间,称为 $V$ 上的双线性函数空间。
\end{definition}

\begin{theorem}
    设特征不为 $2$ 的域 $F$ 上的线性空间,则
    \[ T_2(V) = S_2(V) \oplus A_2(V) \]
\end{theorem}



\section{欧几里得空间}

\begin{definition}[正定的]\index{zhengdingde@正定的}
    设实线性空间 $V$ 上的对称双线性函数,如果对任意 $\alpha \in V$ 有 $f(\alpha,\alpha) \geqslant 0$,等号成立当且仅当 $\alpha = 0$,那么称 $f$ 是正定的。
\end{definition}

\begin{definition}[内积] \index{neiji@内积}
    设实数域 $\RR$ 上的一个线性空间 $V$,$V$ 上的一个正定的对称双线性函数称为 $V$ 上的一个内积。
\end{definition}

习惯上把内积 $f(\alpha,\beta)$ 记作 $(\alpha,\beta)$。

\begin{definition}
    设实数域 $\RR$ 上的一个线性空间 $V$,若给定了 $V$ 上的一个内积,那么称 $V$ 是一个实内积空间。有限维的实内积空间称为欧几里得 Euclid 空间,并且把线性空间 $V$ 的维数称为 Euclid 空间 $V$ 的维数。
\end{definition}

\paragraph{实内积空间中的度量概念}

\begin{definition}
    非负实数 $\sqrt{(\alpha,\alpha)}$ 称为向量 $\alpha$ 的长度,记作 $|\alpha|$。
\end{definition}

长度为 $1$ 的向量称为单位向量。如果 $\alpha \ne 0$,那么 $\mfrac{\alpha}{|\alpha|}$ 是一个单位向量。把 $\alpha$ 变成 $\mfrac{\alpha}{|\alpha|}$ 称为把 $\alpha$ 单位化。

\begin{theorem}[Cauchy - Schwarz 不等式]\index{cauchyschwarzbudengshi@Cauchy - Schwarz 不等式}
    在实内积空间 $V$ 对任意向量 $\alpha,\beta$ 有
    \[ |(\alpha,\beta)| \leqslant |\alpha| |\beta| \]
    等号成立当且仅当 $\alpha,\beta$ 线性相关。
\end{theorem}

\begin{proposition}[Cauchy 不等式]
    对于任意的两组实数 $\ji{\alpha}{n}$ 与  $\ji{\beta}{n}$ 有
    \[ |a_1b_1 + \cdots + a_nb_n| \leqslant \sqrt{a_1^2 + \cdots + a_n^2}\sqrt{b_1^2 + \cdots + b_n^2} \]
    等号成立当且仅当 $(\ji{\alpha}{n})$ 与 $(\ji{\beta}{n})$ 线性相关。
\end{proposition}

\begin{proposition}[Schwarz 不等式]
    对于任意的 $f,g \in C[a,b]$ 有
    \[ \left| \int_a^b f(x)g(x) \dd x \right|^2 \leqslant \left| \int_a^b f^2(x) \dd x \right| \left| \int_a^b g^2(x) \dd x \right| \]
    等号成立当且仅当 $(\ji{\alpha}{n})$ 与 $(\ji{\beta}{n})$ 线性相关。
\end{proposition}

\begin{definition}
    实内积空间 $V$ 中,两个非零向量 $\alpha$ 与 $\beta$ 的夹角 $\langle\alpha,\beta\rangle$ 规定为
    \[ \langle\alpha,\beta\rangle \coloneqq \arccos \frac{(\alpha,\beta)}{|\alpha||\beta|} \]
\end{definition}

\begin{definition}
    在实内积空间 $V$ 中,若 $(\alpha,\beta) = 0$,那么称 $\alpha$ 与 $\beta$ 正交,记作 $\alpha \bot \beta$。
\end{definition}

\begin{proposition}
    在实内积空间 $V$ 中,三角形不等式成立,即对于任意的 $\alpha,\beta \in V$ 有
    \[ |\alpha+\beta| \leqslant |\alpha| + |\beta| \]
\end{proposition}

\begin{proposition}
    在实内积空间 $V$ 中,勾股定理成立,即如果 $\alpha$ 与 $\beta$ 正交,则
    \[ |\alpha + \beta|^2 = |\alpha|^2 + |\beta|^2 \]
\end{proposition}

\begin{proposition}
    在实内积空间 $V$ 中,余弦定理成立,即对于三个非零向量 $\alpha,\beta,\gamma$ 满足 $\gamma = \beta - \alpha$,则
    \[ |\gamma|^2  = |\alpha|^2 + |\beta|^2 - 2 |\alpha| |\beta| \cos\langle\alpha,\beta\rangle \]
\end{proposition}

\begin{definition}
    设 $E$ 是一个非空集合,若其上存在映射 $d : E \times E \to \RR$,如果对任意 $x,y,z\in E$ 都有
    \num{1} 对称性:$d(x,y) = d(y,x)$。
    \num{2} 正定性:$d(x,y) \geqslant $0,等号成立当且仅当 $x = y$。
    \num{3} 三角形不等式:$d(x,z) \leqslant d(x,y) + d(y,z)$。
    那么称 $d$ 是一个距离,称集合 $E$ 是一个度量空间,把 $d(x,y)$ 称为 $x$ 与 $y$ 之间的距离。
\end{definition}

\begin{definition}
    在 $n$ 维 Euclid 空间 $V$ 中,由 $n$ 个两两正交的非零向量组成的基称为 $V$ 的一个正交基,若该基皆为单位向量,称为 $V$ 的一个标准正交基。
\end{definition}

由于内积是正定的对称双线性函数,因此 $V$ 中存在一个基 $\ji{\eta}{n}$ 使得内积在此基下的度量矩阵为单位矩阵 $I$。从而
\[ (\eta_i,\eta_j) = \delta_{ij} \]
因此 $\ji{\eta}{n}$ 是 $V$ 的一个标准正交基。

更具体的,取 $V$ 的一个基 $\ji{\alpha}{n}$ 令 $\beta_1 = \alpha$,且
\[ \beta_n = \alpha_n - \sum_{j=1}^{n-1} \frac{(\alpha_n,\beta_j)}{(\beta_j,\beta_j)} \beta \]
则 $\ji{\beta}{n}$ 是 $V$ 的一个正交基。令
\[ \eta_i = \frac{\beta}{|\beta|} \]
则 $\ji{\eta}{n}$ 是 $V$ 的一个标准正交基。

第一步称为 Schmitdt 正交化,第二步称为单位化。\index{schmitdtzhengjiaohua@Schmitdt 正交化}

\begin{theorem}
    设 $\ji{\eta}{n}$ 是 $n$ 维 Euclid 空间 $V$ 的一个标准正交基,则对于任意的 $\alpha \in V$ 有
    \[ \alpha = \sum_{i=1}^n (\alpha,\eta_i)\eta_i \]
\end{theorem}

即 $\alpha$ 在标准正交基 $\ji{\eta}{n}$ 下的坐标的第 $i$ 个分量等于 $(\alpha,\eta_i)$。此式称为 $\alpha$ 的 Fourier 展开,其中每个系数 $(\alpha,\eta_i)$ 都称为 $\alpha$ 的 Fourier 系数。

\begin{definition}
    设实内积空间 $V,V'$,若存在双射 $\sigma : V \to V'$ 使得对于任意的 $\alpha,\beta \in V$ 和 $k \in \RR$ 有
    \begin{equation*}
        \begin{aligned}
            \sigma(\alpha + \beta) &= \sigma(\alpha) + \sigma(\beta)\\
            \sigma(k\alpha) &= k\sigma(\alpha)\\
            (\alpha,\beta) &= (\sigma(\alpha),\sigma(\beta))
        \end{aligned}
    \end{equation*}
    那么称 $\sigma$ 是 $V \to V'$ 的一个同构映射,此时称 $V$ 与 $V'$ 同构,记作 $V \cong V'$。
\end{definition}

若 $\sigma$ 保持加法和数量乘法,那么称为线性同构;若线性同构还保持内积,则称它为一个保距同构。

\begin{theorem}
    两个 Euclid 空间同构的充要条件是它们的维数相同。
\end{theorem}

\section{正交补,正交投影}

\begin{definition}
    设实内积空间 $V$ 的一个非空子集 $S$。把 $V$ 中与 $S$ 中每一个向量都正交的所有向量的全体称作 $S$ 的正交补,记作 $S^\bot$。即
    \[ S^\bot \coloneqq \{ \alpha \in V \mid (\alpha,\beta) = 0 \} \]
\end{definition}

\begin{theorem}
    设实内积空间 $V$ 的一个有限维子空间 $U$,则
    \[ V = U \oplus U^\bot \]
\end{theorem}

设实内积空间 $V$ 的一个子空间,若 $V = U \oplus U^\bot$,那么有平行于 $U^\bot$ 在 $U$ 上的投影 $\bfP_U$。我们把这个投影 $\bfP_U$ 称作 $V$ 在 $U$ 上的正交投影;把  $\alpha$ 在 $\bfP_U$ 下的像 $\alpha_1$ 称为 $\alpha$ 在 $U$ 上的正交投影。此时 $\alpha = \alpha_1 + \alpha_2$。

即
\[ \bfP_U(\alpha) = \alpha_1 \Leftrightarrow \alpha - \alpha_1 \in U^\bot \]

\begin{theorem}
    设实内积空间 $V$ 的一个子空间 $U$,且 $V = U \oplus U^\bot$,则对于 $\alpha \in V$,则 $\alpha_1 = \bfP_U(\alpha) \in U$ 的充要条件是
    \[ d(\alpha,\alpha_1) \leqslant d(\alpha,\gamma) \]
\end{theorem}

\begin{definition}\index{zuijiabijinyuan@最佳逼近元}
    设实内积空间 $V$ 的子空间 $U$,若对于 $\alpha \in V$ 存在 $\delta \in U$ 有
    \[ d(\alpha,\delta) \leqslant d(\alpha,\gamma) \]
    那么称 $\delta$ 是 $\alpha$ 在 $U$ 上的最佳逼近元。
\end{definition}

设实内积空间 $V$ 的一个无限维子空间 $U$,如果 $\alpha \in V$ 在 $U$ 上的最佳逼近元存在(此时必唯一),那么称 $\delta$ 为 $\alpha$ 在 $U$ 上的正交投影。如果 $V$ 中每个向量 $\alpha$ 都有在 $U$ 上的正交投影 $\delta$,那么把 $\alpha$ 对应到 $\delta$ 的映射称为 $V$ 在 $U$ 上的正交投影。

\section{正交变换与对称变换}

\begin{definition}
    设实内积空间 $V$ 到自身的满射 $\bfA$,如果保持向量的内积不变,即
    \[ (\bfA\alpha,\bfA\beta) = (\alpha,\beta) \]
    那么称 $\bfA$ 是 $V$ 上的一个正交变换。
\end{definition}

\begin{proposition}
    正交变换 $\bfA$ 具有特性:

    \num{1} 保持向量长度。

    \num{2} 保持两个非零向量的夹角不变。

    \num{3} 保持正交性。

    \num{4} 一定是线性变换。

    \num{5} 保持向量间的距离不变。

    \num{6} 一定是单射,一定可逆。
\end{proposition}

\begin{theorem}
    设 $n$ 为 Euclid 空间 $V$ 上的线性变换 $\bfA$,$\bfA$ 在 $V$ 的标准正交基下的矩阵为 $A$,则下列描述等价:

    \num{1} $\bfA$ 是正交变换。

    \num{2} $\bfA$ 把 $V$ 的标准正交基映成标准正交基。

    \num{3} $A$ 是正交矩阵。
\end{theorem}

由于正交矩阵 $A$ 的行列式为 $\pm 1$,则把行列式为 $1$ 的矩阵称为第一类(或旋转),行列式等于 $-1$ 的称为第二类的。

$n$ 维线性空间的任意一个 $n-1$ 维子空间称为一个超平面。

\begin{definition}
    设 $n$ 维 Euclid 空间 $V$ 中的一个单位向量,$\bfP$ 是 $V$ 在 $\langle \eta \rangle$ 上的正交投影,令
    \[ \bfA = \bfI - 2 \bfP \]
    则 $\bfA$ 称为关于超平面 $\langle \eta \rangle^\bot$ 的镜面反射。
\end{definition}

\begin{definition}
    实内积空间 $V$ 上的变换 $\bfA$ 如果满足
    \[ (\bfA\alpha,\beta) = (\alpha,\bfA\beta) \]
    那么称 $\bfA$ 是 $V$ 上的对称变换。
\end{definition}

\section{*酉空间}

在复数域中引入度量概念。若复线性空间 $V$ 上的双线性函数 $f$ 具有性质
\[ f(\alpha,\beta) = \overline{f(\beta,\alpha)} \]
这个性质称为 Hermite 性。

\section{*正交空间与辛空间}

\begin{definition}
    设域 $F$ 上的线性空间 $V$ 上的一个对称双线性函数 $f$,那么称 $f$ 是 $V$ 上的一个内积(或度量),称 $V$ 是一个正交空间。用 $(V,f)$ 表示。
\end{definition}

如果 $f$ 是非退化的,则称 $(V,f)$ 是正则的,否则称为非正则的。

\begin{definition}
    在正交空间 $(V,f)$ 中,如果 $f(\alpha,\beta) = 0$,那么称 $\alpha$ 与 $\beta$ 正交,记作 $\alpha \bot \beta$。
\end{definition}

在正交空间 $(V,f)$ 中,一个非零向量 $\alpha$ 称为迷向的,如果 $f(\alpha,\alpha) = 0$,否则称为非迷向的。

正交空间 $(V,f)$ 包含一个迷向向量,则 $(V,f)$ 称为迷向的,否则为非迷向的。若 $V$ 中所有非零向量都是迷向的,则称为全迷向的。

\section{*正交群,酉群,辛群}

\begin{definition}
    设 $G$ 是一个非空集合,若 $G$ 上的乘法运算满足

    \num{1} 结合律:$a(bc) = (ab)c$。

    \num{2} 单位元:存在 $e \in G$ 使得 $ea = ae = a$。

    \num{3} 逆元:任取 $a \in G$ 总存在 $b \in G$ 使得 $ab = ba = 1$。

    那么称 $G$ 是一个群。
\end{definition}

如果群 $G$ 的运算还满足交换律,那么称 $G$ 为交换群,或 Abel 群。











