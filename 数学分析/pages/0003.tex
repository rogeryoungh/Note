\chapter{函数极限}

\section{函数极限的概念}

\begin{definition}
    设 $f$ 为定义在 $[a,+\infty)$ 上的函数,$A$ 为定数。若对任给的 $\varepsilon>0$,存在正数 $M=M(\varepsilon) \geqslant a$,使得当 $x>M$ 时,有
    $$|f(x)-A| < \varepsilon$$
    则称函数 $f$ 当 $x$ 趋于 $+\infty$ 时以 $A$ 为极限,记作
    $$\lim_{x\to +\infty}f(x) = A\ \text{或}\ f(x) \to A(x\to +\infty)$$
\end{definition}

类似的有 $\displaystyle\lim_{x\to -\infty}f(x)$ 和 $\displaystyle\lim_{x\to \infty}f(x)$。

不难证明
$$\lim_{x \to \infty}f(x) = A \Leftrightarrow \lim_{x \to -\infty}f(x)=\lim_{x \to +\infty}f(x)=A$$

\begin{definition}
    设函数 $f$ 在 $U^\circ(x_0;\delta')$ 内有定义,$A$ 为定数。若对任给的 $\varepsilon>0$,存在正数 $\delta<\delta'$,使得当 $0<|x-x_0|<\delta$ 时,有 $|f(x)-A|<\varepsilon$,则称函数 $f$ 当 $x$ 趋于 $x_0$ 时以 $A$ 为极限,记作
    $$\lim_{x\to x_0}f(x) = A\ \text{或}\ f(x)\to A(x\to x_0)$$
\end{definition}

\begin{definition}
    设函数 $f$ 在 $U_+^\circ(x_0;\delta')$ 内有定义,$A$ 为定数。若对任给的 $\varepsilon>0$,存在正数 $\delta<\delta'$,使得当 $x_0<x<x_0+\delta$ 时,有 $|f(x)-A|<\varepsilon$,则称函数 $f$ 当 $x$ 趋于 $x_0^+$ 时以 $A$ 为极限,记作 
    $$\lim_{x\to x_0^+}f(x) = A\ \text{或}\ f(x)\to A(x\to x_0^+)$$
\end{definition}

类似的还有左极限 $\displaystyle\lim_{x\to x_0^-}f(x)$,统称为单侧极限。又可记为
$$f(x_0+0) = \lim_{x\to x_0^+}f(x)\ \text{与}\ f(x_0-0) = \lim_{x\to x_0^-}f(x)$$

同理还有
$$\lim_{x \to x_0}f(x) = A \Leftrightarrow \lim_{x \to x_0^+}f(x)=\lim_{x \to x_0^-}f(x)=A$$

\section{函数极限的性质}

\begin{theorem}[唯一性]
    若极限 $\displaystyle\lim_{x\to x_0}f(x)$ 存在,则此极限是唯一的。
\end{theorem}

\begin{theorem}[局部有界性]
    若极限 $\displaystyle\lim_{x\to x_0}f(x)$ 存在,则 $f$ 在 $x_0$ 的某空心邻域 $U^\circ(x_0)$ 上有界。
\end{theorem}

\begin{theorem}[保不等式性]
    设 $\lim_{x\to x_0}f(x)$ 与 $\lim_{x\to x_0}g(x)$ 均存在。若存在正数 $N_0$,使得当 $n>N_0$ 时,有 $a_n\leqslant b_n$,则 $\displaystyle\lim_{n\to \infty}a_n \leqslant \lim_{n\to \infty}b_n$。
\end{theorem}

\begin{theorem}[迫敛性]
    设 $\displaystyle\lim_{x\to x_0}f(x) = \lim_{x\to x_0}g(x) = A$,且在某 $U^\circ(x_0;\delta')$ 上有
    $$f(x)\leqslant h(x) \leqslant g(x)$$
    则 $\lim_{x\to x_0}h(x) = A$。
\end{theorem}

\begin{theorem}[四则运算法则]
    若 $\displaystyle\lim_{x\to x_0}f(x)$ 与 $\displaystyle\lim_{x\to x_0}g(x)$ 均存在,则
    $$\lim_{x\to x_0}[f(x)\pm g(x)] = \lim_{x\to x_0}f(x) + \lim_{x\to x_0}g(x)$$
    $$\lim_{x\to x_0}[f(x)g(x)] = \lim_{x\to x_0}f(x) \cdot \lim_{x\to x_0}g(x)$$
    若 $\displaystyle\lim_{x\to x_0}g(x)\ne 0$,则
    $$\lim_{x\to x_0}\frac{f(x)}{g(x)} = \frac{\lim_{x\to x_0}f(x)}{\lim_{x\to x_0}g(x)}$$
\end{theorem}

\section{函数极限存在的条件}

\begin{theorem}[海涅(Heine)定理,归结原则]
    若 $f(x)$ 在 $U^\circ(x_0;\delta')$ 上有定义。$\displaystyle\lim_{x\to x_0}f(x)$ 存在的充要条件是:任何含于 $U^\circ(x_0;\delta')$ 且以 $x_0$ 为极限的数列 $\{x_n\}$,极限 $\displaystyle\lim_{x\to x_0}f(x_n)$ 都存在且相等。
\end{theorem}

即若对任何 $x_n\to x_0(n\to \infty)$ 有 $\displaystyle\lim_{n\to \infty}f(x_n) = A$,则 $\displaystyle\lim_{x\to x_0}f(x)=A$。

\begin{theorem}
    设 $f(x)$ 在点 $x_0$ 的某空心右邻域 $U_+^\circ(x_0)$ 有定义,则 $\displaystyle\lim_{x\to x_0^+}f(x)=A$ 的充要条件是:对任何以 $x_0$ 为极限的递减数列 $\{x_n\}\subset U_+^\circ(x_0)$,有 $\displaystyle\lim_{n\to \infty}f(x_n) = A$。
\end{theorem}

\begin{theorem}
    设 $f(x)$ 为定义在 $U_+^\circ(x_0)$ 上的单调有界函数,则右极限 $\displaystyle\lim_{x\to x_0^+}f(x)=A$ 存在。
\end{theorem}

\begin{theorem}[Cauchy 准则]
    设 $f(x)$ 在 $U^\circ(x_0;\delta')$ 上有定义,则 $\displaystyle\lim_{x\to x_0}f(x)$ 存在的充要条件是:任给 $\varepsilon > 0$,存在正数 $\delta(<\delta')$,使得对任何 $x',x''\in U^\circ(x_0,\delta)$,有 $|f(x')-f(x'')|<\varepsilon$。
\end{theorem}

\section{两个重要的极限}

$$\lim_{x\to 0}\frac{\sin x}{x} = 1$$
$$\lim_{x\to \infty}\left(1+\frac{1}{x}\right)^x = \ee$$

\section{无穷小量与无穷大量}

\begin{definition}[无穷小量]
    设函数 $f$ 在某 $U^\circ(x_0)$ 上有定义,若 $\displaystyle\lim_{x\to x_0}f(x)=0$,则称 $f$ 为当 $x\to x_0$ 时的无穷小量。
\end{definition}

\begin{definition}[有界量]
    设函数 $f$ 在某 $U^\circ(x_0)$ 上有界,则称 $f$ 为当 $x\to x_0$ 时的有界量。
\end{definition}

无穷小量收敛于 $0$ 的速度有快有慢。设当 $x\to x_0$ 时,$f$ 与 $g$ 均为无穷小量。

若 $\displaystyle\lim_{x\to x_0}\mfrac{f(x)}{g(x)} = 0$,则称当 $x\to x_0$ 时 $f$ 为 $g$ 的高阶无穷小量,或称 $g$ 为 $f$ 的低阶无穷小量。

记作
$$f(x)=o(g(x))(x\to x_0)$$
特别地,$f$ 为当 $x\to x_0$ 时的无穷小量记作
$$f(x)=o(1)(x\to x_0)$$

若存在正数 $K$ 和 $L$,使得在某 $U^\circ(x_0)$ 上有
$$K\leqslant \left|\frac{f(x)}{g(x)}\right| \leqslant L$$
则称 $f$ 与 $g$ 为当 $x\to x_0$ 时的同阶无穷小量。特别当
$$\lim_{x\to x_0}\frac{f(x)}{g(x)} = c \ne 0$$
时,$f$ 与 $g$ 必为同阶无穷小量。

若 $\displaystyle\lim_{x\to x_0}\mfrac{f(x)}{g(x)} = 1$ 则称 $f$ 与 $g$ 是当 $x\to x_0$ 时的等价无穷小量。记作
$$f(x) \sim g(x) (x\to x_0)$$

注意并不是任何两个无穷小量都可以进行这种阶的比较。例如 $x\to 0$ 时,$x\sin\dfrac{1}{x}$ 和 $x^2$ 都是无穷小量,但它们的比都不是有界量。

\begin{theorem}
    设函数 $f,g,h$ 在 $U^\circ(x_0)$ 上有定义,且有 $f(x) \sim g(x)(x\to x_0)$,则

    1.若 $\displaystyle\lim_{x\to x_0}f(x)h(x) = A$,则 $\displaystyle\lim_{x\to x_0}g(x)h(x) = A$。

    2.若 $\displaystyle\lim_{x\to x_0}\frac{h(x)}{f(x)}=B$,则 $\displaystyle\lim_{x\to x_0}\frac{h(x)}{g(x)}=B$
\end{theorem}

\begin{definition}[无穷大量]
    设函数 $f$ 在某 $U^\circ(x_0)$ 上有定义,若对任给的 $G>0$,存在 $\delta>0$,使得当 $x\in U^\circ(x_0;\delta)\subset U^\circ(x_0)$ 时,有 $|f(x)|>G$,则称函数 $f$ 当 $x\to x_0$ 时有非正常极限 $\infty$,记作 $\displaystyle\lim_{x\to x_0}f(x) = \infty$。
\end{definition}

\section{常见等价无穷小}

实际上这些等价无穷小就是泰勒展开。

% \begin{equation*}
%     \begin{aligned}
%         \frac{1}{1-x} &= \sum_{k=0}^\infty x^n,(-1,1) \\
%         &= 1 + x + x^2 + x^3 + x^4 + x^5 + x^6 + O(x^7)\\
%         \ln(1+x) &= \sum_{k=0}^\infty\frac{(-1)^k}{k+1}x^{k+1},(-1,1] \\
%         &= x - \frac{x^2}{2} + \frac{x^3}{3} - \frac{x^4}{4} + \frac{x^5}{5} - \frac{x^6}{6} + \frac{x^7}{7} + O(x^8)\\
%         \sin x &= \sum_{k=0}^\infty \frac{(-1)^k}{(2k+1)!}x^{2k+1},\RR \\
%         &= x - \frac{x^3}{6} + \frac{x^5}{120} - \frac{x^7}{5040} + \frac{x^9}{362880}+ O(x^{11})\\
%         \cos x &= \sum_{k=0}^\infty \frac{(-1)^k}{(2k)!}x^{2k},\RR \\
%         &= 1 - \frac{x^2}{2} + \frac{x^4}{24} - \frac{x^6}{720} + \frac{x^8}{40320} + \frac{x^{10}}{3628800} + O(x^{12})\\
%         \ee^x &= \sum_{k=0}^\infty\frac{1}{n!}x^n,\RR \\
%         &= 1 + x + \frac{x^2}{2} + \frac{x^3}{6} + \frac{x^4}{24} + \frac{x^5}{120} + \frac{x^6}{720} + \frac{x^7}{5040} + \frac{x^8}{40320} + O(x^{10})\\
%         \tan x &= \sum_{k=1}^\infty \frac{(-4)^n(1-4^n)B_{2n}}{(2n)!}x^{2n-1},(-\frac{\pi}{2},\frac{\pi}{2}) \\
%         &= x + \frac{x^3}{3} + \frac{2x^5}{15} + \frac{17x^{7}}{315} + \frac{67x^9}{2835} + O(x^{11})
%     \end{aligned}
% \end{equation*}

\begin{equation*}
    \begin{aligned}
        \frac{x}{1-x} &= x + x^2 + x^3 + x^4 + x^5 &+ O(x^6)\\
        \ln(1+x)      &= x - \frac{x^2}{2} + \frac{x^3}{3} - \frac{x^4}{4} + \frac{x^5}{5} &+ O(x^6)\\
        \sin x        &= x - \frac{x^3}{6} + \frac{x^5}{120} &+ O(x^{7})\\
        1- \cos x     &= \frac{x^2}{2} - \frac{x^4}{24} &+ O(x^{6})\\
        \ee^x-1       &= x + \frac{x^2}{2} + \frac{x^3}{6} + \frac{x^4}{24} + \frac{x^5}{120} &+  O(x^{6})\\
        \tan x        &= x + \frac{x^3}{3} + \frac{2x^5}{15} &+ O(x^{7})\\
        \sqrt{x+1}-1  &= \frac{x}{2}-\frac{x^2}{8} +\frac{x^3}{16}-\frac{5 x^4}{128} +\frac{7 x^5}{256} &+  O(x^{6})\\
        \arcsin x     &= x + \frac{x^3}{6} + \frac{3x^5}{40} &+ O(x^7)\\
        \arctan x     &= x - \frac{x^3}{3} + \frac{x^5}{5} &+ O(x^7)
    \end{aligned}
\end{equation*}