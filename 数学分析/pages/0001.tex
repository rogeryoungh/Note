\chapter{实数集与函数}

我初次用的书是华师的数分,后面还会加一些别的书的内容。可能会有点乱,有空了做整理。

集合论与函数和映射视作熟知的。若无额外说明,皆在 $\RR$ 下。

\section{实数}

有理数和无理数统称实数,有理数可用分数形式 $\mfrac{p}{q}$ 表示,也可用有限十进小数或无限十进循环小数表示;而无限十进不循环小数则成为无理数。

为了让任意实数都可用一个确定的无限小数来表示,如下规定:

对于正有限小数(包括正整数) $x$,当 $x=a_0.a_1a_2\cdots{}a_n$ 时,其中 $0\ge a_i \ge 9,i=1,2,\cdots,n,a_n\neq 0$,$a_0$ 为非负整数,即 $$x=a_0.a_1a_2\cdots{}(a_n-1)9999\cdots,$$ 而当 $x=a_0$ 为正整数时,则记 $$x=(a_0-1).9999\cdots$$

对于负有限小数(包括负整数)$y$,则先将 $-y$ 表示为无限小数,再在所得无限小数之前加负号。并规定 $0$ 表示为 $0.0000\cdots$。

%\begin{definition}
	%给定两个非负实数 $$x=a_0.a_1a_2\cdots{}a_n\cdots,\quad y=b_0.b_1b_2\cdots{}b_n\cdots,$$ 其中 $a_0,b_0$ 为非负整数,$a_k,b_k(k=1,2,\cdots)$ 为整数,$0\ge a_k\ge9,0\ge b_k\ge9$。若有 $$a_k=b_k,k=0,1,2,\cdots,$$ 则称 $x$ 与 $y$ 相等,记为 $x=y$;若 $a_0>b_0$ 或存在非负实数 $l$,使得 $$a_k=b_k(k=0,1,2,\cdots,l)\text{而}a_{l+1}>b_{l+1},$$ 则称 $x$ 大于 $y$ 或 $y$ 小于 $x$,分别记为 $x>y$ 或 $y<x$。

	%对于负实数 $x,y$,若按上述规定有 $-x=-y$,称作 $x=y$。若按上述有 $-x>-y$,则
%\end{definition}

实数有以下性质:

\begin{enumerate}
	\item 实数集 $\RR$ 对加、减、乘、除(除数不为 $0$)四则运算是封闭的。
	\item 实数集是有序的:任意 $a,b$ 必满足三个关系之一($a<b,a=b,a>b$)。
	\item 实数的大小关系具有传递性:若 $a>b,b>c$,则有 $a>c$。
	\item 实数具有阿基米德性:对任何 $a,b\in\RR$,若 $b>a>0$,则存在正整数 $n$,使得 $na>b$。
	\item 实数集具有稠密性:任意 $a,b$ 之间必存在另一个实数,可以是有理数,也可以是无理数。
\end{enumerate}

\subsection{数集·确界原理}

区间分为无限区间和有限区间。

设实数 $a<b$,则称数集 $\{x \mid a<x<b\}$ 为开区间,记作 $(a,b)$;数集 $\{x \mid a\leqslant x \leqslant b\}$ 称为闭区间,记作 $[a,b]$;数集 $\{x \mid a\leqslant x<b\}$ 和 $\{x \mid a < x \leqslant b\}$ 都称为半开半闭区间,分别记作 $[a,b)$ 和 $(a,b]$。以上几类区间统称为有限区间。

满足关系式 $x\geqslant a$ 的全体实数 $x$ 的集合记作 $[a,+\infty)$,类似地,有 $(-\infty,a],(a,\infty),(-\infty,a)$。特殊地 $\RR = (-\infty,+\infty)$。这几类区间统称为无限区间。

设 $\delta > 0$,满足 $|x-a|<\delta$ 的 $x$ 的集合称为点 $a$ 的 $\delta$ 邻域,记作 $U(a;\delta)$,或简单的记作 $U(a)$,即有
$$U(a;\delta) = (a-\delta,a+\delta)$$
点 $a$ 的空心 $\delta$ 邻域定义为
$$U^\circ (a;\delta) = \{x \mid 0<|x-a|<\delta\}$$
也可以简单的记作 $U^\circ(a)$。

此外,常用的邻域还有:

点 $a$ 的 $\delta$ 右邻域 $U_+(a;\delta) = [a,a+\delta)$,左邻域 $U_-(a;\delta)$。以及点 $a$ 的空心 $\delta$ 左、右邻域 $U_{-}^\circ(a)$ 与 $U_{+}^{\circ}(a)$。

以及 $\infty$ 邻域 $U(\infty) = \{x \mid |x|>M\}$,其中 $M$ 为充分大的正数。类似的还有 $U(+\infty) = \{x \mid x>M\}$ 和 $U(-\infty) = \{x \mid x<-M\}$。

\begin{definition}[有界集]
	设 $S$ 为一个非空数集,若存在数 $M\in$ 使得 $\forall x\in S$
	
	(1) 都有 $x\leqslant M$,则称 $M$ 是 $S$ 的一个上界。
	
	(2) 都有 $x\geqslant M$,则称 $M$ 是 $S$ 的一个下界。
\end{definition}

若数集 $S$ 既有上界又有下界,则称 $S$ 为有界集,反之称为无界集。

\begin{definition}[上确界]
	设 $S$ 是一个数集,若数 $\beta$ 满足:
	
	(1) $\beta$ 是 $S$ 的上界:$\forall x\in S$,有 $x\leqslant \beta$。
	
	(2) 任何小于 $\beta$ 的数不是数集 $S$ 的上界:$\forall \mu<\beta, \exists x_0\in S$ 使得 $x_0>\mu$。
	
	则称数 $\beta$ 为数集 $S$ 的上确界,记作 $\sup S$。
\end{definition}

\begin{definition}[下确界]
	设 $S$ 是一个数集,若数 $\alpha$ 满足:
	
	(1) $\alpha$ 是 $S$ 的下界:$\forall x\in S$,有 $x\geqslant \alpha$。
	
	(2) 任何大于 $\alpha$ 的数不是数集 $S$ 的下界:$\forall \mu>\alpha, \exists x_0\in S$ 使得 $x_0<\mu$。
	
	则称数 $\alpha$ 为数集 $S$ 的下确界,记作 $\inf S$。
\end{definition}

上确界与下确界统称为确界。应注意,数集 $S$ 的确界可能属于 $S$,也可能不属于 $S$。

\begin{theorem}[确界原理]
	设 $S$ 为非空数集,若 $S$ 有上界,则 $S$ 必有上确界;若 $S$ 有下界,则 $S$ 必有下确界。
\end{theorem}

若把 $\pm \infty$ 看作非正常上下确界,前文定义视为正常上(下)确界,那么任一非空数集必有上下确界。

\section{函数的上下界}

\begin{definition}
	设 $f$ 为定义在 $D$ 上的函数。若存在数 $M(L)$,使得对每一个 $x\in D$,有 $f(x)\leqslant M(f(x) \ge L)$,则称 $f$ 为 $D$ 上的有上(下)界函数,$M(L)$ 称为 $f$ 在 $D$ 上的一个上(下)界。

	反之,若存在数 $M(L)$,使得对每一个 $x\in D$,有 $f(x)\geqslant M(f(x) \leqslant L)$,则称 $f$ 为 $D$ 上的有无上(下)界函数。
\end{definition}

\begin{definition}
	设 $f$ 为定义在 $D$ 上的函数。若存在正数 $M$,使得对每一个 $x\in D$,有 $|f(x)|\leqslant M$,则称 $f$ 为 $D$ 上的有界函数。

	反之,若存在正数 $M$,使得对每一个 $x\in D$,有 $|f(x)|\ge M$,则称 $f$ 为 $D$ 上的无界函数。
\end{definition}

记函数 $f$ 在 $D$ 上的上确界为 $\sup_{x\in D} f(x)$,类似的有 $\inf_{x\in D} f(x)$。

\begin{definition}
	设 $f$ 为定义在 $D$ 上的函数,若对任何 $x_1,x_2\in D$,当 $x_1<x_2$ 时:

	(1) 总有 $f(x_1) \leqslant f(x_2)$,则称 $f$ 为 $D$ 上的增函数,若成立严格不等式 $f(x_1) < f(x_2)$ 时,称 $f$ 为 $D$ 上的严格增函数。

	(2) 总有 $f(x_1) \geqslant f(x_2)$,则称 $f$ 为 $D$ 上的减函数,若成立严格不等式 $f(x_1) > f(x_2)$ 时,称 $f$ 为 $D$ 上的严格减函数。
\end{definition}

增函数和减函数统称为单调函数,严格增函数和严格减函数统称为严格单调函数。

严格单调函数必有反函数,其也为严格单调函数。

\begin{definition}
	设 $D$ 为对称于原点的数集, 函数 $f$ 为定义在 $D$ 上的函数。若对每一个 $x\in D$:

	(1) 有 $f(-x) = -f(x)$,则称 $f$ 为 $D$ 上的奇函数。

	(2) 有 $f(-x) = f(x)$,则称 $f$ 为 $D$ 上的偶函数。
\end{definition}

\section{实数系的构造}

\begin{definition}[Dedekind 分割]
	设 $A$ 为 $Q$ 的子集,若满足以下三个条件

	(1) $A \ne \varnothing,A \ne \QQ$;

	(2) 当 $p\in A,p \in A^c$ 时,$p<q$;

	(3) 任给 $p \in A$,存在 $q \in A$,使得 $p<q$;

	则称 $A$ 为 $\QQ$ 的一个分割,分割的全体组成集合为 $\RR$。
\end{definition}

