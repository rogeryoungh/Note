\chapter{实数集与函数}

我初次用的书是华师的数分,现在发觉基础部分有相当多的细节。仍待调整。

若无额外说明,皆在 $\RR$ 下。

实数集与函数的基础是集合和映射,仍有很多术语不曾了解。

参考自李逸的《基本分析讲义》。

\section{集合}

集合的交并补是熟知的。

定义有序对为 $(a,b) \coloneqq  \{\{a\},\{a,b\}\}$,其中 $a$ 称为有序对的第一坐标,而 $b$ 称为第二坐标。特殊的,$(a,a) = \{a\}$。

定义集合的笛卡尔 Cartesian 乘积为
$$A \times B \coloneqq  \{(a,b) \mid a\in A \text{且} b\in B\}$$
一般 $A \times B \ne B \times A$。同样可以推广到多个集合
$$\prod X_i \coloneqq  X_1 \times X_2 \times \cdots \times X_n = (X_1 \times \cdots \times X_{n-1}) \times X_n$$
其元素 $x$ 是多层嵌套,我们可以简记为
$$x = (\cdots(x_1,x_2),x_3),\cdots,x_n) = (x_1,\cdots,x_n)$$
称 $x_i \coloneqq  \mathrm{pr}_i(x)$ 为 $x$ 的第 $i$ 个分量,$\mathrm{pr}$ 是投影映射。当所有 $X_i$ 都等于 $X$ 时,上述乘积记为 $X^n$。

\section{映射}

设 $C$ 和 $D$ 为两个给定的集合。

\begin{definition}[赋值法则]
	设 $R$ 是 $C\times D$ 的一个子集,若满足当 $(c,d_1)\in R$ 且 $(c,d_2)\in R \Rightarrow d_1=d_2$,称 $R$ 是一个赋值法则。
\end{definition}

赋值法则的定义域 Domain 和像域 Image Set 约定如下
\begin{equation*}
	\begin{aligned}
		\dom(R) \coloneqq  \{c\in C \mid \exists d \in D, (c,d) \in R\}\\
		\im(R) \coloneqq  \{d\in D \mid \exists c \in C, (c,d) \in R\}\\
	\end{aligned}
\end{equation*}

\begin{definition}
	设 $R$ 为一个赋值法则,$B$ 为满足 $\im(R) \subseteq B$ 的一个集合,记二元对 $(R,B)$ 为一个映射,$B$ 称为值域。定义 $f$ 的定义域 $A$ 和像域为 $R$ 的定义域和像域。记作
	$$f:A\to B, a\mapsto f(a)$$
\end{definition}

称 $f$ 的图为
$$\graph(f) \coloneqq  \{(a,f(a)) \in A\times B \mid a \in A\}$$

对任意给定的 $A$ 的子集 $A_0$,定义 $f$ 在 $A_0$ 上的限制为映射
$$f \mid_{A_0} = f : A_0 \to B$$

称映射 $f$ 和 $g$ 的复合为
$$g\circ f: A \to C, a \mapsto g(f(a))$$
显然 $g \circ f$ 仅当 $\im(f) \subseteq \dom(g)$ 时有定义。$f\circ g$ 一般与 $g \circ f$ 不相等。

若映射 $f$ 满足

\num{1} $f(a_1) = f(a_2) \Rightarrow a_1,a_2$,称 $f$ 为单射。

\num{2} 对任意的 $b\in B$ 存在 $a\in A$ 满足 $f(a)=b$,称 $f$ 为满射。

\num{3} $f$ 既是单射又是满射,称 $f$ 为双射。

若 $f$ 为双射,我们定义它的逆映射 $f^{-1}$ 为
$$f^{-1}(b) =  a \Leftrightarrow f(a) = b$$

映射 $\diamond$:$X\times X \to X$ 通常也称为集合 $X$ 上的运算,此时我们把 $\diamond(x,y)$ 记做 $x \diamond y$。对 $X$ 中的非空子集定义
$$A \diamond B \coloneqq  \diamond(a\times B) = \{A \diamond b \mid a \in A, b\in B\}$$

\begin{definition}
	\num{1} 若 $A \diamond A \subseteq A$,则称 $A$ 在运算 $\diamond$ 下封闭。

	\num{2} 若 $x \diamond (y \diamond z) = (x \diamond y) \diamond z$,则称运算 $\diamond$ 是结合的。

	\num{3} 若 $x \diamond y = y \diamond x$,则称运算 $\diamond$ 是交换的。

	\num{4} 若存在 $e \in X$ 使得 $e \diamond x = x \diamond e = x$,则称 $e$ 为 $X$ 上的单位元。
\end{definition}

如果映射定义中的 $B$ 是一个数域,则把映射称为函数。

\section{关系}

简要提及一下关系的部分知识。称集合 $S\times S$ 的子集 $C$ 为关系。把 $(x,y)\in C$ 记作 $xCy$。

\begin{definition}[等价关系]
	若集合 $S$ 上的关系 $C$ 满足

	\num{1} 自反性:对任意的 $x\in S$ 有 $xCx$。
	
	\num{2} 对称性:若 $xCy$ 则 $yCx$。

	\num{3} 传递性:若 $xCy$ 且 $yCz$ 则 $xCz$。

	则称 $C$ 为等价关系,一般用 $\sim$ 来表示等价关系。
\end{definition}

记 $x\in A$ 的等价类:
$$[x] \coloneqq  \{y \in A \mid y \sim x\}$$ 

\begin{definition}[序关系]
	若集合 $S$ 上的关系 $C$ 满足

	\num{1} 相容性:对任意的 $x,y\in S$ 且 $x\ne y$ 满足要么 $xCy$ 要么 $yCx$。
	
	\num{2} 非自反性:不存在 $x\in S$ 使得 $xCx$。

	\num{3} 传递性:若 $xCy$ 且 $yCz$ 则 $xCz$。

	则称 $C$ 为序关系,一般用 $<$ 来表示序关系,称 $(S,C)$ 为有序集。
\end{definition}

$x<y$ 也可以记作 $y>x$。记号 $x\leqslant y$ 指的是 $x<y$ 或 $x=y$,即是 $x>y$ 的否定。

若在 $S$ 内定义了一种序,便是有序集。

\begin{definition}[上界]
	设有序集 $S$ 的非空子集 $E$,若存在 $a \in S$ 使得任取 $x\in E$ 满足 $x \leqslant a$,我们称 $E$ 有上界,$a$ 称为 $E$ 的一个上界。 
\end{definition}

同样可以得到下界。

\begin{definition}[上确界]
	设有序集 $S$ 的非空子集 $E$,且 $E$ 有上界。若存在一个数 $a \in S$ 满足:
	
	\num{1} $a$ 是 $E$ 的上界:$\forall x\in E$,有 $x\leqslant a$。
	
	\num{2} 任何小于 $a$ 的数不是数集 $E$ 的上界:$\forall \mu<a, \exists x_0\in E$ 使得 $x_0>\mu$。
	
	则称数 $a$ 为数集 $E$ 的上确界,记作 $\sup E$。
\end{definition}

类似的可定义有界集的下确界 $a=\inf E$。

\begin{definition}[最小上界性]
	如果有序集 S'的任何非空有上界子集 $E$ 有最小上界,则称 $S$有最小上界性。
	设 $E$ 是有序集 $S$ 的子集,若对任意的有上界的 $E$ 都有 $\sup E \in S$,那么称 $S$ 有最小上界性。
\end{definition}

例如 $\QQ \cap (0,\sqrt2)$ 是 $\QQ$ 的非空子集,其上界 $\sqrt2$ 并不在 $\QQ$ 内。

同样的有最小下界性。可以证明,有最小上界性的一定有最大下界性。展开描述即

\begin{theorem}
	设 $B$ 是具有最小上界性的集合 $S$ 的子集,则对任意的有下界的 $B$ 都有 $\inf B \in S$。
\end{theorem}

\begin{proof}
	对于每个 $B$,构造 $L$ 为 $B$ 的下界组成的集合,显然每个 $B$ 中的元素都是 $L$ 的上界。由最小上界性知,存在 $\sup L \in S$。

	尝试证明 $\inf B=\sup L$。
	
	对于任意的 $x\in B$,若 $x<\sup L$,则存在比 $\sup L$ 小的 $L$ 的上界 $x$,矛盾。故 $x \geqslant \sup L$,即 $\sup L$ 是 $B$ 的下界。

	设 $B$ 的下界 $x$ 有 $x>\sup L$,那么 $x\in L$,则存在比 $\sup L$ 大的 $L$ 元素,矛盾。故不存在比 $\sup L$ 大的 $B$ 的下界。

	综上,$B$ 的下界存在且 $\inf B=\sup L \in S$。
\end{proof}

\section{域公理}

集合 $F$ 上定义的加法 $+$ 和 乘法 $\cdot$ 若满足 $(x,y,z\in F)$

F1 加法结合率:$x+(y+z) = (x+y)+z$。

F2 加法交换律:$x+y = y+x$。

F3 存在加法单位元:存在 $0\in F$,使得对任意 $x\in F$ 有 $0+x=x$。

F4 加法逆元的存在性:对任意 $x\in F$,总存在 $-x\in F$,使得 $x+(-x)=0$。

F5 乘法结合率:$x\cdot (y\cdot z) = (x\cdot y)\cdot z$。

F6 乘法交换律:$x\cdot y = y\cdot x$。

F7 存在乘法单位元:存在 $1\in F$,使得 $1\ne 0$ 且对任意 $x\in F$ 有 $1\cdot x=x$。

F8 乘法逆元的存在性:对任意 $x\in F-\{0\}$,总存在 $x^{-1}\in F$,使得 $x\cdot x^{-1} =0$。

F9 乘法分配律:$x \cdot (y+z) = x\cdot y + x\cdot z$。

则称 $(F,+,\cdot)$ 为一个域。

注意 $-x$ 只是一个记号。同样的,$x^{-1}$ 也只是一个记号。

\begin{definition}[有序域]
	若域 $F$ 是满足如下条件的有序集

	\num{1} 当 $x,y,z\in F$ 且 $y<z$ 时,$x+y<x+z$。

	\num{2} 如果 $x,y\in F$,且 $x>0,y>0$,则 $xy>0$。

	那么称 $F$ 是一个有序域。
\end{definition}

例如 $\QQ$ 是有序域。

\begin{theorem}[存在定理]
	具有最小上界性的有序域 $\RR$ 存在,且包容着 $\QQ$ 作为子域。
\end{theorem}

这个命题的证明较为复杂,是从 $\QQ$ 出发构造 $\RR$,而且其中有很多重要的信息,决定单独一章,这里略过。

\begin{theorem}[Achimedes 原理]
	对于 $x,y \in \RR$ 且 $x>0$,那么必定存在正整数 $n$,使得 $nx>y$。
\end{theorem}
\begin{proof}
	设 $A = \{nx \mid n \in \NN^+ \}$,若不存在 $n$ 则 $y$ 将是 $A$ 的一个上界,由最小上界性可知 $A$ 的上确界存在。
	
	又因为小于上确界的数 $\sup A-x$ 不是上确界,即存在 $m\in \NN^+$ 使得 $\sup A -x <mx$,即 $\sup A < (m+1)x$,矛盾。

	故必定存在 $n$ 使得 $nx>y$。
\end{proof}

\begin{definition}[度量空间]
	称集合 $X$ 的元素为点,若存在 $X$ 上双变量的函数 $d:X \times X \to \RR$,满足($x,y,z\in R$)

	\num{1} 若 $x\ne y$,则 $d(x,y)$;仅 $d(x,x)=0$。

	\num{2} 对于任意的 $x,y$ 都有 $d(x,y) = d(y,x)$。

	\num{3} 对于任意的 $z$,都有 $d(x,y) \leqslant d(x,z) + d(z,y)$。

	就称 $(X,d)$ 是一个度量空间(度量空间),函数 $d$ 称作其上的距离函数。
\end{definition}

这里的空间的含义是线性空间。

对于 $X$ 的子集 $Y$,定义其距离函数
$$d_Y: Y \times Y \to \RR, (y_1,y_2) \mapsto d_Y(y_1,y_2)=d(y_1,y_2)$$
则 $(Y,d_Y)$ 仍是度量空间,称 $d_Y$ 是 $d$ 在 $Y$ 上的诱导度量。$(Y,d_Y)$ 称作是 $(X,d)$ 的子(度量)空间。

\begin{definition}[稠密性]
	给定度量空间 $(X,d)$,$Y$ 是 $X$ 的子集。如果对任意的 $x\in X$ 和任意小的 $\eps>0$,都存在 $y\in Y$,使得 $d(y,x)<\eps$,我们就称 $Y$ 在 $X$ 中是稠密的。
\end{definition}

\begin{example}
	$\QQ$ 在 $\RR$ 中稠密:对于 $x,y \in \RR$ 且 $x<y$,那么必定存在 $p\in\QQ$,使得 $x<p<y$。
\end{example}
\begin{proof}
	由 Achimedes 原理,可设存在 $n\in \NN^+$ 使得 $n(y-x)>1$。
	
	再设存在 $m_1,m_2\in \NN^+$,使得 $m_1>nx,m_2>-nx$。于是
	$$-m_2<nx<m_1$$
	因此存在 $m\in \NN^+$ 有 $-m_2\leqslant m \leqslant m_1$ 使得
	$$m-1\leqslant nx < m \leqslant 1+nx < ny$$
	从而存在 $p=m/n$ 使得 $x<p<y$。
\end{proof}


\section{数系的构造}

直到我读了陶哲轩的《实分析》时,才感到接受了实数理论。实数的定义是公理化的,不是构造性的。

更具体的说,我们不需要知道实数是什么,只需知道这些对象有什么性质,我们就可以抽象的处理它们。从其他的数学对象出发来构造实数是可能的,有多种多样的模型,只要它们服从所有的公理并正确的运作,都是满足的。

实数究竟有多少性质?从自然数开始。

\begin{axiom}[Peano 公理]
	若集合 $N$ 和其上的映射 $v(n)$ 满足

	\num{1} $0\in N$。

	\num{2} 若 $n\in N$,则 $v(n) \in N$。

	\num{3} 对于任意的 $n\in N$,$v(n) \ne 0$。

	\num{4} 若 $v(m) \ne v(n)$,则 $m\ne n$。

	\num{5} 【归纳原理】设 $P(n)$ 是关于自然数的性质,假设只要 $P(n)$ 为真,则 $P(v(n))$ 也为真;且 $P(0)$ 为真。那么对 $N$ 中所有的元素 $P$ 都为真。

	那么称 $N$ 为自然数,记作 $\NN$,$v(n)$ 称为后继函数。
\end{axiom}

\subsection{自然数}

设 $m,n\in \NN$,定义 $\NN$ 上的加法 $+$ 和乘法 $\cdot$ 为
\begin{equation*}
	\begin{aligned}
		0+m\coloneqq m&,\quad v(n)+m\coloneqq v(n+m)\\
		0\cdot m\coloneqq m&,\quad v(n)\cdot m\coloneqq n \cdot m + m
	\end{aligned}
\end{equation*}

我们可以利用归纳原理推出我们熟悉的一些性质。

\begin{theorem}[$\NN$ 的代数算律]
	对于 $a,b,c\in \NN$ 有

	\num{1} 加法是结合的和交换的,且有单位元 $0$。

	\num{2} 乘法是结合的和交换的,且有单位元 $1$。

	\num{3} 分配律:$(a+b) \cdot  c = a \cdot c + b\cdot c$。
\end{theorem}

\begin{definition}[$\NN$ 的序]
	设 $m,n\in \NN$。

	\num{1} 若存在 $a\in \NN$,使得 $n=m+a$,称 $m$ 小于等于 $n$,记作 $m \leqslant n$。

	\num{2} 若 $n\geqslant m$ 且 $n\ne m$,则称 $m$ 严格小于 $n$,记作 $m < n$。
\end{definition}

可以验证,$<$ 和 $\leqslant$ 是 $\NN$ 上的序关系。

\begin{theorem}[加法保序]
	对于 $a,b\in \NN$,若 $a>b$,则 $a+c>b+c$。
\end{theorem}

\subsection{整数}

接下来几节,都是记 $a,b,c$ 为当前集合的元素,$x,y,z$ 都是被构造的集合的元素。

为了表达整数,定义二元组 $(a,b)$,其中 $a,b \in \NN$。记全体二元组的集合为 $Z$。我们约定
$$(a,b) = (c,d) \Leftrightarrow a+d=b+c$$
因为自然数的序是已定义的,于是定义 $Z$ 上的序关系
$$(a,b) \leqslant (c,d) \Leftrightarrow a+d \leqslant b+c$$
然后是定义 $N$ 上的加法和乘法
\begin{equation*}
	\begin{aligned}
		(a,b) + (c,d) &\coloneqq  (a+c,b+d)\\
		(a,b) \cdot (c,d) &\coloneqq  (a c,b d)
	\end{aligned}
\end{equation*}

可以验证,$(n,0)$ 与 $n$ 有相同的性状,我们可以令其相等,从而把自然数嵌入到整数内。至此,我们可以着手验证整数是否满足我们预想的性质。

\begin{theorem}[$\ZZ$ 的代数算律]
	对于 $x,y,z\in \ZZ$ 有

	\num{1} 加法是结合的和交换的,且有单位元 $0$,逆元存在。

	\num{2} 乘法是结合的和交换的,且有单位元 $1$。

	\num{3} 分配律:$(x+y) \cdot  z = x \cdot z + y\cdot z$。
\end{theorem}

即 $\ZZ$ 是一个交换环。于是

\begin{theorem}[$\ZZ$ 是有序域]
	\num{1} 加法保序:当 $x,y,z\in \ZZ$ 且 $y<z$ 时,$x+y<x+z$。

	\num{2} 乘法保序:如果 $x,y\in \ZZ$,且 $x>0,y>0$,则 $xy>0$。
\end{theorem}

我们有理由相信,$(a,b)$ 符合我们对整数的一切想象。因此 $Z = \ZZ$。

另外的,定义整数的负运算为 $-(a,b) = (b,a)$,以此定义减法
$$x - y \coloneqq  x + (-y)$$
可以验证
$$(a,0) - (b,0) = (a,b) = a - b$$

\subsection{有理数}

类似的,记整数的二元组 $(a,b)$,其中 $a,b\in \ZZ,b\ne 0$,记全体二元组的集合为 $Q$。我们约定
$$(a,b) = (c,d) \Leftrightarrow ad = bc$$
因为整数的序是已定义的,于是定义 $Q$ 上的序关系
$$(a,b) \leqslant (c,d) \Leftrightarrow ad \leqslant bc$$
于是定义 $Q$ 上的加法和乘法
\begin{equation*}
	\begin{aligned}
		(a,b) + (c,d) &\coloneqq  (ad+bc,b+d)\\
		(a,b) \cdot (c,d) &\coloneqq  (a \cdot c,b \cdot d)\\
	\end{aligned}
\end{equation*}
定义加法逆元为 $-(a,b) \coloneqq  (-a,b)$。可以验证,$(a,1)$ 与 $a$ 有相同的性状,我们可以令其相等,从而把整数嵌入到有理数内。

至此,我们可以着手验证有理数是否满足我们预想的性质。

\begin{theorem}[$\QQ$ 的代数算律]
	对于 $x,y,z\in \QQ$ 有

	\num{1} 加法是结合的和交换的,且有单位元 $0$,逆元存在。

	\num{2} 乘法是结合的和交换的,且有单位元 $1$,非零元逆元存在。

	\num{3} 分配律:$(x+y) \cdot  z = x \cdot z + y\cdot z$。
\end{theorem}

即 $\QQ$ 是一个域。

\begin{theorem}[$\QQ$ 是有序域]
	\num{1} 加法保序:当 $x,y,z\in \QQ$ 且 $y<z$ 时,$x+y<x+z$。

	\num{2} 乘法保序:如果 $x,y\in \QQ$,且 $x>0,y>0$,则 $xy>0$。
\end{theorem}

我们有理由相信,$(a,b)$ 符合我们对有理数的一切想象。因此 $Q = \QQ$。

另外,定义倒数 $(a,b)^{-1} = (b,a)$,显然 $a,b\ne 0$。从而定义除法
$$x/y \coloneqq  x \cdot y^{-1}$$
可以验证,
$$(a,1)/(b,1) = (a,b) = a/b$$

\subsection{实数\ ·\ Dedekind 分割}



\begin{definition}[Dedekind 分割]
	设 $A \subset \QQ, A' = \complement_\QQ A$,若满足以下三个条件

	\num{1} $A \ne \varnothing,A' \ne \varnothing$;

	\num{2} 当 $p\in A,q \in A'$ 时,$p<q$;

	\num{3} 任给 $p \in A$,存在 $q \in A$,使得 $p<q$;

	则称 $A$ 为 $\QQ$ 的一个分割,记分割的全体为 $R$。
\end{definition}

集合的相等即是 $R$ 上的等价关系。$R$ 上的序关系定义是
$$A \subseteq B \Leftrightarrow A\leqslant B$$
定义加法
$$A+B \coloneqq  \{ a+b \mid a\in A,b\in B\}$$
于是可以定义负运算
$$-A \coloneqq  \{ s \in \QQ \mid \exists r>0,-s-r\in \complement_\QQ A\}$$
$$-A \coloneqq  \{ s \in \QQ \mid \exists r\in \complement_\QQ A,s < -r\}$$


然而乘法因为负数的问题,我们需要分类讨论。$R$ 中存在加法单位元 $0^* = \{x \in \QQ \mid x \geqslant 0\}$,对于正实数 $A,B\geqslant 0^*$,定义乘法
$$A \cdot B \coloneqq  \{ p \in \QQ \mid \text{存在}\ 0<a\in A,\ \text{存在}\ 0<b\in B, p<ab\}$$
同时
\begin{equation*}
	A \cdot B \coloneqq \begin{cases}
		-((-A) \cdot B), &A<0^*, B\geqslant 0^*\\
		-(A \cdot (-B)), &A\geqslant 0^*, B<0^*\\
		-((-A) \cdot (-B)), &A<0^*, B<0^*\\
	\end{cases}
\end{equation*}

当 $A > 0^*$ 时,定义乘法逆元
$$A^{-1} \coloneqq  \{s \in \QQ \mid \exists r \in \complement_\QQ A, s < r^{-1}\}$$
当 $A < 0^*$ 时,定义乘法逆元为 $A^{-1} \coloneqq  -(-A^{-1})$。

至此,我们可以着手验证实数是否满足我们预想的性质。

\begin{theorem}[$\RR$ 的代数算律]
	对于 $x,y,z\in \RR$ 有

	\num{1} 加法是结合的和交换的,且有单位元 $0$,逆元存在。

	\num{2} 乘法是结合的和交换的,且有单位元 $1$,非零元逆元存在。

	\num{3} 分配律:$(x+y) \cdot  z = x \cdot z + y\cdot z$。
\end{theorem}

即 $\RR$ 是一个域。

\begin{theorem}[$\RR$ 是有序域]
	\num{1} 加法保序:当 $x,y,z\in \RR$ 且 $y<z$ 时,$x+y<x+z$。

	\num{2} 乘法保序:如果 $x,y\in \RR$,且 $x>0,y>0$,则 $xy>0$。
\end{theorem}

我们有理由相信,$\LIM(a_n)$ 符合我们对实数性质的一切想象,从而 $R = \RR$。

实数和有理数的最基本的一个区别就是有最小上界性。

\begin{theorem}
	$\RR$ 有最小上界性。
\end{theorem}
\begin{proof}
	设 $\alpha$ 是 $R$ 的非空子集,设 $\alpha$ 存在上界 $B$。令
	$$S = \bigcup_{A \in \alpha} A$$
	下证 $S = \sup A$。

	首先证明 $S$ 是分割。显然 $S$ 非空,又因为对任意的 $A\in \alpha$ 都有 $A \subset B$,故 $S \subset B$,因此 $S \ne \QQ$。取 $p \in S,q\notin S$,于是存在 $A_0 \in \alpha$ 使得 $p \in A_0$,此时 $q \notin A_0$,故 $p<q$。设 $p \in S$,于是存在 $A_0 \in \alpha$ 使得 $p \in A_0$,此时存在 $q\in A_0$ 使得 $p < q$,且 $q \in S$。

	其次,对任意的 $A \in \alpha$ 必然 $A \leqslant S$,故 $S$ 是 $A$ 的一个上界。若 $S' < S$,则有 $s \in S, s \notin S'$,同时存在 $A_0 \in \alpha$ 使得 $s \in A_0$,故 $S' < A_0$,故 $S'$ 不是 $A$ 的上界。

	因此 $S = \sup A$。
\end{proof}

\subsection{实数\ ·\ Cauchy 序列}

我们试图得到实数,是因为有理数还不足以表示所有的数,比如 $x^2=2$ 的解。得到实数和前面的方法有所不同,要复杂的多。

一个有理数上的序列 $\{a_n\}$,是一个从集合 $\NN$ 到 $\QQ$ 的一个映射,即我们以前说的数列。

对于 $\QQ$ 上的无限序列 $\{a_n\}$,若对于任意的 $\eps > 0$ 存在 $N \geqslant 0$ 使得当 $j,k \geqslant N$ 时有 
$$d(a_j,a_k) < \eps$$
则称序列 $\{a_n\}$ 为 Cauchy 序列,记作 $\LIM(a_n)$。记 Cauchy 序列的全体为集合 $R$。

对于 Cauchy 序列 $\LIM(a_n),\LIM(b_n)$,若对于任意的 $\eps > 0$ 存在 $N \geqslant 0$ 使得当 $n \geqslant N$ 时有
$$d(a_n,b_n) < \eps$$
则记作 $\LIM(a_n) = \LIM(b_n)$。

定义 $R$ 的序关系,对于实数 $x,y$,若存在 Cauchy 序列满足 $x=\LIM(a_n),y=\LIM(b_n)$,对于 $n\geqslant 1$ 有 $a_n \leqslant b_n$,则 $\LIM(a_n) \leqslant \LIM(b_n)$。

于是定义 $R$ 上的加法和乘法
\begin{equation*}
	\begin{aligned}
		\LIM(a_n) + \LIM(b_n) &\coloneqq  \LIM(a_n+b_n)&\\
		\LIM(a_n) \cdot \LIM(b_n) &\coloneqq  \LIM(a_nb_n)\\
	\end{aligned}
\end{equation*}
定义负运算 $-\LIM(a_n) \coloneqq  \LIM(-a_n)$。

定义倒数时会因为恼人的 $0$ 出现了一些困难,解决的方法即是把 $0$ 排出。若存在 $c\in \QQ$ 满足 $c > 0$ 使得 $d(a_n,0) \geqslant c$,则称 $\{a_n\}$ 为限制离开零的序列。若 $x$ 为不为零的实数,则必存在一个限制离开零的 Cauchy 序列 $\LIM(a_n) = x$。

于是我们可以定义,设 $x$ 为一个不为零的实数,则存在限制离开零的 Cauchy 序列 $x=\LIM(a_n)$,定义倒数为
$$x^{-1} \coloneqq  \LIM(a_n^{-1})$$

可以验证,常数 Cauchy 序列 $\{a_n\}$ 与 $a$ 具有相同的性状,因此可以令它们相等,从而使有理数嵌入到实数中。

至此,我们可以着手验证实数是否满足我们预想的性质,在 Dedekind 分割中提过了,这里不再重复。

另外,定义 $R$ 上的 Cauchy 序列,若对于任意的实数 $\eps > 0$ 存在 $N \geqslant 0$ 使得当 $j,k \geqslant N$ 时有 
$$d(a_j,a_k) \leqslant \eps$$

可以证明,$R$ 上的 Cauchy 序列与 $\QQ$ 上的 Cauchy 序列等价。

若存在实数 $L$ 满足,存在 $N>0$ 使得当 $n \geqslant N$ 时,都有 $d(a_n,L) \leqslant \eps$,则 $a_n$ 收敛于 $L$,记作
$$\lim_{n\to \infty} a_n = L$$

可以验证
$$\LIM(a_n) = \lim_{n\to \infty} a_n$$

\subsection{复数}

记实数的二元组 $(a,b)$,其中 $a,b\in R$,记全体二元组的集合为 $C$。我们约定
\[(a,b) = (c,d) \Leftrightarrow a=b \land  c=d\]
复数没有序关系。定义 $C$ 上的加法和乘法
\[\begin{aligned}
		(a,b) + (c,d) &\coloneqq  (a+c,b+d)\\
		(a,b) \cdot (c,d) &\coloneqq  (ac-bd,ad+bc)
\end{aligned}\]

定义加法逆元为 $-(a,b) \coloneqq  (-a,-b)$。可以验证,$(a,0)$ 与 $a$ 具相同的性状,我们可以令其相等,从而把实数嵌入到复数域内。

定义非零数的乘法逆元 $(a,b)^{-1} \coloneqq  \left(\frac{a}{a^2+b^2},-\frac{b}{a^2+b^2}\right)$。









%\subsection{数集·确界原理}
%
%区间分为无限区间和有限区间。
%
%设实数 $a<b$,则称数集 $\{x \mid a<x<b\}$ 为开区间,记作 $(a,b)$;数集 $\{x \mid a\leqslant x \leqslant b\}$ 称为闭区间,记作 $[a,b]$;数集 $\{x \mid a\leqslant x<b\}$ 和 $\{x \mid a < x \leqslant b\}$ 都称为半开半闭区间,分别记作 $[a,b)$ 和 $(a,b]$。以上几类区间统称为有限区间。
%
%满足关系式 $x\geqslant a$ 的全体实数 $x$ 的集合记作 $[a,+\infty)$,类似地,有 $(-\infty,a],(a,\infty),(-\infty,a)$。特殊地 $\RR = (-\infty,+\infty)$。这几类区间统称为无限区间。
%
%设 $\delta > 0$,满足 $|x-a|<\delta$ 的 $x$ 的集合称为点 $a$ 的 $\delta$ 邻域,记作 $U(a;\delta)$,或简单的记作 $U(a)$,即有
%$$U(a;\delta) = (a-\delta,a+\delta)$$
%点 $a$ 的空心 $\delta$ 邻域定义为
%$$U^\circ (a;\delta) = \{x \mid 0<|x-a|<\delta\}$$
%也可以简单的记作 $U^\circ(a)$。
%
%此外,常用的邻域还有:
%
%点 $a$ 的 $\delta$ 右邻域 $U_+(a;\delta) = [a,a+\delta)$,左邻域 $U_-(a;\delta)$。以及点 $a$ 的空心 $\delta$ 左、右邻域 $U_{-}^\circ(a)$ 与 $U_{+}^{\circ}(a)$。
%
%以及 $\infty$ 邻域 $U(\infty) = \{x \mid |x|>M\}$,其中 $M$ 为充分大的正数。类似的还有 $U(+\infty) = \{x \mid x>M\}$ 和 $U(-\infty) = \{x \mid x<-M\}$。
%
%\begin{definition}[有界集]
%	设 $S$ 为一个非空数集,若存在数 $M\in$ 使得 $\forall x\in S$
%	
%	\num{1} 都有 $x\leqslant M$,则称 $M$ 是 $S$ 的一个上界。
%	
%	\num{2} 都有 $x\geqslant M$,则称 $M$ 是 $S$ 的一个下界。
%\end{definition}
%
%若数集 $S$ 既有上界又有下界,则称 $S$ 为有界集,反之称为无界集。
%
%\begin{theorem}[确界原理]
%	设 $S$ 为非空数集,若 $S$ 有上界,则 $S$ 必有上确界;若 $S$ 有下界,则 $S$ 必有下确界。
%\end{theorem}
%
%若把 $\pm \infty$ 看作非正常上下确界,前文定义视为正常上(下)确界,那么任一非空数集必有上下确界。
%
%\section{函数的上下界}
%
%\begin{definition}
%	设 $f$ 为定义在 $D$ 上的函数。若存在数 $M(L)$,使得对每一个 $x\in D$,有 $f(x)\leqslant M(f(x) \ge L)$,则称 $f$ 为 $D$ 上的有上(下)界函数,$M(L)$ 称为 $f$ 在 $D$ 上的一个上(下)界。
%
%	反之,若存在数 $M(L)$,使得对每一个 $x\in D$,有 $f(x)\geqslant M(f(x) \leqslant L)$,则称 $f$ 为 $D$ 上的有无上(下)界函数。
%\end{definition}
%
%\begin{definition}
%	设 $f$ 为定义在 $D$ 上的函数。若存在正数 $M$,使得对每一个 $x\in D$,有 $|f(x)|\leqslant M$,则称 $f$ 为 $D$ 上的有界函数。
%
%	反之,若存在正数 $M$,使得对每一个 $x\in D$,有 $|f(x)|\ge M$,则称 $f$ 为 $D$ 上的无界函数。
%\end{definition}
%
%记函数 $f$ 在 $D$ 上的上确界为 $\sup_{x\in D} f(x)$,类似的有 $\inf_{x\in D} f(x)$。
%
%\begin{definition}
%	设 $f$ 为定义在 $D$ 上的函数,若对任何 $x_1,x_2\in D$,当 $x_1<x_2$ 时:
%
%	\num{1} 总有 $f(x_1) \leqslant f(x_2)$,则称 $f$ 为 $D$ 上的增函数,若成立严格不等式 $f(x_1) < f(x_2)$ 时,称 $f$ 为 $D$ 上的严格增函数。
%
%	\num{2} 总有 $f(x_1) \geqslant f(x_2)$,则称 $f$ 为 $D$ 上的减函数,若成立严格不等式 $f(x_1) > f(x_2)$ 时,称 $f$ 为 $D$ 上的严格减函数。
%\end{definition}
%
%增函数和减函数统称为单调函数,严格增函数和严格减函数统称为严格单调函数。
%
%严格单调函数必有反函数,其也为严格单调函数。
%
%\begin{definition}
%	设 $D$ 为对称于原点的数集, 函数 $f$ 为定义在 $D$ 上的函数。若对每一个 $x\in D$:
%
%	\num{1} 有 $f(-x) = -f(x)$,则称 $f$ 为 $D$ 上的奇函数。
%
%	\num{2} 有 $f(-x) = f(x)$,则称 $f$ 为 $D$ 上的偶函数。
%\end{definition}