\chapter{初等数论}

注意我们的理论基础是整数,尽量通过分类讨论的方式得到结论。而且也要把握脉络,抓住重点,不要迷失于无谓的细节中。

自然数 $\NN$ 、正整数 $\NN^+$ 和整数 $\ZZ$ 我们是熟知的。

\section{整数公理}

整数的公理

我们熟知一些整数的代数算律

结合律:$(a+b)+c = (a+b)+c$。

交换律:$a+b = b+a$。

消去律:

\begin{definition}
	对于整数 $a,b$,其中 $a\ne 0$,若存在整数 $c$,它使得
	$$b=ac$$
	则 $b$ 叫做 $a$ 的倍数,$a$ 叫做 $b$ 的因数,记作 $a \mid b$。
\end{definition}

有时也称作 $a$ 能整除 $b$,或 $b$ 能被 $a$ 整除,或 $a$ 能除尽 $b$,或 $b$ 能被 $a$ 除尽。

若 $a$ 不能整除 $b$,我们就记作 $a \nmid b$。

\begin{lemma}
	如果对于整数 $a,b$ 满足 $a \mid b$,则有
	$$(-a) \mid b,\quad a \mid (-b),\quad (-a) \mid (-b),\quad |a| \mid |b|$$
\end{lemma}

这个比较显然,由定义知存在 $c$ 使得 $b=ac$,再构造验证即可。

\begin{lemma}
	对于整数 $a,b,c$ 有 $a \mid b,b \mid c$,则有 $a \mid c$。
\end{lemma}

\begin{proof}
	因为 $a \mid b,b \mid c$,故存在整数 $d,e$ 使得 $b=ad,c=be$。

	因此存在整数 $f=de$ 使得 $c=af=ade$,故 $a \mid c$。
\end{proof}

\begin{lemma}
	对于整数 $a,b$ 有 $|a| \mid |b|$,若 $|a|<|b|$ 则有 $a=0$。
\end{lemma}

\begin{proof}
	因为 $|a| \mid |b|$,则存在整数 $c$ 使得 $|a|=|b|c$。那么有
	$$0 \leqslant |a|=|b|c<|b|$$
	即 $0\leqslant c<1$,又 $c$ 为整数,故 $c=a=0$。
\end{proof}

\begin{theorem}
	对于整数 $a,b$,若 $b\ne 0$ 则一定存在唯一一对 $q,r$ 使得
	$$a=bq+r,\quad 0 \leqslant r< |b|$$
\end{theorem}

\begin{proof}
	先证明存在性。
	
	(1) 若恰 $b \mid a$,则必存在 $c$ 使得 $a=bc$,此时有 $q=c,r=0$。

	(2) 否则一定存在 $n$ 使得 $n|b|<a<(n+1)|b|$,即存在 $0<r<|b|$ 使得 $a=|b|n+r$。
	
	当 $b>0$ 时,令 $q=n$;当 $b<0$ 时,令 $q=-n$ 则有
	$$a=bq+r,\quad 0 \leqslant r< |b|$$

	再证明唯一性。设存在两对 $q_1,r_1$ 和 $q_2,r_2$ 使得
	$$a=bq_1+r_1=bq_2+r_2,\quad 0 \leqslant r_1,r_2< |b|$$
	相减有
	$$a-a=b(q_1-q_2)+r_1-r_2=0$$
	即 $r_1-r_2=-b(q_1-q_2)$,因此有 $b \mid (r_1-r_2)$。而 $|r_1-r_2|<|b|$,又引理知有 $|r_1-r_2|=0$。故
	$$r_1=r_2,q_1=r_2$$
	即两对相同。
\end{proof}

\begin{definition}[素数]
	若一个大于 $1$ 的正整数,只能被 $1$ 和它本身整除,不能被其他正整数整除,这样的数叫做素数。
\end{definition}

若能被其他正整数整除,则称为合数。因此一个正整数必然是素数、合数或 $1$。
