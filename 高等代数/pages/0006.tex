\chapter{二次型 · 矩阵的合同}

\section{二次型及其标准型}

\begin{definition}
    数域 $K$ 上的一个 $n$ 元二次型型是系数在 $K$ 中的 $n$ 个变量的齐次多项式,它的一般形式是
    \[f(\ji{x}{n}) = \sum_{i=1}^n\sum_{j=1}^na_{ij}x_ix_j\]
    其中 $a_{ij} = a_{ji}$。
\end{definition}

把二次型的系数按原来顺序排成一个 $n$ 级矩阵 $A$,则称 $A$ 是二次型 $f(\ji{x}{n})$ 的矩阵,它是对称矩阵。

再令 $\vX = \transpose{(\ji{x}{n})}$,则二次型可以写作
\[f(\ji{x}{n}) = \transpose{X}AX\]

令 $\vY = \transpose{(y_1,\cdots,y_n)}$,设 $C$ 是数域 $K$ 上的 $n$ 级可逆矩阵,则关系式
\[\vX = C\vY\]
称为变量 $\ji{x}{n}$ 到变量 $y_1,\cdots,y_n$ 的一个非退化线性替换。

\begin{definition}
    数域上两个 $n$ 元二次型 $\transpose{\vX}A\vX$ 与 $\transpose{\vY}A\vY$,如果存在一个非退化线性替换 $\vX = C\vY$,把 $\transpose{\vX}A\vX$ 变成 $\transpose{\vY}B\vY$ 那么称二次型 $\transpose{\vX}A\vX$ 与 $\transpose{\vY}B\vY$ 等价,记作 $\transpose{\vX}A\vX \cong \transpose{\vY}B\vY$。
\end{definition}

\begin{definition}
    数域 $K$ 上两个 $n$ 级矩阵 $A$ 与 $B$,如果存在 $K$ 上的一个 $n$ 级可逆矩阵 $C$,使得
    \[\transpose{C}AC = B\]
    那么称 $A$ 与 $B$ 合同,记作 $A\simeq B$。
\end{definition}

如果二次型 $\transpose{\vX} A \vX$ 等价于一个只含平方项的二次型,那么这个只含平方项的二次型称为 $\transpose{\vX}A\vX$ 的一个标准形。

如果 $T$ 是正交矩阵,那么变量的替换 $\vX = T\vY$ 称为正交替换。

\begin{theorem}
    设 $A,B$ 都是数域 $K$ 上 $n$ 级矩阵,则 $A$ 合同于 $B$ 当且仅当 $A$ 经过一系列成对初等行、列变换可以变成 $B$,此时对 $I$ 只作其中的初等列变换得到的可逆矩阵 $C$,就使得 $\transpose{C}AC=B$。
\end{theorem}

\begin{theorem}
    数域 $K$ 上的任一对称矩阵都合同于一个对角矩阵。
\end{theorem}

这意味着数域 $K$ 上任一 $n$ 元二次型都等价于一个只含平方项的二次型。

二次型 $\transpose{\vX}A\vX$ 的矩阵 $A$ 的秩就称为二次型 $\transpose{\vX}A\vX$ 的秩。

\section{实二次型的规范形}

实数域上的二次型简称为实二次型,$n$ 元实二次型 $\transpose{\vX}A\vX$ 经过一个适当的非退化线性替换 $\vX = C\vY$ 可以化成下述形式的标准形
\[d_1y_1^2+\cdots+d_py_p^2-d_{p+1}y_{p+1}^2-\cdots-d_ry_r^2\]
其中 $d_i>0,i=1,\cdots,r$。再做一次非退化线性替换可以变成
\[z_1^2+\cdots+z_p^2-z_{p+1}^2-\cdots-z_r^2\]

\begin{theorem}
    $n$ 元实二次型 $\transpose{\vX}A\vX$ 的规范形是唯一的。
\end{theorem}

\begin{definition}
    在实二次型 $\transpose{\vX}A\vX$ 的规范形中,系数为 $+1$ 的平方项个数 $p$ 称为 $\transpose{\vX}A\vX$ 的正惯性指数,系数为 $-1$ 的平方项个数 $r-p$ 称为 $\transpose{\vX}A\vX$ 的负惯性指数;正惯性指数减去负惯性指数所得的差 $2p-r$ 称为 $\transpose{\vX}A\vX$ 的符号差。
\end{definition}

任一 $n$ 级实对称矩阵合同于对角矩阵 $\diag\{1,\cdots,1,-1,\cdots,-1,0,\cdots,0\}$,其中 $0$ 的个数等于 $\transpose{\vX}A\vX$ 的正惯性指数,$-1$ 的个数等于 $\transpose{\vX}A\vX$ 的负惯性指数(也分别称作 $A$ 的惯性指数),这个对角举着称为 $A$ 的合同规范形。

现讨论复数域上的二次型,简称为复二次型。设 $n$ 元复二次型 $\transpose{\vX}A\vX$ 经过一个适当的非退化线性替换 $\vX = C\vY$ 变成下述形式的标准形
\[d_1y_1^2+\cdots+d_ry_r^2\]
其中 $d_i\ne 0,i=1,\cdots,n$,$r$ 是这个二次型的秩。再做一个非退化线性替换可得
\[z_1^2+\cdots+z_r^2\]
把这个标准形叫做复二次型 $\transpose{\vX}A\vX$ 的规范形,显然其完全由其秩决定,故只有一种形式。

\section{正定二次型与正定矩阵}

\begin{definition}
    $n$ 元实二次型 $\transpose{\vX}A\vX$ 称为正定的,如果对于 $\RR^n$ 中任一非零列向量 $\valpha$,都由 $\transpose{\valpha}A\valpha>0$。
\end{definition}

\begin{theorem}
    $n$ 元实二次型 $\transpose{\vX}A\vX$ 是正定的当且仅当它的正惯性系数等于 $n$。
\end{theorem}

\begin{definition}
    实对称矩阵 $A$ 称为正定的,如果实二次型 $\transpose{\vX}A\vX$ 是正定的 。
\end{definition}

正定的实对称矩阵简称为正定矩阵。

\begin{theorem}
    实对称矩阵 $A$ 是正定的充分必要条件是 $A$ 的说有顺序主子式全大于 $0$。
\end{theorem}

\begin{definition}
    实对称矩阵 $A$ 称为半正定(负定,半负定,不定)的,如果实二次型对于 $\RR^n$ 中任一非零列向量 $\valpha$,都有
    \[\transpose{\valpha}A\valpha \geqslant 0 \quad (\transpose{\valpha}A\valpha<0,\transpose{\valpha}A\valpha\leqslant 0)\]
    如果 $\transpose{\vX}A\vX$ 既不是半正定的,又不是半负定的,那么称它是不定的。
\end{definition}

\begin{definition}
    实对称矩阵 $A$ 称为半正定(负定,半负定,不定)的,如果实二次型 $\transpose{\vX}A\vX$ 是半正定(负定,半负定,不定)的。
\end{definition}

\begin{theorem}
    $n$ 级实对称矩阵 $A$ 是半正定的,当且仅当 $A$ 的所有主子式全非负。
\end{theorem}

\begin{theorem}
    实对称矩阵 $A$ 负定的充分必要条件是:它的奇数阶顺序主子式全小于 $0$,偶数阶顺序主子式全大于 $0$。
\end{theorem}

\begin{theorem}
    设二元实值函数 $F(x,y)$ 有一个稳定点 $\valpha=(x_0,y_0)$ (即 $F(x,y)$ 在 $(x_0,y_0)$ 处的一阶偏导数全为 $0$)。设 $F(x,y)$ 在 $(x_0,y_0)$ 的一个邻域内有 3 阶连续偏导数。令
    \[H = \left(\begin{matrix}
        F_{xx}''(x_0,y_0) & F_{xy}''(x_0,y_0)\\
        F_{yx}''(x_0,y_0) & F_{yy}''(x_0,y_0)
    \end{matrix}\right)\]
    称 $H$ 是 $F(x,y)$ 在 $(x_0,y_0)$ 处的黑塞(Hesse)矩阵。如果 $H$ 是正定的,那么 $F(x,y)$ 在 $(x_0,y_0)$ 处达到极小值。如果 $H$ 是负定的,那么 $F(x,y)$ 在 $(x_0,y_0)$ 处达到极大值。
\end{theorem}

其可推广到 $n$ 元函数的情形:设 $F(\ji{x}{n})$ 有一个稳定点 $\valpha = (\ji{a}{n})$,设 $F(\ji{x}{n})$ 在 $\valpha$ 的一个邻域内有 3 阶连续偏导数,令
\[H = (F_{x_ix_j}''(\valpha))\]
称 $H$ 是 $F(\ji{x}{n})$ 在 $\valpha$ 处的黑塞矩阵。如果 $H$ 是正定的,那么 $F$ 在 $(x_0,y_0)$ 处达到极小值。如果 $H$ 是负定的,那么 $F$ 在 $(x_0,y_0)$ 处达到极大值。