\chapter{矩阵的相抵与相似}

\section{等价关系与集合的划分}

等价关系还是记录一下。

\begin{definition}
    设 $S$ 是一个非空集合,我们把 $S\times S$ 的一个子集 $W$ 叫做 $S$ 上的一个二元关系。如果 $(a,b)\in W$,那么称 $a$ 与 $b$ 有 $W$ 关系;如果 $(a,b)\notin W$,那么称 $a$ 与 $b$ 没有 $W$ 关系。
\end{definition}

当 $a$ 与 $b$ 有 $W$ 关系时,记作 $aWb$,或 $a\sim b$。

\begin{definition}
    集合 $S$ 上的一个二元关系 $\sim$ 如果具有下述性质:$\forall a,b,c\in S$,有

    (1) 反身性 $a\sim a$

    (2) 对称性 $a\sim b \Rightarrow b\sim a$

    (3) 传递性 $a\sim b$ 且 $b\sim c \Rightarrow a\sim c$ 

    那么称 $\sim$ 是 $S$ 上的一个等价关系。
\end{definition}

\begin{definition}
    设 $\sim$ 是集合 $S$ 上的一个等价关系,$a\in S$ ,令
    $$\overline{a} := \{x\in S \mid x\in a\}$$
    称 $\overline{a}$ 是由 $a$ 确定的等价类,$a$ 称为等价类 $\overline{a}$ 的一个代表。
\end{definition}

\begin{definition}
    如果集合 $S$ 是一些非空子集 $S_i$ ($i\in I$,这里 $I$ 表示指标集)的并集,并且其中不相等的子集一定不相交,那么称集合 $\{S
    _i \mid i\in I\}$ 是 $S$ 的一个划分,记作 $\pi(S)$。
\end{definition}

\begin{theorem}
    设 $\sim$ 是集合 $S$ 上的一个等价关系,则所有等价类组成的集合是 $S$ 的一个划分,记作 $\pi_\sim(S)$。
\end{theorem}

\begin{definition}
    设 $\sim$ 是集合 $S$ 上的一个等价关系。由所有等价类组成的集合称为 $S$ 对于关系 $\sim$ 的商集,记作 $S/\sim$。
\end{definition}

注意,$S$ 的商集 $S/\sim$ 里的元素是 $S$ 的子集,不是 $S$ 的元素。

设 $\sim$ 是集合 $S$ 上的一个等价关系,一种量或一种表达式如果对于同一个等价类里的元素是相等的,那么称这种量或表达式是一个不变量。恰好能完全决定等价类的一组不变量称为完全不变量。

\section{矩阵的相抵}

数域 $K$ 上所有 $s\times n$ 矩阵组成的集合记作 $M_{s\times n}(K)$,当 $s=n$ 时简记为 $M_n(K)$。

\begin{definition}
    对于数域 $K$ 上的 $s\times n$ 矩阵 $A$ 和 $B$,如果从 $A$ 经过一系列初等行变换和初等列变换能变成矩阵 $B$,那么称 $A$ 与 $B$ 是相抵的,记作 $A\overset{\text{相抵}}{\sim}B$。
\end{definition}

相抵是集合 $M_{s\times n}(K)$ 上的一个二元关系,容易验证相抵是一个等价关系,其下矩阵 $A$ 的等价类称为 $A$ 的相抵类。

\begin{theorem}
    设数域 $K$ 上 $s\times n$ 矩阵 $A$ 的秩为 $r$,如果 $r>0$,那么 $A$ 相抵于下述形式的矩阵
    $$\left(\begin{matrix}
        E_r & 0\\
        0   & 0
    \end{matrix}\right)$$
    称其为 $A$ 的相抵标准形;如果 $r=0$,那么相抵标准形是零矩阵。
\end{theorem}

\begin{theorem}
    数域 $K$ 上 $s\times n$ 矩阵 $A$ 与 $B$ 相抵当且仅当它们的秩相等,即矩阵的秩是相抵关系下的完全不变量。
\end{theorem}

\section{广义逆矩阵}

\begin{theorem}
    设 $A$ 是数域 $K$ 上 $s\times n$ 非零矩阵,则矩阵方程
    $$AXA = A$$
    一定有解。如果 $\rank(A) = r$,并且
    $$A = P\left(\begin{matrix}
        E_r & 0\\
        0   & 0
    \end{matrix}\right)Q$$
    其中 $P,Q$ 分别是 $K$ 上 $s$ 级、$n$ 级可逆矩阵,那么矩阵方程的通解为
    $$X = Q^{-1}\left(\begin{matrix}
        E_r & B\\
        C   & D
    \end{matrix}\right)P^{-1}$$
    其中 $B,C,D$ 分别是数域 $K$ 上任意的 $r\times (s-r),(n-r)\times r,(n-r)\times (s-r)$ 矩阵。
\end{theorem}

\begin{definition}
    设 $A$ 是数域 $K$ 上 $s\times n$ 非零矩阵,则矩阵方程 $AXA = A$ 的每一个解都称为 $A$ 的一个广义逆矩阵,简称广义逆。用 $A^-$ 表示。
\end{definition}

\begin{theorem}
    非齐次线性方程组 $A\XXX = \bbb$ 有解的充分必要条件是
    $$\bbb = AA^-\bbb$$
\end{theorem}

\begin{theorem}
    非齐次线性方程组 $A\XXX = \bbb$ 有解时,它的通解为
    $$\XXX  = A^-\bbb$$
\end{theorem}

\begin{theorem}
    数域 $K$ 上 $n$ 元齐次线性方程组 $A\XXX = 0$ 的通解为
    $$\XXX  = (I_n - A^-A)\boldsymbol{Z}$$
    其中 $A^-$ 是 $A$ 的任意一个广义逆,$\boldsymbol{Z}$ 取遍 $K^n$ 中任意列向量。
\end{theorem}

% \begin{definition}
%     设 $A$ 是复数域上 $s\times n$ 矩阵,矩阵方程组
%     $$\begin{cases}
%         AXA = A,\\
%         XAX = X,\\
%         (AX)^* = AX,\\
%         (XA)^* = XA,
%     \end{cases}$$
%     称为 $A$ 的 Penrose 方程组,它的解称为 $A$ 的 Moore-Penrose 广义逆,记作 $A^+$。
% \end{definition}44

\section{矩阵的相似}

\begin{definition}
    设 $A$ 与 $B$ 都是数域 $K$ 上 $n$ 级矩阵,如果存在数域 $K$ 上一个 $n$ 级可逆矩阵 $P$,使得
    $$P^{-1}AP = B$$
    那么称 $A$ 与 $B$ 是相似的,记作 $A\sim B$。
\end{definition}

同样相似是集合 $M_{s\times n}(K)$ 上的一个二元关系,容易验证相似是一个等价关系,其下矩阵 $A$ 的等价类称为 $A$ 的相似类。

相似的矩阵具有相等的行列式和秩。

\begin{definition}
    $n$ 级矩阵 $A=(a_{ij})$ 的主对角线上元素的称为 $A$ 的迹,记作 $\tr(A)$。
\end{definition}

容易验证矩阵的迹都有如下性质
\begin{equation*}
    \begin{aligned}
        \tr(A+B) &= \tr(A) + \tr(B)\\
         \tr(kA) &= k\tr(A)\\
         \tr(AB) &= \tr(BA)
    \end{aligned}
\end{equation*}

\begin{theorem}
    相似的矩阵都有相同的迹。
\end{theorem}

这表明,矩阵的行列式、秩、迹都是相似关系下的不变量,简称为相似不变量。

如果 $n$ 级矩阵相似于一个对角矩阵,那么称 $A$ 可对角化。

\begin{theorem}
    数域 $K$ 上 $n$ 级矩阵 $A$ 可对角化的充分必要条件是,$K^n$ 中有 $n$ 个线性无关的列向量 $\aaa_1,\cdots,\aaa_n$,以及 $K$ 中有 $n$ 个数 $\lambda_1,\cdots,\lambda_n$(它们之中有些可能相等),使得
    $$A\aaa_i = \lambda_i \aaa_i, i=1,\cdots,n$$
    这时,令 $P = (\aaa_1,\cdots,\aaa_n)$,则
    $$P^{-1}AP = \diag\{\lambda_1,\cdots,\lambda_n\}$$
\end{theorem}

\section{矩阵的特征值和特征向量}

\begin{definition}
    设 $A$ 是数域 $K$ 上的一个 $n$ 级矩阵,如果 $K^n$ 中有非零列向量 $\aaa$ 使得
    $$A \aaa = \lambda_0\aaa,\text{且}\ \lambda_0\in K$$
    那么称 $\lambda_0$ 是 $A$ 的一个特征值,称 $\aaa$ 是 $A$ 的属于特征值 $\lambda_0$ 的一个特征向量。
\end{definition}

注意零向量不是特征向量。

\begin{theorem}
    设 $A$ 是数域 $K$ 上的 $n$ 级矩阵,则

    (1) $\lambda_0$ 是 $A$ 的一个特征值当且仅当 $\lambda_0$ 是 $A$ 的特征多项式 $|\lambda E-A|$ 在 $K$ 中的一个根。

    (2) $\aaa$ 是 $A$ 的属于特征值 $\lambda_0$ 的一个特征向量当且仅当 $\aaa$ 是齐次线性方程组 $(\lambda_0E-A)\XXX = \ling$  的一个解。
\end{theorem}

设 $\lambda_j$ 是 $A$ 的一个特征值,把齐次线性方程组 $(\lambda_jE-A)\XXX = \ling$ 的解空间称为 $A$ 的属于 $\lambda_j$ 的特征子空间,其中的全部非零向量都是 $A$ 的属于 $\lambda_j$ 的全部特征向量。

\begin{theorem}
    相似的矩阵有相等的特征多项式。
\end{theorem}

因此矩阵的特征多项式和特征值都是相似不变量。

\begin{definition}
    设 $A$ 是数域 $K$ 上的 $n$ 级矩阵,$\lambda_1$ 是 $A$ 的一个特征值。把 $A$ 的属于 $\lambda_1$ 的特征子空间的维数叫做特征值 $\lambda_1$ 的几何重数,而把 $\lambda_1$ 作为 $A$ 的特征多项式根的重数叫做 $\lambda_1$ 的代数重数。
\end{definition}

代数重数简称为重数。

\begin{theorem}
    设 $\lambda_1$ 是数域 $K$ 上的 $n$ 级矩阵 $A$ 的一个特征值,则 $\lambda_1$ 的几何重数不超过它的代数重数。
\end{theorem}

\section{矩阵可对角化的条件}

\begin{theorem}
    数域 $K$ 上 $n$ 级矩阵 $A$ 可对角化的充分必要条件是 $A$ 有 $n$ 个线性无关的特征向量 $\aaa_1,\cdots,\aaa_n$,此时令
    $$P=(\aaa_1,\cdots,\aaa_n)$$
    则
    $$P^{-1}AP = \diag\{\lambda_1,\cdots,\lambda_n\}$$
    其中 $\lambda_i$ 是 $\aaa_i$ 所属的特征值。上述对角矩阵称为 $A$ 的相似标准形。
\end{theorem}

\begin{theorem}
    设 $\lambda_1,\lambda_2$ 是数域 $K$ 上 $n$ 级矩阵 $A$ 的不同特征值,$\aaa_1,\cdots,\aaa_s$ 与 $\bbb_1,\cdots,\bbb_r$ 分别是 $A$ 的属于 $\lambda_1,\lambda_2$ 的线性无关的特征向量,则 $\aaa_1,\cdots,\aaa_s,\bbb_1,\cdots,\bbb_r$ 线性无关。
\end{theorem}

\begin{theorem}
    设 $\lambda_1,\cdots,\lambda_m$ 是数域 $K$ 上 $n$ 级矩阵 $A$ 的不同特征值,$\aaa_{j1},\cdots,\aaa_{jr_j}$ 是 $A$ 的属于 $\lambda_j$ 的线性无关的特征向量,$j=1,\cdots,m$,则向量组
    $$\aaa_{11},\cdots,\aaa_{1r_1},\cdots,\aaa_{m1},\cdots,\aaa_{1r_m}$$
    线性无关。
\end{theorem}

\begin{theorem}
    数域 $K$ 上 $n$ 级矩阵 $A$ 可对角化的充分必要条件是:$A$ 的特征多项式的全部复根都属于 $K$,并且 $A$ 的每个特征值的几何重数等于它的代数重数。
\end{theorem}

\section{实对称矩阵的对角化}

若对于 $n$ 级矩阵 $A,B$,存在一个 $n$ 级正交矩阵 $T$,使得 $T^{-1}AT=B$,那么称 $A$ 正交相似于 $B$。

\begin{theorem}
    实对称矩阵的特征多项式的每一个复根都是实数,从而它们都是特征值。
\end{theorem}

\begin{theorem}
    实对称矩阵 $A$ 的属于不同特征值的特征向量是正交的。
\end{theorem}

\begin{theorem}
    实对称矩阵一定正交相似于对角矩阵。
\end{theorem}

对于 $n$ 级实对称矩阵 $A$,找一个正交矩阵 $T$,使得 $T^{-1}AT$ 为对角矩阵的步骤如下。

1. 计算 $|\lambda I- A|$,求出它的全部不同的根:$\lambda_1,\cdots,\lambda_m$,它们是 $A$ 的特征值。

2. 对于每一个特征值 $\lambda_j$,求 $(\lambda_jI-A)\XXX = \ling$ 的一个基础解系 $\aaa_{j1},\cdots,\aaa_{jr_j}$;然后把它们施密特正交化和单位化,得到 $\eee_{j1},\cdots,\eee_{jr_j}$。它们也是 $A$ 的属于 $\lambda_j$ 的一个特征向量。

3. 令
$$T=(\eee_{11},\cdots,\eee_{1r_1},\cdots,\eee_{m1},\cdots,\eee_{mr_m})$$
则 $T$ 是 $n$ 级正交矩阵,且
$$T^{-1}AT = \diag\{\lambda_{1},\cdots,\lambda_{1},\cdots,\lambda_{m},\cdots,\lambda_{m}\}$$