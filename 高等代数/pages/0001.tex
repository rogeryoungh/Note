\chapter{线性方程组的解法}

\section{矩阵消元法}

形如这样左端都是未知量 $x_n$ 的一次齐次式,右端是常数,
\[a_1x_1+a_2x_2+\cdots+a_nx_n=b\]
像这样的方程称为线性方程。每个未知量前面的数称为系数,右端的项称为常数项。

含 $n$ 个未知量的线性方程组称为 $n$ 元线性方程组,它的一般形式是
\begin{equation*}
	\left\{
		\begin{matrix}
			a_{11}x_1+a_{12}x_2+\cdots+a_{1n}x_n=b_1\\
			a_{21}x_1+a_{22}x_2+\cdots+a_{2n}x_n=b_2\\
			\cdots\qquad\cdots\qquad\cdots\\
			a_{s1}x_1+a_{s2}x_2+\cdots+a_{sn}x_n=b_s
		\end{matrix}
	\right.
\end{equation*}
方程的个数 $s$ 与未知量的个数 $n$ 可以相等,也可以不等。

\begin{definition}[线性方程组的初等变换]
	线性方程组的初等变换有三种,分别为:
	
	\num{1} 把一个方程的倍数加到另一个方程上。

	\num{2} 互换两个方程的位置。

	\num{3} 用一个非零数乘某一个方程。
\end{definition}

对于线性方程组,若 $\ji{x}{n}$ 分别用数 $c_1,\cdots,c_n$ 代入后,每个方程都变成恒等式,那么称 $n$ 元有序组 $(c_1,\cdots,c_n)$ 是线性方程组的一个解。方程组所有解组成的集合称为这个线性方程组的解集,符合实际要求的解称为可行解。

通过初等变换能够使线性方程组变为阶梯形方程组,进一步可以变为简化阶梯形方程组,此种形式可以较方便的看出方程组的解。

\begin{theorem}
	初等变换不改变线性方程组的解。
\end{theorem}

可以把原线性方程组的系数和常数项按次序排成一张表,称为方程组的增广矩阵;而只列出系数的方程组称为系数矩阵。

\begin{definition}
	由 $sn$ 个数排成的 $s$ 行(横的)$n$ 列(纵的)表
	\begin{equation*}
		\left(
			\begin{matrix}
				a_{11}&a_{12}&\ldots&a_{1n}\\
				a_{21}&a_{22}&\ldots&a_{2n}\\
				\vdots&\vdots&&\vdots\\a_{s1}&a_{s2}&\ldots&a_{sn}\\
			\end{matrix}
		\right)
	\end{equation*}
	称为一个 $s\times n$ 矩阵,记作 $A_{s\times n}$ 或 $A=(a_{ij})$,它的 $(i,j)$ 元也记作 $A(i;j)$。
\end{definition}

特殊的,如果矩阵 $A$ 的行数和列数相等皆为 $n$,则称它为 $n$ 级方阵或方阵。元素全为 $0$ 的矩阵称为零矩阵,记作 $0_{s\times n}$ 或 $0$。

\begin{definition}[初等行变换]\index{chudenghangbianhuan@初等行变换}
	矩阵的初等行变换有三种,分别为:
	
	\num{1} 把一行的倍数加到另一行上。

	\num{2} 互换两行的位置。

	\num{3} 用一个非零数乘某一行。
\end{definition}

矩阵经过初等行变换,可变成阶梯形矩阵,并可进一步化简成简化行阶梯形矩阵。

阶梯形矩阵的特点为 \num{1}元素全为 $0$ 的行(零行)在下方(如果有的话);\num{2}元素不全为 $0$ 的行(非零行),左起第一个不为 $0$ 的元素(主元),他们的列指标随着行指标递增而严格增大。

简化行阶梯形矩阵(行最简形矩阵)的特点为 \num{1}它是阶梯形矩阵;\num{2}每个非零行的主元都是 $1$;\num{3}每个主元所在的列的其余元素都是 $0$。

在解线性方程组时,可以通过一系列初等行变换,它的增广矩阵化为阶梯形矩阵,甚至继续化简为简化行阶梯形矩阵,都可简化求解过程。

\begin{theorem}
	任意矩阵都可以经过一系列初等行变换化为阶梯形矩阵,也可以变成简化行阶梯形矩阵。
\end{theorem}

\section{线性方程组的解的情况及其判别准则}


由于初等变换不改变线性方程组的解,其总可以化为阶梯形方程组。因此设阶梯形方程组有 $n$ 个未知量,它的增广矩阵 $J$ 有 $r$ 个非零行,$J$ 有 $n+1$ 列。

1. 若阶梯形方程组中出现 $0=d$(其中 $d$ 为非零数)这种方程,即最后一个非零行的主元位于 $n+1$ 列,则阶梯形方程组无解。

2. 最后一个非零行的主元不位于 $n+1$ 列。

2(1). $r=n$ 时,阶梯形方程恰有唯一解。

2(2). $r<n$ 时,有无穷多组解。

\begin{theorem}
	系数为有理数(实数、复数)的 $n$ 元线性方程组的解的情况只有三种可能:无解,有唯一解,有无穷多组解。
\end{theorem}

若一个线性方程组有解,则称它是相容的;否则称它是不相容的。

\section{数域}

\begin{definition}
	复数集的一个子集 $K$ 是一个数域,那么满足:
	
	\num{1} $0,1 \in K$;
	
	\num{2} $a,b \in K \Rightarrow a \pm b,ab \in K$;
	
	\num{3} $a,b \in K$,且 $b \ne 0 \Rightarrow \dfrac{a}{b} \in K$。
\end{definition}

其中,$\QQ,\RR,\CC$ 都是数域,但整数集 $\mathrm{Z}$ 不是数域。

有理数域是最小的数域。

\begin{theorem}
	任意数域都包含有理数域。
\end{theorem}


