\chapter{线性空间}

\section{域 F 上的线性空间的基与维数}

\begin{definition}
    设非空集合 $V$;设数域 $F$。在其上有向量的加法和纯量与向量的乘法。若满足以下公理

    A1 加法结合律:$(\aaa+\bbb) + \ggg = \aaa +(\bbb+\ggg)$。
    
    A2 加法交换律:$\aaa + \bbb = \bbb + \aaa$。
    
    A3 加法存在单位元 $\ling$:$\aaa + \ling = \aaa$。
    
    A4 加法逆元的存在性:对任意的 $\aaa \in V$ 总存在 $-\aaa\in V$,使得 $\aaa + (-\aaa) = \ling$。
    
    M1 乘法结合律:$\lambda(\mu \aaa) = (\lambda \mu) \aaa$。
    
    M2 乘法存在单位元 $\mathbf{1}$:$\mathbf{1} \aaa = \mathbf{1}$。
    
    D1 分配律 1:$\lambda(\aaa+\bbb) = \lambda \aaa + \lambda \bbb$。
    
    D2 分配律 2:$(\lambda + \mu)\aaa = \lambda \aaa + \mu \aaa$。

    那么称 $V$ 是域 $F$ 上的一个线性空间,$V$ 的元素称为向量,$F$ 的元素称为纯量。
\end{definition}