\chapter{数学}

\section{GCD 和 LCM}

\lstinputlisting[style=cpp,caption=/数学/gcdlcm.cpp]{数学/gcdlcm.cpp}

\section{EXGCD}

对于方程 
\[ax+by=\gcd(a,b)\]
可通过 $\rm exgcd$ 求出一个整数解。

\lstinputlisting[style=cpp,caption=/数学/exgcd.cpp]{数学/exgcd.cpp}

方程 $ax+by=c$ 有解的充要条件是 $\gcd(a,b) \mid c$。

\lstinputlisting[style=cpp,caption=/数学/liEu.cpp]{数学/liEu.cpp}

\section{乘法逆元}

方程 $ax \equiv 1 \pmod p$ 有解的充要条件是 $\gcd(a,p) = 1$。

容易想到它与方程 $ax + py = c$ 等价,于是可以利用 $\rm exgcd$ 求最小正解。

\lstinputlisting[style=cpp,caption=/数学/逆元/exgcd法.cpp]{数学/逆元/exgcd法.cpp}

仅当 $p$ 为质数时,由 Fermat 小定理知 $x \equiv a^{p-2} \pmod p$。

\lstinputlisting[style=cpp,caption=/数学/逆元/快速幂法.cpp]{数学/逆元/快速幂法.cpp}

\section{筛法}

\paragraph{Eratosthenes 筛}

复杂度 $O(n\log \log n)$。

\lstinputlisting[style=cpp,caption=/数学/筛法/Eratosthenes.cpp]{数学/筛法/Eratosthenes.cpp}

\paragraph{Eular 筛}

复杂度 $O(n)$,每个合数只会被筛一次。

\lstinputlisting[style=cpp,caption=/数学/筛法/Eular.cpp]{数学/筛法/Eular.cpp}

\section{素性测试}

\subsection{试除法}

\lstinputlisting[style=cpp,caption=/数学/试除法.cpp]{数学/试除法.cpp}

\subsection{Miller Rabbin}

如果 $n\leqslant 2^{32}$,那么 $ppp$ 取 $2,7,61$;如果 $ppp$ 选择 $2,3,7,61,24251$,那么 $10^{16}$ 内只有唯一的例外。如果莫名 WA 了,就多取点素数吧。

\lstinputlisting[style=cpp,caption=/数学/Miller\_Rabbin.cpp]{数学/Miller_Rabbin.cpp}

\section{Lucas 定理}

当 $n,m$ 很大而 $p$ 较小的时候,有
\[  
\binom{n}{m}\bmod p = \binom{\left\lfloor n/p \right\rfloor}{\left\lfloor m/p\right\rfloor}\cdot\binom{n\bmod p}{m\bmod p}\bmod p 
\]

\lstinputlisting[style=cpp,caption=/数学/Lucas.cpp]{数学/Lucas.cpp}

\section{约瑟夫 Josephus 问题}

对 $n$ 个人进行标号 $0,\cdots,n-1$,顺时针站一圈。从 $0$ 号开始,每一次从当前的人继续顺时针数 $k$ 个,然后让这个人出局,如此反复。

设最后剩下的人的编号为 $J(n,k)$,有递推式
\[J(n+1,k) = (J(n,k)+k) \bmod (n+1)\]
踢出第一个人 $k$ 后,剩下就转化为 $J(n,k)$ 的情景,还原编号只需增加相对位移 $k$。

\lstinputlisting[style=cpp,caption=/数学/Josephus.cpp]{数学/Josephus.cpp}

\section{中国剩余定理}

若 $n_i$ 中任意两个互质,求方程组的解
\[\begin{cases}
    x \equiv a_1 &\pmod {n_1} \\ x \equiv a_2 &\pmod {n_2} \\ &\vdots \\ x \equiv a_k &\pmod {n_k} 
\end{cases}\]

\lstinputlisting[style=cpp,caption=/数学/china.cpp]{数学/china.cpp}

\section{博弈}

下面都是石子游戏,轮流取走物品。方便起见,称场上 $n$ 堆石子 $a_1,\cdots,a_n$ 为局势。先手必输的局势称为奇异局势

\subsection{Nim 博弈}

有 $n$ 堆分别有 $a_i$ 个物品,两人轮流取走任意一堆的任意个物品,不能不取,最后取光者获胜。奇异局势判定
\[a_1 \oplus \cdots \oplus a_n =0\]

\subsection{Wythoff 博奕}

两堆分别有 $a,b$ 各物品,两个人轮流从某一堆或同时从两堆中取同样多的物品,不可不取,最后取光者获胜。

\lstinputlisting[style=cpp,caption=/数学/博弈/Wythoff.cpp]{数学/博弈/Wythoff.cpp}

特点:所有自然数都出现在奇异局势中,不重不漏。