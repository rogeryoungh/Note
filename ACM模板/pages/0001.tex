\chapter{上号}

\section{头文件}

\lstinputlisting[style=cpp,caption=/上号/头文件.hpp]{上号/头文件.hpp}

\section{预编译}

头文件引入方式改为如下,可以把头文件放入 \lstinline[style=cpp]{lab.hpp} ,然后使用 \lstinline[language=bash]{clang++ lab.hpp} 预编译。

实际编译使用 \lstinline[language=bash]{clang++ lab.cpp -D RYLOCAL} 添加条件编译参数。

\begin{lstlisting}[style=cpp]
#ifdef RYLOCAL
#include "lab.hpp"
#else  //#include <bits/stdc++.h>
#include <iostream>  #include <cstring>    #include <functional>
#include <cmath>     #include <cstdio>     #include <algorithm>
#endif
\end{lstlisting}

\section{进制转换}

\lstinputlisting[style=cpp,caption=/上号/进制转换.cpp]{上号/进制转换.cpp}

\section{常见技巧}

向上取整 $p/q$ 为 \lstinline[style=cpp]{(p-1)/q+1} 。

预计算 $\log_n$,只需 \lstinline[style=cpp]{_fora(i, n, MN) logn[i] = logn[i/n] + 1;}。

字典序 \lstinline[style=cpp]{strcmp(x,y) < 0}。

\section{二分查找}

\paragraph*{STL 二分}

在 $[l,r)$ 查找 $\geqslant value$ 中最前的一个,找不到则返回 $r$ 。支持 cmp 函数。

\begin{lstlisting}[style=cpp]
ForwardIt lower_bound(ForwardIt l, ForwardIt r, const T& value);
\end{lstlisting}

在 $[l,r)$ 查找 $> value$ 中最前的一个,找不到则返回 $r$ 。支持 cmp 函数。

\begin{lstlisting}[style=cpp]
ForwardIt upper_bound(ForwardIt l, ForwardIt r, const T& value);
\end{lstlisting}

手写二分,在单增(单减)数组中查找 $\geqslant x(\leqslant x)$ 的数中最前的一个。

\lstinputlisting[style=cpp,caption=/上号/二分/01.cpp]{上号/二分/01.cpp}

在单增(单减)数组中查找 $\leqslant x(\geqslant x)$ 的数中最后的一个。

\lstinputlisting[style=cpp,caption=/上号/二分/02.cpp]{上号/二分/02.cpp}

对于上凸($\wedge$ 形)函数,可以使用三分法来查找最大值。对于下凸($\vee$ 形)变号即可

\lstinputlisting[style=cpp,caption=/上号/三分法.cpp]{上号/三分法.cpp}

\section{矩阵乘法}

构建一个 $p$ 行 $q$ 列的矩阵。

\lstinputlisting[style=cpp,caption=/上号/矩阵乘法.cpp]{上号/矩阵乘法.cpp}

矩阵的输入、输出。

\begin{lstlisting}[style=cpp]
void read(Mtx& mtx) {
    _fora(i, 1, mtx.p) _fora(j, 1, mtx.q)
        mtx.m[i][j] = (MOD + rr()) % MOD;
}
void pr(Mtx mtx) {
    _fora(i, 1, mtx.p) {
        printf("%lld",mtx.m[i][1]);
        _fora(j, 2, mtx.q)
            printf(" %lld",mtx.m[i][j]);
        putchar('\n');
    }
}
\end{lstlisting}

\section{快速幂}

\lstinputlisting[style=cpp,caption=/上号/快速幂.cpp]{上号/快速幂.cpp}

\lstinputlisting[style=cpp,caption=/上号/矩阵快速幂.cpp]{上号/矩阵快速幂.cpp}

\section{快速排序}

\lstinputlisting[style=cpp,caption=/上号/快速排序.cpp]{上号/快速排序.cpp}

\section{\texorpdfstring{第 $k$ 大数}{第 k 大数}}

\lstinputlisting[style=cpp,caption=/上号/第k大数.cpp]{上号/第k大数.cpp}
