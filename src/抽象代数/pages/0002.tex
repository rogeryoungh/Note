\chapter{群}

\section{置换}

\newcommand{\NNn}{\mathbb{N}_n^{+}}

为方便起见,本节简记集合 $\{1,\cdots,n\}$ 为 $\NNn$。

\begin{definition}
	设 $X$ 是一个集合,则 $X$ 中的一个表是指函数 $f:\NNn \to X$。若 $X$ 中的表 $f$ 是双射,则称 $f$ 为 $X$ 的一个排列。
\end{definition}

因此,$X$ 的排列是 $X$ 的所有元素组成的一个无重复的 $n$ 元组。显然 $n$ 元集恰有 $n^n$ 个表和 $n!$ 个排列。

\begin{definition}[置换]
	设 $X$ 是一个集合(可能是无限集),$X$ 的一个置换是指双射 $\alpha : X \to X$。
\end{definition}

给定一个有限集 $X$,$|X|=n$, 设 $\phi : \NNn \to X$ 是一个排列,若 $f : \NNn \to X$ 是 $X$ 的一个排列,则 $f \circ \phi^{-1} : X \to X$ 是 $X$ 的一个置换。反之,若 $\alpha : X \to X$ 是 $X$ 的一个置换,则 $\alpha \circ \phi : \NNn \to X$ 是 $X$ 的一个排列。

即排列和置换只是描述同一事物的两种不同方法,使用置换而不是排列,其好处是置换可做合成运算。

若 $X = \NNn$,则我们可以使用一个二行记号来表示置换 $\alpha$:
\[\alpha = \left(\begin{matrix}
			1         & 2         & \cdots & j         & \cdots & n          \\
			\alpha(1) & \alpha(2) & \cdots & \alpha(j) & \cdots & \alpha(n)
		\end{matrix}\right)\]
其底行是排列 $\alpha(1) , \alpha(2) , \cdots , \alpha(n)$。

\begin{definition}[对称群]
	集合 $X$ 的所有置换构成的族,记为 $S_X$,称为 $X$ 上的对称群。当 $X = \NNn$ 时, $S_X$ 通常记为 $S_n$,并称为 $n$ 次对称群。
\end{definition}

注意到,有些置换是交换的,有些置换又不是交换的。







\let\NNn\relax
