\chapter{初等数论}

注意我们的理论基础是整数,尽量通过分类讨论的方式得到结论。而且也要把握脉络,抓住重点,不要迷失于无谓的细节中。

自然数 $\NN$ 、正整数 $\NN^+$ 和整数 $\ZZ$ 我们是熟知的。

\section{整除}

\subsection{整数公理}

整数的公理

我们熟知一些整数的代数算律

结合律:$(a+b)+c = (a+b)+c$。

交换律:$a+b = b+a$。

消去律:

\begin{definition}
	对于整数 $a,b$,其中 $a\ne 0$,若存在整数 $c$,它使得
	$$b=ac$$
	则 $b$ 叫做 $a$ 的倍数,$a$ 叫做 $b$ 的因数,记作 $a \mid b$。
\end{definition}

有时也称作 $a$ 能整除 $b$,或 $b$ 能被 $a$ 整除,或 $a$ 能除尽 $b$,或 $b$ 能被 $a$ 除尽。

若 $a$ 不能整除 $b$,我们就记作 $a \nmid b$。

\begin{lemma}
	如果对于整数 $a,b$ 满足 $a \mid b$,则有
	$$(-a) \mid b,\quad a \mid (-b),\quad (-a) \mid (-b),\quad |a| \mid |b|$$
\end{lemma}

这个比较显然,由定义知存在 $c$ 使得 $b=ac$,再构造验证即可。

\begin{lemma}
	对于整数 $a,b,c$ 有 $a \mid b,b \mid c$,则有 $a \mid c$。
\end{lemma}

\begin{proof}
	因为 $a \mid b,b \mid c$,故存在整数 $d,e$ 使得 $b=ad,c=be$。

	因此存在整数 $f=de$ 使得 $c=af=ade$,故 $a \mid c$。
\end{proof}

\begin{lemma}
	对于整数 $a,b$ 有 $|a| \mid |b|$,若 $|a|<|b|$ 则有 $a=0$。
\end{lemma}

\begin{proof}
	因为 $|a| \mid |b|$,则存在整数 $c$ 使得 $|a|=|b|c$。那么有
	$$0 \leqslant |a|=|b|c<|b|$$
	即 $0\leqslant c<1$,又 $c$ 为整数,故 $c=a=0$。
\end{proof}

\begin{theorem}
	对于整数 $a,b$,若 $b\ne 0$ 则一定存在唯一一对 $q,r$ 使得
	$$a=bq+r,\quad 0 \leqslant r< |b|$$
\end{theorem}

\begin{proof}
	先证明存在性。
	
	(1) 若恰 $b \mid a$,则必存在 $c$ 使得 $a=bc$,此时有 $q=c,r=0$。

	(2) 否则一定存在 $n$ 使得 $n|b|<a<(n+1)|b|$,即存在 $0<r<|b|$ 使得 $a=|b|n+r$。
	
	当 $b>0$ 时,令 $q=n$;当 $b<0$ 时,令 $q=-n$ 则有
	$$a=bq+r,\quad 0 \leqslant r< |b|$$

	再证明唯一性。设存在两对 $q_1,r_1$ 和 $q_2,r_2$ 使得
	$$a=bq_1+r_1=bq_2+r_2,\quad 0 \leqslant r_1,r_2< |b|$$
	相减有
	$$a-a=b(q_1-q_2)+r_1-r_2=0$$
	即 $r_1-r_2=-b(q_1-q_2)$,因此有 $b \mid (r_1-r_2)$。而 $|r_1-r_2|<|b|$,又引理知有 $|r_1-r_2|=0$。故
	$$r_1=r_2,q_1=r_2$$
	即两对相同。
\end{proof}

\begin{definition}[素数]
    设整数 $p \ne 0,\pm 1$,若它除了 $\pm 1, \pm p$ 外没有其他的因数,则称 $p$ 是素数;否则称 $p$ 是合数。
\end{definition}

我们讲到素数时,一般指正的。把素数的集合记作 $\PP$。

\begin{theorem}
    若 $a$ 是合数,则必存在素数 $p$ 使得 $p \mid a$。
\end{theorem}

此时称该素数为 $a$ 的素因数。

\begin{theorem}
    设整数 $a \geqslant 2$,那么 $a$ 一定可以分解为素数的乘积,即
    \[ a = p_1p_2\cdots p_s \]
    其中 $p_j \in \PP$。
\end{theorem}

OI 中,经常会求符合命题 $P(k)$ 的数 $k$ 有多少个,此时我们有记号 $[P(k)]$,当命题成立时其值为 $1$,命题为假时值为 $0$。

\subsection{公因数与公倍数}

\begin{definition}[公因数]
    设 $a_1,a_2$ 是两个整数,若 $d \mid a_1 $ 且 $ d \mid a_2$,则称 $d$ 是 $a_1,a_2$ 的公因数。一般的,若对于一组整数 $a_1,\cdots,a_k$,有 $d \mid a_i$,则称 $d$ 是 $a_1,\cdots,a_k$ 的公因数。
\end{definition}

把 $a_1,a_2$ 的正的公因数中最大的,称作最大公因数,记作 $(a_1,a_2)$ 或 $\gcd(a_1,a_2)$。

由定义易知,若 $(a_1,a_2) = d$,则 $(a_1/d, a_2/d) = 1$。

\begin{definition}[互素]
    若 $(a_1,a_2) = 1$,则称 $a_1,a_2$ 是互素的。
\end{definition}

类似的,对于多个数也类似的有最大公因数和互素等概念。

\begin{definition}[公倍数]
    设 $a_1,a_2$ 是两个整数,多 $a_1 \mid l$ 且 $a_2 \mid l$,则称 $l$ 是 $a_1,a_2$ 的公倍数。一般的,若对于一组整数 $a_1,\cdots,a_k$,有 $a_j \mid l$,则称 $l$ 是 $a_1,\cdots,a_k$ 的公倍数。
\end{definition}

把 $a_1,a_2$ 的正的公倍数中最小的,称作最小公因数,记作 $[a_1,a_2]$ 或 $\lcm(a_1,a_2)$。

由定义易知,对于 $m > 0$ 有 $[ma_1,ma_2] = m[a_1,a_2]$。

\subsection{带余除法}

\begin{theorem}
    设整数 $a,b$ 且 $a \ne 0$,则一定存在唯一的一对整数 $q,r$ 使得
    \[ b = qa + r, 0 \leqslant r < |a| \]
\end{theorem}

更一般的,对于任意的 $d$ 总存在一对 $q,r$ 使得
\[ b = qa + r, d \leqslant r < |a| + d \]

当 $d = 0$ 时,称 $r$ 为最小非负余数,$d = 1$ 时称 $r$ 为最小正余数。计算机一般是 $d = 0$。

\begin{lemma}
    设 $a > 0$,则任意整数被 $a$ 除后所得的最小非负余数只可能是 $0,\cdots,a-1$ 中的一个。 
\end{lemma}

于是我们可以按余数对整数进行分类。