\chapter{概率论}

\section{随机事件和概率}

\subsection{大纲要求}

1. 了解样本空间(基本事件空间)的概念,理解随机事件的概念,掌握事件的关系及运算。

2. 理解概率、条件概率的概念,掌握概率的基本性质,会计算古典型概率和几何型概率,掌握概率的加法公式、减法公式、乘法公式、全概率公式以及贝叶斯(Bayes)公式。

3. 理解事件独立性的概念,掌握用事件独立性进行概率计算;理解独立重复试验的概念,掌握计算有关事件概率的方法。


\section{随机变量及其分布}

\subsection{大纲要求}

1. 理解随机变量的概念,理解分布函数
$$
F(x) = P\{X \leqslant x\}(-\infty < x < +\infty)
$$
的概念及性质,会计算与随机变量相联系的事件的概率。

2. 理解离散型随机变量及其概率分布的概念,掌握 $0-1$ 分布、二项分布 $B(n, p)$、几何分布、超几何分布、泊松(Poisson)分布 $P(\lambda)$ 及其应用。

3. 了解泊松定理的结论和应用条件,会用泊松分布近似表示二项分布。

4. 理解连续型随机变量及其概率密度的概念,掌握均匀分布 $U(a, b)$、 正态分布 $N(\mu, \sigma^2)$、指数分布及其应用,其中参数为 $\lambda(\lambda > 0)$ 的指数分布 $E(\lambda)$ 的概率密度为
$$
f(x) = \begin{cases} \lambda e^{-\lambda x}, & x > 0 \\ 0, &x \leqslant 0 \end{cases}
$$
5. 会求随机变量函数的分布。


\section{多维随机变量及其分布}

\subsection{大纲要求}

1. 理解多维随机变量的概念,理解多维随机变量的分布的概念和性质,理解二维离散型随机变量的概率分布、边缘分布和条件分布,理解二维连续型随机变量的概率密度、边缘密度和条件密度,会求与二维随机变量相关事件的概率。

2. 理解随机变量的独立性及不相关性的概念,掌握随机变量相互独立的条件。

3. 掌握二维均匀分布,了解二维正态分布 $N(\mu_1, \mu_2; \sigma_1^2, \sigma_2^2; \rho)$ 的概率密度,理解其中参数的概率意义。

4. 会求两个随机变量简单函数的分布,会求多个相互独立随机变量简单函数的分布。


\section{随机变量的数字特征}

\subsection{大纲要求}

1. 理解随机变量数字特征(数学期望、方差、标准差、矩、协方差、相关系数)的概念,会运用数字特征的基本性质,并掌握常用分布的数字特征。

2. 会求随机变量函数的数学期望。


\section{大数定律和中心极限定理}

\subsection{大纲要求}

1. 了解切比雪夫不等式。

2. 了解切比雪夫大数定律、伯努利大数定律和辛钦大数定律(独立同分布随机变量序列的大数定律)。

3. 了解棣莫弗-拉普拉斯定理(二项分布以正态分布为极限分布)和列维-林德伯格定理(独立同分布随机变量序列的中心极限定理)。


\section{数理统计的基本概念}

\subsection{大纲要求}

1. 理解总体、简单随机样本、统计量、样本均值、样本方差及样本矩的概念,其中样本
方差定义为
\[ S^2 = \frac{1}{n-1} \sum_{t=1}^n (\chi_1 = \chi) \cal \]
2. 了解 $\chi^2$ 分布、$t$ 分布、$F$ 分布的概念及性质,了解上侧 $\alpha$ 分位数的概念并会查表计算。

3. 了解正态总体的常用抽样分布。


\section{参数估计}

\subsection{大纲要求}

1. 理解参数的点估计、估计量与估计值的概念。

2. 掌握矩估计法(一阶矩、二阶矩)和最大似然估计法。

3. 了解估计量的无偏性、有效性(最小方差性)和一致性(相合性)的概念,并会验证估计量的无偏性。

4. 理解区间估计的概念,会求单个正态总体的均值和方差的置信区间,会求两个正态总
体的均值差和方差比的置信区间。


\section{假设检验}

\subsection{大纲要求}

1. 理解显著性检验的基本思想,掌握假设检验的基本步骤,了解假设检验可能产生的两类错误。

2. 掌握单个及两个正态总体的均值和方差的假设检验
