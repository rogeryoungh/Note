\chapter{实数集与函数}

我初次用的书是华师的数分,现在发觉基础部分有相当多的细节。后参考自李逸的《基本分析讲义》。

若无额外说明,皆在 $\mathbb{R}$ 下。仍有很多术语不曾了解。

\section{前言}

实数理论确实很难以理解。

我初次接触华师数分时,实数看的云里雾里,不知道为什么要罗列一大堆定理,索性直接跳过,读的倒也算通顺。初入门径后,又读了几本其他的书,才感觉到实数理论的意义;但是感觉公理又多又乱,在脑子里缠起来了。继续读下去,观点再高了一点,终于敢说懂了一点。

我尝试讲一讲。按照 Bourbaki 的观点,序结构连同拓扑和代数结构一道组成了数学结构的三大母体。具体的说,如果只有一个光秃秃的集合,我们做不了太多事情。为了在这个集合上展开进一步的讨论,我们需要对它装备一些结构:
\footnote{\href{https://www.zhihu.com/question/47999353/answer/1012530744}{知乎:如何理解数学中的序结构,代数结构和拓扑结构?}}

1. 序结构:元素和元素的排序,比如实数上的大小关系、集合的包含关系。

2. 代数结构:元素和元素的运算,比如加法和乘法。

3. 拓扑结构:子集和子集之间的关系,比如点集的邻近性、敛散性、连续性。

学习实数,主要是抓住这三个方面的性质。接触了更多具体的例子,会对抽象的定义有更多的感悟。

\section{集合和映射}

集合的交并补是熟知的。

定义有序对为 $(a,b) \coloneqq  \{\{a\},\{a,b\}\}$,其中 $a$ 称为有序对的第一坐标,而 $b$ 称为第二坐标。特殊的,$(a,a) = \{a\}$。

定义集合的笛卡尔 Cartesian 乘积为
\[ A \times B \coloneqq  \{(a,b) \mid a\in A \text{且} b\in B\}\]
一般 $A \times B \ne B \times A$。同样可以推广到多个集合
\[ \prod X_i \coloneqq  X_1 \times X_2 \times \cdots \times X_n = (X_1 \times \cdots \times X_{n-1}) \times X_n\]
其元素 $x$ 是多层嵌套,我们可以简记为
\[ x = (\cdots(x_1,x_2),x_3),\cdots,x_n) = (\seq{x}{n})\]
称 $x_i \coloneqq  \mathrm{pr}_i(x)$ 为 $x$ 的第 $i$ 个分量,$\mathrm{pr}$ 是投影映射。当所有 $X_i$ 都等于 $X$ 时,上述乘积记为 $X^n$。

设 $C$ 和 $D$ 为两个给定的集合。

\begin{definition}[赋值法则]
	设 $R$ 是 $C\times D$ 的一个子集,若满足当 $(c,d_1)\in R$ 且 $(c,d_2)\in R \Rightarrow d_1=d_2$,称 $R$ 是一个赋值法则。
\end{definition}

赋值法则的定义域 Domain 和像域 Image Set 约定如下
\[ \begin{aligned}
		\operatorname{dom}(R) \coloneqq  \{c\in C \mid \exists d \in D, (c,d) \in R\} \\
		\operatorname{Im}(R) \coloneqq  \{d\in D \mid \exists c \in C, (c,d) \in R\}  \\
	\end{aligned} \]

\begin{definition}
	设 $R$ 为一个赋值法则,$B$ 为满足 $\operatorname{Im}(R) \subseteq B$ 的一个集合,记二元对 $(R,B)$ 为一个映射,$B$ 称为值域。定义 $f$ 的定义域 $A$ 和像域为 $R$ 的定义域和像域。记作
	\[ f:A\to B, a\mapsto f(a) \]
\end{definition}

称 $f$ 的图为
\[ \operatorname{graph}(f) \coloneqq  \{(a,f(a)) \in A\times B \mid a \in A\}\]

对任意给定的 $A$ 的子集 $A_0$,定义 $f$ 在 $A_0$ 上的限制为映射
\[ f \mid_{A_0} = f : A_0 \to B\]

称映射 $f$ 和 $g$ 的复合为
\[ g\circ f: A \to C, a \mapsto g(f(a))\]
显然 $g \circ f$ 仅当 $\operatorname{Im}(f) \subseteq \operatorname{dom}(g)$ 时有定义。$f\circ g$ 一般与 $g \circ f$ 不相等。

\begin{itemize}
	\item 若映射 $f$ 满足 $f(a_1) = f(a_2) \Rightarrow a_1,a_2$,称 $f$ 为单射。
	\item 若映射 $f$ 满足对任意的 $b\in B$ 存在 $a\in A$ 满足 $f(a)=b$,称 $f$ 为满射。
	\item 若映射 $f$ 满足$f$ 既是单射又是满射,称 $f$ 为双射。
\end{itemize}

若 $f$ 为双射,我们定义它的逆映射 $f^{-1}$ 为
\[ f^{-1}(b) =  a \Leftrightarrow f(a) = b\]

映射 $\ast: X\times X \to X$ 通常也称为集合 $X$ 上的运算,此时我们把 $\ast(x,y)$ 记做 $x \ast y$。对 $X$ 中的非空子集定义
\[ A \ast B \coloneqq  \ast(a\times B) = \{A \ast b \mid a \in A, b\in B\}\]

如果映射定义中的 $B$ 是一个数域,则把映射称为函数。

\section{序关系}

称集合 $S\times S$ 的子集 $\odot$ 为关系。把 $(x,y) \in \odot$ 记作 $x \odot y$。

\begin{definition}[等价关系]
	若集合 $S$ 上的关系 $\simeq$ 满足

	\begin{itemize}
		\item 自反性:对任意的 $x\in S$ 有 $x \simeq x$。

		\item 对称性:若 $x \simeq y$ 则 $y \simeq x$。

		\item 传递性:若 $x \simeq y$ 且 $y \simeq z$ 则 $x \simeq z$。
	\end{itemize}

	则称 $\simeq$ 为等价关系,一般记作 $\sim$ 或 $=$。
\end{definition}

记 $x\in A$ 的等价类:
\[ [x] \coloneqq  \{y \in A \mid y \sim x\}\]

\begin{definition}[全序关系]
	若集合 $S$ 上的关系 $\preceq$ 满足

	\begin{itemize}
		\item 反对称性:若 $x \preceq y$ 且 $y \preceq x$ 则 $x = y$。

		\item 传递性:若 $x \preceq y$ 且 $y \preceq z$ 则 $x \preceq z$。

		\item 完全性:对任意的 $x,y\in S$ 要么 $x \preceq y$ 要么 $y \preceq x$。(注意蕴含了自反性 $x \preceq x$)。
	\end{itemize}
	则称 $\preceq$ 为全序关系, $(S,\preceq)$ 为全序集。一般用 $\leqslant$ 来表示全序关系.
\end{definition}

对于每一非严格的全序关系 $\leqslant$,定义其对应的严格全序关系 $<$ 为:$a < b$ 等价于 $a \leqslant b$ 且 $a \neq b$。我们对四个关系 $<, \leqslant, >, \geqslant$ 都可以定义全序集。

\begin{definition}
	设全序集 $S' \subseteq S$,若 $x \in S$ 是:

	\begin{itemize}
		\item $S$ 的最大元:若不存在 $y \in S$ 使得 $x < y$,则称 $x$ 是最大元。

		\item $S'$ 的上界:若对于任取的 $x' \in S'$ 满足 $x' \leqslant x$,则称 $x$ 是 $S'$ 的上界。

		\item $S'$ 的上确界:对于任取的 $S'$ 的上界 $y$ 都有 $x \leqslant y$,且 $x$ 是 $S'$ 的上界,则称 $x$ 是 $S'$ 的上确界。
	\end{itemize}
\end{definition}

类似的可定义全序集的最小元、下界、下确界。上、下确界分别记作 $\sup S'$ 和 $\inf S'$。比如取 $A = A' = (0, 1)$,其上下确界都不存在。若取 $A = \mathbb{R}$,上下确界存在也不一定在 $A'$ 里面。

\begin{definition}[最小上界性]
	如果集合 $S$ 的任何非空有上界子集 $S'$ 有最小上界,则称 $S$ 有最小上界性。

	换句话说,若任取非空子集 $S' \subseteq S$,若 $S'$ 在 $S$ 内存在上界,那么 $S'$ 在 $S$ 内存在上确界。
\end{definition}



例如 $\mathbb{Q} \cap (0,\sqrt2)$ 是 $\mathbb{Q}$ 的非空子集,其上界 $\sqrt2$ 并不在 $\mathbb{Q}$ 内。

同样的有最小下界性。可以证明,有最小上界性的一定有最大下界性。展开描述即

\begin{theorem}
	设 $B$ 是具有最小上界性的集合 $S$ 的子集,则对任意的有下界的 $B$ 都有 $\inf B \in S$。
\end{theorem}

\begin{proof}
	对于每个 $B$,构造 $L$ 为 $B$ 的下界组成的集合,显然每个 $B$ 中的元素都是 $L$ 的上界。由最小上界性知,存在 $\sup L \in S$。

	尝试证明 $\inf B=\sup L$。

	对于任意的 $x\in B$,若 $x<\sup L$,则存在比 $\sup L$ 小的 $L$ 的上界 $x$,矛盾。故 $x \geqslant \sup L$,即 $\sup L$ 是 $B$ 的下界。

	设 $B$ 的下界 $x$ 有 $x>\sup L$,那么 $x\in L$,则存在比 $\sup L$ 大的 $L$ 元素,矛盾。故不存在比 $\sup L$ 大的 $B$ 的下界。

	综上,$B$ 的下界存在且 $\inf B=\sup L \in S$。
\end{proof}

\section{代数初步}

首先给出一些性质。

\begin{definition}
	给定集合 $A$,设元素 $x, y, z \in A$,定义如下性质

	\begin{itemize}
		\item 封闭性:若 $A \ast A \subseteq A$,则称 $A$ 在运算 $\ast$ 下封闭。

		\item 结合性:若 $x \ast (y \ast z) = (x \ast y) \ast z$,则称运算 $\ast$ 是结合的。

		\item 交换性:若 $x \ast y = y \ast x$,则称运算 $\ast$ 是交换的。

		\item 单位元(幺元):若存在 $e \in A$ 使得 $e \ast x = x \ast e = x$,则称 $e$ 为 $A$ 上的单位元。

		\item 逆元(么元):若存在 $x^{-1} \in A$ 使得 $x \ast x^{-1} = x^{-1} \ast x = e$,则称 $x^{-1}$ 为 $x$ 在 $A$ 上的逆元。
	\end{itemize}

\end{definition}

注意 $-x$ 和 $x^{-1}$ 只是记号,不代表我们定义出了减法和除法运算。

\begin{definition}[群,半群]
	给定集合 $G$ 和其上的二元运算 $\ast: G \times G \to G$,若:

	\begin{itemize}
		\item 满足结合律,称为半群。

		\item 满足单位元、交换性的半群,称为幺半群、交换半群。

		\item 每个元素都可逆的幺半群,称为群。

		\item 满足交换律的群,称为交换群(Abelian 群)。
	\end{itemize}
\end{definition}

\begin{example}
	群中单位元和逆元唯一。
\end{example}

\begin{proof}
	设 $x$ 存在两个逆元 $y_1, y_2$,有
	\[ y_1 = y_1 \ast e = y_1 \ast x \ast y_2 = e \ast y_2 = y_2 \]
	类似的,设存在两个单位元 $e_1, e_2$,有
	\[ e_1 = e_1 \ast e_2 = e_2 \]
\end{proof}

\begin{definition}[环,域]
	给定集合 $R$ 和其上的二元运算 $(R, +, \ast)$,若:

	\begin{itemize}
		\item 满足 $(R, +)$ 是交换群,$(R, \ast)$ 是幺半群,且乘法关于加法满足分配率,称为环。

		\item 环 $R$ 中 $(R, \ast)$ 可交换,称为交换环。

		\item 除零元外皆可逆的环,称为除环。

		\item 交换除环称为域。
	\end{itemize}
\end{definition}

\section{拓扑初步}

可以想象开集的基为开区间,闭集的基为闭区间。

\begin{definition}[拓扑空间]
	设 $\mathcal{T}$ 是集合 $X$ 子集的集族:

	(O1) 若 $\varnothing \in \mathcal{T}$ 且 $X \in \mathcal{T}$。

	(O2) 若有限个 $U_i \in \mathcal{T}$,则 $\bigcap U_i \in \mathcal{T}$。

	(O3) 若任意个 $U_\alpha \in \mathcal{T}$,则 $\bigcup {U_\alpha} \in \mathcal{T}$。

	则称 $(X, \mathcal{T})$ 为拓扑空间,其中 $\mathcal{T}$ 是此空间的拓扑,$\mathcal{T}$ 的元素称为开集。在无歧义的情况下,也称 $X$ 是拓扑空间。
\end{definition}

\begin{definition}[邻域]
	对于空间 $X$ 和空间内一点 $x \in X$

	1. 若子集 $U$ 包含着某一开集且开集包含着 $x$,则称 $U$ 为 $x$ 的邻域。

	2. 若子集 $U$ 是开集且 $x \in U$,则称 $U$ 为 $x$ 的开邻域。
\end{definition}

\begin{definition}[闭集]
	对于空间 $X$,若其子集 $F$ 满足 $X-F$ 是开集,则称 $F$ 是闭集。
\end{definition}

\begin{example}
	换句话说:

	1. 任意多开集的交集还是开集,有限个开集的并集还是开集。

	2. 任意多闭集的交集还是闭集,有限个闭集的并集还是闭集。
\end{example}

\begin{definition}[闭包,内核]
	设空间 $X$ 和其子集 $A$,记

	1. 所有包含 $A$ 的闭集的交为 $A$ 的闭包 $\overline{A}$,即包含 $A$ 的最小闭集,其中的元素称为 $A$ 的接触点。

	2. 所有包含 $A$ 的开集的并为 $A$ 的内核 $A^{\circ}$,即包含 $A$ 的最大开集。
\end{definition}

\begin{definition}
	对于空间 $X$ 和其子集 $A$,若

	1. $\overline{A} = X$ 则称集合 $A$ 稠密于空间 $X$。

	2. $(\overline{A})^{\circ} = \varnothing$ 则称集合 $A$ 疏(无处稠密)于空间 $X$。
\end{definition}

显然 $\mathbb{N}$ 和 $\mathbb{Z}$ 疏于 $\mathbb{R}$,$\mathbb{Q}$ 和 $\mathbb{R} - \mathbb{Q}$ 稠密于 $\mathbb{R}$。

虽然我们在小学二年级学过无理数的存在性,但我们仍对 $\mathbb{Q}$ 和 $\mathbb{R}$ 性状上的具体区别知之甚少。

\begin{definition}[连通]
	若拓扑空间 $X$ 不能表示为两个不相交闭集的并,则称 $X$ 是连通空间。其子集 $X'$ 若是连通空间,则称 $X'$ 是连通的。
\end{definition}

可以得到 $\mathbb{R}$ 是连通的,$\mathbb{Q}$ 是不连通的。

依定义,闭区间是 $\mathbb{R}$ 内的开集,其任意数量的交都是闭集。假想一个不断变小的区间列,一层套一层。

\begin{definition}[闭区间套]
	给定有限的一列闭区间 $\{ I_i = [a_i, b_i] \}$,若

	(1) 其是下降的
	\[ I_1 \supseteq I_2  \supseteq I_3  \supset \cdots \];

	(2) 区间长度 $\lim\limits_{n \to \infty} (b_i - a_i) = 0$;

	那么称这一列区间是闭缩区间套,简称区间套。
\end{definition}

\begin{definition}[有序域]
	若域 $F$ 是满足如下条件的有序集

	(1) 当 $x,y,z\in F$ 且 $y<z$ 时,$x+y<x+z$。

	(2) 如果 $x,y\in F$,且 $x>0,y>0$,则 $xy>0$。

	那么称 $F$ 是一个有序域。
\end{definition}

例如 $\mathbb{Q}$ 是有序域。

\begin{theorem}[存在定理]
	具有最小上界性的有序域 $\mathbb{R}$ 存在,且包容着 $\mathbb{Q}$ 作为子域。
\end{theorem}

这个命题的证明较为复杂,是从 $\mathbb{Q}$ 出发构造 $\mathbb{R}$,而且其中有很多重要的信息,决定单独一章,这里略过。

\begin{theorem}[Achimedes 原理]
	对于 $x,y \in \mathbb{R}$ 且 $x>0$,那么必定存在正整数 $n$,使得 $nx>y$。
\end{theorem}
\begin{proof}
	设 $A = \{nx \mid n \in \mathbb{N}^+ \}$,若不存在 $n$ 则 $y$ 将是 $A$ 的一个上界,由最小上界性可知 $A$ 的上确界存在。

	又因为小于上确界的数 $\sup A-x$ 不是上确界,即存在 $m\in \mathbb{N}^+$ 使得 $\sup A -x <mx$,即 $\sup A < (m+1)x$,矛盾。

	故必定存在 $n$ 使得 $nx>y$。
\end{proof}

\begin{definition}[度量空间]
	称集合 $X$ 的元素为点,若存在 $X$ 上双变量的函数 $d:X \times X \to \mathbb{R}$,满足($x,y,z\in R$)

	\begin{itemize}
		\item 若 $x\ne y$,则 $d(x,y)$;仅 $d(x,x)=0$。

		\item 对于任意的 $x,y$ 都有 $d(x,y) = d(y,x)$。

		\item 对于任意的 $z$,都有 $d(x,y) \leqslant d(x,z) + d(z,y)$。
	\end{itemize}

	就称 $(X,d)$ 是一个度量空间(度量空间),函数 $d$ 称作其上的距离函数。
\end{definition}

这里的空间的含义是线性空间。

对于 $X$ 的子集 $Y$,定义其距离函数
\[ d_Y: Y \times Y \to \mathbb{R}, (y_1,y_2) \mapsto d_Y(y_1,y_2)=d(y_1,y_2)\]
则 $(Y,d_Y)$ 仍是度量空间,称 $d_Y$ 是 $d$ 在 $Y$ 上的诱导度量。$(Y,d_Y)$ 称作是 $(X,d)$ 的子(度量)空间。

\begin{definition}[稠密性]
	给定度量空间 $(X,d)$,$Y$ 是 $X$ 的子集。如果对任意的 $x\in X$ 和任意小的 $\eps>0$,都存在 $y\in Y$,使得 $d(y,x)<\eps$,我们就称 $Y$ 在 $X$ 中是稠密的。
\end{definition}

\begin{example}
	$\mathbb{Q}$ 在 $\mathbb{R}$ 中稠密:对于 $x,y \in \mathbb{R}$ 且 $x<y$,那么必定存在 $p\in\mathbb{Q}$,使得 $x<p<y$。
\end{example}
\begin{proof}
	由 Achimedes 原理,可设存在 $n\in \mathbb{N}^+$ 使得 $n(y-x)>1$。

	再设存在 $m_1,m_2\in \mathbb{N}^+$,使得 $m_1>nx,m_2>-nx$。于是
	\[ -m_2<nx<m_1 \]
	因此存在 $m\in \mathbb{N}^+$ 有 $-m_2\leqslant m \leqslant m_1$ 使得
	\[ m-1\leqslant nx < m \leqslant 1+nx < ny \]
	从而存在 $p=m/n$ 使得 $x<p<y$。
\end{proof}


\section{数系的构造}

直到我读了陶哲轩的《实分析》时,才感到接受了实数理论。实数的定义是公理化的,不是构造性的。

更具体的说,我们不需要知道实数是什么,只需知道这些对象有什么性质,我们就可以抽象的处理它们。从其他的数学对象出发来构造实数是可能的,有多种多样的模型,只要它们服从所有的公理并正确的运作,都是满足的。

实数究竟有多少性质?从自然数开始。

\begin{axiom}[Peano 公理]
	若集合 $N$ 和其上的映射 $v(n)$ 满足

	(1) $0\in N$。

	(2) 若 $n\in N$,则 $v(n) \in N$。

	(3) 对于任意的 $n\in N$,$v(n) \ne 0$。

	(4) 若 $v(m) \ne v(n)$,则 $m\ne n$。

	(5) 【归纳原理】设 $P(n)$ 是关于自然数的性质,假设只要 $P(n)$ 为真,则 $P(v(n))$ 也为真;且 $P(0)$ 为真。那么对 $N$ 中所有的元素 $P$ 都为真。

	那么称 $N$ 为自然数,记作 $\mathbb{N}$,$v(n)$ 称为后继函数。
\end{axiom}

\subsection{自然数}

设 $m,n\in \mathbb{N}$,定义 $\mathbb{N}$ 上的加法 $+$ 和乘法 $\cdot$ 为
\[
	\begin{aligned}
		0+m\coloneqq m      & ,\quad v(n)+m\coloneqq v(n+m)             \\
		0\cdot m\coloneqq m & ,\quad v(n)\cdot m\coloneqq n \cdot m + m
	\end{aligned}
\]

我们可以利用归纳原理推出我们熟悉的一些性质。

\begin{theorem}[$\mathbb{N}$ 的代数算律]
	对于 $a,b,c\in \mathbb{N}$ 有

	(1) 加法是结合的和交换的,且有单位元 $0$。

	(2) 乘法是结合的和交换的,且有单位元 $1$。

	(3) 分配律:$(a+b) \cdot  c = a \cdot c + b\cdot c$。
\end{theorem}

\begin{definition}[$\mathbb{N}$ 的序]
	设 $m,n\in \mathbb{N}$。

	(1) 若存在 $a\in \mathbb{N}$,使得 $n=m+a$,称 $m$ 小于等于 $n$,记作 $m \leqslant n$。

	(2) 若 $n\geqslant m$ 且 $n\ne m$,则称 $m$ 严格小于 $n$,记作 $m < n$。
\end{definition}

可以验证,$<$ 和 $\leqslant$ 是 $\mathbb{N}$ 上的序关系。

\begin{theorem}[加法保序]
	对于 $a,b\in \mathbb{N}$,若 $a>b$,则 $a+c>b+c$。
\end{theorem}

\subsection{整数}

接下来几节,都是记 $a,b,c$ 为当前集合的元素,$x,y,z$ 都是被构造的集合的元素。

为了表达整数,定义二元组 $(a,b)$,其中 $a,b \in \mathbb{N}$。记全体二元组的集合为 $Z$。我们约定
\[ (a,b) = (c,d) \Leftrightarrow a+d=b+c\]
因为自然数的序是已定义的,于是定义 $Z$ 上的序关系
\[ (a,b) \leqslant (c,d) \Leftrightarrow a+d \leqslant b+c\]
然后是定义 $N$ 上的加法和乘法
\[
	\begin{aligned}
		(a,b) + (c,d)     & \coloneqq  (a+c,b+d) \\
		(a,b) \cdot (c,d) & \coloneqq  (a c,b d)
	\end{aligned}
\]

可以验证,$(n,0)$ 与 $n$ 有相同的性状,我们可以令其相等,从而把自然数嵌入到整数内。至此,我们可以着手验证整数是否满足我们预想的性质。

\begin{theorem}[$\mathbb{Z}$ 的代数算律]
	对于 $x,y,z\in \mathbb{Z}$ 有

	(1) 加法是结合的和交换的,且有单位元 $0$,逆元存在。

	(2) 乘法是结合的和交换的,且有单位元 $1$。

	(3) 分配律:$(x+y) \cdot  z = x \cdot z + y\cdot z$。
\end{theorem}

即 $\mathbb{Z}$ 是一个交换环。于是

\begin{theorem}[$\mathbb{Z}$ 是有序域]
	(1) 加法保序:当 $x,y,z\in \mathbb{Z}$ 且 $y<z$ 时,$x+y<x+z$。

	(2) 乘法保序:如果 $x,y\in \mathbb{Z}$,且 $x>0,y>0$,则 $xy>0$。
\end{theorem}

我们有理由相信,$(a,b)$ 符合我们对整数的一切想象。因此 $Z = \mathbb{Z}$。

另外的,定义整数的负运算为 $-(a,b) = (b,a)$,以此定义减法
\[ x - y \coloneqq  x + (-y)\]
可以验证
\[ (a,0) - (b,0) = (a,b) = a - b\]

\subsection{有理数}

类似的,记整数的二元组 $(a,b)$,其中 $a,b\in \mathbb{Z},b\ne 0$,记全体二元组的集合为 $Q$。我们约定
\[ (a,b) = (c,d) \Leftrightarrow ad = bc\]
因为整数的序是已定义的,于是定义 $Q$ 上的序关系
\[ (a,b) \leqslant (c,d) \Leftrightarrow ad \leqslant bc\]
于是定义 $Q$ 上的加法和乘法
\[
	\begin{aligned}
		(a,b) + (c,d)     & \coloneqq  (ad+bc,b+d)           \\
		(a,b) \cdot (c,d) & \coloneqq  (a \cdot c,b \cdot d) \\
	\end{aligned}
\]
定义加法逆元为 $-(a,b) \coloneqq  (-a,b)$。可以验证,$(a,1)$ 与 $a$ 有相同的性状,我们可以令其相等,从而把整数嵌入到有理数内。

至此,我们可以着手验证有理数是否满足我们预想的性质。

\begin{theorem}[$\mathbb{Q}$ 的代数算律]
	对于 $x,y,z\in \mathbb{Q}$ 有

	(1) 加法是结合的和交换的,且有单位元 $0$,逆元存在。

	(2) 乘法是结合的和交换的,且有单位元 $1$,非零元逆元存在。

	(3) 分配律:$(x+y) \cdot  z = x \cdot z + y\cdot z$。
\end{theorem}

即 $\mathbb{Q}$ 是一个域。

\begin{theorem}[$\mathbb{Q}$ 是有序域]
	(1) 加法保序:当 $x,y,z\in \mathbb{Q}$ 且 $y<z$ 时,$x+y<x+z$。

	(2) 乘法保序:如果 $x,y\in \mathbb{Q}$,且 $x>0,y>0$,则 $xy>0$。
\end{theorem}

我们有理由相信,$(a,b)$ 符合我们对有理数的一切想象。因此 $Q = \mathbb{Q}$。

另外,定义倒数 $(a,b)^{-1} = (b,a)$,显然 $a,b\ne 0$。从而定义除法
\[ x/y \coloneqq  x \cdot y^{-1}\]
可以验证,
\[ (a,1)/(b,1) = (a,b) = a/b\]

\subsection{实数\ ·\ Dedekind 分割}



\begin{definition}[Dedekind 分割]
	对于给定的空间 $S$, $A \subset S, A' = \complement_S A$,若满足以下三个条件

	(D1) $A \ne \varnothing, A \neq S(A' \ne \varnothing)$;

	(D2) 当 $p\in A,q \in A'$ 时,$p<q$;

	(D3) 不存在最大数:任给 $p \in A$,存在 $q \in A$,使得 $p<q$;

	则称 $A$ 为 $S$ 的一个分割。
\end{definition}

直观的来说,我们把整个 $S$ 划分成了下组 $A$ 和上组 $A'$。

记 $\mathbb{Q}$ 上 Dedekind 分割的全体为 $R$,集合的相等即是 $R$ 上的等价关系,$R$ 上的序关系定义是
\[ A \subseteq B \Leftrightarrow A\leqslant B \]
定义加法
\[ A+B \coloneqq  \{ a+b \mid a\in A,b\in B\} \]
于是可以定义负运算
\[ \begin{aligned}
		-A & \coloneqq  \{ s \in \mathbb{Q} \mid \exists r>0,-s-r\in \complement_\mathbb{Q} A\} \\
		-A & \coloneqq  \{ s \in \mathbb{Q} \mid \exists r\in \complement_\mathbb{Q} A,s < -r\}
	\end{aligned} \]


然而乘法因为负数的问题,我们需要分类讨论。$R$ 中存在加法单位元 $0^* = \{x \in \mathbb{Q} \mid x \geqslant 0\}$,对于正实数 $A,B\geqslant 0^*$,定义乘法
\[ A \cdot B \coloneqq  \{ p \in \mathbb{Q} \mid \text{存在}\ 0<a\in A,\ \text{存在}\ 0<b\in B, p<ab\}\]
同时
\[
	A \cdot B \coloneqq \begin{cases}
		-((-A) \cdot B),    & A<0^*, B\geqslant 0^* \\
		-(A \cdot (-B)),    & A\geqslant 0^*, B<0^* \\
		-((-A) \cdot (-B)), & A<0^*, B<0^*          \\
	\end{cases}
\]

当 $A > 0^*$ 时,定义乘法逆元
\[ A^{-1} \coloneqq  \{s \in \mathbb{Q} \mid \exists r \in \complement_\mathbb{Q} A, s < r^{-1}\}\]
当 $A < 0^*$ 时,定义乘法逆元为 $A^{-1} \coloneqq  -(-A^{-1})$。

至此,我们可以着手验证实数是否满足我们预想的性质。

\begin{theorem}[$\mathbb{R}$ 的代数算律]
	对于 $x,y,z\in \mathbb{R}$ 有

	(1) 加法是结合的和交换的,且有单位元 $0$,逆元存在。

	(2) 乘法是结合的和交换的,且有单位元 $1$,非零元逆元存在。

	(3) 分配律:$(x+y) \cdot  z = x \cdot z + y\cdot z$。
\end{theorem}

即 $\mathbb{R}$ 是一个域。

\begin{theorem}[$\mathbb{R}$ 是有序域]
	(1) 加法保序:当 $x,y,z\in \mathbb{R}$ 且 $y<z$ 时,$x+y<x+z$。

	(2) 乘法保序:如果 $x,y\in \mathbb{R}$,且 $x>0,y>0$,则 $xy>0$。
\end{theorem}

我们有理由相信,$R$ 符合我们对实数性质的一切想象,从而 $R = \mathbb{R}$。

\begin{theorem}[Dedekind 原理]
	设 $A$ 为 $\mathbb{R}$ 上的 Dedekind 分割,$A' = \complement_\mathbb{R} A$,对于任给的 $a\in A,a' \in A'$,存在 $r \in \mathbb{R}$ 使得 $a < r \leqslant a'$。
\end{theorem}

\begin{proof}
	由于 $a, a'$ 也是 $\mathbb{Q}$ 上的分割,下面使用集合的语言。显然 $a' \in A'$ 是 $A$ 的一个上界。构造
	\[ b = \bigcup_{a \in A} a \]
	下证 $a \subset b \subseteq a'$。

	首先证明 $b$ 是 $\mathbb{Q}$ 上分割。

	\begin{itemize}
		\item D1:显然 $b$ 非空,又因为对于任意的 $a \in A, a' \in A'$ 都有 $a \subset a'$,故 $b \subseteq a'$(注意可取等),即 $b \neq \mathbb{Q}$。
		\item D2:接下来取 $\beta \in b, \beta' \in \complement_\mathbb{Q} b$,于是存在 $a_0 \in A$ 使得 $\beta \in a_0$,此时 $\beta' \notin a_0$ 即 $\beta' \in a_0'$,故 $\beta < \beta'$。
		\item D3:对于任意的 $\beta_1 \in b$,存在 $a_0 \in b$ 使得 $\beta_1 \in a_0$,此时存在 $\alpha_1 \in a_0$ 使得 $\beta_1 < \alpha_1$ 且 $\alpha_1 \in b$。
	\end{itemize}


	接下来证明 $b \notin A$。假设 $b \in A$ 即存在 $a_0$ 使得 $b = a_0$,但由 D3 总是存在 $b = a_0 \subset a_1$,与 $b$ 是全体并集矛盾。

	因此 $b \in A'$,故 $a \subset b \subseteq a'$。
\end{proof}

实数和有理数的最基本的一个区别就是有最小上界性。

\begin{theorem}[确界原理]
	$\mathbb{R}$ 具有最小上界性。即对于 $\mathbb{R}$ 的任何子集 $S$,若 $S$ 在 $\mathbb{R}$ 内存在上界,那么 $S$ 在 $\mathbb{R}$ 内存在上确界。
\end{theorem}
\begin{proof}
	设 $B'$ 是 $S$ 全体上界组成的集合,即 $B' = \{x \mid\forall s \in S, x \geqslant s \}$,令
	\[ B = \complement_\mathbb{R} B' = \{ x \mid \exists s \in S, x < s\} \]
	试证 $B$ 是 $\mathbb{R}$ 的一个分割。D1 和 D2 比较显然。显然对于任意的 $b \in B$ 存在 $s \in S$ 使得 $x < s$,那么总是可以取 $b_2 = \frac{b+s}{2} \in B$ 使 $x < b_2$。

	由上文所证的 Dedekind 原理知,总存在 $u \in \mathbb{R}(B')$ 使得任取上确界 $b' \in B'$ 使得 $u \leqslant b'$,故 $u$ 是上确界。
\end{proof}

\subsection{实数\ ·\ Cauchy 序列}

\newcommand{\LIM}{\operatorname{{LIM}}}

我们试图得到实数,是因为有理数还不足以表示所有的数,比如 $x^2=2$ 的解。得到实数和前面的方法有所不同,要复杂的多。

一个有理数上的序列 $\{a_n\}$,是一个从集合 $\mathbb{N}$ 到 $\mathbb{Q}$ 的一个映射,即我们以前说的数列。

对于 $\mathbb{Q}$ 上的无限序列 $\{a_n\}$,若对于任意的 $\eps > 0$ 存在 $N \geqslant 0$ 使得当 $j,k \geqslant N$ 时有
\[ d(a_j,a_k) < \eps\]
则称序列 $\{a_n\}$ 为 Cauchy 序列,记作 $\LIM(a_n)$。记 Cauchy 序列的全体为集合 $R$。

对于 Cauchy 序列 $\LIM(a_n),\LIM(b_n)$,若对于任意的 $\eps > 0$ 存在 $N \geqslant 0$ 使得当 $n \geqslant N$ 时有
\[ d(a_n,b_n) < \eps\]
则记作 $\LIM(a_n) = \LIM(b_n)$。

定义 $R$ 的序关系,对于实数 $x,y$,若存在 Cauchy 序列满足 $x=\LIM(a_n),y=\LIM(b_n)$,对于 $n\geqslant 1$ 有 $a_n \leqslant b_n$,则 $\LIM(a_n) \leqslant \LIM(b_n)$。

于是定义 $R$ 上的加法和乘法
\[ \begin{aligned}
		\LIM(a_n) + \LIM(b_n)     & \coloneqq  \LIM(a_n+b_n) & \\
		\LIM(a_n) \cdot \LIM(b_n) & \coloneqq  \LIM(a_nb_n)    \\
	\end{aligned} \]
定义负运算 $-\LIM(a_n) \coloneqq  \LIM(-a_n)$。

定义倒数时会因为恼人的 $0$ 出现了一些困难,解决的方法即是把 $0$ 排出。若存在 $c\in \mathbb{Q}$ 满足 $c > 0$ 使得 $d(a_n,0) \geqslant c$,则称 $\{a_n\}$ 为限制离开零的序列。若 $x$ 为不为零的实数,则必存在一个限制离开零的 Cauchy 序列 $\LIM(a_n) = x$。

于是我们可以定义,设 $x$ 为一个不为零的实数,则存在限制离开零的 Cauchy 序列 $x=\LIM(a_n)$,定义倒数为
\[ x^{-1} \coloneqq  \LIM(a_n^{-1})\]

可以验证,常数 Cauchy 序列 $\{a_n\}$ 与 $a$ 具有相同的性状,因此可以令它们相等,从而使有理数嵌入到实数中。

至此,我们可以着手验证实数是否满足我们预想的性质,在 Dedekind 分割中提过了,这里不再重复。

另外,定义 $R$ 上的 Cauchy 序列,若对于任意的实数 $\eps > 0$ 存在 $N \geqslant 0$ 使得当 $j,k \geqslant N$ 时有
\[ d(a_j,a_k) \leqslant \eps\]

可以证明,$R$ 上的 Cauchy 序列与 $\mathbb{Q}$ 上的 Cauchy 序列等价。

若存在实数 $L$ 满足,存在 $N>0$ 使得当 $n \geqslant N$ 时,都有 $d(a_n,L) \leqslant \eps$,则 $a_n$ 收敛于 $L$,记作
\[ \lim_{n\to \infty} a_n = L\]

可以验证
\[ \LIM(a_n) = \lim_{n\to \infty} a_n\]

\let\LIM\relax

\subsection{复数}

记实数的二元组 $(a,b)$,其中 $a,b\in R$,记全体二元组的集合为 $C$。我们约定
\[(a,b) = (c,d) \Leftrightarrow a=b \land  c=d\]
复数没有序关系。定义 $C$ 上的加法和乘法
\[\begin{aligned}
		(a,b) + (c,d)     & \coloneqq  (a+c,b+d)     \\
		(a,b) \cdot (c,d) & \coloneqq  (ac-bd,ad+bc)
	\end{aligned}\]

定义加法逆元为 $-(a,b) \coloneqq  (-a,-b)$。可以验证,$(a,0)$ 与 $a$ 具相同的性状,我们可以令其相等,从而把实数嵌入到复数域内。

定义非零数的乘法逆元 $(a,b)^{-1} \coloneqq  \left(\frac{a}{a^2+b^2},-\frac{b}{a^2+b^2}\right)$。

\section{实数的完备性}

如上所见,实数可以有多种完全不同的定义。换句话说我们可以承认一些公理,使得些都是等价的公理,以下举出几个例子。

\begin{itemize}
	\item R0 Dedekind 原理;
	\item R1 确界原理;
	\item R2 单调有界原理;
	\item R3 区间套原理;
	\item R4 有限覆盖原理;
	\item R5 聚点原理;
	\item R6 致密性原理;
	\item R7 柯西收敛原理;
	\item R8 介值定理;
	\item R9 连通性原理;
	\item R10 Achimedes 原理;
\end{itemize}

这些定理是彼此等价的,其逻辑关系是
\[ \mathrm{R0} \Leftrightarrow \mathrm{R1} \Leftrightarrow\mathrm{R2} \Leftrightarrow\mathrm{R3} + \mathrm{R10} \Leftrightarrow\mathrm{R4} \Leftrightarrow\mathrm{R5} \Leftrightarrow\mathrm{R6} \Leftrightarrow\mathrm{R7}+\mathrm{R10} \Leftrightarrow\mathrm{R8} \Leftrightarrow\text{R9}  \]
因此都可以选作实数的完备性(连续性)公理。互相推导很适合作为课后练习,可以查看 \href{https://zhuanlan.zhihu.com/p/48859870}{知乎:实数的完备性定理}。

确界原理由于其简明的性质,我们将在其上构建极限与收敛的体系。







%\subsection{数集·确界原理}
%
%区间分为无限区间和有限区间。
%
%设实数 $a<b$,则称数集 $\{x \mid a<x<b\}$ 为开区间,记作 $(a,b)$;数集 $\{x \mid a\leqslant x \leqslant b\}$ 称为闭区间,记作 $[a,b]$;数集 $\{x \mid a\leqslant x<b\}$ 和 $\{x \mid a < x \leqslant b\}$ 都称为半开半闭区间,分别记作 $[a,b)$ 和 $(a,b]$。以上几类区间统称为有限区间。
%
%满足关系式 $x\geqslant a$ 的全体实数 $x$ 的集合记作 $[a,+\infty)$,类似地,有 $(-\infty,a],(a,\infty),(-\infty,a)$。特殊地 $\mathbb{R} = (-\infty,+\infty)$。这几类区间统称为无限区间。
%
%设 $\delta > 0$,满足 $|x-a|<\delta$ 的 $x$ 的集合称为点 $a$ 的 $\delta$ 邻域,记作 $U(a;\delta)$,或简单的记作 $U(a)$,即有
%\[U(a;\delta) = (a-\delta,a+\delta)\]
%点 $a$ 的空心 $\delta$ 邻域定义为
%\[U^\circ (a;\delta) = \{x \mid 0<|x-a|<\delta\}\]
%也可以简单的记作 $U^\circ(a)$。
%
%此外,常用的邻域还有:
%
%点 $a$ 的 $\delta$ 右邻域 $U_+(a;\delta) = [a,a+\delta)$,左邻域 $U_-(a;\delta)$。以及点 $a$ 的空心 $\delta$ 左、右邻域 $U_{-}^\circ(a)$ 与 $U_{+}^{\circ}(a)$。
%
%以及 $\infty$ 邻域 $U(\infty) = \{x \mid |x|>M\}$,其中 $M$ 为充分大的正数。类似的还有 $U(+\infty) = \{x \mid x>M\}$ 和 $U(-\infty) = \{x \mid x<-M\}$。
%
%\begin{definition}[有界集]
%	设 $S$ 为一个非空数集,若存在数 $M\in$ 使得 $\forall x\in S$
%	
%	(1) 都有 $x\leqslant M$,则称 $M$ 是 $S$ 的一个上界。
%	
%	(2) 都有 $x\geqslant M$,则称 $M$ 是 $S$ 的一个下界。
%\end{definition}
%
%若数集 $S$ 既有上界又有下界,则称 $S$ 为有界集,反之称为无界集。
%
%\begin{theorem}[确界原理]
%	设 $S$ 为非空数集,若 $S$ 有上界,则 $S$ 必有上确界;若 $S$ 有下界,则 $S$ 必有下确界。
%\end{theorem}
%
%若把 $\pm \infty$ 看作非正常上下确界,前文定义视为正常上(下)确界,那么任一非空数集必有上下确界。
%
%\section{函数的上下界}
%
%\begin{definition}
%	设 $f$ 为定义在 $D$ 上的函数。若存在数 $M(L)$,使得对每一个 $x\in D$,有 $f(x)\leqslant M(f(x) \ge L)$,则称 $f$ 为 $D$ 上的有上(下)界函数,$M(L)$ 称为 $f$ 在 $D$ 上的一个上(下)界。
%
%	反之,若存在数 $M(L)$,使得对每一个 $x\in D$,有 $f(x) \geqslant M(f(x) \leqslant L)$,则称 $f$ 为 $D$ 上的有无上(下)界函数。
%\end{definition}
%
%\begin{definition}
%	设 $f$ 为定义在 $D$ 上的函数。若存在正数 $M$,使得对每一个 $x\in D$,有 $|f(x)|\leqslant M$,则称 $f$ 为 $D$ 上的有界函数。
%
%	反之,若存在正数 $M$,使得对每一个 $x\in D$,有 $|f(x)|\ge M$,则称 $f$ 为 $D$ 上的无界函数。
%\end{definition}
%
%记函数 $f$ 在 $D$ 上的上确界为 $\sup_{x\in D} f(x)$,类似的有 $\inf_{x\in D} f(x)$。
%
%\begin{definition}
%	设 $f$ 为定义在 $D$ 上的函数,若对任何 $x_1,x_2\in D$,当 $x_1<x_2$ 时:
%
%	(1) 总有 $f(x_1) \leqslant f(x_2)$,则称 $f$ 为 $D$ 上的增函数,若成立严格不等式 $f(x_1) < f(x_2)$ 时,称 $f$ 为 $D$ 上的严格增函数。
%
%	(2) 总有 $f(x_1) \geqslant f(x_2)$,则称 $f$ 为 $D$ 上的减函数,若成立严格不等式 $f(x_1) > f(x_2)$ 时,称 $f$ 为 $D$ 上的严格减函数。
%\end{definition}
%
%增函数和减函数统称为单调函数,严格增函数和严格减函数统称为严格单调函数。
%
%严格单调函数必有反函数,其也为严格单调函数。
%
%\begin{definition}
%	设 $D$ 为对称于原点的数集, 函数 $f$ 为定义在 $D$ 上的函数。若对每一个 $x\in D$:
%
%	(1) 有 $f(-x) = -f(x)$,则称 $f$ 为 $D$ 上的奇函数。
%
%	(2) 有 $f(-x) = f(x)$,则称 $f$ 为 $D$ 上的偶函数。
%\end{definition}