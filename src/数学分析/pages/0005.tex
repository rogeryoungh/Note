\chapter{导数和微分}

\section{导数的定义}

\begin{definition}
	设函数 $y=f(x)$ 在点 $x_0$ 的某邻域有定义,若极限
	$$\lim_{x\to x_0}\frac{f(x)-f(x_0)}{x-x_0}$$
	存在,则称函数 $f$ 在点 $x_0$ 可导,并称该极限为函数 $f$ 在点 $x_0$ 的导数,记作 $f'(x_0)$。
\end{definition}

\begin{theorem}
	若函数 $f$ 在点 $x_0$ 可导,则 $f$ 在点 $x_0$ 连续。
\end{theorem}

\begin{definition}
	设函数 $y=f(x)$ 在点 $x_0$ 的某右邻域 $[x_0,x_0+\delta)$ 上有定义,若右极限
	$$\lim_{\Delta x\to 0^+}\frac{\Delta y}{\Delta x} = \lim_{\Delta x\to 0^+}\frac{f(x_0+\Delta x)-f(x_0)}{\Delta x},(0<\Delta x<\delta)$$
	存在,则称该极限值为 $f$ 在点 $x_0$ 的右导数,记作 $f_+'(x)$。同理有左导数。
\end{definition}

左导数和右导数统称为单侧导数。

\begin{theorem}
	若函数 $y=f(x)$ 在点 $x_0$ 的某邻域上有定义,则 $f'(x_0)$ 存在的充要条件是 $f_-'(x)$ 与 $f_+'(x)$ 都存在且相等。
\end{theorem}

若函数 $f$ 在区间 $I$ 上每一点都可导(对区间端点,仅考虑相应的单侧导数),则称 $f$ 为 $I$ 上的可导函数。此时对每一个 $x\in I$,都有 $f$ 的一个导数 $f'(x)$ (或单侧导数)与之对应。

这样就定义了一个在 $I$ 上的函数,称为导函数,简称为导数。记作 $f',y',\dfrac{\dd y}{\dd x}$,即

$$f'(x) = \lim_{\Delta x \to 0}\frac{f(x+\Delta)-f(x)}{\Delta},x\in I$$

有时 $f'(x_0)$ 也可写作 $y'\mid_{x=x_0}$ 或 $\mfrac{\dd y}{\dd x}\mid_{x=x_0}$。

曲线 $y = f(x)$ 在点 $(x_0,y_0)$ 的切线方程是
$$y-y_0 = f'(x_0)(x-x_0)$$

\begin{definition}
	若函数 $f$ 在点 $x_0$ 的某邻域 $U(x_0)$ 上对一切 $x\in U(x_0)$ 有
	$$f(x_0) \geqslant f(x)$$
	则称 $f$ 在点 $x_0$ 取得极大值,称点 $x_0$ 为极大值点。同理有极小值点。
\end{definition}

极大值、极小值统称为极值,极大值点、极小值点统称为极值点。

\begin{theorem}[费马定理]
	设函数 $f$ 在点 $x_0$ 的某邻域上有定义,且在点 $x_0$ 可导。若点 $x_0$ 为极值点,则必有 $f'(x_0)=0$。
\end{theorem}

\section{求导法则}

\begin{theorem}
	若函数 $u(x)$ 和 $v(x)$ 在点 $x_0$ 可导,则函数 $f(x)=u(x)\pm v(x)$ 在点 $x_0$ 也可导,且
	$$f'(x_0) = u'(x_0)\pm v'(x_0)$$
	函数 $f(x)=u(x)v(x)$ 在点 $x_0$ 也可导,且
	$$f'(x_0) = u'(x_0)v'(x_0)$$
	若 $v(x)\ne 0$,则函数 $f(x)=\dfrac{u(x)}{v(x)}$ 在点 $x_0$ 也可导,且
	$$f'(x_0) = \frac{u'(x_0)v(x_0)-u(x_0)v'(x_0)}{v(x_0)^2}$$
\end{theorem}

\begin{theorem}
	设 $y=f(x)$ 为 $x=\phi(x)$ 的反函数,若 $\phi(y)$ 在点 $y_0$ 的某邻域上连续、严格单调且 $\phi'(y_0)\ne0$,则 $f(x)$ 在点 $x_0=\phi(y_0)$ 可导,且
	$$f'(x_0)=\frac{1}{\phi'(y_0)}$$
\end{theorem}

\begin{theorem}
	设 $u=\phi(x)$ 在点 $x_0$ 可导,$y=f(u)$ 在点 $u_0=\phi(x_0)$ 可导,则复合函数 $f\circ \phi$ 在点 $x_0$ 可导,且
	$$(f\circ \phi)'(x_0) = f'(u_0)\phi'(x_0) = f'(\phi(x_0))\phi'(x_0)$$
\end{theorem}

\subsection{基本求导法则}

1. $(u\pm v)' = u'\pm v'$

2. $(uv)' = u'v+uv'$

3. $\left(\dfrac{u}{v}\right) = \dfrac{u'v-uv'}{v^2}$

4. $(u\pm v)' = u'\pm v'$

\subsection{基本初等函数导数公式}

1. $(c)' = 0$($c$ 为常数)

2. $(x^a)' = ax^{a-1}$($a$ 为任意实数)

3. $(\sin x)' = \cos x,(\cos x)'=-\sin x,(\tan x)'=\sec^2x$

4. $(\cot x)' = -\csc^2 x,(\sec x)'=\sec x \tan x,(\csc x)'=-\csc x \cot x$

5. $(\arcsin x)'=\dfrac{1}{\sqrt{1-x^2}},(\arccos x)' = -\dfrac{1}{\sqrt{1-x^2}},(\arctan x)'=\dfrac{1}{1+x^2}$

6. $(a^x)'=a^x \ln a(a>0\text{且}a\ne 1)$

7. $(\log_a|x|)'=\dfrac{1}{x\ln a}(a>0\text{且}a\ne 1)$

\section{单调性与导数}

\begin{theorem}
	设 $f$ 在区间 $I$ 上可导,则 $f(x)$ 在 $I$ 上递增(减) 的充要条件时
	$$f'(x) \geqslant 0(\leqslant 0)$$
\end{theorem}

\begin{theorem}[介值定理]
	设 $f$ 为 $[a,b]$ 上的连续函数,$\mu$ 时严格介于 $f(a)$ 和 $f(b)$ 之间的数,则存在 $\xi\in (a,b)$,使得 $f(\xi)=\mu$。
\end{theorem}