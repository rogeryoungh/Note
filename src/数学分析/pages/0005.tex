\chapter{积分理论}

\section{不定积分}

先放这,稍后做整理。

\begin{definition}
	设函数 $f$ 与 $F$ 在区间 $I$ 上都有定义。若
	\[ F'(x) = f(x), x\in I \]
	则称 $F$ 为 $f$ 在区间 $I$ 上的一个原函数。
\end{definition}

\begin{theorem}
	设 $F_1$ 和 $F_2$ 是 $f$ 在区间 $I$ 上的两个个原函数,则 $F_1(x) - F_2(x)$ 是与 $x$ 无关的常数。
\end{theorem}
\begin{proof}
	显然
	\[ G'(x) = \left[ F_1(x)-F_2(x) \right]' = f(x) - f(x) = 0, x \in I \]
	根据 Lagrange 中值定理,有
	\[ F(x) - G(x) \equiv C, x\in I \]
\end{proof}

\begin{definition}
	函数 $f$ 在区间 $I$ 上的全体原函数称为 $f$ 在 $I$ 上的不定积分,记作
	\[ \int f(x) \d x = F(x) + C \]
	其中 $\displaystyle\int$ 称为积分号,$f(x)$ 为被积函数,$f(x)\d x$ 为被积表达式,$x$ 称为积分变量。
\end{definition}

\subsection{基本不定积分表}

以下是一些常见函数的原函数

\[ \begin{aligned}
		\int {x}^{m} \d x                  & = \frac{1}{m + 1} x^{m+1} + C, \quad m \neq 1        \\
		\int x^{-1} \d x                   & = \ln |x| + C                                        \\
		\int \frac{1}{1 + x^2} \d x        & = \arctan x + C                                      \\
		\int \frac{1}{1 - x^2} \d x        & = \frac{1}{2} \ln \left| \frac{1+x}{1-x} \right| + C \\
		\int \frac{1}{\sqrt{1 + x^2}} \d x & = \ln(x+\sqrt{1 + x^2}) + C                          \\
		\int \frac{1}{\sqrt{1 - x^2}} \d x & = \arcsin x + C                                      \\
	\end{aligned} \]

\section{定积分}

设闭区间 $[a,b]$ 上有 $n-1$ 个点,依次为
\[ a = x_0 < x_1 < \cdots < x_{n-1} < x_n = b \]
它们把 $[a,b]$ 分为 $n$ 个小区间 $\Delta_i = [x_{i-1},x_i], i=1,\cdots,n$。这些分点或这些闭子区间构成对 $[a,b]$ 的一个分割,记为
\[ T=\{x_0,\cdots,x_n\}\ \text{或}\ \{\seq{\Delta}{n}\} \]
小区间 $\Delta_i$ 的长度为 $\Delta x_i = x_i-x_{i-1}$,并记
\[ \| T \| = \max_{1 \leqslant i \leqslant n}\{\Delta x_i\} \]
称为分割的模。

\begin{definition}
	设 $f$ 是定义在 $[a,b]$ 上的一个函数。对于 $[a,b]$ 的一个分割 $T=\{\seq{\Delta}{n}\}$,任取点 $\xi\in\Delta_i,i=1,\cdots,n$,并作和式
	\[ \sum_{i=1}^nf(\xi_i)\Delta x_i \]
	称此和式为函数 $f$ 在 $[a,b]$ 上的一个积分和,也称黎曼和。
\end{definition}

显然,积分既与分割 $T$ 与有关,又与所选取的点集 $\{\xi_i\}$ 有关。

\begin{definition}
	设 $f$ 是定义在 $[a,b]$ 上的一个函数,$J$ 是一个确定的实数。若对任给的正数 $\epsilon$,总存在某一正数 $\delta$,使得对 $[a,b]$ 的任何分割 $T$,以及在其上任意选取的点集 $\{\xi_i\}$,只要 $\| T \| < \delta$,就有
	\[ \left| \sum_{i=1}^nf(\xi_i)\Delta x_i - J \right| < \eps \]
	则称函数 $f$ 在区间 $[a,b]$ 上可积或黎曼可积;数 $J$ 称为 $f$ 在 $[a,b]$ 上的定积分或黎曼积分,记作
	\[ J = \int_a^b f(x) \d x = \lim_{\| T \| \to 0} \sum_{i=1}^nf(\xi_i)\Delta x_i \]
\end{definition}

其中 $f$ 称为被积函数,$x$ 称为积分变量,$[a,b]$ 称为积分区间,$a,b$ 分别称为这个定积分的下限和上限。

\begin{theorem}[Newton - Leibniz 公式]
	若函数 $f$ 在 $[a,b]$ 上连续,且存在原函数 $F$,即 $F'(x) = f(x), x\in[a,b]$,则 $f$ 在 $[a,b]$ 上可积,且
	\[ \int_a^bf(x)\d x = F(b) - F(a) \]
\end{theorem}

\begin{proof}
	即证对于任给的 $\eps > 0$,存在 $\delta>0$ 使得当 $\| T \| < \delta$ 时有
	\[ \left| \sum_{i=1}^nf(\xi_i)\Delta x_i - [F(b)-F(a)] \right| < \eps \]
	对于任意分割 $T$,在每个小区间 $[x_{i-1},x_i]$ 上对 $F(x)$ 使用 Lagrange 中值定理,则分别存在 $\eta_i \in (x_{i-1},x_i),i=1,\cdots,n$,使得
	\[ F(b)-F(a) = \sum_{i=1}^n F'(\eta_i)\Delta x_i = \sum_{i=1}^nf(\eta_i)\Delta x_i \]
	又因为 $f$ 在 $[a,b]$ 上一致连续,因此存在 $\delta > 0$ 当 $x_1,x_2\in[a,b]$ 且 $|x_1-x_2| < \delta$ 时,有
	\[ f(x_1) - f(x_2)| < \frac{\eps}{b-a} \]
	由 $\Delta x_i \leqslant \| T \| < \delta$ 时,任取 $\xi_i \in [x_{i-1},x_i]$,便有 $|\xi_i-\eta_i|<\delta$,于是
	\[ LHS = \left| \sum_{i=1}^n [f(\xi_i)-f(\eta_i)]\Delta x_i \right| \leqslant \frac{\eps}{b-a}\sum_{i=1}^n\Delta x_i = \eps \]
	于是 $f$ 在 $[a,b]$ 上可积。
\end{proof}

\subsection{可积条件}

\begin{theorem}
	若函数 $f$ 在 $[a,b]$ 上可积,则 $f$ 在 $[a,b]$ 上必定有界。
\end{theorem}
\begin{proof}
	反证,若 $f$ 在 $[a,b]$ 上无界,则对于即对于 $[a,b]$ 的任意分割 $T$,必存在属于 $T$ 的某区间 $\Delta_k$,使 $f$ 在其上无界。在 $i\ne k$ 的各个区间 $\Delta_i$ 上取定 $\xi_i$,记
	\[ G = \left| \sum_{i\ne k}f(\xi)\Delta x_i \right| \]
	任意大的正数 $M$,存在 $\xi_k\in \Delta_k$,使得
	\[ |f(\xi_k)| > \frac{M+G}{\Delta x_k} \]
	于是有
	\[ \left| \sum_{i=1}^nf(\xi)\Delta x_i \right| \geqslant |f(\xi_k)\Delta x_k| - \left| \sum_{i\ne k}f(\xi)\Delta x_i \right| > \frac{M+G}{\Delta x_k}\cdot \Delta x_k - G = M \]
\end{proof}

有界函数不一定黎曼可积,比如 Dirichlet 函数。

设 $T$ 为对 $[a,b]$ 的任意分割。由 $f$ 在 $[a,b]$ 上有界,它在每个 $\Delta_i$ 上存在上、下确界:
\[ M_i=\sup_{x\in\Delta_i}f(x),m_i = \inf_{x=\Delta_i}f(x),i=1,\cdots,n \]
作和
\[ S(T) = \sum_{i=1}^nM_i\Delta x_i, s(T) = \sum_{i=1}^n m_i \Delta x_i \]
分别称为 $f$ 关于分割 $T$ 的上和与下和(或称达布上和与达布下和,统称达布和)。任给 $\xi_i = \Delta_i,i=1,\cdots,n$,显然有
\[ m(b-a) \leqslant s(T) \leqslant \sum_{i=1}^n f(\xi_i)\Delta x_i \leqslant S(T) \leqslant M(b-a) \]
与积分和相比较,达布和只与分割 $T$ 有关,而与点集 $\{\xi_i\}$ 无关。

\begin{proposition}
	给定分割 $T$,对于任何点集 $\{\xi_i\}$ 而言,上和时所有积分和的上确界,下和是所有积分和的下确界。
\end{proposition}
\begin{proof}
	设 $\Delta_i$ 中 $M_i$ 是 $f(x)$ 的上确界,故可选取点 $\xi=\Delta_i$,使 $f(\xi_i)>M_i-\frac{\eps}{b-a}$,于是有
	\[ \sum_{i=1}^nf(\xi_i)\Delta x_i > \sum_{i=1}^nM_i\Delta x_i-\frac{\eps}{b-a}\sum_{i=1}^n\Delta x_i = S(T)-\eps \]
	即 $S(T)$ 是全体积分和的上确界。类似可证 $s(T)$ 是全体积分和的下确界。
\end{proof}


\begin{proposition}
	设 $T'$ 为分割 $T$ 添加 $p$ 个新分点后所得到的分割,则有

\end{proposition}

\begin{theorem}
	函数 $f$ 在 $[a,b]$ 上可积的充要条件是:人格 $\eps >0$,总存在相应的一个分割 $T$ 使得
	\[ S(T) - s(T) < \sum_{i=1}^n\omega_i\Delta x_i = \eps \]
\end{theorem}

其中 $\omega$ 称为 $f$ 在 $\Delta_i$ 上的振幅。

由充要条件,我们可以得到一系列的可积函数类。

\begin{theorem}
	若 $f$ 为 $[a,b]$ 上的连续函数,则 $f$ 在 $[a,b]$ 上可积。
\end{theorem}
\begin{proof}
	由于 $f$ 在闭区间 $[a,b]$ 上一致连续,即任给 $\eps > 0$,存在 $\delta>0$,对 $[a,b]$ 中任意两点 $x_1,x_2$,只要 $|x_1-x_2|<\delta$,便有
	\[ |f(x_1)-f(x_2)| < \frac{\eps}{b-a} \]
	所以对于在 $[a,b]$ 的分割 $T$ 满足 $\|T\| < \delta$,在 $T$ 所属的任一小区间 $\Delta_i$ 上,都有
	\[ \omega_i = M_i-m_i = \sup_{x_1,x_2\in\Delta_i}|f(x_1)-f(x_2)| \leqslant \frac{\eps}{b-a} \]
	从而
	\[ \sum_{T}\omega_i\Delta x_i \leqslant \frac{\eps}{b-a}\sum_T\Delta x_i = \eps \]
\end{proof}



\section{不定积分}

这部分我的参考书是《积分的方法与技巧》(金玉明等)。

\subsection{分项积分法}

若干微分式的和或差的不定积分,等于每个微分式的各自积分的和或差。
\[ \int\left( f(x)+g(x)-h(x) \right)\d x = \int f(x)\d x + \int g(x)\d x - \int h(x)\d x \]
因此多项式的积分可以简单的通过积分各个单项式得到。

当分母可以被拆成低次式时,可以考虑使用分项积分法。

如果一个分式的分母为多项式,则可把它化成最简单的分式再积分。如
\[ \frac{1}{x^2-a^2} = \frac{1}{2a}\left( \frac{1}{x-a}-\frac{1}{x+a} \right) \]
这里可以通过通分后待定系数得到。于是其积分为
\[ \int \frac{\d x}{x^2-a^2} = \frac{1}{2a}\ln\left|\frac{x-a}{x+a}\right|+C \]
对于更复杂的真分式的情况,若要计算的是
\[ \int \frac{mx+n}{x^2+px+q} \d x \]
分母不一定能直接分解,但总能进行配方
\[ x^2+px+q = \left(x+\frac{p}{2}\right)^2+q-\frac{p^2}{4} = t^2 \pm a^2 \]
再令 $A=m,B=n-\frac12mp$,可得
\[ \int \frac{mx+n}{x^2+px+q} \d x = \int \frac{At+B}{t^2 \pm a^2} \d
	= A\int \frac{t\d t}{t^2 \pm a^2} + B\int \frac{\d t}{t^2 \pm a^2}\]
其中

\[
	\begin{aligned}
		A\int \frac{t \d t}{t^2 \pm a^2} & = \frac{A}{2}\int \frac{\d(t^2 \pm a^2)}{t^2 \pm a^2} = \frac{A}{2} \ln|t^2 \pm a^2| +C \\
		B\int \frac{\d t}{t^2 + a^2}     & = \frac{B}{a}\arctan\frac{t}{a}+C                                                       \\
		B\int \frac{t \d t}{t^2 - a^2}   & = \frac{B}{2a}\ln\left|\frac{t-a}{t+a}\right|+C
	\end{aligned}
\]

因此当 $p^2<4q$ 时,可以得到
\[
	\begin{aligned}
		\int \frac{mx+n}{x^2+px+q} & = \frac{A}{2} \ln|t^2 + a^2| + \frac{B}{a}\arctan\frac{t}{a} + C                               \\
		                           & =\frac{m}{2}\ln|x^2+px+q| + \frac{2n-mp}{\sqrt{4q-p^2}}\arctan{\frac{2x+p}{\sqrt{4q-p^2}}} + C
	\end{aligned}
\]

当 $p^2>4q$ 时,可以得到
\[
	\begin{aligned}
		\int \frac{mx+n}{x^2+px+q} & = \frac{A}{2} \ln|t^2 - a^2| + \frac{B}{2a}\ln\left|\frac{t-a}{t+a}\right|                                                  \\
		                           & =\frac{m}{2}\ln|x^2+px+q| + \frac{2n-mp}{2\sqrt{4q-p^2}}\ln\left| \frac{2x+p-\sqrt{p^2-4q}}{2x+p+\sqrt{p^2-4q}} \right| + C
	\end{aligned}
\]

\subsection{分部积分法}

根据乘积的微分法则
\[ \d(uv) = u \d v + v \d u \]
显然有
\[ \int u \d v = uv - \int v \d u \]

更进一步的,假设
\[ \begin{aligned}
		\int u v^{(n+1)} \d x & = \int u \d v^{(n)} = uv^{(n)} - \int u' v(n) \d x                                                                 \\
		                      & = u v^{(n)} - u^{(1)} v^{(n-1)} + u^{(2)} v^{(n-2)} + \cdots + (-1)^n u^{(n)} v + (-1)^{n+1} \int u^{(n+1)} \d v x
	\end{aligned} \]

例如当 $P(x)$ 是关于 $x$ 的多项式时,假设计算
\[ \int P(x) \ee^{ax} \d x = \ee^{ax} \left( \frac{P}{a} - \frac{P'}{a^2} + \frac{P''}{a^3} + cdots \right) \]
另一个形式是
\[ \int P(x) \sin ax \d x = \sin ax \left( \frac{P'}{a^2} - \frac{P^{(3)}}{a^4} + \cdots  \right) - \cos ax \left( \frac{P}{a} - \frac{P''}{a^3} + \cdots \right) \]
对于 $\cos ax$,类似的有
\[ \int P(x) \cos ax \d x = \sin ax \left( \frac{P'}{a^2} - \frac{P^{(3)}}{a^4} + \cdots  \right) + \cos ax \left( \frac{P}{a} - \frac{P''}{a^3} + \cdots \right) \]

这种一般答案比被积函数高一阶,试一试系数就行。有时候会两种形式交错,比如
\[ \int \ee^{ax} \sin bx \d x = \frac{\ee^{ax}}{a^2+b^2} (a \sin bx - b \cos bx) + C\]


\subsection{三角换元积分法}

这种题需要灵感并积累一些套路。

三角换元一般隐藏这一些直角三角形,建议草稿纸上画出来,分析的清楚一点。

\paragraph{三角函数及双曲三角函数}

回顾一下高中知识。除了常见的 $\sin x, \cos x, \tan x$,其倒数也有名字
\[ \sin x \csc x = 1, \quad \cos x \sec x = 1, \quad \tan x \cot x = 1 \]
常见公式 $\sin^2 x + \cos^2 x = 1$ 有变形
\[ 1 + \tan^2 x = \sec^2 x, \quad 1 + \cot^2 x = \csc^2 x \]
三角函数的和差关系
\[ \begin{aligned}
		\sin (x \pm y) & = \sin x \cos y \pm \cos x \sin y               \\
		\cos (x \pm y) & = \cos x \cos y \mp \sin x \sin y               \\
		\tan (x \pm y) & = \frac{\tan x \pm \tan y}{1 \mp \tan x \tan y}
	\end{aligned} \]
令 $t = \tan \frac{x}{2}$,有万能公式
\[ \sin x = \frac{2 t}{1 + t^2}, \quad \cos x = \frac{1 - t^2}{1 + t^2}, \quad \tan x = \frac{2t}{1-t^2} \]

双曲三角函数的定义是
\[ \sinh x = \frac{\ee^x - \ee^{-x}}{2}, \quad \cosh x = \frac{\ee^x + \ee^{-x}}{2}, \quad \tanh = \frac{\ee^x - \ee^{-x}}{\ee^x + \ee^{-x}} \]
有公式 $\sinh^2 x - \cosh^2 x = 1$,变形即
\[ 1 - \tanh^2 x = \frac{1}{\cosh^2 x}, \quad 1 - \coth^2 x = \frac{1}{\sinh^2 x} \]

类似的有反双曲三角函数,即
\[ \arcsinh x = \ln(x+\sqrt{x^2+1}), \quad \arccosh x = \ln(x+\sqrt{x^2-1}), \quad \arctanh x = \frac{1}{2} \ln \left| \frac{1+x}{1-x}\right| \]

\paragraph{形式一} 如果没有特殊形式,可以尝试齐次化(或者硬代万能公式),强行化有理式。

例如
\[ \int \frac{\d x}{a^2 \sin^2 x + b^2 \cos^2 x} \]
尝试 $\tan x = t$,有
\[ \d t = \frac{\d x}{\cos^2 x} = (1 + t^2) \d x \]
因此
\[ LHS = \int \frac{\d t}{a^2 t^2 + b^2} = \frac{1}{ab} \arctan \left( \frac{a}{b} \tan x \right) + C \]


假如计算
\[ \int \frac{\d x}{\sin 2x} = \int \frac{t^2+1}{2t} \d x = \frac{1}{2} \ln | \tan x| + C \]


\paragraph{形式二} 形如 $\sqrt{a^2 - x^2}$ 的积分,设 $x = a \sin \theta$,则
\[ \d x = a \cos \theta \d \theta, \quad \sqrt{a^2 - x^2} = a \cos \theta \]

例如
\[ \int \sqrt{a^2 - x^2} \d x = a^2 \int \cos^2 \theta \d \theta = \frac{x}{2}\sqrt{a^2 - x^2} + \frac{a^2}{2} \arcsin \frac{x}{a} + C \]

\paragraph{形式三} 形如 $\sqrt{a^2 + x^2}$ 的积分,设 $x = a \tan \theta$,则
\[ \d x = a \sec^2 \theta \d \theta, \quad \sqrt{a^2+x^2} = a \sec \theta \]

双元法是一类特殊的三角换元,来自虚调子\footnote{\href{https://zhuanlan.zhihu.com/p/443599480}{一小时学会双元教程!}}。其本质就是 $x\d x \pm y \d y = 0$ 更直观,在部分题中能够避免被根号绕进去。

\subsection{欧拉替换法}

对于二次根式 $\sqrt{a x^2 + b x + c}$。

\paragraph{形式一} 令
\[ \sqrt{a^2x^2+bx+c} = t - ax \]
两边平方可得
\[ x = \frac{t^2 - c}{2at + b}, \quad 2\frac{at^2 + bt + ca}{(2at+b)^2} \d t \]

\section{定积分的基本性质}

\subsection{中值定理}

\begin{theorem}[积分第一中值定理]
	设 $f$ 在 $[a,b]$ 上可积且连续,则存在 $\xi \in [a, b]$ 使得
	\[ \int_{a}^{b} f(x) \d x = f(\xi) (b-a) \]
\end{theorem}

\begin{proof}
	设 $m, M$ 分别是 $f$ 在 $[a,b]$ 上的最小值和最大值,显然存在 $\mu \in [m, M]$ 使得
	\[ m \leqslant \mu = \frac{1}{b-a}\int_{a}^{b} f(x) \d x \leqslant M \]
	由连续性显然存在 $f(\xi) = \mu$。
\end{proof}

\begin{theorem}[广义积分第一中值定理]
	假设 $f, g$ 在 $[a, b]$ 上可积且连续,在 $[a,b]$ 上恒有 $g \geqslant 0$ 或 $g \leqslant 0$,则存在 $\xi \in [a,b]$ 使得
	\[ \int_{a}^{b} f(x)g(x) \d x = f(\xi) \int_{a}^{b} g(x) \d x \]
\end{theorem}

\begin{proof}
	不妨令 $g \geqslant 0$。设 $m, M$ 为 $f$ 在 $[a, b]$ 上的最小值和最大值,如果
	\[ \int_{a}^{b} g(x) \d x \neq 0 \]
	则存在 $\mu \in [m, M]$ 使得
	\[ m \leqslant \mu = \frac{ \int_{a}^{b} f(x)g(x) \d x}{\int_{a}^{b} g(x) \d x} \leqslant M \]
	显然此时存在 $\xi = \mu$。假如积分为 $0$,则显然有 $g = 0$,任意 $\xi$ 都成立。
\end{proof}

\begin{theorem}[积分第二中值定理]
	假设 $f, g$ 在 $[a,b]$ 上可积,且 $g \geqslant 0$。如果 $g$ 单调递减,则存在 $\xi \in [a, b]$ 满足
	\[ \int_{a}^{b} f(x) g(x) \d x = g(a) \int_{a}^{\xi} f(x) \d x \]
	如果 $g$ 单调递增,则存在 $\eta \in [a, b]$ 满足
	\[ \int_{a}^{b} f(x) g(x) \d x = g(b) \int_{\eta}^{b} f(x) \d x \]
\end{theorem}

\begin{proof}
	只证第一个命题。我们给出两种证法。

	(1) 设 $F(x)$ 为 $f$ 的原函数,首先分部积分法
	\[ \int_{a}^{b} f(x) g(x) \d x = F(b)g(b) - F(a)g(a) - \int_{a}^{b} F(x)g'(x) \d x \]
	由 Lagrange 中值定理,知存在
	\[ \int_{a}^{b} F(x)g'(x) \d x = F(\xi) (g(b) - g(a)) \]
	带入整理得
	\[ \int_{a}^{b} f(x) g(x) \d x = g(a) (F(\xi) - F(a)) + g(b) (F(b) - F(\xi)) \]
	即广义积分第二中值定理。注意到取 $g_1 = g$ 且在 $x=b$ 处设置间断点 $g(b) = 0$,以上证明仍成立。

	(2)用到了 Abel 变换,有点复杂。TODO
\end{proof}

\begin{theorem}[广义积分第二中值定理]
	假设 $f, g$ 在 $[a,b]$ 上可积,且 $g$ 单调,则存在 $\xi \in [a, b]$ 满足
	\[ \int_{a}^{b} f(x) g(x) \d x = g(a) \int_{a}^{\xi} f(x) \d x + g(b) \int_{\xi}^{b} f(x) \d x \]
\end{theorem}

\begin{proof}
	不妨设递增,令 $g_1 = g(b) - g(x)$,应用积分第二中值定理即可。
\end{proof}

\begin{theorem}[Cauchy 积分不等式]
	如果 $f, g$ 在 $[a, b]$ 上可积,那么我们有
	\[ \left(\int_{a}^{b} f(x) g(x) \d x \right)^2 \leqslant \left(\int_{a}^{b} f^2(x)\d x \right)\left(\int_{a}^{b} g^2(x) \d x\right) \]
\end{theorem}
\begin{proof}
	对任意划分 $T$ 有
	\[ \left(\sum f(x_i) g(x_i) \Delta x_i\right)^2 \leqslant \left(\sum f^2(x_i) \Delta x_i\right)\left(\sum g^2(x_i) \Delta x_i\right) \]
	故当 $\|T\| \to 0$ 时也成立。
\end{proof}

\begin{theorem}[Jensen 积分不等式]
	如果 $f$ 在 $[a, b]$ 上是凸且连续的,且 $\varphi(x)$ 在 $\mathbb{R}$ 上连续,那么我们有
	\[ f\left(\frac{1}{c} \int_{0}^{c} \varphi(t) \d t \right) \leqslant \frac{1}{c} \int_{0}^{c} f(\varphi(t))\d t \]
\end{theorem}
\begin{proof}
	对任意划分 $T$ 有
	\[ f\left(\frac{1}{c} \sum \varphi(t) \Delta x_i \right) \leqslant \frac{1}{c} \sum f(\varphi(t))\Delta x_i \]
	故当 $\|T\| \to 0$ 时也成立。
\end{proof}

\begin{theorem}[Hadamard 积分不等式]
	如果 $f$ 在 $[a, b]$ 上是凸且连续的,那么有
	\[ f\left(\frac{a+b}{2}\right) \leqslant \frac{1}{b - a} \int_{a}^{b} f(t) \d t \leqslant \frac{f(a) + f(b)}{2} \]
\end{theorem}

\begin{proof}
	(1) 假如 $f$ 可导,设
	\[ F_1(x) = \int_{a}^{x} f(t) \d t - (x-a)f\left(\frac{a+x}{2}\right) \]
	求导,由 Taylor 公式得
	\[ F_1'(x) = f(x) - f\left(\frac{a+x}{2}\right) - \frac{x-a}{2}f'\left(\frac{a+x}{2}\right) = \frac{f''(\xi_1)}{2} \left(\frac{x-a}{2}\right)^2 \geqslant 0 \]
	因此 $F_1(x) \geqslant 0$。再构造
	\[ F_2(x) = (x - a)(f(x) + f(a)) - 2 \int_{a}^{x} f(t) \d t \]
	求导得
	\[ F_2'(x) = (x-a) f'(x) + f(a) - f(x), \quad F_2''(x) = (x-a)f''(x) \geqslant 0 \]
	故 $F_2'(x) \geqslant F_2'(a) = 0$,因此 $F_2(x) \geqslant 0$。

	(2)前面的证明是不严谨的,因为 $f$ 不一定可导。构造 $F(\lambda) = f(a + \lambda(b - a))$,注意到
	\[ \begin{aligned}
			F(\lambda_1 + \mu(\lambda_2 - \lambda_1)
			 & = f(\mu(a + \lambda_1(b-a)) + (1 - \mu)(a + \lambda_2(b-a))) \\
			 & \leqslant \mu F(\lambda_1) + (1 - \mu) F(\lambda_2)
		\end{aligned}  \]
	故 $F$ 是凸的。因此根据 Jensen 不等式得
	\[ f\left(\frac{a+b}{2}\right) = F\left(\int_{0}^{1} \lambda \d \lambda \right) \leqslant \int_{0}^{1} F(\lambda) \d \lambda = \frac{1}{b-a}\int_{a}^{b} f(t) \d t \]
	另一方面
	\[ \int_{0}^{1} F(\lambda) \d \lambda = \frac{1}{b-a}\int_{a}^{b} f(t) \d t \leqslant \int_{0}^{1} \left((1 - \lambda)f(a) + \lambda f(b)\right) \d \lambda = \frac{f(a) + f(b)}{2} \]
\end{proof}

\subsection{变量替换法}

\begin{theorem}
	假设函数 $\varphi : [\alpha, \beta] \to [a, b]$ 连续可导,且 $\varphi(\alpha) = a, \varphi(\beta) = b$,则对任意的 $[a, b]$ 上的可积函数 $f$ 有 $f(\varphi(t))\varphi'(t)$ 可积,且
	\[ \int_{a}^{b} f(x) \d x = \int_{\alpha}^{\beta} f(\varphi(t)) \varphi'(t) \d t \]
	假如 $\varphi$ 严格单调,则有
	\[ \int_{\varphi(\alpha)}^{\varphi(\beta)} f(x) \d x = \int_{\alpha}^{\beta} f(\varphi(t)) \varphi'(t) \d t \]

\end{theorem}

\section{反常积分}

\begin{definition}
	假设函数 $f : [a, \infty) \to \mathbb{R}$ 可积。如果极限存在,则记
	\[ \int_{a}^{+\infty} f(x) \d x = \lim_{A \to +\infty} \int_{a}^{A} f(x) \d x \]
	称为 $f$ 在 $[a, +\infty)$ 上的无穷积分或第一类反常积分。
\end{definition}

类似的我们可以定义 $(-\infty, a]$ 上的无穷积分。考虑函数 $f : (-\infty, +\infty) \to \mathbb{R}$,若对任意的 $c$ 下面的积分都收敛,则记
\[ \int_{-\infty}^{+\infty} f(x) \d x = \int_{-\infty}^{c} f(x) \d x + \int_{c}^{+\infty}f(x) \d x \]
当然,只要存在 $c$ 使得收敛,将对全体 $\mathbb{R}$ 收敛。

一个常见的误区是认为
\[ \lim_{A \to +\infty} \int_{-A}^{A} f(x) \d x \]
收敛,则 $(-\infty, +\infty)$ 上的反常积分存在。反例是 $f = \sin x$。

\subsection{收敛判别法}

接下来我们考察
\[ I(f) = \int_{a}^{+\infty} f(x) \d x \]
的收敛性。

\begin{theorem}[有界判别法]
	积分
	\[ I_A(f) = \int_{a}^{A} f(x) \d x \]
	有界,是 $I(f)$ 收敛的充分必要条件。
\end{theorem}

\begin{theorem}[比较判别法]
	如果 $0 \leqslant f(x) \leqslant g(x)$。若 $I(f)$ 发散则 $I(g)$ 发散。若 $I(g)$ 收敛则 $I(f)$ 收敛。

	对于极限形式
	\[ \lim_{x \to +\infty} \frac{f(x)}{g(x)} = K \geqslant 0 \]
	若 $K = 0$ 且 $I(g)$ 收敛则 $I(f)$ 收敛。若 $K = +\infty$ 且 $I(g)$ 发散则 $I(f)$ 发散。若 $K$ 为正实数,则 $I(f)$ 收敛是 $I(g)$ 收敛的充要条件。
\end{theorem}

\begin{proof}
	非极限形式易证。对于极限形式,若 $K$ 是正实数,则存在 $a_1$ 使得对 $x > a_1$ 有
	\[ \frac{K}{2} g(x) \leqslant f(x) \leqslant \frac{3K}{2} g(x) \]
	成立,套用非极限形式即证。
\end{proof}

\begin{theorem}[Cauchy 判别法]
	假定 $a > 0$,若
	\[ f(x) \leqslant \frac{K}{x^p}, \quad K > 0, p > 1 \]
	则 $I(f)$ 收敛。若
	\[ f(x) \geqslant \frac{K}{x^p}, \quad K > 0, p \leqslant 1 \]
	则 $I(f)$ 发散。

	极限形式:若
	\[ \lim_{x \to +\infty} x^p f(x) = K \geqslant 0 \]
	当 $K$ 是非负实数且 $p > 1$ 时收敛。当 $K$ 是正实数或 $+\infty$ 且 $p \leqslant 1$ 时 $I(f)$ 发散。
\end{theorem}

\begin{proof}
	在比较判别法中取 $g = x^{-p}$ 即可。
\end{proof}

\begin{theorem}[Abel 判别法]
	假设反常积分
	\[ \int_{a}^{+\infty} f(x) \d x \]
	收敛且 $g(x)$ 单调有界,则反常积分
	\[ \int_{a}^{+\infty} f(x) g(x) \d x \]
	收敛。
\end{theorem}

\begin{theorem}[Dirichlet 判别法]
	假如函数
	\[ F(A) = \int_{a}^{A} f(x) \d x \]
	有界且 $g(x)$ 单调同时 $g \to 0$ 则反常积分
	\[ \int_{a}^{+\infty} f(x) g(x) \d x \]
\end{theorem}

\section{常见的积分}

基本积分表之前已经列出,我们这里列一些组合的。



\begin{example}[有理式]
	\[ \begin{aligned}
			\int \frac{1}{(x+a)(x+b)} \d x & = \frac{1}{b-a} \ln \left| \frac{x+b}{x+a} \right| + C                \\
			\int \frac{1}{x^2(x+a)}   \d x & = \frac{1}{a^2} \ln \left| 1 + \frac{a}{x} \right| - \frac{1}{ax} + C \\
			\int \frac{1}{x(x^2+a)}   \d x & = \frac{1}{2a} \ln \left| \frac{x^2}{x^2+a}\right| + C
		\end{aligned} \]
\end{example}

\begin{example}[带 $a^2 \pm x^2$]
	\[ \begin{aligned}
			\int \frac{1}{a^2+x^2} \d x                     & = \frac{1}{a} \arctan \frac{x}{a} + C                                                        \\
			\int \frac{1}{a^2 - x^2} \d x                   & = \frac{1}{a} \arctanh \frac{x}{a} + C  = \frac{1}{2a} \ln \left| \frac{a+x}{a-x}\right| + C \\
			\int \frac{1}{\sqrt{a^2 - x^2}} \d x            & = \arcsin \frac{x}{a} + C                                                                    \\
			\int \frac{1}{\sqrt{a^2 + x^2}} \d x            & = \arcsinh \frac{x}{a} + C  = \ln \left| \sqrt{a^2 + x^2} + x^2 \right|  + C                 \\
			\int \frac{x}{\sqrt{a^2 \pm x^2}} \d x          & = \pm\sqrt{a^2 \pm x^2} + C                                                                  \\
			\int \frac{1}{(a^2 \pm x^2)^{\frac{3}{2}}} \d x & = \frac{x}{a^2\sqrt{x^2 \pm a^2}} + C                                                        \\
			\int \frac{x}{(a^2 \pm x^2)^{\frac{3}{2}}} \d x & = \mp \frac{x}{a^2\sqrt{x^2 \pm a^2}} + C                                                    \\
			\int \sqrt{a^2 + x^2} \d x                      & = \frac{x}{2}  \sqrt{a^2+x^2}+\frac{a^2}{2} \int \frac{\d x}{\sqrt{a^2-x^2}}                 \\
			\int \frac{1}{(a^2+x^2)^n} \d x                 & = \frac{x}{2(n-1)a^2(x^2+a^2)^{n-1}} + \frac{2n-3}{2(n-1)a^2} \int \frac{\d x}{(x^2+a^2)^n}
		\end{aligned} \]
\end{example}

\begin{example}[根式]
	\[ \begin{aligned}
			\int x \sqrt{x+a} \d x           & = \frac{2}{15} (3x - 2a) (x+a)^{\frac{3}{2}} + C                                                                   \\
			\int \frac{x}{\sqrt{x+a}} \d x   & = \frac{2}{3} (x - 2a) \sqrt{x+a} + C                                                                              \\
			\int \frac{1}{x\sqrt{x+a}} \d x  & = -\frac{1}{\sqrt{a}} \ln \left| \frac{\sqrt{a + x} + \sqrt{a}}{\sqrt{a + x} - \sqrt{a}} \right| + C , \quad a > 0 \\
			\int \frac{1}{x\sqrt{x-a}} \d x  & = \frac{2}{\sqrt{a}}\arctan \sqrt{\frac{x}{a}-1}, \quad a > 0                                                      \\
			\int \sqrt{\frac{a+x}{a-x}} \d x & = \arcsin \frac{x}{a} - \sqrt{a^2 - x^2} + C, \quad a > 0                                                          \\
			\int \sqrt{\frac{a-x}{a+x}} \d x & = \arcsin \frac{x}{a} + \sqrt{a^2 + x^2} + C, \quad a > 0                                                          \\
			\int \frac{\sqrt{x+a}}{x} \d x   & = 2 \sqrt{x+a} + a \int \frac{\d x}{x \sqrt{x+a}}
		\end{aligned} \]
\end{example}
