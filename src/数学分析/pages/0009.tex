\chapter{概率论}

\newcommand{\bbP}{\mathbb{P}}

考研中遇到的概率论很少,暂放于此。

\section{随机事件与概率}

假定某个试验有有限个可能的结果 $\seq{e}{N}$,其出现机会是等可能的。假定事件 $E$ 包含了其中的 $M$ 个结果,则称事件 $E$ 的概率为
\[ \bbP(E) = M / N \]
这是古典概型。另一些时候,我们把结果扩展到无限的情况。我们把度量(可以通俗的理解为面积)相同的事件称为等可能的,这是几何概型。

称集合 $\Omega$ 随机事件的样本空间,其元素 $\omega$ 称为基本事件。事件 $A$ 是 $\Omega$ 的一个子集,赋给每个事件一个实数值 $\bbP(A)$,称为概率。其满足要求:

\begin{itemize}
	\item 非负性:$\bbP(A) \geqslant 0$。
	\item 规范性:$\bbP(\Omega) = 1, \bbP(\varnothing) = 0$。
	\item 加法公理。
\end{itemize}

若两个事件不能在同一次试验中同时发生,则称为互斥的。若一些事件中任意两个都互斥,则称为两两互斥。可以进一步导出对立事件的概念。可以把集合的交并补也引入。

\begin{theorem}[加法公理]
	若干个互斥事件之和的概率,等于各事件的概率之和:
	\[ \bbP\left(\bigcup A_i\right) = \sum \bbP(A_i) \]
\end{theorem}

\begin{note}
	这条公理其实是在古典定义、统计定义下是可证明的,但是为什么是公理呢?因为我们的确可以建立一种新的概率理论,在其中加法公理不成立。类似于平行公设。
\end{note}

\begin{definition}
	设 $A,B$ 是两个事件且 $\bbP(A) > 0$,我们称在已知 $A$ 发生的条件下事件 $B$ 发生的概率为条件概率,记作
	\[ \bbP(B | A) = \frac{\bbP(AB)}{\bbP(A)} \]
\end{definition}

假如 $\bbP(B \mid A) > \bbP(B)$,我们可以说事件 $A$ 促进了事件 $B$ 的发生。反之 $\bbP(B \mid A) = \bbP(A)$,则 $B$ 的发生对 $A$ 无影响。

形式的说:

\begin{definition}[条件概率]
	若两事件 $A, B$ 满足
	\[ \bbP(A B) = \bbP(A) \bbP(B) \]
	则称 $A, B$ 独立。
\end{definition}

变换形式是为了避免 $0$ 的讨论。

设一列事件 $A_1, A_2, \cdots$,假如从中取出任意有限个都成立
\[ \bbP(A_{i1} \cdots A_{im}) = \bbP(A_{i1}) \cdots \bbP(A_{im}) \]
那么称事件 $A_1, A_2, \cdots$ 相互独立。注意与两两独立的区别。


\begin{theorem}[全概率公式]
	设 $B_1, \cdots$ 为一列事件,他们两两互斥且每次试验至少发生一个。有时称这种性质为“完备事件群”。那么对任意事件 $A$ 有
	\[ \bbP(A) = \sum_{i=1}^n \bbP(B_i)\bbP(A \mid B_i) \]
\end{theorem}

\begin{proof}
	显然
	\[ \bbP(A) = \sum_{i=1}^n \bbP(B_i A) = \sum_{i=1}^n \bbP(B_i)\bbP(A \mid B_i) \]
\end{proof}


\begin{theorem}[全概率公式]
	贝叶斯公式:对 $n$ 个两两不相容事件 $\seq{A}{N}$,则对事件 $B$ 有
	\[ \bbP\left(A_j | B\right) = \frac{\bbP(A_j)\bbP(B | A_j)}{\sum\limits_{i=1}^n \bbP(A_i)\bbP(B | A_i) } \]
\end{theorem}

\begin{proof}
	显然
	\[ \bbP\left(A_j | B\right) = \frac{\bbP(A_j B)}{\bbP(B)} = \frac{\bbP(A_j)\bbP(B | A_j)}{\sum\limits_{i=1}^n \bbP(A_i)\bbP(B | A_i) } \]
\end{proof}

\begin{note}
	若 $\bbP(AB) = 0$,不意味着 $AB = \varnothing$。比如 $[0,1]\cap [1,2] = \{1\}$,但概率是 $0$。
\end{note}

\subsection{摸球不放回模型}

袋中有 $a$ 个黑球,$b$ 个白球,若不放回的取 $n$ 个球,其中恰好 $k$ 个白球的概率为
\[ \bbP(A_k) = \frac{\binom{b}{k} \binom{a}{n-k}}{\binom{a+b}{n}} \]

\begin{theorem}[抽签原理]
	袋中有 $a$ 个黑球,$b$ 个白球,若不放回的依次取球,每次抽中白球的概率都为 $\frac{b}{a+b}$。
\end{theorem}

\begin{proof}
	对于第 $i$ 次抽检,可以看成从 $a+b$ 个球中抽出 $i$ 个球排成一排,其可能数是 $A_{a+b}^i$。而第 $i$ 个位置是白球的概率为 $b A_{a+b-1}^{i-1}$,故
	\[ \bbP(A_i) = \frac{b A_{a+b-1}^{i-1}}{A_{a+b}^i} = \frac{b}{a+b} \]
\end{proof}

\subsection{摸球放回模型}

袋中有 $a$ 个黑球,$b$ 个白球,若有放回的取出 $n$ 个球,其中恰好 $k$ 个白球的概率为
\[ \bbP(B_k) = \binom{n}{k} \frac{a^{n-k} b^k}{(a+b)^n} = [x^k] \left(\frac{a + bx}{a+b}\right)^n \]

\section{一维随机变量及其分布}

随机变量就是“其值会随机而定”的变量,是一个实值单值函数。设随机实验的样本空间为 $\Omega$,若对于任意 $\omega \in \Omega$ 都有唯一实数 $X(\omega)$ 与其对应,且对任意实数 $x$,$\{\omega \mid X(\omega) \leqslant x, \omega \in \Omega\}$ 是随机时间,则称定义在 $\Omega$ 上的实值单值函数为随机变量。

\paragraph{离散型随机变量}

研究一个随机变量,不只是查看其能取哪些值,更要看其取各种值的概率如何。

\begin{definition}
	设 $X$ 是离散型随机变量,其全部可能值为 $\{a_1, \cdots\}$,则称
	\[ p_i = \bbP\{X = a_i\} \]
	为其概率函数。称
	\[ F(x) = \bbP\{X \leqslant x\} = \sum_{a_i \leqslant x} p_i \]
	是其分布函数。记为 $X \sim F(x)$,称 $X$ 服从 $F(x)$ 分布。
\end{definition}

其具有以下的性质:

\begin{itemize}
	\item $F(x)$ 对 $x$ 单调不减。
	\item $F(x)$ 是 $x$ 的右连续函数。
	\item $F(- \infty) = \lim\limits_{i \to -\infty} F(x) = 0$,$F(+\infty) = \lim\limits_{x \to + \infty} F(x) = 1$。
	\item $\bbP\{X \leqslant a\} = F(a)$,$\bbP\{X < a\} = F(a - 0)$,$\bbP\{X = a\} = F(a) - F(a - 0)$。
\end{itemize}

如果随机变量只取有限的可列值,则称为离散型随机变量,可以写分布列。

\paragraph{连续型随机变量}

如果随机变量的分布函数是 $F(x)$,则 $f(x) = F'(x)$ 是其的概率密度函数。

其具有以下的性质:

\begin{itemize}
	\item $f(x) \geqslant 1$。
	\item 对任何常数 $a, b$ 有
	      \[ \bbP\{a \leqslant X \leqslant b\} = F(b) - F(a) = \int_{a}^{b} f(x) \d x, \qquad \int_{-\infty}^{\infty} f(x) \d x = 1 \]
\end{itemize}

\subsection{常见随机分布}

\paragraph{二项分布}
如果 $X$ 的概率分布为
\[ \bbP\{X = k\} = \binom{n}{k} p^k(1 - p)^{n-k}. \quad k = 0, 1, \cdots, n \]
则称 $X$ 服从参数为 $(n, p)$ 的二项分布,记为 $X \sim B(n, p)$。特别的,当 $n=1$ 时称为二项分布。二项分布也是 $n$ 重伯努利实验中事件 $A$ 发生的次数,其中 $\bbP(A) = p$。

\paragraph{泊松分布}
如果 $X$ 的概率分布为
\[ \bbP\{X = k\} = \frac{\lambda^k}{k!} \ee^{-\lambda} , \quad k = 0, 1, \cdots \]
则记为 $X \sim \bbP(\lambda)$。

\paragraph{几何分布}
如果 $X$ 的概率分布为
\[ \bbP\{X=k\} = (1 - p)^{k-1}p, \quad k = 1, 2, \cdots \]
则记为 $X \sim G(p)$。

\paragraph{超几何分布}
如果 $X$ 的概率分布为
\[ \bbP\{X = k\} = \frac{\binom{M}{k} \binom{N - M}{n - k}}{\binom{N}{n}}, \quad \max(0, n - N + M) \leqslant k \leqslant \min(M, n) \]
则记为 $X \sim H(n, N, M)$。设有 $N$ 个产品组成的整体,其中有 $M$ 个不合格产品,从中取出 $n$ 个,次品数为 $k$,这就是组合意义。

\paragraph{均匀分布}
如果 $X$ 的概率密度函数为
\[ f(x) = \frac{1}{b - a}, \quad a < x < b \]
则记为 $X \sim U(a, b)$。

\paragraph{指数分布}
如果 $X$ 的概率密度函数为
\[ f(x) = \lambda \ee^{-\lambda x}, \quad x > 0 \]
则记为 $X \sim E(\lambda)$。

\paragraph{正态分布}
如果 $X$ 的概率密度为
\[ f(x) = \frac{1}{\sqrt{2 \pi}  \sigma} \exp \left(- \frac{(x - \mu)^2}{2\sigma^2}  \right) \]
则记为 $X \sim N(\mu, \sigma^2)$。

\section{多维随机变量}

设 $X = (\seq{X}{N})$,其中每个分量都是一维随机变量,则 $X$ 是一个 $n$ 维随机变量。

\paragraph{离散型}

假如每个分量是离散型的,那么称 $X$ 是 $n$ 维离散型随机变量。

\begin{definition}
	以 $\{a_{i1}, \cdots\}$ 记 $X_i$ 的全部可能值,则事件的概率
	\[ p(\seq{j}{n}) = \bbP\{X_1 = a_{1j_1}, \cdots, X_n = a_{n j_n}\} \]
	为随机变量的概率函数。显然应该满足条件
	\[ p(j_1, \cdots) \geqslant 0, \quad \sum_{j_n} \cdots \sum_{j_1} p(j_1, \cdots)  =1 \]
\end{definition}

\paragraph{连续型}
连续型随机变量不能简单的定义为各分量都是“一维连续型随机变量的那种”。考虑 $X_1 = X_2$,则 $(X_1, X_2)$ 仅在对角线处有值,故不可能存在概率密度函数。

\begin{definition}
	设 $f(\seq{x}{n})$ 是定义在 $\mathbb{R}^n$ 上的非负函数,使得对 $\mathbb{R}^n$ 中的任何集合 $A$,有
	\[ \bbP\{X \in A\} = \int_A \cdots \int f(\seq{x}{n}) \d x_1 \cdots \d x_n \]
	则称 $f$ 是 $X$ 的概率密度函数。显然当 $\bbP\{ X \in \mathbb{R}^n\} = 1$。
\end{definition}

也可以定义分布函数
\[ F(\seq{x}{n}) = \bbP\{X_1 \leqslant x_1, \cdots, X_n \leqslant x_n \} \]

对于二维 $(X, Y)$,若 $f$ 在点 $(x, y)$ 处连续,则
\[ \frac{\partial^2 F(x, y)}{\partial x \partial y} = f(x, y) \]
反之若 $F(x, y)$ 连续可导,则此式是其概率密度。

\paragraph{边缘分布}

对任意的 $n$ 个实数,$x_1, x_2, \cdots, x_n$,称 $n$ 元函数
\[ F(\seq{x}{n}) = \bbP\{X_1 \leqslant x_1, X_2 \leqslant x_2, \cdots. X_n \leqslant x_n\} \]
为多维随机变量 $(\seq{x}{n})$ 的联合分布函数,记为 $(\seq{x}{n}) \sim F(\seq{x}{n})$。

\begin{itemize}
	\item $F(x, y)$ 是 $x, y$ 的单调不减函数。
	\item $F(x, y)$ 是 $x, y$ 的右连续函数。
	\item $F(-\infty, y) = F(x, -\infty) = F(-\infty, -\infty) = 0$,$F(+\infty, +\infty) = 1$。
	\item 对任意 $x_1 < x_2, y_1 < y_2$,有
	      \[ \bbP\{x_1 < X \leqslant x_2, y_1 Y \leqslant y_2\} = F(x_2, y_2) - F(x_2, y_1) - F(x_1, y_2) + F(x_1, y_1) \geqslant 0 \]
\end{itemize}

设其联合分布函数为 $F(x,y)$,定义边缘分布函数
\[ F_X(x) = \bbP\{X \leqslant x\} = F(x, +\infty) \]
同理,有 $F_Y(y) = F(+\infty, y)$。


\subsection{常见的二维分布}

\paragraph{二维均匀分布} 称 $(X,Y)$ 在平面有界区域 $D$ 上服从均匀分布,如果 $(X, Y)$ 的概率密度为
\[ f(x, y) = \frac{1}{S_D}, \quad (x, y) \in D \]

\paragraph{二维正态分布} TODO。

\subsection{二维随机变量的独立性}

\paragraph{条件概率分布}

对于二维离散型随机向量 $(X, Y)$,其联合概率分布为
\[ p_{i,j} = \bbP\{X= x_i, Y = y_j\} \]
记为 $(X,Y) \sim p_{i,j}$。依条件概率的定义,有
\[ \bbP\{ X = x_i \mid Y = y_i \} = \frac{\bbP\{X = x_i, Y = y_j\}}{\bbP\{Y=y_j\}} = \frac{p_{i, j}}{p_{, j}} \]

设二维连续型随机向量 $(X, Y)$,其概率密度是 $f(x, y)$。假设 $y \in [a, b]$,依条件概率的定义,有
\[ \bbP\{X \leqslant c \mid a \leqslant Y \leqslant b\} = \frac{\bbP\{X \leqslant c , a \leqslant Y \leqslant b\}}{\bbP\{a \leqslant y \leqslant b \}} = \frac{\int_{-\infty}^c \int_{a}^b f(x, y) \d y \d x }{\int_a^b f_Y(y) \d y }\]
对 $c$ 求导,即得条件密度函数
\[ f_{X \mid Y}(x \mid a \leqslant y \leqslant b) = \frac{\int_{a}^b f(x, y) \d y}{\int_{a}^b f_{Y}(y) \d t} \]
考虑极限,$a, b$ 收敛于 $y$ 处,有
\[ f_{X \mid Y}(x \mid y) = \frac{f(x, y)}{f_{Y}(y)} \]

\paragraph{独立性}

与一维情况类似,我们可以推广为 $n$ 维。

\begin{definition}
	设 $n$ 维随机向量 $(\seq{x}{n})$ 的联合密度函数为 $f(\seq{x}{n})$,而 $f_i$ 是 $X_i$ 的边缘密度函数,若
	\[ f(\seq{x}{n}) = f_1(x_1) \cdots f_n(x_n) \]
	则称随机变量 $\seq{x}{n}$ 相互独立。
\end{definition}

\paragraph{随机变量函数的概率分布}

离散的情况是容易的,实在不行手算。

设连续型随机变量 $X$ 有密度函数 $f(x)$,对于函数 $g$,考虑 $Y=g(X)$ 下的情况。假设 $g$ 严格上升,设 $h = g^{-1}$,有
\[ \bbP\{Y \leqslant y\} = \bbP\{X \leqslant h(y)\} = \int_{-\infty}^{h(y)} f(t) \d t \]
则 $y$ 的密度函数为
\[ l(y) = f(h(y)) h'(y) \]
同理,对严格递减也可以进行讨论,得到结论
\[ l(y) = f(h(y)) |h'(y)| \]

例如 $Y = aX+b$,则 $Y$ 的密度函数为
\[ l(y) = \frac{1}{|a|}f\left(\frac{y-b}{a}\right) \]

现考虑二元情况 $(X_1, X_2)$ 的密度函数为 $f(x_1, x_2)$,并有变量
\[ Y_1 = g_1(X_1, X_2), \quad Y_2 = g_2(X_1, X_2) \]
假定存在逆变换
\[ X_1 = h_1(Y_1, Y_2), \quad X_2 = h_2(Y_1, Y_2) \]
此时 Jacobi 行列式
\[ J(y_1, y_2) = \left|\begin{matrix}
		\partial h_1 / \partial y_1 & \partial h_1 / \partial y_2 \\
		\partial h_2 / \partial y_1 & \partial h_2 / \partial y_2
	\end{matrix}\right| \]
不为 $0$。在 $(Y_1, Y_2)$ 的平面上任取一个区域 $A$,对应于 $(X_1, X_2)$ 上的区域是 $B$。即
\[ \begin{aligned}
		\bbP\{(Y_1, Y_2) \in A\} & = \bbP\{(X_1, X_2) \in B\}                                                      \\
		                      & = \iint_{B} f(x_1, x_2) \d x_1 \d x_2                                        \\
		                      & = \iint_{A} f(h_1(y_1, y_2), h_2(y_1, y_2)) \cdot |J(y_1, y_2) \d y_1 \d y_2
	\end{aligned} \]
因此 $(Y_1, Y_2)$ 的概率密度为
\[ l(y_1, y_2) = f(h_1(y_1, y_2), h_2(y_1, y_2)) | J(y_1, y_2)| \]

比较常见的例子是线性变换
\[ Y_1 = a_{11} X_1 + a_{12} X_2, \quad Y_2 = a_{21} X_1 + a_{22} X_2 \]
当其行列式不为 $0$ 时,设其存在逆变换
\[ X_1 = b_{11} Y_1 + b_{12} Y_2, \quad X_2 = b_{21} Y_1 + b_{22} Y_2 \]
可得
\[ l(y_1, y_2) = f(b_{11} y_1 + b_{12} y_2, b_{21} y_1 + b_{22} y_2) | b_{11} b_{22} - b_{12} b_{21} | \]

\section{随机变量的数字特征}

设 $X$ 是随机变量,其分布列为 $p_i = \bbP\{X = x_i\}$,记
\[ E(X) = \sum_{i=1}^\infty x_i p_i \]
为随机变量 $X$ 的数学期望。若 $X$ 是连续型随机变量,则记
\[ E(X) = \int_{-\infty}^{+\infty} x f(x) \d x \]
为其期望。

其拥有线性性。比如设 $X, Y$ 相互独立,有
\[ E(X Y) = E(X) E(Y), \quad E(X \pm Y) = E(X) \pm E(Y) \]

我们记 $E[(X - E(X))^2]$ 为 $X$ 的方差,有
\[ D(X) = E[(X - E(X))^2] = E(X^2) - (E(X))^2 \]
称 $\sqrt{D(X)}$ 为 $X$ 的标准差,或者均方差,记为 $\sigma(X)$。

若 $X$ 是离散型随机变量,则
\[ D(X) = \sum_{i=1}^{\infty} (x_i - E)^2 p_i \]
连续性随机变量,则
\[ D(X) = \int_{-\infty}^{\infty} (x-E)^2 f(x) \d x \]

\begin{theorem}[切比雪夫不等式]
	如果随机变量 $X$ 的期望 $E(X)$ 和方差 $D(X)$ 存在,则对任意 $\eps > 0$ 有
	\[ \bbP\{|X - E(X)| < \eps\} \geqslant 1 - \frac{D(X)}{\eps^2} \]
\end{theorem}

我们定义 $(X, Y)$ 的协方差为
\[ \operatorname{Cov}(X, Y) = E[(X - E(X)(Y - E(Y)))] = E(XY) - E(X) E(Y) \]

称 $\rho_{XY} = \frac{\operatorname{Cov}(X, Y)}{\sqrt{D(X) D(Y)}}$ 为 $X, Y$ 的相关系数。

\section{大数定律与中心极限定理}

设随机变量 $X$ 与随机变量序列 $\{X_n\}$,如果对任意的 $\eps > 0$ 有
\[ \lim_{n \to \infty} \bbP\{|X_n - X| < \eps \} =1 \]
则称随机变量序列 $\{X_n\}$ 依概率收敛于随机变量 $X$,记为
\[ \lim_{n \to \infty} X_n = X(\bbP), \quad \text{或}\ X_n \stackrel{\bbP}{\longrightarrow} X(n \to \infty) \]

\begin{theorem}[切比雪夫大数定律]
	设 $\{X_n\}$ 是相互独立的随机变量序列,如果方差 $D(X)$ 存在且有一致有上界,则 $\{X_n\}$ 服从大数定律
	\[ \frac{1}{n} \sum_{i=1}^n X_i \stackrel{P}{\longrightarrow} \frac{1}{n} \sum_{i=1}^n E(X_i) \]
\end{theorem}

\begin{theorem}[伯努利大数定律]
	假设 $\mu_n$ 是 $n$ 重伯努利实验中时间 $A$ 发生的次数,在每次实验中 $A$ 发生的概率为 $p(0 < p < 1)$,则
	\[ \frac{\mu_n}{n} \stackrel{P}{\longrightarrow} p \]
\end{theorem}

\begin{theorem}[辛钦大数定律]
	设 $\{X_n\}$ 是独立同分布的随机变量序列,如果 $E(X_i) = \mu$ 存在,则
	\[ \frac{1}{n} \sum_{i=1}^n X_i \stackrel{P}{\longrightarrow} \mu \]
\end{theorem}

\begin{theorem}[列维 - 林德伯格定理]
	假设 $\{X_n\}$ 是独立同分布的随机变量序列,如果
	\[ E(X_i) = \mu, D(X_i) = \sigma^2 > 0 \]
	存在,则对任意的实数 $x$ 有
	\[ \lim_{n \to \infty} \bbP\left\{ \frac{\sum_{i=1}^n X_i - n \mu}{\sigma \sqrt{n}} \leqslant x \right\} = \frac{1}{\sqrt{2 \pi}} \int_{-\infty}^x \exp\left(-\frac{t^2}{2}\right) \d t = \Phi(x)  \]
\end{theorem}

\begin{theorem}[棣莫弗 - 拉普拉斯定理]
	设随机变量 $Y_n \sim B(n, p)$,其中 $0 < p < 1$ 且 $n > 1$,则对任意的实数 $x$,有
	\[ \lim_{n \to \infty} \bbP\left\{ \frac{Y_n - np}{\sqrt{np(1 - p)}} \leqslant x \right\} = \frac{1}{\sqrt{2 \pi}} \int_{-\infty}^x \exp\left(-\frac{t^2}{2}\right) \d t = \Phi(x)  \]
\end{theorem}

\section{数理统计}

研究对象的全体称为总体,组成总体的每一个元素称为个体。我们把总体和 $X$ 等同起来,所谓总体的分布就是指 $X$ 的分布。

$n$ 个相互独立且与总体 $X$ 具有相同概率分布的随机变量 $\seq{x}{n}$ 所组成的总体为 $(\seq{x}{n})$ 称为来自总体 $X$ 容量为 $n$ 的一个简单随机样本,简称样本。一次抽样结果的 $n$ 个具体数值称为 $\seq{x}{n}$ 的一个观测值(样本值)。

假设总体 $X$ 的分布函数为 $F$,则 $(\seq{x}{n})$ 的分布函数为
\[ F(\seq{x}{n}) = \prod_{i=1}^n F(x_i) \]

设 $\seq{x}{n}$ 为来自总体 $X$ 的一个样本,$g$ 为仅与 $x$ 有关的 $n$ 元函数,则称 $g$ 为样本的一个统计量。若 $(\seq{x}{n})$ 为样本值,则 $g(\seq{x}{n})$ 为观测值。

样本均值
\[ \overline{X} = \frac{1}{n} \sum_{i=1}^n X_i \]
样本方差
\[ S^2 = \frac{1}{n-1} \sum_{i=1}^n (X_i - \overline{X})^2 \]
样本 $k$ 阶(原点)矩
\[ A_k = \frac{1}{n} \sum_{i=1}^n X_i^k \]
样本 $k$ 阶中心矩
\[ B_k = \frac{1}{n} \sum_{i=1}^n (X_i - \overline{X})^k \]

将 $n$ 个观测量从小到大的顺序排列,记随机变量 $X_{(k)}$ 为第 $k$ 顺序统计量。

常用统计量:

\[
	\begin{aligned}
		E(X_i)          & = E(X)             \\
		D(X_i)          & = D(X)             \\
		E(\overline{X}) & = E(X)             \\
		D(\overline{X}) & = \frac{1}{n} D(X) \\
		E(S^2)          & = D(X)
	\end{aligned}
\]

\subsection{三大分布}

\newcommand{\calX}{\mathcal{X}}

\paragraph{$\calX^2$ 分布}

若随机变量 $\seq{x}{n}$ 相互独立且都服从标准正态分布,则随机变量 $X = \sum X_i^2$ 服从自由度为 $n$ 的 $\calX^2$ 分布,记为 $X \sim \calX^2(n)$。

对于给定的 $\alpha(0 < \alpha < 1)$,称满足
\[ \bbP\left\{\calX^2 > \calX_\alpha^2(n)\right\} = \int_{\calX_\alpha^2(n)}^n f(x) \d x = \alpha  \]
的 $\calX_\alpha^2(n)$ 为 $\calX^2(n)$ 分布的上 $\alpha$ 分位点。

\paragraph{$t$ 分布}

设随机变量 $X \sim N(0, 1), Y \sim \calX^2(n)$,$X$ 与 $Y$ 互相独立,则随机变量 $t = \frac{X}{\sqrt{Y / n}}$ 服从自由度为 $n$ 的 $t$ 分布,记为 $t \sim t(n)$.

\paragraph{$F$ 分布}

设随机变量 $X \sim \calX^2(n_1), y \sim \calX^2(n_2)$,且 $X$ 与 $Y$ 相互独立,则 $F = \frac{X / n_1}{Y / n_2}$ 服从自由度为 $(n_1, n_2)$ 的 $F$ 分布,记为 $F \sim F(n_1, n_2)$。

\subsection{参数的点估计}

设总体 $X$ 的分布函数为 $F(x; \theta)$,其中 $\theta$ 是一个未知参数,$\seq{x}{n}$ 是取自总体 $X$ 的一个样本。由样本构造一个适当的统计量 $\hat{\theta}(\seq{x}{n})$ 作为参数 $\theta$ 的估计,则称 $\hat{\theta}$ 为其估计量。

如果 $\seq{x}{n}$ 是样本的一个观察值,将其带入估计量得值 $\hat{\theta}(\seq{x}{n})$ 并以此值作为未知参数的近似值,则称为 $\theta$ 的估计值。

\paragraph{矩估计法}

设总体 $X$ 分布有 $n$ 个样本,有 $k$ 个未知参数。若 $X$ 的原点矩存在,我们令样本矩等于总体矩
\[ \frac{1}{n} \sum_{i=1}^{N} X_i^l = E(X^l), \quad l = 1, \cdots, k \]
这是包含 $k$ 个参数的 $k$ 个方程,由此解得矩估计量和矩估计值。

一般约定:用矩法方程求总体未知参数的估计量时,从低阶开始。

\paragraph{最大似然估计法}

最大似然原理:对未知参数 $\theta$ 进行估计时,在该参数可能的取值范围 $I$ 内选取,用使”样本获得观测值 $\seq{x}{n}$ 的概率最大的参数值 $\hat{\theta}$ 作为 $\theta$ 的估计。

假设 $X$ 是离散型随机变量,其概率分布为 $\bbP\{X = x\} = p(x; \theta)$,那么求其取值概率
\[ L(\seq{x}{n};\theta) = \bbP\{X_1 = x_1, \cdots, X_n = x_n\} = \prod_{i=1}^n \bbP\{X_i=x_i\} = \prod_{i=1}^n p(x_i; \theta) \]
称为样本的似然函数。若存在 $\hat{theta}$ 使得 $L$ 取到最大值,则称 $\hat{theta}$ 为最大似然估计值,对应的统计量是 $\theta$ 的最大似然估计量。

同理,连续型随机变量也有
\[ L(\seq{x}{n}; \theta) = \prod_{i=1}^n f(x_i; \theta) \]



