\chapter{数列极限}

\section{数列极限的概念}

\begin{definition}[数列极限的 $\eps - N$ 定义]
	设 $\{a_n\}$ 为数列,$A$ 为定数。若对任给的正数 $\eps$,总存在正整数 $N=N(\eps)$,使得当 $n>N$ 时有
	\[|a_n - A| < \eps\]
	则称数列 $\{a_n\}$ 收敛于 $A$,或称 $A$ 为数列 $\{a_n\}$ 的极限,记作
	\[\lim\limits_{n\to \infty} a_n = A \text{,或}\ a_n \to a(n \to \infty)\]
\end{definition}

若数列 $\{a_i\}$ 存在 $A \in \mathbb{R}$ 使得 $a_n \to A$ 成立,则称为收敛的。反之称为发散的,逻辑展开即:对任意 $A$ 都有 $a_n$ 不收敛至 $A$。

在使用 $\eps - N$ 语言时,$N(\eps)$ 的选取是非常有技巧的,需要多加练习才能感悟到。

特殊地,若 $\lim\limits_{n\to \infty} a_n = 0$,则称 $\{a_n\}$ 为无穷小数列。

\begin{definition}[无穷大数列]
	若数列 $\{a_n\}$ 满足:对任意正数 $M>0$,存在正整数 $N$,使得当 $n>N$ 时,

	(1) $a_n>M$,则称数列 $\{a_n\}$ 发散于正无穷大,记作 $\lim\limits_{n\to \infty} a_n = +\infty$,或 $a_n \to +\infty$。

	(2) 有 $a_n<M$,则称数列 $\{a_n\}$ 发散于负无穷大,记作 $\lim\limits_{n\to \infty} a_n = -\infty$,或 $a_n \to -\infty$。

	两者合称无穷大数列。
\end{definition}

\begin{example}
	证明数列 $a_n = \sin n$ 发散。
\end{example}

\begin{proof}
	不妨假设其极限为 $A$,任取 $\eps$ 存在 $N$ 使得当 $n > N$ 时有 $|\sin n - A| < \eps$。注意到
	\[ |2 \sin 1 \cos n| = |\sin(n+1) - \sin(n-1)| < 2 \eps \]
	可以得到 $\cos n \to 0$,又
	\[ |\sin 2n| = 2 |\sin n \cos n| < 2 |\cos n| < \frac{2\eps}{\sin 1} \]
	从而 $\sin n \to 0$。显然有矛盾
	\[ |\sin^2 2n + \cos^2 2n| < \frac{5 \eps^2}{\sin^2 1} < 1 \]
	故不存在极限,即发散。
\end{proof}

\section{收敛数列的性质}

\begin{theorem}[唯一性]
	若数列 $\{a_n\}$ 收敛,则它只有一个极限。
\end{theorem}

\begin{proof}
	如果数列 $\{a_n\}$ 同时以 $A,B$ 为极限,即任给 $\eps>0$,总存在 $N_1,N_2$,使得
	\[|a_n-A|<\eps,n>N_1;\quad |a_n-B|<\eps,n>N_2\]
	那么当 $n>\max\{N_1,N_2\}$ 时需要恒成立
	\[2\eps > |a_n-A|+|a_n-B| \geqslant |A-B|\]
	当 $A\ne B$ 时,对于 $2\eps <|A-B|$ 不恒成立,因此只能 $A=B$。
\end{proof}

\begin{theorem}[有界性]
	若数列 $\{a_n\}$ 收敛,则 $\{a_n\}$ 有界。
\end{theorem}

\begin{proof}
	不妨设 $\lim\limits_{n\to \infty} a_n = A$。令 $\eps = 1$,那么存在 $n>N$ 使得
	\[|a_n-A| \leqslant 1\]
	令
	\[M = \{|a_1|,\cdots,|a_N|,|A-1|,|A+1|\}\]
	那么对任意正整数 $n$,总有 $|a_n|\leqslant M$。
\end{proof}

\begin{theorem}[保序性]
	设 $\lim\limits_{n\to \infty} a_n = A,\lim\limits_{n\to \infty} b_n = B$,则有

	(1) 如果存在 $N$ 使得当 $n>N$ 时有 $a_n\geqslant b_n$ 恒成立,则 $A\geqslant B$。

	(2) 反之,如果 $A>B$,则存在 $N$ 使得当 $n>N_1$ 时 $a_n>b_n$ 恒成立。
\end{theorem}
\begin{proof}
	(1) 如果设 $B-A=2\delta>0$,那么存在 $N_2,N_3>N$
	\[|a_n-A|<\delta,n>N_2;\qquad |b_n-B|<\delta,n>N_3\]
	于是当 $n>\max\{N_2,N_3\}$ 时有
	\[a_n<A+\delta=B-\delta<b_n\]
	因此矛盾,故 $A\geqslant B$。

	(2) 设 $A-B=2\delta>0$,那么存在 $N_2,N_3$
	\[|a_n-A|<\delta,n>N_2;\qquad |b_n-B|<\delta,n>N_3\]
	于是存在 $N_1=\max\{N_2,N_3\}$,当 $n>N_1$ 时有
	\[a_n>A-\delta=B+\delta>b_n\]
\end{proof}

若 $b_n$ 是常数列,$A\ne 0$,我们还可得到推论:存在 $N$,使得当 $n>N$ 时,有
\[ \frac{1}{2}|A| < |a_n| < \frac{3}{2}|A| \]

\begin{theorem}[迫敛性,夹逼定理]
	设数列 $\{a_n\},\{b_n\},\{c_n\}$ 满足当 $n>N_0$ 有 $a_n\leqslant c_n\leqslant b_n$。若
	\[\lim_{n\to \infty}a_n = A = \lim_{n\to \infty}c_n\]
	则 $\lim\limits_{n\to \infty}b_n = A$。
\end{theorem}
\begin{proof}
	即对于任给的 $\eps>0$,存在 $N_1,N_2$,使得当 $n>N_1$ 有
	\[A-\eps<a_n<A+\eps\]
	当 $n>N_2$ 有
	\[A-\eps<c_n<A+\eps\]
	因此当 $n>\max\{N_0,N_1,N_2\}$ 时,有
	\[A-\eps < a_n \leqslant b_n \leqslant c_n < A+\eps\]
\end{proof}

\begin{example}
	如果 $a_1, \cdots, a_k > 0$,那么有
	\[ \lim_{n \to \infty} \sqrt[n]{a_1^n + \cdots + a_k^n} = \max\{a_1, \cdots, a_k\} \]
\end{example}

\begin{proof}
	不妨设 $a_1 = \max\{a_1, \cdots, a_k\}$,那么有
	\[ a_1 < \sqrt[n]{a_1^n + \cdots + a_k^n} < \sqrt[n]{k a_1^n} \to a_1 \]
	由夹逼原理知原式成立。
\end{proof}

\begin{theorem}[四则运算]
	设 $\lim\limits_{n\to \infty} a_n = A,\lim\limits_{n\to \infty} b_n = B$,则有

	(1) $\{\alpha a_n+\beta b_n\}$ 收敛到 $\alpha A+\beta B$,其中 $\alpha,\beta$ 为常数。

	(2) $\{a_nb_n\}$ 收敛到 $AB$。

	(3) 当 $B\ne 0$ 时,$\{a_n/b_n\}$ 收敛到 $A/B$。
\end{theorem}
\begin{proof}
	(1) 任给 $\eps >0$,存在 $N_1,N_2$ 使得
	\[|a_n-A|<\frac{\eps}{2|\alpha|+1},n>N_1;\qquad |b_n-B|<\frac{\eps}{2|\beta|+1},n>N_2\]
	则当 $n>\max\{N_1,N_2\}$ 时有
	\[
		\begin{aligned}
			|(\alpha a_n+\beta b_n)-(\alpha A+\beta B)| & \leqslant |\alpha||a_n-A|+|\beta||b_n-B|                          \\
			                                            & < \frac{\eps|\alpha|}{2|\alpha|+1}+\frac{\eps|\beta|}{2|\beta|+1} \\
			                                            & < \frac{\eps}2+\frac{\eps}2=\eps
		\end{aligned}
	\]

	(2) 由收敛数列的有界性,存在 $M$ 使得 $|a_n|\leqslant M$,那么
	\[0\leqslant |a_nb_n-AB|=|(a_n-A)b_n+A(b_n-B)|\leqslant M|a_n-A|+|A||b_n-B|\]
	由迫敛性知 $\lim\limits_{n\to \infty}|a_nb_n-AB| =0$。

	(3) 由保号性的推论,存在 $N$ 使得当 $n>N$ 时有 $|b_n|>\dfrac{|B|}{2}$,那么
	\[0 \leqslant \left|\frac{1}{b_n}-\frac{1}{B}\right| = \frac{|b_n-B|}{|b_n||B|} \leqslant \frac{2}{|B|^2}{|b_n-B|}\]
	由迫敛性知 $\lim\limits_{n\to \infty}\left|\frac{1}{b_n}-\frac{1}{B}\right| =0$。
\end{proof}

\subsection{Stolz 定理}

Stolz 定理主要是用来处理 $\infty/\infty$ 型和 $0/0$ 型极限,可以认为是洛必达的替代。

\begin{theorem}
	对于任意的 $1 \leqslant k \leqslant n$,设 $b_k>0$ 且 $m \leqslant \dfrac{a_k}{b_k} \leqslant M$,则有
	\[m \leqslant \frac{\sum a_n}{\sum b_n} \leqslant M\]
\end{theorem}

\begin{theorem}[Stolz 定理一]
	设数列 $\{x_n\},\{y_n\}$,且 $\{y_n\}$ 严格单调地趋于 $+\infty$,如果
	\[\lim_{n\to \infty}\frac{x_n-x_{n-1}}{y_n-y_{n-1}}=A\]
	则
	\[\lim_{n\to \infty} \frac{x_n}{y_n} = A\]
\end{theorem}

\begin{proof}
	分类讨论 Todo……
\end{proof}

\begin{theorem}[Stolz 定理二]
	设数列 $\{y_n\}$ 严格单调地趋于 $0$,且数列 $\{x_n\}$ 也收敛到 $0$,那么如果
	\[\lim_{n\to \infty}\frac{x_n-x_{n-1}}{y_n-y_{n-1}}=A\]
	则
	\[\lim_{n\to \infty} \frac{x_n}{y_n} = A\]
\end{theorem}

\begin{proof}
	分类讨论 Todo……
\end{proof}


\section{数列收敛的判别法则}

我们需要一些更方便的判别法则。

\subsection{单调数列}

若数列 $\{a_n\}$ 各项满足关系式 $a_n \leqslant a_{n+1}(a_n \geqslant a_{n+1})$,则称 $\{a_n\}$ 为递增(递减)数列,统称为单调数列。

\begin{theorem}[单调有界定理]
	单调有界数列必有极限。
\end{theorem}

\begin{proof}
	不妨设 $\{a_i\}$ 为有上界的单调递增序列,由确界原理知存在上确界 $\beta$。按其定义,任给 $\eps > 0$ 都存在 $N$ 使得 $\beta - \eps < a_N < \beta$(否则 $\beta - \eps$ 就是新的上确界)。故 $|a_n - \beta| < \eps$,即收敛至 $\beta$。
\end{proof}

\subsection{子列}

设 $\{a_n\}$ 为数列,如果 $\{n_k\}$ 是一列严格递增的正整数,则数列 $\{a_{n_k}\}$ 称为数列 $\{a_n\}$ 的一个子列。子列显然有性质 $n_k \geqslant k$,归纳易证。

特殊的子列 $\{a_{2k}\}$ 和 $\{a_{2k-1}\}$ 分别称为偶子列与奇子列。数列本身也是其自己的子列。

\begin{theorem}[Weierstrass 致密性定理]
	任何有界数列必定有收敛的子列。
\end{theorem}

\begin{proof}
	不妨设数列包含无数个不同的 $a_n$,否则显然成立。假设数列有界,设其值域为 $[A_0, B_0]$。注意到我们可以对分区间为 $[A_0, \frac{A_0 + B_0}{2}]$ 和 $[\frac{A_0 + B_0}{2}, B_0]$,至少其中之一包含无穷多个 $a_n$,记为 $[A_1, B_1]$。我们可以不断划分,得到一闭缩区间套
	\[ [A_0, B_0] \supset [A_1, B_1] \subset \cdots   \]
	总是可以在区间中找到下标递增的项,即我们要求的子列。
\end{proof}

\begin{theorem}
	数列 $\{a_n\}$ 收敛的充要条件:$\{a_n\}$ 的任何子列都收敛。
\end{theorem}

\subsection{Cauchy 准则}

\begin{definition}
	设 $\{a_n\}$ 为数列,如果任给 $\eps>0$,均存在 $N(\eps)$ 使任取 $m,n>N(\eps)$ 有
	\[|a_m-a_n| < \eps\]
	则称 $\{a_n\}$ 为 Cauchy 数列或基本列。
\end{definition}

反之,若存在 $\eps > 0$ 使得任给 $N$ 都存在 $n, m > N$ 使得
\[ |a_n - a_m| \geqslant \eps \]
则称该数列为非 Cauchy 的。

\begin{theorem}
	Cauchy 数列必定是有界数列。
\end{theorem}
\begin{proof}
	取 $\eps=1$,则存在 $N$ 使得当 $m,n>N$ 时有
	\[|a_m-a_n| < 1\]
	令 $M = \max\{|a_k|+1 \mid 1 \leqslant k \leqslant N+1\}$,则当 $n\leqslant N$ 时显然有 $|a_n|\leqslant M$,而当 $n>N$ 时有
	\[|a_n| \leqslant |a_n-a_{N+1}| + |a_{N+1} < 1+ |a_{N+1}| \leqslant M\]
	这说明 $\{a_n\}$ 是有界数列。
\end{proof}

\begin{theorem}[Cauchy 收敛准则]
	$\{a_n\}$ 为 Cauchy 数列当且仅当它是收敛的。
\end{theorem}
\begin{proof}
	(1) 充分性:设 $\{a_n\}$ 收敛到 $A$,则任给 $\eps >0$ 存在 $N$,当 $n>N$ 时有
	\[|a_n-A|\leqslant \frac{\eps}{2}\]
	因此当 $m,n>N$ 时有
	\[|a_m-a_n| \leqslant |a_m-A| + |A-a_n| < \frac{\eps}{2}+\frac{\eps}{2}=\eps\]
	这说明 $a_n$ 为 Cauchy 数列。

	(2) 必要性:已证 Cauchy 列有界,则必存在收敛子列 $\{a_{u_k}\}$,因此任给 $\eps > 0$ 存在 $N_1$ 使得当 $u_i > N_1$ 时有
	\[ | a_{u_i} - A| < \frac{\eps}{2} \]
	又由定义知存在 $N_2$ 使得任取 $n, m > N_2$ 有
	\[ |a_n - a_m| < \frac{\eps}{2} \]
	因此取 $N = \max\{N_1, N_2\}$,取 $u_k > N$,则当 $n > N$ 时有
	\[ |a_n - A| \leqslant |a_n - a_{u_k}| + | a_{u_k} - A| \leqslant \eps \]
\end{proof}

\begin{theorem}[不动点原理]
	设递推数列 $a_{n+1} = f(a_n)$,假设 $a_n \subset (\alpha, \beta)$,若存在常数 $L \in (0, 1)$ 使得对任意 $x,y \in (\alpha, \beta)$ 有
	\[ |f(x) - f(y)| \leqslant L|x-y| \]
	则数列收敛。
\end{theorem}

\begin{proof}
	首先类似于等比
	\[ |a_{n+1} - a_n| \leqslant \cdots \leqslant L^{n-1}|a_2 - a_1| \]
	从而
	\[ |a_{n+k} - a_{n}| \leqslant \sum_{i=1}^{k} |a_{n + i} - a_{n + i - 1}| \leqslant \sum_{i=1}^{k} L^{n+i} |a_{2} - a_{1}| \leqslant \frac{L^{n-1}}{1-L}|a_2 - a_1|  \]
	由 Cauchy 收敛准则知数列收敛。
\end{proof}

\section{常见数列}

首先我们定义三个数列
\[ a_n = \left(1 + \frac{1}{n}\right)^n, \quad b_n = \left(1 + \frac{1}{n}\right)^{n+1}, e_n = \sum_{i=0}^{n}\frac{1}{k!} \]
我们通常定义 $a_n$ 的极限为 $\ee$。下证三者极限存在且相同。

其中 $e_n$ 的单调性是显然的。我们先证:
\[ a_n < a_{n+1},\quad b_n > b_{n+1} \]
首先
\[ \begin{aligned}
	a_n &= \sum_{k=0}^{n} \binom{n}{k} \frac{1}{n^k} \\
	&= 1 + \sum_{k=1}^n \frac{1}{k!} \prod_{j=1}^{k-1} \left(1 - \frac{j}{n}\right) \\
	&< 1 + \sum_{k=1}^n \frac{1}{k!} \prod_{j=1}^{k-1} \left(1 - \frac{j}{n+1}\right) \\
	&< 1 + \sum_{k=1}^{n+1} \frac{1}{k!} \prod_{j=1}^{k-1} \left(1 - \frac{j}{n+1}\right) = a_{n+1} \\
\end{aligned} \]
再借助 Bernoulli 不等式
\[ \begin{aligned}
	\frac{b_{n-1}}{b_n} &= \frac{\left(1 + \frac{1}{n-1}\right)^n}{\left(1 + \frac{1}{n}\right)^{n+1}} \\
	&= \left(1+\frac{1}{n^2 - 1}\right)^n \frac{1}{1 + \frac{1}{n}} \\
	&> \left(1+\frac{n}{n^2 - 1}\right) \frac{1}{1 + \frac{1}{n}} \\
	&= 1 + \frac{1}{(n+1)^2(n-1)} > 1
\end{aligned} \]

注意到当 $n > 2$ 时
\[ a_n \leqslant 1 + \sum_{k=1}^n \frac{1}{k!} = e_n \leqslant 2 + \sum_{k=2}^n \frac{1}{k(k-1)} = 3 - \frac{1}{n} < 3 \]
故 $a_n$ 和 $b_n$ 单调有界,故必有极限。再注意到
\[ b_n = \left(1 + \frac{1}{n}\right) a_n \]
由极限的四则运算,故 $b_n$ 的极限也存在,且 $a_n$ 和 $b_n$ 收敛于同一个值。

我们定义 $a_n$ 的极限为 $\ee$,下证 $e_n \to \ee$。注意到固定 $u$ 有
\[ \begin{aligned}
	a_n &= 1 + \sum_{k=1}^n \frac{1}{k!} \prod_{j=1}^{k-1} \left(1 - \frac{j}{n}\right) \\
	&> 1 + \sum_{k=1}^u \frac{1}{k!} \prod_{j=1}^{k-1} \left(1 - \frac{j}{n}\right)
\end{aligned} \]
那么令 $n \to \infty$,有
\[ e_k = \sum_{k=0}^u \frac{1}{k!}\leqslant a_n \to \ee \]
故由夹逼定理知 $e_n \to \ee$。

