\chapter{数列极限}

\section{数列极限的概念}

\begin{definition}[数列极限的 $\eps - N$ 定义]
	设 $\{a_n\}$ 为数列,$A$ 为定数。若对任给的正数 $\eps$,总存在正整数 $N=N(\eps)$,使得当 $n>N$ 时有
	$$|a_n - A| < \eps$$
	则称数列 $\{a_n\}$ 收敛于 $A$,或称 $A$ 为数列 $\{a_n\}$ 的极限,记作
	$$\displaystyle\lim_{n\to \infty} a_n = A \text{,或}\ a_n \to a(n \to \infty)$$
\end{definition}

等价定义:任给 $\eps > 0$,若在 $U(A;\eps)$ 之外数列 $\{a_n\}$ 中的项至多只有有限个,则称 $\{a_n\}$ 收敛于极限 $A$。

若对于数列 $\{a_n\}$,不存在 $A$ 使得 $a_n\to A$,则称数列 $\{a_n\}$ 发散。

特殊地,若 $\displaystyle\lim_{n\to \infty} a_n = 0$,则称 $\{a_n\}$ 为无穷小数列。

\begin{definition}[无穷大数列]
	若数列 $\{a_n\}$ 满足:对任意正数 $M>0$,存在正整数 $N$,使得当 $n>N$ 时,

	(1) $a_n>M$,则称数列 $\{a_n\}$ 发散于正无穷大,记作 $\displaystyle\lim_{n\to \infty} a_n = +\infty$,或 $a_n \to +\infty$。

	(2) 有 $a_n<M$,则称数列 $\{a_n\}$ 发散于负无穷大,记作 $\displaystyle\lim_{n\to \infty} a_n = -\infty$,或 $a_n \to -\infty$。
\end{definition}

\section{收敛数列的性质}

\begin{theorem}[唯一性]
	若数列 $\{a_n\}$ 收敛,则它只有一个极限。
\end{theorem}

\begin{proof}
	如果数列 $\{a_n\}$ 同时以 $A,B$ 为极限,即任给 $\eps>0$,总存在 $N_1,N_2$,使得
	$$|a_n-A|<\eps,n>N_1;\quad |a_n-B|<\eps,n>N_2$$
	那么当 $n>\max\{N_1,N_2\}$ 时需要恒成立
	$$2\eps > |a_n-A|+|a_n-B| \geqslant |A-B|$$
	当 $A\ne B$ 时,对于 $2\eps <|A-B|$ 不恒成立,因此只能 $A=B$。
\end{proof}

\begin{theorem}[有界性]
	若数列 $\{a_n\}$ 收敛,则 $\{a_n\}$ 有界。
\end{theorem}

\begin{proof}
	不妨设 $\displaystyle\lim_{n\to \infty} a_n = A$。令 $\eps = 1$,那么存在 $n>N$ 使得
	$$|a_n-A| \leqslant 1$$
	令
	$$M = \{|a_1|,\cdots,|a_N|,|A-1|,|A+1|\}$$
	那么对任意正整数 $n$,总有 $|a_n|\leqslant M$。
\end{proof}

\begin{theorem}[保不等式性,保序性]
	设 $\displaystyle\lim_{n\to \infty} a_n = A,\displaystyle\lim_{n\to \infty} b_n = B$,则有

	(1) 如果存在 $n>N$ 使得 $a_n\geqslant b_n$ 恒成立,则 $A\geqslant B$。

	(2) 反之,如果 $A>B$,则存在 $n>N_1$ 使得 $a_n>b_n$ 恒成立。
\end{theorem}
\begin{proof}
	(1) 如果设 $B-A=2\delta>0$,那么存在 $N_2,N_3>N$
	$$|a_n-A|<\delta,n>N_2;\qquad |b_n-B|<\delta,n>N_3$$
	于是当 $n>\max\{N_2,N_3\}$ 时有
	$$a_n<A+\delta=B-\delta<b_n$$
	因此矛盾,故 $A\geqslant B$。

	(2) 设 $A-B=2\delta>0$,那么存在 $N_2,N_3$
	$$|a_n-A|<\delta,n>N_2;\qquad |b_n-B|<\delta,n>N_3$$
	于是存在 $N_1=\max\{N_2,N_3\}$,当 $n>N_1$ 时有
	$$a_n>A-\delta=B+\delta>b_n$$
\end{proof}

若 $b_n$ 是常数列,$A\ne 0$,我们还可得到推论:存在 $N$,使得当 $n>N$ 时,有
$$\frac{1}{2}|A| < |a_n| < \frac{3}{2}|A|$$

\begin{theorem}[迫敛性,夹逼定理]
	设数列 $\{a_n\},\{b_n\},\{c_n\}$ 满足当 $n>N_0$ 有 $a_n\leqslant c_n\leqslant b_n$。若
	$$\lim_{n\to \infty}a_n = A = \lim_{n\to \infty}c_n$$
	则 $\displaystyle\lim_{n\to \infty}b_n = A$。
\end{theorem}
\begin{proof}
	即对于任给的 $\eps>0$,存在 $N_1,N_2$,使得当 $n>N_1$ 有
	$$A-\eps<a_n<A+\eps$$
	当 $n>N_2$ 有
	$$A-\eps<c_n<A+\eps$$
	因此当 $n>\max\{N_0,N_1,N_2\}$ 时,有
	$$A-\eps < a_n \leqslant b_n \leqslant c_n < A+\eps$$
\end{proof}

\begin{theorem}[四则运算]
	设 $\displaystyle\lim_{n\to \infty} a_n = A,\displaystyle\lim_{n\to \infty} b_n = B$,则有

	(1) $\{\alpha a_n+\beta b_n\}$ 收敛到 $\alpha A+\beta B$,其中 $\alpha,\beta$ 为常数。

	(2) $\{a_nb_n\}$ 收敛到 $AB$。

	(3) 当 $B\ne 0$ 时,$\{a_n/b_n\}$ 收敛到 $A/B$。
\end{theorem}
\begin{proof}
	(1) 任给 $\eps >0$,存在 $N_1,N_2$ 使得
	$$|a_n-A|<\frac{\eps}{2|\alpha|+1},n>N_1;\qquad |b_n-B|<\frac{\eps}{2|\beta|+1},n>N_2$$
	则当 $n>\max\{N_1,N_2\}$ 时有
	\begin{equation*}
		\begin{aligned}
			|(\alpha a_n+\beta b_n)-(\alpha A+\beta B)| & \leqslant |\alpha||a_n-A|+|\beta||b_n-B|                          \\
			                                            & < \frac{\eps|\alpha|}{2|\alpha|+1}+\frac{\eps|\beta|}{2|\beta|+1} \\
			                                            & < \frac{\eps}2+\frac{\eps}2=\eps
		\end{aligned}
	\end{equation*}

	(2) 由收敛数列的有界性,存在 $M$ 使得 $|a_n|\leqslant M$,那么
	$$0\leqslant |a_nb_n-AB|=|(a_n-A)b_n+A(b_n-B)|\leqslant M|a_n-A|+|A||b_n-B|$$
	由迫敛性知 $\displaystyle\lim_{n\to \infty}|a_nb_n-AB| =0$。

	(3) 由保号性的推论,存在 $N$ 使得当 $n>N$ 时有 $|b_n|>\dfrac{|B|}{2}$,那么
	$$0 \leqslant \left|\frac{1}{b_n}-\frac{1}{B}\right| = \frac{|b_n-B|}{|b_n||B|} \leqslant \frac{2}{|B|^2}{|b_n-B|}$$
	由迫敛性知 $\displaystyle\lim_{n\to \infty}\left|\frac{1}{b_n}-\frac{1}{B}\right| =0$。
\end{proof}

\begin{definition}[数列的子列]
	设 $\{a_n\}$ 为数列,如果 $\{n_k\}$ 是一列严格递增的正整数,则数列 $\{a_{n_k}\}$ 称为数列 $\{a_n\}$ 的一个子列。
\end{definition}

特殊的子列 $\{a_{2k},a_{2k-1}\}$ 分别称为偶子列与奇子列。

\begin{theorem}
	数列 $\{a_n\}$ 收敛的充要条件:$\{a_n\}$ 的任何子列都收敛。
\end{theorem}

\section{数列极限存在的条件}

若数列 $\{a_n\}$ 各项满足关系式 $a_n \leqslant a_{n+1}(a_n \geqslant a_{n+1})$,则称 $\{a_n\}$ 为递增(递减)数列,统称为单调数列。

\begin{theorem}[单调有界定理]
	单调有界数列必有极限。
\end{theorem}

\begin{proof}
	不妨设 $\{a_i\}$ 为有上界的单调递增序列,
\end{proof}

\begin{theorem}[致密性定理]
	任何有界数列必定有收敛的子列。
\end{theorem}

\section{Cauchy 准则}

\begin{definition}
	设 $\{a_n\}$ 为数列,如果任给 $\eps>0$,均存在 $N(\eps)$ 使当 $m,n>N(\eps)$ 时有
	$$|a_m-a_n| < \eps$$
	则称 $\{a_n\}$ 为 Cauchy 数列或基本列。
\end{definition}

\begin{theorem}
	Cauchy 数列必定时有界数列。
\end{theorem}
\begin{proof}
	取 $\eps=1$,则存在 $N$ 使得当 $m,n>N$ 时有
	$$|a_m-a_n| < 1$$
	令 $M = \max\{|a_k|+1 \mid 1 \leqslant k \leqslant N+1\}$,则当 $n\leqslant N$ 时显然有 $|a_n|\leqslant M$,而当 $n>N$ 时有
	$$|a_n| \leqslant |a_n-a_{N+1}| + |a_{N+1} < 1+ |a_{N+1}| \leqslant M$$
	这说明 $\{a_n\}$ 是有界数列。
\end{proof}

\begin{theorem}[Cauchy 收敛准则]
	$\{a_n\}$ 为 Cauchy 数列当且仅当它是收敛的。
\end{theorem}
\begin{proof}
	(1) 充分性:设 $\{a_n\}$ 收敛到 $A$,则任给 $\eps >0$ 存在 $N$,当 $n>N$ 时有
	$$|a_n-A|\leqslant \frac{\eps}{2}$$
	因此当 $m,n>N$ 时有
	$$|a_m-a_n| \leqslant |a_m-A| + |A-a_n| < \frac{\eps}{2}+\frac{\eps}{2}=\eps$$
	这说明 $a_n$ 为 Cauchy 数列。

	(2) 必要性:Todo ……
\end{proof}

Cauchy 收敛准则的条件称为 Cauchy 条件。

\section{Stolz 公式}

\begin{theorem}
	对于任意的 $1 \leqslant k \leqslant n$,设 $b_k>0$ 且 $m \leqslant \dfrac{a_k}{b_k} \leqslant M$,则有
	$$m \leqslant \frac{\sum a_n}{\sum b_n} \leqslant M$$
\end{theorem}

\begin{theorem}[Stolz 公式一]
	设数列 $\{x_n\},\{y_n\}$,且 $\{y_n\}$ 严格单调地趋于 $+\infty$,如果
	$$\lim_{n\to \infty}\frac{x_n-x_{n-1}}{y_n-y_{n-1}}=A$$
	则
	$$\lim_{n\to \infty} \frac{x_n}{y_n} = A$$
\end{theorem}

\begin{proof}
	分类讨论 Todo……
\end{proof}

\begin{theorem}[Stolz 公式二]
	设数列 $\{y_n\}$ 严格单调地趋于 $0$,且数列 $\{x_n\}$ 也收敛到 $0$,那么如果
	$$\lim_{n\to \infty}\frac{x_n-x_{n-1}}{y_n-y_{n-1}}=A$$
	则
	$$\lim_{n\to \infty} \frac{x_n}{y_n} = A$$
\end{theorem}

\begin{proof}
	分类讨论 Todo……
\end{proof}
