%% \newcommand{\mfrac}[2]{\frac{#1}{#2}}

\chapter{函数极限}

\section{函数极限的概念}

\begin{definition}
    设 $f$ 为定义在 $[a,+\infty)$ 上的函数,$A$ 为定数。若对任给的 $\eps>0$,存在正数 $M=M(\eps) \geqslant a$,使得当 $x>M$ 时,有
    $$|f(x)-A| < \eps$$
    则称函数 $f$ 当 $x$ 趋于 $+\infty$ 时以 $A$ 为极限,记作
    $$\lim_{x \to +\infty}f(x) = A\ \text{或}\ f(x) \to A(x \to +\infty)$$
\end{definition}

类似的有 $\displaystyle\lim_{x \to -\infty}f(x)$ 和 $\displaystyle\lim_{x \to \infty}f(x)$。

不难证明
$$\lim_{x \to \infty}f(x) = A \Leftrightarrow \lim_{x \to -\infty}f(x)=\lim_{x \to +\infty}f(x)=A$$

\begin{definition}
    设函数 $f$ 在 $U^\circ(x_0;\delta')$ 内有定义,$A$ 为定数。若对任给的 $\eps>0$,存在正数 $\delta<\delta'$,使得当 $0<|x-x_0|<\delta$ 时,有 $|f(x)-A|<\eps$,则称函数 $f$ 当 $x$ 趋于 $x_0$ 时以 $A$ 为极限,记作
    $$\lim_{x \to x_0}f(x) = A\ \text{或}\ f(x)\to A(x \to x_0)$$
\end{definition}

\begin{definition}
    设函数 $f$ 在 $U_+^\circ(x_0;\delta')$ 内有定义,$A$ 为定数。若对任给的 $\eps>0$,存在正数 $\delta<\delta'$,使得当 $x_0<x<x_0+\delta$ 时,有 $|f(x)-A|<\eps$,则称函数 $f$ 当 $x$ 趋于 $x_0^+$ 时以 $A$ 为极限,记作
    $$\lim_{x \to x_0^+}f(x) = A\ \text{或}\ f(x)\to A(x \to x_0^+)$$
\end{definition}

类似的还有左极限 $\displaystyle\lim_{x \to x_0^-}f(x)$,统称为单侧极限。又可记为
$$f(x_0+0) = \lim_{x \to x_0^+}f(x)\ \text{与}\ f(x_0-0) = \lim_{x \to x_0^-}f(x)$$

同理还有
$$\lim_{x \to x_0}f(x) = A \Leftrightarrow \lim_{x \to x_0^+}f(x)=\lim_{x \to x_0^-}f(x)=A$$

\section{函数极限的性质}

\begin{theorem}[唯一性]
    若极限 $\displaystyle\lim_{x \to x_0}f(x)$ 存在,则此极限是唯一的。
\end{theorem}

\begin{theorem}[局部有界性]
    若极限 $\displaystyle\lim_{x \to x_0}f(x)$ 存在,则 $f$ 在 $x_0$ 的某空心邻域 $U^\circ(x_0)$ 上有界。
\end{theorem}

\begin{theorem}[保不等式性]
    设 $\lim_{x \to x_0}f(x)$ 与 $\lim_{x \to x_0}g(x)$ 均存在。若存在正数 $N_0$,使得当 $n>N_0$ 时,有 $a_n\leqslant b_n$,则 $\displaystyle\lim_{n\to \infty}a_n \leqslant \lim_{n\to \infty}b_n$。
\end{theorem}

\begin{theorem}[迫敛性]
    设 $\displaystyle\lim_{x \to x_0}f(x) = \lim_{x \to x_0}g(x) = A$,且在某 $U^\circ(x_0;\delta')$ 上有
    $$f(x)\leqslant h(x) \leqslant g(x)$$
    则 $\lim_{x \to x_0}h(x) = A$。
\end{theorem}

\begin{theorem}[四则运算法则]
    若 $\displaystyle\lim_{x \to x_0}f(x)$ 与 $\displaystyle\lim_{x \to x_0}g(x)$ 均存在,则
    $$\lim_{x \to x_0}[f(x)\pm g(x)] = \lim_{x \to x_0}f(x) + \lim_{x \to x_0}g(x)$$
    $$\lim_{x \to x_0}[f(x)g(x)] = \lim_{x \to x_0}f(x) \cdot \lim_{x \to x_0}g(x)$$
    若 $\displaystyle\lim_{x \to x_0}g(x)\ne 0$,则
    $$\lim_{x \to x_0}\frac{f(x)}{g(x)} = \frac{\lim_{x \to x_0}f(x)}{\lim_{x \to x_0}g(x)}$$
\end{theorem}

\section{函数极限存在的条件}

\begin{theorem}[海涅(Heine)定理,归结原则]
    若 $f(x)$ 在 $U^\circ(x_0;\delta')$ 上有定义。$\displaystyle\lim_{x \to x_0}f(x)$ 存在的充要条件是:任何含于 $U^\circ(x_0;\delta')$ 且以 $x_0$ 为极限的数列 $\{x_n\}$,极限 $\displaystyle\lim_{x \to x_0}f(x_n)$ 都存在且相等。
\end{theorem}

即若对任何 $x_n\to x_0(n\to \infty)$ 有 $\displaystyle\lim_{n\to \infty}f(x_n) = A$,则 $\displaystyle\lim_{x \to x_0}f(x)=A$。

\begin{theorem}
    设 $f(x)$ 在点 $x_0$ 的某空心右邻域 $U_+^\circ(x_0)$ 有定义,则 $\displaystyle\lim_{x \to x_0^+}f(x)=A$ 的充要条件是:对任何以 $x_0$ 为极限的递减数列 $\{x_n\}\subset U_+^\circ(x_0)$,有 $\displaystyle\lim_{n\to \infty}f(x_n) = A$。
\end{theorem}

\begin{theorem}
    设 $f(x)$ 为定义在 $U_+^\circ(x_0)$ 上的单调有界函数,则右极限 $\displaystyle\lim_{x \to x_0^+}f(x)=A$ 存在。
\end{theorem}

\begin{theorem}[Cauchy 准则]
    设 $f(x)$ 在 $U^\circ(x_0;\delta')$ 上有定义,则 $\displaystyle\lim_{x \to x_0}f(x)$ 存在的充要条件是:任给 $\eps > 0$,存在正数 $\delta(<\delta')$,使得对任何 $x',x''\in U^\circ(x_0,\delta)$,有 $|f(x')-f(x'')|<\eps$。
\end{theorem}

\section{两个重要的极限}

$$\lim_{x \to 0}\frac{\sin x}{x} = 1$$
$$\lim_{x \to \infty}\left(1+\frac{1}{x}\right)^x = \ee$$

\section{无穷小量与无穷大量}

\begin{definition}[无穷小量]
    设函数 $f$ 在某 $U^\circ(x_0)$ 上有定义,若 $\displaystyle\lim_{x \to x_0}f(x)=0$,则称 $f$ 为当 $x \to x_0$ 时的无穷小量。
\end{definition}

\begin{definition}[有界量]
    设函数 $f$ 在某 $U^\circ(x_0)$ 上有界,则称 $f$ 为当 $x \to x_0$ 时的有界量。
\end{definition}

无穷小量收敛于 $0$ 的速度有快有慢。设当 $x \to x_0$ 时,$f$ 与 $g$ 均为无穷小量。

若 $\displaystyle\lim_{x \to x_0}\mfrac{f(x)}{g(x)} = 0$,则称当 $x \to x_0$ 时 $f$ 为 $g$ 的高阶无穷小量,或称 $g$ 为 $f$ 的低阶无穷小量。

记作
$$f(x)=o(g(x))(x \to x_0)$$
特别地,$f$ 为当 $x \to x_0$ 时的无穷小量记作
$$f(x)=o(1)(x \to x_0)$$

若存在正数 $K$ 和 $L$,使得在某 $U^\circ(x_0)$ 上有
$$K\leqslant \left|\frac{f(x)}{g(x)}\right| \leqslant L$$
则称 $f$ 与 $g$ 为当 $x \to x_0$ 时的同阶无穷小量。特别当
$$\lim_{x \to x_0}\frac{f(x)}{g(x)} = c \ne 0$$
时,$f$ 与 $g$ 必为同阶无穷小量。

若 $\displaystyle\lim_{x \to x_0}\mfrac{f(x)}{g(x)} = 1$ 则称 $f$ 与 $g$ 是当 $x \to x_0$ 时的等价无穷小量。记作
$$f(x) \sim g(x) (x \to x_0)$$

注意并不是任何两个无穷小量都可以进行这种阶的比较。例如 $x \to 0$ 时,$x\sin\dfrac{1}{x}$ 和 $x^2$ 都是无穷小量,但它们的比都不是有界量。

\begin{theorem}
    设函数 $f,g,h$ 在 $U^\circ(x_0)$ 上有定义,且有 $f(x) \sim g(x)(x \to x_0)$,则

    1.若 $\displaystyle\lim_{x \to x_0}f(x)h(x) = A$,则 $\displaystyle\lim_{x \to x_0}g(x)h(x) = A$。

    2.若 $\displaystyle\lim_{x \to x_0}\frac{h(x)}{f(x)}=B$,则 $\displaystyle\lim_{x \to x_0}\frac{h(x)}{g(x)}=B$
\end{theorem}

\begin{definition}[无穷大量]
    设函数 $f$ 在某 $U^\circ(x_0)$ 上有定义,若对任给的 $G>0$,存在 $\delta>0$,使得当 $x\in U^\circ(x_0;\delta)\subset U^\circ(x_0)$ 时,有 $|f(x)|>G$,则称函数 $f$ 当 $x \to x_0$ 时有非正常极限 $\infty$,记作 $\displaystyle\lim_{x \to x_0}f(x) = \infty$。
\end{definition}

\section{常见等价无穷小}

实际上这些等价无穷小就是 Talor 展开。

% \begin{equation*}
%     \begin{aligned}
%         \frac{1}{1-x} &= \sum_{k=0}^\infty x^n,(-1,1) \\
%         &= 1 + x + x^2 + x^3 + x^4 + x^5 + x^6 + O(x^7)\\
%         \ln(1+x) &= \sum_{k=0}^\infty\frac{(-1)^k}{k+1}x^{k+1},(-1,1] \\
%         &= x - \frac{x^2}{2} + \frac{x^3}{3} - \frac{x^4}{4} + \frac{x^5}{5} - \frac{x^6}{6} + \frac{x^7}{7} + O(x^8)\\
%         \sin x &= \sum_{k=0}^\infty \frac{(-1)^k}{(2k+1)!}x^{2k+1},\RR \\
%         &= x - \frac{x^3}{6} + \frac{x^5}{120} - \frac{x^7}{5040} + \frac{x^9}{362880}+ O(x^{11})\\
%         \cos x &= \sum_{k=0}^\infty \frac{(-1)^k}{(2k)!}x^{2k},\RR \\
%         &= 1 - \frac{x^2}{2} + \frac{x^4}{24} - \frac{x^6}{720} + \frac{x^8}{40320} + \frac{x^{10}}{3628800} + O(x^{12})\\
%         \ee^x &= \sum_{k=0}^\infty\frac{1}{n!}x^n,\RR \\
%         &= 1 + x + \frac{x^2}{2} + \frac{x^3}{6} + \frac{x^4}{24} + \frac{x^5}{120} + \frac{x^6}{720} + \frac{x^7}{5040} + \frac{x^8}{40320} + O(x^{10})\\
%         \tan x &= \sum_{k=1}^\infty \frac{(-4)^n(1-4^n)B_{2n}}{(2n)!}x^{2n-1},(-\frac{\pi}{2},\frac{\pi}{2}) \\
%         &= x + \frac{x^3}{3} + \frac{2x^5}{15} + \frac{17x^{7}}{315} + \frac{67x^9}{2835} + O(x^{11})
%     \end{aligned}
% \end{equation*}

\begin{equation*}
    \begin{aligned}
        \frac{x}{1-x} & = x + x^2 + x^3 + x^4 + x^5                                                      & + O(x^6)    \\
        \ln(1+x)      & = x - \frac{x^2}{2} + \frac{x^3}{3} - \frac{x^4}{4} + \frac{x^5}{5}              & + O(x^6)    \\
        \sin x        & = x - \frac{x^3}{6} + \frac{x^5}{120}                                            & + O(x^{7})  \\
        1- \cos x     & = \frac{x^2}{2} - \frac{x^4}{24}                                                 & + O(x^{6})  \\
        \ee^x-1       & = x + \frac{x^2}{2} + \frac{x^3}{6} + \frac{x^4}{24} + \frac{x^5}{120}           & +  O(x^{6}) \\
        \tan x        & = x + \frac{x^3}{3} + \frac{2x^5}{15}                                            & + O(x^{7})  \\
        \sqrt{x+1}-1  & = \frac{x}{2}-\frac{x^2}{8} +\frac{x^3}{16}-\frac{5 x^4}{128} +\frac{7 x^5}{256} & +  O(x^{6}) \\
        \arcsin x     & = x + \frac{x^3}{6} + \frac{3x^5}{40}                                            & + O(x^7)    \\
        \arctan x     & = x - \frac{x^3}{3} + \frac{x^5}{5}                                              & + O(x^7)
    \end{aligned}
\end{equation*}

\section{函数的连续性}

\begin{definition}[连续性]
    设函数 $f$ 在某 $U(x_0)$ 上有定义。若
    $$\lim_{x\to x_0}f(x) = f(x_0)$$
    则称 $f$ 在点 $x_0$ 连续。
\end{definition}

记 $\Delta x = x-x_0$,称为自变量 $x$ 在点 $x_0$ 的增量或改变量。设 $y_0=f(x_0)$,相应的函数 $y$ 在点 $x_0$ 的增量记为
$$\Delta y = f(x)-f(x) = f(x+\Delta)-f(x_0) = y-y_0$$

连续性的 $\eps-\delta$ 形式定义:若对任给的 $\eps>0$,存在 $\delta>0$,使得当 $|x-x_0|<\delta$ 时,有 $|f(x)-f(x_0)|<\eps$,则称函数 $f$ 在点 $x_0$ 连续。

或者进一步表示为
$$\lim_{x\to x_0}f(x) = f\left(\lim_{x\to x_0}x\right)$$

\begin{definition}
    设函数 $f$ 在某 $U_+(x_0)$ 上有定义。若
    $$\lim_{x\to x_0^+}f(x) = f(x_0)$$
    则称 $f$ 在点 $x_0$ 右连续。同理左连续。
\end{definition}

因此函数 $f$ 在点 $x_0$ 连续的充要条件是:$f$ 在点 $x_0$ 既是左连续,又是右连续。

\begin{definition}[间断点]
    设函数 $f$ 在某 $U^\circ(x_0)$ 上有定义。若 $f$ 在点 $x_0$ 无定义,或 $f$ 在点 $x_0$ 有定义而不连续,则称点 $x_0$ 为函数 $f$ 的间断点或不连续点。
\end{definition}

若 $\displaystyle\lim_{x\to x_0}f(x)=A$,而 $f$ 在点 $x_0$ 无定义,或有定义但 $f(x_0)\ne A$,则称点 $x_0$ 为 $f$ 的可去间断点。

若函数 $f$ 在点 $x_0$ 的左、右极限都存在,但 $\displaystyle\lim_{x\to x_0^+}f(x) \ne \lim_{x\to x_0^-}f(x)$,则称点 $x_0$ 为函数 $f$ 的跳跃间断点。

可去间断点与跳跃间断点统称为第一类间断点,所有其他形式的间断点统称为第二类间断点。

若函数 $f$ 在区间 $I$ 上的每一点都连续,则称 $f$ 为 $I$ 上的连续函数。对于闭区间或半开区间的端点,函数在这些点上连续是指左连续或右连续。

\subsection{连续函数的性质}

\begin{theorem}[局部有界性]
    若函数 $f$ 在点 $x_0$ 连续,则 $f$ 在某 $U(x_0)$ 上有界。
\end{theorem}

\begin{theorem}[局部保号性]
    若函数 $f$ 在点 $x_0$ 连续,且 $f(x_0)>0$,则对任何正数 $r<f(x_0)$,存在某 $U(x_0)$,使得对一切 $x\in U(x_0)$,有 $f(x)>r$。
\end{theorem}

\begin{theorem}[四则运算]
    若函数 $f,g$ 在点 $x_0$ 连续,则 $f\pm g,f\cdot g,f/g$ 也都在点 $x_0$ 连续。
\end{theorem}

\begin{theorem}
    若函数 $f$ 在点 $x_0$ 连续,$g$ 在点 $u_0$ 连续,$u_0=f(x_0)$,则复合函数 $g\circ f$ 在 $x_0$ 连续。
\end{theorem}

\begin{definition}
    设 $f$ 为定义在数集 $D$ 上的函数。若存在 $x_0\in D$,使得对一切 $x\in D$,有 $f(x_0)\ge f(x)$,则称 $f$ 在 $D$ 上有最大值,并称 $f(x_0)$ 为 $f$ 在 $D$ 上的最大值。
\end{definition}

\begin{theorem}[最大、最小值定理]
    若函数 $f$ 在闭区间 $[a,b]$ 上连续,则 $f$ 在闭区间 $[a,b]$ 上有最大值与最小值。
\end{theorem}

\begin{theorem}[介值定理]
    若函数 $f$ 在闭区间 $[a,b]$ 上连续,且 $f(a)\ne f(b)$。若 $\mu$ 为介于 $f(a)$ 和 $f(b)$ 之间的任何实数。则至少存在一点 $x_0\in (a,b)$ 使得 $f(x_0)=\mu$。
\end{theorem}

\begin{theorem}
    若函数 $f$ 在 $[a,b]$ 上严格单调并连续,则反函数 $f^{-1}$ 在其定义域 $[\min\{f(a),f(b)\},\max\{f(a),f(b)\}]$ 上连续。
\end{theorem}

\begin{definition}
    设 $f$ 是定义在区间 $I$ 上的函数。若对任给的 $\eps>0$ 存在 $\delta=\delta(\eps)>0$ 使得对任何 $x',x''\in I$,只要 $|x'-x''|<\delta$ 就有
    $$|f(x')-f(x'')|<\eps$$
    就称函数 $f$ 在区间 $I$ 上一致连续。
\end{definition}

\begin{theorem}[一致连续性]
    若函数 $f$ 在闭区间 $[a,b]$ 上连续,则 $f$ 在 $[a,b]$ 上一致连续。
\end{theorem}

\subsection{初等函数的连续性}

\begin{theorem}
    设 $p>0$,$a,b$ 为任意两个实数,则有
    $$p^a\cdot p^b = p^{a+b},(p^a)^b=p^{ab}$$
\end{theorem}

\begin{theorem}
    指数函数 $a^x(a>0)$ 在 $\RR$ 上是连续的。
\end{theorem}


