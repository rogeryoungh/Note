\chapter{级数}

\section{数项级数的敛散性}

给定一个无穷数列 $\{a_n\}$,称形式和
\[ \sum_{i=1}^\infty a_i = a_1 + a_2 + a_3 + \cdots \]
为无穷级数,其中 $a_i$ 记作通项或一般项。对于前 $n$ 项和,我们称为第 $n$ 个部分和。

若 $\lim_{n \to \infty} S_n = S$ 存在且有限,则称级数级数 $\sum a_i$ 收敛,否则就称级数发散。

显然若级数 $\sum a_i$ 收敛,则通项
\[ a_n = S_n - S_{n-1} \to S - S = 0 \quad(n \to \infty) \]

\begin{theorem}[Cauchy 准则]
	给定级数 $\sum a_i$,存在 $N = N(\varepsilon)$ 使得当 $n > N$ 时有
	\[ |S_{n+p} - S_{n}| = |a_{n+p} + \cdots + a_{n+1}| < \varepsilon \]
\end{theorem}

显然级数的有限项不影响级数的敛散性,以下讨论均舍弃。

\section{正项级数的敛散性}

若级数各项的符号都相同,则称为同号级数;进一步的,若皆为正号,则称为正项级数。由于 $0$ 并不影响收敛性,故以下讨论的正项级数可含 $0$。

\begin{theorem}[基本判别法]
	正项级数 $\sum a_i$ 收敛的充要条件是:部分和数列 $\{S_n\}$ 有界。
\end{theorem}

\begin{theorem}[比较原则]
	设正项级数 $\sum a_i, \sum b_i$,若存在正数 $N$,使得
	\[ a_i \leqslant b_i, \quad \forall n \geqslant N \]
	那么

	(1) 若级数 $\sum b_i$ 收敛,则级数 $\sum a_i$ 收敛。

	(2) 若级数 $\sum a_i$ 发散,则级数 $\sum b_i$ 发散。
\end{theorem}

\begin{proof}
	设其部分和分别为 $A_i, B_i$。不妨假设 $N=1$,即对数列的每项都成立。

	(1) 显然有
	\[ A_i \leqslant B_i \leqslant B \]
	即 $A_i$ 为单调有界数列,故必有极限 $A$,因此级数 $\sum a_i$ 收敛。

	(2) 其为(1)的逆否命题,也成立。
\end{proof}

因此我们可以通过比较、放缩等手段,将待判级数和已知敛散性的级数进行比较。

一个重要的已知级数是等比级数,我们有三种判断手段。

\begin{theorem}[比式判别法]
	设 $\sum a_i$ 是正项级数,却存在某正数 $N$ 和 $q(0 < q < 1)$ 使得
	\[ a_{n+1} / a_n \leqslant q, \quad \forall n \geqslant N \]
	则级数 $\sum a_i$ 收敛;若
	\[ a_{n+1} / a_n \geqslant 1, \quad \forall n \geqslant N \]
	则级数 $\sum a_i$ 发散。


	(极限形式)且
	\[ \lim_{n \to \infty} \frac{a_{n+1}}{a_n} = q \]
	则当 $q < 1$ 时,级数收敛;$q > 1$ 时,级数发散。
\end{theorem}

\begin{proof}
	(1) 不妨设 $N = 1$,则有 $a_n \leqslant q^{n-1}a_1$。而等比级数显然是收敛的,故收敛。

	(2) 显然当 $n > N_0$ 时 $a_n \geqslant n_0 > 0$,其每项的极限不趋于 $0$,故发散。

	(极限形式)由极限知,我们取 $\varepsilon = \frac{|1-q|}{2}$,存在正数 $N$ 使得当 $n > N$ 时有
	\[ q - \varepsilon < \frac{a_{n+1}}{a_n} < q + \varepsilon \]

	当 $q<1$ 时,显然 $q + \varepsilon = \frac{1+q}{2} < 1$,由比式判别法知收敛。

	当 $q > 1$ 时,$q - \varepsilon = \frac{1+q}{2} > 1$,由比式判别法知发散。
\end{proof}

\begin{theorem}[Cauchy 判别法(根式判别法)]
	设 $\sum a_i$ 是正项级数,且存在正数 $N$ 和正常数 $q$ 使得
	\[ \sqrt[n]{a_n} \leqslant q < 1, \forall n > N \]
	则级数收敛;若
	\[ \sqrt[n]{a_n} \geqslant 1, \forall n > N \]
	则级数发散。
\end{theorem}

\begin{proof}
	(1) 不妨设 $N=1$,显然 $a_n \leqslant q^n$,由等比级数收敛知原级数也收敛。

	(2) 显然 $a_n \geqslant 1^n = 1$,故发散。
\end{proof}

\begin{theorem}[积分判别法]
	设 $f$ 是 $[1, \infty)$ 上的减函数,则级数 $\sum f(i)$ 收敛的充分必要条件是反常积分 $\int_{1}^{+\infty} f(x) \d x$ 收敛。
\end{theorem}

\begin{proof}
	假设级数收敛于 $S$,由收敛的性质知,$\lim_{n \to \infty} f(n) = 0$,故 $f$ 非负。进而对任意正整数 $N$ 有
	\[ \int_{1}^{N} f(x) \d x = \sum_{n=2}^N \int_{n -1}^{n} f(x) \d x \leqslant \sum_{n=1}^{N-1} f(n) \leqslant \sum_{n=1}^{\infty} f(n) = S \]
	考虑对 $M \in (N, N + 1]$,都有
	\[ 0 \leqslant \int_{1}^{M} f(x) \d x \leqslant \int_{1}^{N+1} \leqslant S \]
	因此该反常积分收敛。

	反之,设反常积分。。。不想证了 TODO
\end{proof}

\begin{theorem}[Kummer]
	设 $\sum a_i, \sum b_i$ 为正项级数,若 $n$ 充分大时有
	\[ \frac{1}{b_n} \cdot \frac{a_n}{a_{n+1}} - \frac{1}{b_{n+1}} \geqslant \lambda > 0 \]
	则 $\sum a_n$ 收敛;若
	\[ \frac{1}{b_n} \cdot \frac{a_n}{a_{n+1}} - \frac{1}{b_{n+1}} \leqslant 0 \]
	且 $\sum b_n$ 发散,则 $\sum a_n$ 发散。
\end{theorem}

\begin{proof}
	(1) 条件即
	\[ a_{n+1} \leqslant \frac{1}{\lambda} \left( \frac{a_n}{b_n} - \frac{a_{n+1}}{b_{n+1}} \right) \]
	设 $S_n$ 是 $a_n$ 的和函数,有
	\[ S_{n+1} \leqslant S_N +\frac{1}{\lambda} \left( \frac{a_N}{b_N} - \frac{a_{n+1}}{b_{n+1}} \right) \leqslant S_N + \frac{1}{\lambda} \frac{a_N}{b_N} \]
	从而 $S_N$ 有上界,故收敛。

	(2) 可知
	\[ \frac{1}{b_n} \cdot \frac{a_n}{a_{n+1}} - \frac{1}{b_{n+1}} \leqslant 0 \Longrightarrow \frac{a_1}{b_1} \leqslant \cdots \leqslant \frac{a_n}{b_n} \leqslant \frac{a_{n+1}}{b_{n+1}} \]
	故 $a_n \geqslant \frac{a_1}{b_1} b_n$,由 $\sum b_n$ 知 $\sum a_n$ 也发散。
\end{proof}

当 $b_n = 1$ 时即得比值判别法;取 $b_n = \frac{1}{n}$ 就是 Raabe 判别法;取 $b_n = \frac{1}{n \ln n}$,则得 Gauss 判别法。

\begin{theorem}[Cauchy 凝聚判别法]
	设 $a_n$ 单调递减趋于 $0$。则 $\sum a_n$ 收敛当且仅当 $\sum 2^k a_{2^k}$ 收敛。
\end{theorem}

在 Taylor 公式中,我们经常得到正负项交替出现的级数,我们称为交错级数。

\begin{theorem}[Leibniz 判别法]
	若 $a_n$ 单调递减趋于 $0$,则级数 $\sum (-1)^{n-1} a_n$ 收敛。
\end{theorem}

\begin{proof}
	懒得证了,TODO。大致思想是注意到 $S_{2m-1}$ 和 $S_{2m}$ 分别是单减和单增的,从而 $[S_{2m-1}, S_{2m}]$ 是退缩区间套。
\end{proof}

\section{一般项级数收敛判别法}

若级数 $\sum a_n$ 各项绝对值组成的级数 $\sum |a_n|$ 收敛,那么称原级数绝对收敛;若原级数收敛,绝对值级数不收敛,则称为条件收敛。

更神秘的判别法,比如 Abel 判别法、Dirichlet 判别法感觉不太用得到,跳过。


\section{函数项级数的一致收敛}

设 $f_1, f_2, \cdots$ 是 $E$ 上的函数列,记为 $\{f_i\}$。若带入 $x = x_0$ 使得函数列的值收敛,则称函数列在点 $x_0$ 收敛,$x_0$ 称为函数列的收敛点;反之若发散,则称发散点。

若函数列在 $D \subset E$ 上每一点都收敛,则称其在数集 $D$ 上收敛。此时对函数上每一点 $x$,都有一个极限值与之对应,称为函数列的极限函数。记为
\[ \lim_{n \to \infty} f_n(x) = f(x) \]
函数列的全体收敛点集,称为收敛域。


