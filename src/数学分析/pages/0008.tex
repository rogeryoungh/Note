\chapter{多变量理论}

\section{多变量极限}

我们回忆一下 $n$ 维 Euclidean 空间
\[ \mathbb{R}^n = \{\vbf{x} = (x_1, \cdots, x_n) \mid x_i \in \mathbb{R} \} \]
其中的元素我们记为 $n$ 维向量(vector)。在其上我们定义向量内积的概念
\[ \vbf{x} \cdot \vbf{y} = \langle \vbf{x}, \vbf{y} \rangle= \sum_{i=1}^{n} x_i y_i \]
向量内积具有线性性。并满足 Schwarz 不等式:
\[ (\vbf{x} \cdot \vbf{y})^2 \leqslant (\vbf{x} \cdot \vbf{x})(\vbf{y} \cdot \vbf{y}) \]

我们定义一个向量的范数(norm)为
\[ |\vbf{x}| = \sqrt{\vbf{x} \cdot \vbf{x}} = \sqrt{\sum_{i=1}^{n} |x_i|^2}  \]
从而有三角不等式
\[ |\vbf{x} + \vbf{y}| +\leqslant |\vbf{x}| + |\vbf{y}| \]
事实上还存在一些其他范数,比如 $p$ - 范数
\[ |\vbf{x}|_p = \sqrt[p]{\sum_{i=1}^{n} |x_i|^p}  \]
和 $\infty$ - 范数
\[ |\vbf{x}|_\infty = \max_{i=1}^{n} |x_i|  \]

可以进一步的定义向量的夹角
\[ \theta_{\vbf{x}, \vbf{y}} = \arccos\frac{\vbf{x} \cdot \vbf{y}}{|\vbf{x}||\vbf{y}|} \]
和距离
\[ d(\vbf{x}, \vbf{y}) = |\vbf{x}-\vbf{y}| \]
从而构建出度量空间。

我们可以定义球邻域
\[ \mathbb{B}^n(\vbf{x}, r) = \{ \vbf{y} \mid |\vbf{y}-\vbf{x}| < r \} \]
和方邻域
\[ \mathbb{O}^n(\vbf{a}, r) = \{ \vbf{x} \mid |x_i - a_i| < r, 1 \leqslant i \leqslant n \} \]
由于
\[ \mathbb{B}^n(\vbf{a}, r) \subset \mathbb{O}^n(\vbf{a}, r) \subset \mathbb{B}^n(\vbf{a}, r\sqrt{n})  \]
我们可以不加区分的统称为邻域,记作 $U(\vbf{a}, r)$。

\begin{definition}
	设 $\{\vbf{x}_i\}$ 是 $\mathbb{R}^n$ 中的点列。如果存在 $M > 0$ 使得 $\vbf{x}_i \in \mathbb{B}^n(\vbf{0}, M)$,则称其为有界的。如果对任意 $\eps > 0$ 都存在正整数 $I$ 使得当 $i > I$ 时满足
	\[ |\vbf{i} - \vbf{x}| < \eps \]
	成立,即 $\vbf{x} \in \mathbb{B}^n(\vbf{x}, \eps)$,则记
	\[ \lim_{i \to \infty} \vbf{x}_i = \vbf{x} \]
	假如不收敛到任何点,则称其为发散的。
\end{definition}

考虑到考研对多元部分的考察并不深入,将实数上的连续性搬到多元空间上是自然的,可以信赖直觉。

\begin{definition}
	设 $D \subset \mathbb{R}^n$,且 $\vbf{a} \in D'$ 是 $D$ 的聚点,如果对任意 $\eps > 0$ 存在 $\delta > 0$ 使得当 $\vbf{x} \in U^{\circ}(\vbf{a}, \delta) \cap D$ 有
	\[ |f(\vbf{x} - A)| < \eps \]
	成立,则称 $f(\vbf{x})$ 在 $D$ 上当 $\vbf{x} \to \vbf{a}$ 时以 $A$ 为极限,或者称为 $n$ 重极限,记作
	\[ \lim_{\vbf{x} \to \vbf{a}} f(\vbf{x}) = A \]
\end{definition}

实数上极限的唯一性、有界性等性质也可以搬过来。但是有一些复杂的情况,比如
\[ \lim_{(x, y) \to \vbf{0}} \frac{y}{x} \]
我们沿着 $y = kx$ 趋于 $\vbf{0}$,得到是 $k$ 不是定值,说明极限不存在。

\begin{definition}
	设 $D = D_1 \times D_2 \subset \mathbb{R}$,$x_0, y_0$ 分别是 $D_1, D_2$ 的聚点。如果对每个固定的 $y \in D_2$ 且 $y \neq y_0$ 作为 $x$ 的一元函数,极限
	\[ \lim_{x \to x_0} f(x, y) \]
	存在且极限
	\[ \lim_{y \to y_0} \lim_{x \to x_0} f(x, y) \]
	也存在,则称后者为 $f$ 在 $(x_0, y_0)$ 点先对 $x$ 后对 $y$ 的累次极限。
\end{definition}

\section{多变量导数}

回顾一下一元函数 $f(x)$ 在 $a$ 处的微分
\[ f(a + \Delta x) - f(a) = f'(a) \Delta x + o(|\Delta x|) \]
根据无穷小的定义得到
\[ \lim_{\Delta x \to 0} \frac{f(a + \Delta x) - f(a) - f'(a) \Delta x}{|\Delta x|} = 0 \]

\begin{definition}
	设 $n$ 元函数 $f(\vbf{x})$ 在 $\vbf{a}$ 的邻域内有定义,若存在 $n$ 维向量 $\vbf{b}$ 使得
	\[ \lim_{\Delta \vbf{x} \to 0} \frac{f(\vbf{a} + \Delta \vbf{x}) - f(\vbf{a}) - \vbf{b} \cdot \Delta \vbf{x}}{|\Delta \vbf{x}|} = 0 \]
	成立,即
	\[ f(\vbf{a} + \Delta \vbf{x}) - f(\vbf{a}) = \vbf{b} \cdot \Delta \vbf{x} + o(|\Delta x|) \]
	则称 $f$ 在 $\vbf{a}$ 处可微或可导,向量 $\vbf{b}$ 为函数 $f$ 在 $\vbf{a}$ 处的导数,记为
	\[ \vbf{b} = f'(\vbf{a}) = \nabla f(\vbf{a}) \]
	并记微分为 $\d f = \vbf{b} \cdot \d \vbf{x}$。
\end{definition}

假如考虑对于固定点 $\vbf{a}$,考虑在 $x_i$ 方向上的极限
\[ \lim_{\Delta x_i \to 0} \frac{f(a_1, \cdots, a_i + \Delta x_i, \cdots, a_n) - f(\vbf{a})}{\Delta x_i} \]
若极限存在,则称 $f$ 在 $\vbf{a}$ 关于 $x_i$ 可偏导,记极限为 $f$ 在 $\vbf{a}$ 处关于 $x_i$ 的偏导数并记为
\[ \frac{\partial f}{\partial x_i}(\vbf{a}) = f_{x_i}(\vbf{a}) \] 

注意,偏导数存在不意味着可微。比如
\[ f(x, y) = \frac{xy}{x^2 + y^2}, \quad f(\vbf{0}) = 0 \]
此时 $f_x(\vbf{0}) = f_y(\vbf{0}) = 0$,但并不连续。

\begin{theorem}
	如果函数 $f$ 在 $\vbf{a}$ 处可微,则在 $\vbf{a}$ 处必存在偏导数且满足
	\[ f'(\vbf{a}) = \left( f_{x_1}(\vbf{a}), \cdots, f_{x_n}(\vbf{a}) \right) \]
	和
	\[ \d f(\vbf{a}) = f'(\vbf{a}) \d \vbf{x} = \sum_{i=1}^n f_{x_i} \d x_i \]
\end{theorem}

\begin{proof}
	由可微知存在 $\vbf{b}$ 使得
	\[ f(\vbf{a} + \Delta \vbf{x}) - f(\vbf{a}) = \vbf{b} \cdot \Delta \vbf{x} + o(|\Delta \vbf{x}|) \]
	此时我们可以取仅 $x_i$ 方向有值的无穷小,便可导出偏导。
\end{proof}

\begin{theorem}
	如果 $f$ 在 $\vbf{a}$ 的某个邻域存在偏导数,且偏导数在 $\vbf{a}$ 点连续,则 $f$ 在 $\vbf{a}$ 处可微。
\end{theorem}

考虑一单位向量 $\vbf{\mu}$,如果极限
\[ \frac{\partial f}{\partial \mu}(\vbf{x}_0) = \lim_{t \to 0+} \frac{f(\vbf{x}_0 + t \vbf{\mu}) - f(\vbf{x}_0)}{t} \]
存在,则称函数 $f$ 在 $\vbf{x}_0$ 处沿着方向 $\vbf{\mu}$ 是方向可导的,并称该极限为函数 $f$ 在 $\vbf{x}_0$ 处沿着 $\vbf{\mu}$ 的方向导数。

假如 $f$ 在 $\vbf{x}_0$ 处的偏导数存在,函数 $f$ 在 $\vbf{x}_0$ 处的梯度为
\[ \grad_{\vbf{x}_0} f = \nabla_{\vbf{x}_0} f = \left( f_{x_1}(\vbf{x}_0), \cdots, f_{x_n}(\vbf{x}_n) \right) \]
显然当函数 $f$ 可微时,$\nabla_{\vbf{x}_0} f = f'(\vbf{x}_0)$,且
\[ \frac{\partial f}{\partial \vbf{\mu}} (\vbf{x}_0) = \nabla_{\vbf{x}_0} f \cdot \vbf{\mu} \]

我们记其为 Nabla 算子,即
\[ \nabla = \left( \frac{\partial}{\partial x_1}, \cdots, \frac{\partial}{\partial x_n} \right) \]
和 Laplace 算子
\[ \Delta = \nabla \cdot \nabla = \sum_{i=1}^{n} \frac{\partial^2}{(\partial x_i)^2} \]
如果函数 $f$ 的所有二阶偏导存在且 $\Delta f = 0$,则称其为调和的。

推广偏导数概念到二阶直至更高阶是自然的,但是微分算子是不可交换的。

\begin{theorem}[Clairaut - Euler]
	假设函数 $f$ 的混合二阶偏导数 $f_{x_i x_j}$ 和 $f_{x_j x_i}$ 在 $\vbf{x}_0$ 连续,则必有 $f_{x_i x_j}(\vbf{x}_0) = f_{x_j x_i}(\vbf{x}_0)$。
\end{theorem}

\section{偏导数的应用}

\subsection{链式法则}

我们以二元为例子,考虑二元二维向量值函数 $\vbf{g}(x, y) = (u(x, y), v(x, y))$,得到复合函数
\[ z = (f \circ \vbf{g})(x, y) = f(u(x, y), v(x, y)) \]
我们有链式法则
\[ \begin{aligned}
	\frac{\partial z}{\partial x}(x_0, y_0) = \frac{\partial f}{\partial u}(u_0, v_0) \frac{\partial u}{\partial x}(x_0, y_0) + \frac{\partial f}{\partial v}(u_0, v_0)\frac{\partial v}{\partial x}(x_0, y_0) \\
	\frac{\partial z}{\partial y}(x_0, y_0) = \frac{\partial f}{\partial u}(u_0, v_0) \frac{\partial u}{\partial y}(x_0, y_0) + \frac{\partial f}{\partial v}(u_0, v_0)\frac{\partial v}{\partial y}(x_0, y_0)
\end{aligned} \]

\subsection{微分中值定理}

如果对任意 $\vbf{x}_0, \vbf{x}_1 \in D$ 都有
\[ \overline{\vbf{x}_0 \vbf{x}_1} = \{ (1-t)\vbf{x}_0 + t\vbf{x}_1 \mid t \in (0, 1) \} \subset D \]
则称区域 $D \subset \mathbb{R}^n$ 是凸的。如果对 $\vbf{x}_0$ 有任意的 $\vbf{x}_1$ 都有 $\overline{\vbf{x}_0 \vbf{x}_1} \subset D$,称 $D \subset \mathbb{R}^n$ 是关于 $\vbf{x}_0$ 星形的。

\begin{theorem}
	若多元函数 $f$ 在凸区域 $D \subset \mathbb{R}^n$ 内可微,则对 $D$ 内的任意两点 $\vbf{x}_0$ 和 $\vbf{x}$,存在 $\theta \in (0, 1)$ 使得
	\[ f(\vbf{x}) - f(\vbf{x}_0) = \nabla_{\vbf{x}_0 + \theta(\vbf{x} - \vbf{x}_0)}(f) \cdot (\vbf{x} - \vbf{x}_0) \]
\end{theorem}

类似的,我们可以把一元的 Taylor 各种公式扩展到多元。

\subsection{切线与法平面}

考虑空间中曲线的参数方程
\[ \vbf{r}(t) = (x(t), y(t), z(t)), \quad t \in [a, b] \]
如果 $\vbf{r}'(t)$ 连续且 $\vbf{r}'(t) \neq \vbf{0}$ 则称曲线是光滑的。

固定点 $\vbf{P}_0 = (x_0, y_0, z_0)$,我们可以得到切向量
\[ \vbf{r}'(t_0) = (x'(t_0), y'(t_0), z'(t_0)) \]
和曲线在 $\vbf{P}_0$ 处的切线方程
\[ \frac{x - x_0}{x'(t_0)} = \frac{y - y_0}{y'(t_0)} = \frac{z - z_0}{z'(t_0)} \]
和曲线在 $\vbf{P}_0$ 处的法平面
\[ x'(t_0)(x - x_0) + y'(t_0)(y - y_0) + z'(t_0)(z - z_0) = \vbf{r}'(t_0) \cdot (\vbf{p} - \vbf{P}_0) = 0 \]

考虑曲面的方程
\[ F(x, y, z) = 0 \]
固定点 $\vbf{P}_0 = (x_0, y_0, z_0)$,得到切平面的法向量
\[ \vbf{n} = (F_x(\vbf{P}_0), F_y(\vbf{P}_0), F_z(\vbf{P}_0)) \]
从而得到切平面方程为
\[ F_x(\vbf{P}_0)(x-x_0) + F_y(\vbf{P}_0)(y-y_0) + F_z(\vbf{P}_0)(z-z_0) = \nabla_{\vbf{P}_0}f \cdot (\vbf{p} - \vbf{P}_0) = 0 \]
和法线方程
\[ \frac{x - x_0}{F_x(t_0)} = \frac{y - y_0}{F_y(t_0)} = \frac{z - z_0}{F_z(t_0)} \]

\section{极值问题}

极值问题主要有两种:无条件极值和有条件极值。

对于多元函数 $f$,若存在 $\vbf{x}_0$ 的邻域 $\mathbb{B}^n(\vbf{x}_0, \rho)$ 使得
\[ f(\vbf{x}_0) \geqslant f(\vbf{x}) \]
则称 $\vbf{x}_0$ 为 $f$ 的极大值点。类似的,有极小值点,统称为极值点。如果存在去心邻域 $\mathbb{B}^n(\vbf{x}_0, \rho) - \{\vbf{x}_0\}$ 使得
\[ f(\vbf{x}_0) > f(\vbf{x}) \]
则称为严格极大值点。类似的有严格极小值和严格极值点。

\subsection{无条件极值}

\begin{theorem}[多元函数 Fermat 引理]
	假设 $n$ 元函数 $f$ 在 $\vbf{x}_0$ 处可偏导且 $\vbf{x}_0$ 为其极值点,则 $f'(\vbf{x}_0) = \vbf{0}$。
\end{theorem}

\begin{theorem}
	假设 $\vbf{x}_0$ 为多元函数 $f$ 的驻点,并假设 $f$ 在 $\vbf{x}_0$ 处有二阶连续偏导数,引入 Hessian 矩阵
	\[ \operatorname{Hess}_{\vbf{x}}(f) = [f_{x_i x_j}(\vbf{x})]_{n \times n} \]
	则有如下结论:
	\begin{itemize}
		\item $\operatorname{Hess}_{\vbf{x}}(f)$ 正定,$f(\vbf{x}_0)$ 为严格极小值。
		\item $\operatorname{Hess}_{\vbf{x}}(f)$ 负定,$f(\vbf{x}_0)$ 为严格极大值。
		\item $\operatorname{Hess}_{\vbf{x}}(f)$ 不定,$f(\vbf{x}_0)$ 不是极值。
	\end{itemize}
\end{theorem}

那么对于二元函数 $f$,设 $(x_0, y_0)$ 为其驻点,引入记号
\[ \Delta(x_0, y_0) = (f_{xx} f_{yy} - f_{xy}^2)(x_0, y_0) \]
则有如下结论:
\begin{itemize}
	\item 如果 $\Delta > 0$ 且 $f_xx > 0$,为严格极小值。
	\item 如果 $\Delta > 0$ 且 $f_xx < 0$,为严格极大值。
	\item 如果 $\Delta < 0$,不是极值。
	\item 如果 $\Delta = 0$ 无法判断。
\end{itemize}

无法判断的例子有 $f_1 = x^2y^2, g = -f, h = x^2 y^3$,在 $0$ 处极值情况各不相同。

因此我们求取最值可以通过以下几个步骤:

\begin{itemize}
	\item 求出驻点和不可导点和相应的函数值。
	\item 求出边界上的函数值。
	\item 比较 $1$ 和 $2$ 的情况。
\end{itemize}

\subsection{条件极值问题}

Todo Lagrange 橙子法。

\section{二重积分}


