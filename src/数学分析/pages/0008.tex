\chapter{级数}

\section{数项级数的敛散性}

给定一个无穷数列 $\{a_n\}$,称形式和
\[ \sum_{i=1}^\infty a_i = a_1 + a_2 + a_3 + \cdots \]
为无穷级数,其中 $a_i$ 记作通项或一般项。对于前 $n$ 项和,我们称为第 $n$ 个部分和。

若 $\lim_{n \to \infty} S_n = S$ 存在且有限,则称级数级数 $\sum a_i$ 收敛,否则就称级数发散。

显然若级数 $\sum a_i$ 收敛,则通项
\[ a_n = S_n - S_{n-1} \to S - S = 0 \quad(n \to \infty) \]

\begin{theorem}[Cauchy 准则]
	给定级数 $\sum a_i$,存在 $N = N(\varepsilon)$ 使得当 $n > N$ 时有
	\[ |S_{n+p} - S_{n}| = |a_{n+p} + \cdots + a_{n+1}| < \varepsilon \]
\end{theorem}

显然级数的有限项不影响级数的敛散性,以下讨论均舍弃。

\section{正项级数的敛散性}

若级数各项的符号都相同,则称为同号级数;进一步的,若皆为正号,则称为正项级数。由于 $0$ 并不影响收敛性,故以下讨论的正项级数可含 $0$。

\begin{theorem}[基本判别法]
	正项级数 $\sum a_i$ 收敛的充要条件是:部分和数列 $\{S_n\}$ 有界。
\end{theorem}

\begin{theorem}[比较原则]
	设正项级数 $\sum a_i, \sum b_i$,若存在正数 $N$,使得
	\[ a_i \leqslant b_i, \quad \forall n \geqslant N \]
	那么
	
	(1) 若级数 $\sum b_i$ 收敛,则级数 $\sum a_i$ 收敛。

	(2) 若级数 $\sum a_i$ 发散,则级数 $\sum b_i$ 发散。
\end{theorem}

\begin{proof}
	设其部分和分别为 $A_i, B_i$。不妨假设 $N=1$,即对数列的每项都成立。

	(1) 显然有
	\[ A_i \leqslant B_i \leqslant B \]
	即 $A_i$ 为单调有界数列,故必有极限 $A$,因此级数 $\sum a_i$ 收敛。

	(2) 其为(1)的逆否命题,也成立。
\end{proof}

