\chapter{积分的方法与技巧}

这部分我的参考书是《积分的方法与技巧》(金玉明等)。

\section{积分的存在性}

先放这,稍后做整理。

\begin{definition}
    设函数 $f$ 与 $F$ 在区间 $I$ 上都有定义。若
    $$F'(x) = f(x), x\in I$$
    则称 $F$ 为 $f$ 在区间 $I$ 上的一个原函数。
\end{definition}

\begin{theorem}
    设 $F$ 是 $f$ 在区间 $I$ 上的一个原函数,则
    \num{1} $F+C$ 也是 $f$ 在 $I$ 上的原函数,其中 $C$ 为任意常量函数
    \num{2} $f$ 在 $I$ 上的任意两个原函数之间,只可能相差一个常数。
\end{theorem}
\begin{proof}
    \num{1} 显然
    $$\left[ F(x)+C \right]' = F'(x) = f(x), x\in I$$
    
    \num{2} 设 $F$ 和 $G$ 是 $f$ 在 $I$ 上的任意两个原函数,则有
    $$\left[ F(x)-G(x) \right]' = F'(x) - G'(x) = f(x) - f(x) = 0, x\in I$$
    根据 Lagrange 中值定理,有
    $$F(x) - G(x) \equiv C, x\in I$$
\end{proof}

\begin{definition}
    函数 $f$ 在区间 $I$ 上的全体原函数称为 $f$ 在 $I$ 上的不定积分,记作
    $$\int f(x) \dd x = F(x) + C$$
\end{definition}

其中 $\displaystyle\int$ 称为积分号,$f(x)$ 为被积函数,$f(x)\dd x$ 为被积表达式,$x$ 称为积分变量。

\subsection{定积分}

设闭区间 $[a,b]$ 上有 $n-1$ 个点,依次为
$$a = x_0 < x_1 < \cdots < x_{n-1} < x_n = b$$
它们把 $[a,b]$ 分为 $n$ 个小区间 $\Delta_i = [x_{i-1},x_i], i=1,\cdots,n$。这些分点或这些闭子区间构成对 $[a,b]$ 的一个分割,记为
$$T=\{x_0,\cdots,x_n\}\ \text{或}\ \{\Delta_1,\cdots,\Delta_n\}$$
小区间 $\Delta_i$ 的长度为 $\Delta x_i = x_i-x_{i-1}$,并记 
$$\| T \| = \max_{1 \leqslant i \leqslant n}\{\Delta x_i\}$$
称为分割的模。

\begin{definition}
    设 $f$ 是定义在 $[a,b]$ 上的一个函数。对于 $[a,b]$ 的一个分割 $T=\{\Delta_1,\cdots,\Delta_n\}$,任取点 $\xi\in\Delta_i,i=1,\cdots,n$,并作和式
    $$\sum_{i=1}^nf(\xi_i)\Delta x_i$$
    称此和式为函数 $f$ 在 $[a,b]$ 上的一个积分和,也称黎曼和。
\end{definition}

显然,积分既与分割 $T$ 与有关,又与所选取的点集 $\{\xi_i\}$ 有关。

\begin{definition}
    设 $f$ 是定义在 $[a,b]$ 上的一个函数,$J$ 是一个确定的实数。若对任给的正数 $\epsilon$,总存在某一正数 $\delta$,使得对 $[a,b]$ 的任何分割 $T$,以及在其上任意选取的点集 $\{\xi_i\}$,只要 $\| T \| < \delta$,就有
    $$\left| \sum_{i=1}^nf(\xi_i)\Delta x_i - J \right| < \eps$$
    则称函数 $f$ 在区间 $[a,b]$ 上可积或黎曼可积;数 $J$ 称为 $f$ 在 $[a,b]$ 上的定积分或黎曼积分,记作
    $$J = \int_a^b f(x) \dd x = \lim_{\| T \| \to 0} \sum_{i=1}^nf(\xi_i)\Delta x_i$$
\end{definition}

其中 $f$ 称为被积函数,$x$ 称为积分变量,$[a,b]$ 称为积分区间,$a,b$ 分别称为这个定积分的下限和上限。

\begin{theorem}[Newton - Leibniz 公式]
    若函数 $f$ 在 $[a,b]$ 上连续,且存在原函数 $F$,即 $F'(x) = f(x), x\in[a,b]$,则 $f$ 在 $[a,b]$ 上可积,且
    $$\int_a^bf(x)\dd x = F(b) - F(a)$$
\end{theorem}

\begin{proof}
    即证对于任给的 $\eps > 0$,存在 $\delta>0$ 使得当 $\| T \| < \delta$ 时有
    $$\left| \sum_{i=1}^nf(\xi_i)\Delta x_i - [F(b)-F(a)] \right| < \eps$$
    对于任意分割 $T$,在每个小区间 $[x_{i-1},x_i]$ 上对 $F(x)$ 使用 Lagrange 中值定理,则分别存在 $\eta_i \in (x_{i-1},x_i),i=1,\cdots,n$,使得
    $$F(b)-F(a) = \sum_{i=1}^n F'(\eta_i)\Delta x_i = \sum_{i=1}^nf(\eta_i)\Delta x_i$$
    又因为 $f$ 在 $[a,b]$ 上一致连续,因此存在 $\delta > 0$ 当 $x_1,x_2\in[a,b]$ 且 $|x_1-x_2| < \delta$ 时,有
    $$f(x_1) - f(x_2)| < \frac{\eps}{b-a}$$
    由 $\Delta x_i \leqslant \| T \| < \delta$ 时,任取 $\xi_i \in [x_{i-1},x_i]$,便有 $|\xi_i-\eta_i|<\delta$,于是
    $$LHS = \left| \sum_{i=1}^n [f(\xi_i)-f(\eta_i)]\Delta x_i \right| \leqslant \frac{\eps}{b-a}\sum_{i=1}^n\Delta x_i = \eps$$
    于是 $f$ 在 $[a,b]$ 上可积。
\end{proof}

\subsection{可积条件}

\begin{theorem}
    若函数 $f$ 在 $[a,b]$ 上可积,则 $f$ 在 $[a,b]$ 上必定有界。
\end{theorem}
\begin{proof}
    反证,若 $f$ 在 $[a,b]$ 上无界,则对于即对于 $[a,b]$ 的任意分割 $T$,必存在属于 $T$ 的某区间 $\Delta_k$,使 $f$ 在其上无界。在 $i\ne k$ 的各个区间 $\Delta_i$ 上取定 $\xi_i$,记
    $$G = \left| \sum_{i\ne k}f(\xi)\Delta x_i \right|$$    
    任意大的正数 $M$,存在 $\xi_k\in \Delta_k$,使得
    $$|f(\xi_k)| > \frac{M+G}{\Delta x_k}$$
    于是有
    $$\left| \sum_{i=1}^nf(\xi)\Delta x_i \right| \geqslant |f(\xi_k)\Delta x_k| - \left| \sum_{i\ne k}f(\xi)\Delta x_i \right| > \frac{M+G}{\Delta x_k}\cdot \Delta x_k - G = M$$
\end{proof}

有界函数不一定黎曼可积,比如 Dirichlet 函数。

设 $T$ 为对 $[a,b]$ 的任意分割。由 $f$ 在 $[a,b]$ 上有界,它在每个 $\Delta_i$ 上存在上、下确界:
$$M_i=\sup_{x\in\Delta_i}f(x),m_i = \inf_{x=\Delta_i}f(x),i=1,\cdots,n$$
作和
$$S(T) = \sum_{i=1}^nM_i\Delta x_i, s(T) = \sum_{i=1}^n m_i \Delta x_i$$
分别称为 $f$ 关于分割 $T$ 的上和与下和(或称达布上和与达布下和,统称达布和)。任给 $\xi_i = \Delta_i,i=1,\cdots,n$,显然有
$$m(b-a) \leqslant s(T) \leqslant \sum_{i=1}^n f(\xi_i)\Delta x_i \leqslant S(T) \leqslant M(b-a)$$
与积分和相比较,达布和只与分割 $T$ 有关,而与点集 $\{\xi_i\}$ 无关。

\begin{proposition}
    给定分割 $T$,对于任何点集 $\{\xi_i\}$ 而言,上和时所有积分和的上确界,下和是所有积分和的下确界。
\end{proposition}
\begin{proof}
    设 $\Delta_i$ 中 $M_i$ 是 $f(x)$ 的上确界,故可选取点 $\xi=\Delta_i$,使 $f(\xi_i)>M_i-\frac{\eps}{b-a}$,于是有
    $$\sum_{i=1}^nf(\xi_i)\Delta x_i > \sum_{i=1}^nM_i\Delta x_i-\frac{\eps}{b-a}\sum_{i=1}^n\Delta x_i = S(T)-\eps$$
    即 $S(T)$ 是全体积分和的上确界。类似可证 $s(T)$ 是全体积分和的下确界。
\end{proof}


\begin{proposition}
    设 $T'$ 为分割 $T$ 添加 $p$ 个新分点后所得到的分割,则有
    
\end{proposition}

\begin{theorem}
    函数 $f$ 在 $[a,b]$ 上可积的充要条件是:人格 $\eps >0$,总存在相应的一个分割 $T$ 使得
    $$S(T) - s(T) < \sum_{i=1}^n\omega_i\Delta x_i = \eps$$
\end{theorem}

其中 $\omega$ 称为 $f$ 在 $\Delta_i$ 上的振幅。

由充要条件,我们可以得到一系列的可积函数类。

\begin{theorem}
    若 $f$ 为 $[a,b]$ 上的连续函数,则 $f$ 在 $[a,b]$ 上可积。
\end{theorem}
\begin{proof}
    由于 $f$ 在闭区间 $[a,b]$ 上一致连续,即任给 $\eps > 0$,存在 $\delta>0$,对 $[a,b]$ 中任意两点 $x_1,x_2$,只要 $|x_1-x_2|<\delta$,便有
    $$|f(x_1)-f(x_2)| < \frac{\eps}{b-a}$$
    所以对于在 $[a,b]$ 的分割 $T$ 满足 $\|T\| < \delta$,在 $T$ 所属的任一小区间 $\Delta_i$ 上,都有
    $$\omega_i = M_i-m_i = \sup_{x_1,x_2\in\Delta_i}|f(x_1)-f(x_2)| \leqslant \frac{\eps}{b-a}$$
    从而
    $$\sum_{T}\omega_i\Delta x_i \leqslant \frac{\eps}{b-a}\sum_T\Delta x_i = \eps$$
\end{proof}



\section{分项积分法}

若干微分式的和或差的不定积分,等于每个微分式的各自积分的和或差。
$$\int\left( f(x)+g(x)-h(x) \right)\dd x = \int f(x)\dd x + \int g(x)\dd x - \int h(x)\dd x$$
因此多项式的积分可以简单的通过积分各个单项式得到。

如果一个分式的分母为多项式,则可把它化成最简单的分式再积分。如
$$\frac{1}{x^2-a^2} = \frac{1}{2a}\left( \frac{1}{x-a}-\frac{1}{x+a} \right)$$
这里可以通过通分后待定系数得到。于是其积分为
$$\int \frac{\dd x}{x^2-a^2} = \frac{1}{2a}\ln\left|\frac{x-a}{x+a}\right|+C$$
对于更复杂的真分式的情况,若要计算的是
$$\int \frac{mx+n}{x^2+px+q} \dd x$$
分母不一定能直接分解,但总能进行配方
$$x^2+px+q = \left(x+\frac{p}{2}\right)^2+q-\frac{p^2}{4} = t^2 \pm a^2$$
再令 $A=m,B=n-\mfrac12mp$,可得
$$\int \frac{mx+n}{x^2+px+q} \dd x = \int \frac{At+B}{t^2 \pm a^2} \dd t 
= A\int \frac{t\dd t}{t^2 \pm a^2} + B\int \frac{\dd t}{t^2 \pm a^2}$$
其中

\begin{equation*}
    \begin{aligned}
        A\int \frac{t \dd t}{t^2 \pm a^2} &= \frac{A}{2}\int \frac{\dd(t^2 \pm a^2)}{t^2 \pm a^2} = \frac{A}{2} \ln|t^2 \pm a^2| +C \\
        B\int \frac{\dd t}{t^2 + a^2} &= \frac{B}{a}\arctan\frac{t}{a}+C \\
        B\int \frac{t \dd t}{t^2 - a^2} &= \frac{B}{2a}\ln\left|\frac{t-a}{t+a}\right|+C
    \end{aligned}
\end{equation*}

因此当 $p^2<4q$ 时,可以得到

\begin{equation*}
    \begin{aligned}
        \int \frac{mx+n}{x^2+px+q} &= \frac{A}{2} \ln|t^2 + a^2| + \frac{B}{a}\arctan\frac{t}{a} + C\\
        &=\frac{m}{2}\ln|x^2+px+q| + \frac{2n-mp}{\sqrt{4q-p^2}}\arctan{\frac{2x+p}{\sqrt{4q-p^2}}} + C
    \end{aligned}
\end{equation*}

当 $p^2>4q$ 时,可以得到

\begin{equation*}
    \begin{aligned}
        \int \frac{mx+n}{x^2+px+q} &= \frac{A}{2} \ln|t^2 - a^2| + \frac{B}{2a}\ln\left|\frac{t-a}{t+a}\right|\\
        &=\frac{m}{2}\ln|x^2+px+q| + \frac{2n-mp}{2\sqrt{4q-p^2}}\ln\left| \frac{2x+p-\sqrt{p^2-4q}}{2x+p+\sqrt{p^2-4q}} \right| + C
    \end{aligned}
\end{equation*}