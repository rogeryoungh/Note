\chapter{常微分方程}

如果知道函数及其导数、微分组成的关系式,得到的便是微分方程。

\section{基本概念}

自变量只有一个的方程是\textbf{常微分方程},两个以上则称\textbf{偏微分方程},其中未知函数最高阶导数的阶数称为微分方程的阶数。一般的 $n$ 阶常微分方程具有形式
\[ F\left(x, y, \frac{\d y}{\d x}, \cdots, \frac{\d^n y}{\d x^n} \right) = 0 \]
如方程的左端为 $y$ 及各阶导数的一次有理整式,则称为 $n$ 阶线性微分方程,否则称为非线性微分方程。一般的 $n$ 阶线性微分方程有形式
\[ \frac{\d^n y}{\d x^n} + a_{n-1}(x) \frac{\d^{n-1} y}{\d x^{n-1}} + \cdots + a_1(x) \frac{\d y}{\d x} + a_0(x) y = f(x) \]
这里 $a_i(x), f(x)$ 是关于 $x$ 的已知函数。

如果函数 $y = \varphi(x)$ 带入方程后,能使其变为恒等式,则称函数 $y=\varphi(x)$ 为方程的解。如果隐函数 $\Phi(x, y) = 0$ 是其解,则称隐式解,也可以不加区分的称为解。

含有 $n$ 个独立的任意常数 $\seq{C}{n}$ 的解
\[ y = \varphi(x, \seq{C}{n}) \]
称为 $n$ 阶方程的通解,其中独立性指 $\varphi$ 及各阶偏导数关于 $n$ 个常数 $\seq{C}{n}$ 的 Jacobi 行列式不为 $0$。为了确定微分方程的一个特解,需要给出定解条件。常见的定解条件是初值条件和边值条件,其中初值条件指:当 $x = x_0$ 时,有
\[ y=y_0, y' = y_0^{(1)}, \cdots, y^{(n-1)} = y_0^{(n-1)} \]
满足初值条件的解称为微分方程的特解。

\section{变量分离微分方程}

形如
\[ \frac{\d y}{\d x} = f(x) g(y) \]
的方程,称为变量分离方程,假如 $g(y) \neq 0$ 我们可以很容易的分离它
\[ \frac{\d y}{g(y)} = f(x) \d x \]
两边同时积分即可。另外,不要忘记当 $g(y) = 0$ 时也有解 $y=y_0$。

有两种常见的变式:

(1)形如
\[ \frac{\d y}{\d x} = g\left(\frac{y}{x}\right) \]
的方程,记作齐次微分方程。做变量代换 $u = \frac{y}{x}$,有
\[ \frac{\d u}{\d x} = \frac{1}{x} \left(\frac{\d y}{\d x} -\frac{y}{x} \right) = \frac{g(u) - u}{x} \]
就变成变量分离的了。

(2)形如
\[ \frac{\d y}{\d x} = g \left(\frac{a_1 x + b_1 y + c_1}{a_2 x + b_2 y + c_2}\right) \]
分为三种情况讨论:
\begin{enumerate}
	\item 如果
	      \[ \frac{a_1}{a_2} = \frac{b_1}{b_2} = \frac{c_1}{c_2} = k \]
	      则比较显然。
	\item 如果
	      \[ \frac{a_1}{a_2} = \frac{b_1}{b_2} = k \neq \frac{c_1}{c_2} \]
	      令 $u = a_2 x + b_2 y$,此时有
	      \[ \frac{\d u}{\d x} = g \left(a_2 + b_2 \frac{k u + c_1}{u + c_2}\right) \]
	      是变量分离方程。
	\item 对于剩余的情况,把分子分母看成两条不相交的直线,尝试平移到原点。设交点为 $(x, y) = (x_0, y_0)$,有
	      \[ \frac{\d y}{\d x} = \frac{\d (y - y_0)}{\d (x - x_0)} = g \left(\frac{a_1 (x - x_0) + b_1 (y - y_0)}{a_2(x - x_0) + b_2 (y - y_0)} \right) \]
	      也变成了齐次形式。
\end{enumerate}

\section{一阶线性微分方程}

一阶线性常微分方程的一般形式是
\[ \frac{\d y}{\d x} = p(x) y + q(x) \]
其中 $p(x), q(x)$ 在给定区间上连续的函数。当 $q(x) = 0$ 时,称为一阶线性齐次常微分方程;否则称一阶线性非齐次常微分方程。

以下用大写字母简记为积分,方便记忆。

\paragraph{一阶线性其次常微分方程} 即
\[ \frac{\d y}{\d x} = p(x) y \]
那么我们可以分离变量
\[ \frac{\d y}{y} = p(x) \d x \]
两边积分得到
\[ \ln |y| = \int_{x_0}^{x} p(t) \d t + C_1 \]
其中 $C_1$ 是任意常数;若引入任意非零常数 $C_2 = \pm \ee^{C_1}$,就得到
\[ y(x) = C_2 \ee^{\int_{x_0}^{x} p(t) \d t } = C_2 \ee^{P(x)} \]
显然 $y = 0$ 也是一解,故 $C_2$ 是可以取任意常数的。

这样对于给定初值的问题
\[ \frac{\d y}{\d x} = p(x) y, \quad y(x_0) = y_0 \]
的通解为
\[ y(x) = y_0 \ee^{ \int_{x_0}^{x} p(t) \d t } = y_0 \ee^{P(x)} \]

除此之外,我们有以下性质:

\begin{itemize}
	\item 方程的解要么恒为零,要么恒不为零。
	\item 方程任何有限个解的线性组合仍是解,所有解构成一个一维线性空间。
\end{itemize}

\paragraph{一阶线性非齐次常微分方程} 一阶线性非齐次常微分方程为
\[ \frac{\d y}{\d x} = p(x) y + q(x) \]
考虑套用之前的公式,设
\[ u(x) = y \ee^{-\int_{x_0}^{x} p(t) \d t} = y \ee^{-P(x)} \]
注意到
\[ \frac{\d u}{\d x} \cdot \ee^{\int_{x_0}^{x} p(t) \d t} = y' - p(x) y = q(x) \]
可以得到 $y(x)$ 的解
\[ y(x) = \ee^{\int_{x_0}^{x} p(t) \d t} \left(  \int \ee^{-\int p(s) \d s} q(s) \d s + C \right)
	= \ee^{P(x)} \left( \int \ee^{-P(x)} q(x) + C \right)  \]

可以发现一个有趣的性质,非齐次方程的解可以由齐次形式下的解的“常数变易”得到。因此先考虑解齐次形式,再变易常数为函数 $c(x)$,这种解法记作常数变易法。

\subsection{各种变形}

\paragraph{Bernoulli 微分方程} 形如
\[ \frac{\d y}{\d x} = p(x) y + q(x) y^n, \quad n \neq 0,1 \]
的方程称为 Bernoulli 微分方程。设 $y \neq 0$,得到
\[ y^{-n} \frac{\d y}{\d x} = \frac{\d (y^{1-n})}{(1-n)\d x} = y^{1-n}p(x) + q(x) \]
换元 $u = y^{1-n}$ 即可。

\paragraph{Riccati 方程}
形如
\[ \frac{\d y}{\d x} + p(x) y + q(x)y^2 = r(x) \]
的方程称为 Riccati 微分方程。设 $\phi(x)$ 是其一个特解,令 $u = y - \phi$,得到
\[ \frac{\d u}{\d x} + (p + 2 \phi q) u + q u^2 = 0 \]
即是一个 Bernoulli 方程。

\paragraph{Euler 方程}
形如
\[ x^2 \frac{\d^2 y}{\d x^2} + p x \frac{\d y}{\d x} + q = r(x) \]
当 $x > 0$ 时,令 $x = \ee^u$,则
\[ \frac{\d y}{\d x} = \frac{\d y}{x \d u}, \quad \frac{\d^2 y}{\d x^2} = \frac{1}{x^2} \left( \frac{\d^2 y}{\d u^2} - \frac{\d y}{\d u} \right) \]
因此方程化为
\[ \frac{\d^2 y}{\d u^2} + (p-1) \frac{\d y}{\d u} + qy = r(\ee^u) \]

\subsection{可降阶的二阶微分方程}

\paragraph{不含 $y$} 若微分方程不含 $y$,即方程可表示为
\[ y'' = f(x, y') \]
则可以构造 $y'=p$,从而变成一阶的微分方程。

\paragraph{不含 $x$} 即方程可以表示为
\[ y'' = f(y, y') \]
则可以令 $p = y'$,得到 $y'' = \frac{p \d p}{\d y}$,则原方程变为一阶方程。

\section{恰当微分方程}

有时可以考虑答案是全微分的形式。设
\[ f(x, y) \d x + g(x, y) \d y = 0 \]
并假设 $f, g$ 连续。设其左端恰好是某个二元函数 $u(x, y)$ 的全微分
\[ f(x, y) \d x + g(x, y) \d y = \d u(x, y) = \frac{\partial u}{\partial x} \d x + \frac{\partial u}{\partial y} \d y \]
则称其为恰当微分方程。容易验证其通解为 $u(x, y) = C$。

注意到
\[ \frac{\partial f}{\partial y} = \frac{\partial^2 u}{\partial y \partial x} = \frac{\partial^2 u}{\partial x \partial y} = \frac{\partial g}{\partial x} \]
故其是该方程为恰当微分方程的充要条件。

再考虑如何求出 $u$,先以 $y$ 为参数
\[ u = \int f(x, y) \d x + \varphi(y) \]
回带得到
\[ \frac{\partial u}{\partial y} = g(x, y) = \frac{\d \varphi(y)}{\d y} + \frac{\partial}{\partial y}\int f(x, y) \d x \]
总之有解
\[ u = \int f(x, y) \d x + \int \left( g(x, y) - \frac{\partial}{\partial y}\int f(x, y) \d x  \right) \d y \]

这个形式过于复杂,一般情况下多瞪眼可能更快。下面是一些常见的表

\[ \begin{aligned}
		y \d x + x \d y                   & = \d x y                              \\
		\frac{y \d x - x \d y}{y^2}       & = \d \frac{x}{y}                      \\
		\frac{y \d x - x \d y}{xy}        & = \d \ln\left|\frac{x}{y}\right|      \\
		\frac{y \d x - x \d y}{x^2 + y^2} & = \d \arctan \frac{x}{y}              \\
		\frac{y \d x - x \d y}{x^2 - y^2} & = \d \ln\left| \frac{x-y}{x+y}\right| \\
	\end{aligned} \]

假如一个方程乘以 $\mu(x, y)$ 就成为了恰当微分方程,则称 $\mu$ 是原方程的积分因子。假若方程有解存在,则必有积分因子存在,而且不是唯一的。因此之前的很多方法都可以改写成积分因子法,但从零瞪出一个因子非常有技巧性,实战中不实用。

\section{齐次线性微分方程}

一般的 $n$ 阶线性微分方程有形式
\[ \frac{\d^n y}{\d x^n} + a_{n-1}(x) \frac{\d^{n-1} y}{\d x^{n-1}} + \cdots + a_1(x) \frac{\d y}{\d x} + a_0(x) y = f(x) \]

倘若 $f(x) \equiv 0$ 则是齐次线性微分方程,否则为非齐次。对其分析过程比较复杂,这里直接给出结论:方程的线性无关解最大个数为 $n$,所有解构成一个 $n$ 维线性空间。其非齐次的通解可以将各个系数用常数变易法待定,计算即可。

注意到
\[ \ee^{i x} = \cos x + i \sin x \]
反推得到
\[ \sin x = \frac{1}{2}(\ee^{i x} + \ee^{-i x}), \quad \cos x = \frac{1}{2} (\ee^{ix} - \ee^{-ix}) \]
因此我们在计算时完全可以将指数和三角统一起来,叙述过程时再改成三角。

\subsection{常系数}

设所有系数都是常数,即方程为
\[ L[y] \equiv \frac{\d^n y}{\d x^n} + a_{n-1} \frac{\d^{n-1} y}{\d x^{n-1}} + \cdots + a_1 \frac{\d y}{\d x} + a_0 y = 0 \]
考虑代入
\[ L[\ee^{\lambda x}] = (\lambda^n + a_{n-1} \lambda^{n-1} + \cdots + a_0) \ee^{\lambda x} = F(\lambda) \ee^{\lambda x} \]
其中方程 $F(\lambda)$ 是关于 $\lambda$ 的 $n$ 次多项式,我们求解 $F(\lambda) = 0$ 即可,因此又称为特征方程,其根称为特征根。假如所有的根都不是重根,那么可以直接得到 $n$ 个线性无关的解 $y_i = \ee^{\lambda_i x}$。

假若方程存在 $k$ 重根 $\mu$,即特征方程存在因子 $(\lambda - \mu)^k$,考虑平移代换 $z = y \ee^{\mu x}$,继续代入 $y = \ee^{\lambda x}$ 有
\[ L[y \ee^{\mu x}] = L[\ee^{(\lambda + \mu) x}] = F(\lambda + \mu) \ee^{(\lambda + \mu) x} \]
显然 $F(\lambda + \mu)$ 含有 $k$ 重零根,对比系数可以得到方程 $L[y] = 0$ 的 $0 \sim k-1$ 项系数为 $0$,即方程形如
\[ L[z] = \frac{\d^n y}{\d x^n} + b_{n-1} \frac{\d^{n-1} y}{\d x^{n-1}} + \cdots + b_{k} \frac{\d^k y}{\d x^k} = 0 \]
可以观察出其 $k$ 个线性无关的解 $y = 1, x, x^2, \cdots, x^{k-1}$,故原方程的 $k$ 个解为
\[ z = \ee^{\mu x}, x \ee^{\mu x}, \cdots, x^{k-1} \ee^{\mu x} \]

\subsection{常见二阶微分方程}

常见的二阶常系数微分方程的一般形式是
\[ y'' + p y' + qy = f(x) \]
首先求其对应的齐次微分方程的通解。

令 $\Delta = p^2 - 4q > 0$,设其两个实根为 $r_1, r_2$,可得通解
\[ y = C_1 \ee^{r_1 x} + C_2 \ee^{r_2 x} \]
若 $\Delta = 0$,存在重根 $r_1$,则通解为
\[ y = (C_1 + C_2 x) \ee^{r x} \]
若 $\Delta < 0$,存在一对共轭虚根 $\alpha \pm \beta i$,则通解为
\[ y = \ee^{a x} (C_1 \cos \beta x + C_2 \sin \beta x) \]

当自由项
\[ f(x) = P_n(x) \ee^{ax} \]
时,特解要设为
\[ y^* = x^k \ee^{ax} Q_n(x) \]
其中 $k$ 为特征方程的根和 $a$ 的重合次数(这个规则很神秘,可以从微分算子法的角度记忆),$P_n(x), Q_n(x)$ 为关于 $x$ 的度 $n$ 多项式。

当自由项为
\[ f(x) = \ee^{ax} \left( P_n(x) \sin b x \right) \]
时,特解要设为
\[ y^* = x^k \ee^{ax} (Q_n(x) \cos bx + R_n(x) \sin bx) \]
其中 $k$ 为特征方程的根和 $a \pm bi$ 是否重合(重合一定重俩),$P_n(x), Q_n(x), R_n(x)$ 为关于 $x$ 的度 $n$ 多项式。

\section{Gronwall 定理}

考虑区间 $[a, b]$ 上的微分不等式
\[ y'(x) + p(x) y(x) \leqslant f(x) \]
令 $P(x)$ 为 $p(x)$ 的原函数,类似于普通微分方程的解法,有
\[ \frac{\d }{\d x} \left(y(x) \ee^{P(x)}\right) = (y'(x) + p(x) y) \ee^{P(x)} \leqslant f(x) \ee^{P(x)} \]
选取 $u \in [a, b]$,在 $[a, u]$ 上积分得到
\[ y(u) \ee^{P(u)} - y(a) \leqslant \int_{a}^{u} f(s) \ee^{P(s)} \d s = \ee^{P(u)} \int_{a}^{u} f(s) \ee^{P(s) - P(u)} \d s  \]
从而得到
\[ y(x) \leqslant y(a) \ee^{-P(u)} + \int_{a}^{u} f(s) \ee^{P(s) - P(u)} \d s \]

特殊的,令 $f \equiv 0$,有
\[ y(x) \leqslant y(a) \ee^{-P(x)} \]
如果 $y_1, y_2$ 满足如下微分不等式
\[ y_1' + p(x) y_1 \leqslant y_2' + p(x) y_2, \quad y_1(a) = y_2(a) \]
则一定 $y_1 \equiv y_2$。
