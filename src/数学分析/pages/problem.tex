\chapter{习题}

\section{函数、极限}

\begin{problem}[000001]
设 $a_1=1,a_k=k(a_{k-1}+1)$,试计算
\[ \lim_{n\to \infty}\prod_{k=1}^n\left(1+\frac{1}{a_k}\right)\]
\end{problem}
\begin{solution}
	先变形
	\[ \left(1+\frac{1}{a_k}\right)=\frac{a_{n+1}}{ka_n}\]
	累乘可以化简
	\[ \prod_{k=1}^n\left(1+\frac{1}{a_k}\right) = \frac{a_{n+1}}{(n+1)!}\]
	注意到
	\[ \frac{a_{n+1}}{(n+1)!}-\frac{a_{n}}{n!} = \frac{a_{n+1}-(n+1)a_n}{(n+1)!} = \frac{1}{n!}\]
	故
	\[ \lim_{n\to \infty}\prod_{k=1}^n\left(1+\frac{1}{a_k}\right) = \lim_{n\to \infty}\left(1+\frac{1}{2!}+\cdots+\frac{1}{n!}\right) = \ee\]
\end{solution}

\begin{problem}[000002]
设 $x_1 = 2, x_n + (x_n - 4)x_{n-1} = 3(n = 2, 3, \cdots)$,求 $\displaystyle\lim_{n\to \infty} x_n$。
\end{problem}

\begin{solution}
	显然是考不动点。

	考虑方程
	\[ x + (x-4)x - 3 = x^2 - 3x - 3 = 0 \]
	的解,取其中一解 $x_0 = \frac{3 + \sqrt{21}}{2}$。接下来考察单调性,设 $x_{n-1} \in [2, x_0)$,有
	\[ x_n - x_{n-1} = 4 - x_{n-1} - \frac{1}{x_n - 1} = -\frac{x_{n-1}^2 - 3x_{n-1} - 3}{x_{n-1} - 1} > 0 \]
	故序列 $\{x_n\}$ 单调递增且 $x_n \in [2, x_0)$。设极限为 $A$,解方程
	\[ A^2 - 3A - 3 = 0, \quad A = x_0 = \frac{3 + \sqrt{21}}{2} \]
\end{solution}

\begin{problem}[000003]
是否存在这样的函数,它在区间 $[0,1]$ 上每点取有限值,在此区间的任何点的任一邻域内无界。
\end{problem}
\begin{solution}
	构造
	\[ f(x) = \begin{cases}
			n, & x=\dfrac{m}{n},m,n\ \text{为互质整数} \\
			0, & x\ \text{为无理数}
		\end{cases} \]
\end{solution}

\begin{problem}[000004]
设 $f,g$ 是 $\mathbb{R}$ 上的实函数,且
\[ f(x+y)+f(x-y) = 2f(x)g(y)\]
在 $\mathbb{R}$ 上 $f(x)$ 不恒等于零但有界,试证:$|g(y)|\leqslant 1$
\end{problem}
\begin{solution}
	令 $M=\sup|f(x)|$,则有
	\[ 2M\geqslant |f(x+y)|+|f(x-y)| \geqslant |f(x+y)+f(x-y)| = 2|f(x)||g(y)| \]
	设存在 $y_0$ 使得 $|g(y_0)|=1+2\delta>1$。由上确界的定义知存在 $x_0$ 有
	\[ M \geqslant |f(x_0)| > \frac{M}{\delta+1}\]
	故
	\[ 2|f(x_0)||g(y_0)| > \frac{2(1+2\delta)M}{1+\delta} > 2M\]
	因此矛盾,故恒有 $|g(y)|\leqslant 1$。
\end{solution}


\begin{problem}[000005]
设 $f$ 是闭区间 $[a,b]$ 上的增函数(但不一定连续),如果 $f(a) \geqslant a,f(b) \leqslant b$,试证: $\exists x_0 \in [a,b]$,使得 $f(x_0) = x_0$。
\end{problem}
\begin{solution}
	设 $A=\{x \mid f(x) \geqslant x\}$,由题知 $a\in A$ 故 $A$ 非空。又 $f$ 定义在 $[a,b]$ 上,故 $A$ 有界。因此设 $x_0=\sup A\in [a,b]$ 是有意义的。又 $f(x)\in[a,b]$ 在定义域内,分类讨论如下

	1. 若 $y_0=f(x_0) > x_0$,由单调性知
	\[ f(y_0)=f(f(x_0)) \geqslant f(x_0) = y_0\]
	故 $y_0\in A$。这意味着 $\sup A \geqslant y_0 >x_0$,矛盾。

	2. 若 $y_0=f(x_0) < x_0$,由确界定义知 $\exists x_1\in A$ 使 $y_0<x_1\leqslant x_0$,由单调性知
	\[ f(x_1)\leqslant f(x_0)=y_0 <x_1\]
	这意味着 $x_1\notin A$,矛盾。

	故 $y_0=f(x_0)=x_0$,此时 $x_0 = \sup\{x \mid f(x) \geqslant x\}$。

	注意 $x_0$ 不一定在 $A$ 中,即 $f(x_0) \geqslant x_0$ 不一定成立。
\end{solution}


\begin{problem}[000006]
设 $f(x)$ 是定义在 $\mathbb{R}$ 上的函数且对任意 $x,y$有
\[ |xf(y)-yf(x)| \leqslant M|x|+M|y|\]
其中 $M > 0$。求证:存在常数 $a$ 使得对任意 $x$ 有 $|f(x)-ax| \leqslant M$
\end{problem}
\begin{solution}
	当 $x=0$ 时,有 $|f(0)|\leqslant M$。而当 $xy\ne 0$ 时,恒有
	\[ \left| f(x)-\frac{f(y)}{|y|}x \right| \leqslant M \left(1+\frac{|x|}{|y|}\right)\]
	若 $a$ 不存在,即对任意的 $a$ 存在 $x_0$ 使
	\[ |f(x_0)-ax_0|=M(1+2\delta)>M\]
	那么取 $a = \dfrac{f(y_0)}{|y_0|}$,当 $y_0=\dfrac{|x_0|}{\delta}$ 时,有
	\[ \left| f(x)-\frac{f(y_0)}{|y_0|}x \right| \leqslant M \left(1+ \frac{|x_0|}{|y_0|}\right)=M(1+\delta)\]
	因此矛盾,故存在 $a$。
\end{solution}

\begin{problem}[000007]
设 $\displaystyle\lim_{n\to\infty}a_n=A$,求证:$\displaystyle\lim_{n\to\infty}\frac{\sum a_n}{n}=A$。
\end{problem}

\begin{solution}
	即对于任给的 $\eps>0$,存在 $n>N_1$ 使得
	\[ |a_n-A|<\dfrac{\eps}{2}\]
	那么变形有
	\[ \left|\frac{\sum a_n}{n}-A\right| \leqslant \frac{\sum |a_n-A|}{n} = \frac{\sum_{k=1}^{N_1} |a_k-A|}{n} + \frac{\sum_{k=N_1+1}^{n} |a_k-A|}{n}\]
	注意到 $\sum_{k=1}^{N_1} |a_k-A|$ 已经为定值,从而存在 $n>N_2$ 使得
	\[ \frac{\sum_{k=1}^{N_1}|x_k-A|}{n}<\frac{\eps}{2}\]
	因此当 $n>\max\{N_1,N_2\}$ 时有
	\[ LHS < \frac{\eps}{2}+\frac{n-N_1}{n}\times \frac{\eps}{2} < \frac{\eps}{2}+\frac{\eps}{2} = \eps \]
\end{solution}

\begin{problem}[000009]
设 $f(z)$ 在 $[0,1]$ 上具有一阶连续导数,$f(0) = 0$,证明:存在$\xi \in [0,1]$,使得
\[ f'(\xi) = 2\int_0^1f(x) \d x \]
\end{problem}

\begin{solution}
	TODO。设上下界 $m, M$,中值定理。
\end{solution}

\begin{problem}[000010]
设 $f(x)$ 在 $[0, 1]$ 上连续,在 $(0, 1)$ 内可导,且 $f(0) = 0, f(1) = 1$,证明存在不同的 $\xi_1, \xi_2 \in (0, 1)$,使得
\[ \frac{1}{f'(\xi_1)} + \frac{1}{f'(\xi_2)} = 2 \]
\end{problem}

\begin{solution}
	由于连续,即存在 $x_0 \in (0, 1)$ 使得 $f(x_0) = \frac{1}{2}$。由中值定理,存在 $\xi_1 \in (0, x_0)$ 使得
	\[ f'(\xi_1) = \frac{f(x_0) - f(0)}{x_0 - 0} = \frac{1}{2x_0}
	\]
	同理,存在 $\xi_2 \in (x_0, 1)$ 使得
	\[ f'(\xi_2) = \frac{f(1) - f(x_0)}{1 - x_0} = \frac{1}{2(1 -x_0)} \]
	故
	\[ \frac{1}{f'(\xi_1)} + \frac{1}{f'(\xi_2)} = 2x_0 + 2(1 - x_0) = 2 \]

\end{solution}

\begin{problem}[000011]
设 $f(x)$ 在 $[0, 3]$ 上连续,在 $(0, 3)$ 内可导,且 $f(0) + f(1) + f(2) = 3, f(3) = 1$,证明存在 $\xi \in (0, 3)$,使得 $f'(\xi) = 0$。
\end{problem}

\begin{solution}
	TODO。设 $f(c) = f(3) = 1, c \in (0, 2)$。套中值定理。
\end{solution}

\begin{problem}[000012]
设 $f(x)$ 在 $[0, 1]$ 上连续,在 $(0, 1)$ 内可导,且
\[ f(1) = k \int_{0}^{\frac{1}{k}} x \ee^{1-x} f(x) \d x (k > 1) \]
证明至少存在一点 $\xi \in (0, 1)$,使得 $f'(\xi) = (1 - \xi^{-1})f(\xi)$。
\end{problem}

\begin{solution}
	TODO。构造 $F(x) = x \ee^{1-x} f(x)$。套中值定理。
\end{solution}

\begin{problem}[000013]
设 $f(x)$ 在 $[0, 1]$ 上连续,在 $(0, 1)$ 内二阶可导,过点 $A(0, f(0))$ 与 $B(1, f(1))$ 的直线与曲线 $y=f(x)$ 相交于点 $C(c, f(c))$,其中 $0 < c < 1$,证明存在 $\xi \in (0, 1)$,使得 $f''(\xi) = 0$。
\end{problem}

\begin{solution}
	TODO。构造 $F(x) = (1-x)f(0) + xf(1) - f(x)$,有 $F(0) = F(1) = F(c) = 0$。套中值定理。
\end{solution}

\begin{problem}[000016]
设 $f(x)$ 在 $[a, b]$ 上连续,在 $(a, b)$ 上导函数连续,且存在 $c \in (a, b)$ 使得 $f'(c) = 0$。证明:存在 $\xi \in (a, b)$,使得
\[ f'(\xi) - f(\xi) + f(a) = 0 \]
\end{problem}

\begin{solution}
	几何角度很直观,即 $\frac{f'(x) - f(a)}{f'(x)}$ 存在值为 $1$ 的时候。但是零点很恼人,严谨的说明是要费一番功夫的。

	积分构造
	\[ F(x) = \frac{f(x) - f(a)}{\ee^x}, \quad F'(x) = \frac{f'(x) - f(x) + f(a)}{\ee^x} = \frac{f'(x)}{\ee^x} - F(x) \]
	问题转化为存在 $\xi$ 使得 $F'(\xi) = 0$。

	易知 $F(a) = 0, F'(c) = -F(c)$。因此由 Lagrange 中值定理得,存在 $x_0 \in (a, c)$ 使得
	\[ F'(x_0) = \frac{F(c) - F(a)}{c - a} \]
	由于
	\[ F'(x_0) F'(c) = \frac{-F(c)^2}{c- a} < 0 \]
	因此由介值定理可知,必存在 $\xi \in (x_0, c)$ 使得 $F'(d) = 0$。
\end{solution}

\begin{problem}[000017]
已知 $\lim\limits_{x \to +\infty} f(x)$ 存在,且
\[ f(x) = \frac{x^{1+x}}{(1+x)^x} - \frac{x}{\ee} + 2 \lim_{x \to \infty} f(x) \]
求 $f(x)$。
\end{problem}

\begin{solution}
	显然先取 $t = \frac{1}{x}$,设极限为 $A$,则
	\[ A = 2A + \lim_{t \to 0^+} \left( \frac{1}{t (t+1)^{\frac{1}{t}}} - \frac{1}{t \ee} \right) \]
	故取等价无穷小得
	\[ -A = \lim_{t \to 0^+} \frac{1 - \exp\left( \frac{\ln(t + 1)}{t} - 1\right)}{t \exp\frac{\ln(t + 1)}{t}} = \lim_{t \to 0^+} \frac{1 - \frac{\ln(t + 1)}{t}}{t \ee} = \frac{1}{2 \ee} \]
	因此
	\[ f(x) = \frac{x^{1+x}}{(1+x)^x} - \frac{x + 1}{\ee} \]
\end{solution}

\begin{problem}[000018]
(1)证明:当 $x \geqslant 1$ 时,有
\[ \frac{1}{x+1} < \ln\left(1 + \frac{1}{x}\right) < \frac{1}{x} \]
(2)证明:设 $f(x)$ 在 $[1, \infty)$ 连续可导,且
\[ f'(x) = \frac{1}{1 + f^2(x)} \left( \sqrt{\frac{1}{x}} - \sqrt{\ln\left(1 + \frac{1}{x}\right)} \right) \]
有 $\lim\limits_{x \to +\infty} f(x)$ 存在。
\end{problem}

\begin{solution}
	(1) 显然求导即证。

	(2) 显然 $f'(x) > 0$。由于 $1 + f^2(x) \geqslant 1$,因此
	\[ 0 < f'(x) < \frac{1}{\sqrt{x}} - \frac{1}{\sqrt{x+1}} < \frac{1}{2 x \sqrt{x}} \]
	故 $f(x)$ 在 $[1, \infty)$ 上单调递增,积分有
	\[ f(t) - f(1) = \int_{1}^{t} f'(x) \d x < \int_{1}^{t} \frac{1}{2 x \sqrt{x}} \d x = 1 - \frac{1}{\sqrt{t}} < 1 \]
	由于 $f$ 单调递增且有界,故极限存在。
\end{solution}

\begin{problem}[000019]
设数列 $\{x_n\}$ 满足 $0 < x_n < \frac{\pi}{2}$,且
\[ \cos x_{n+1} - x_{n+1} = \cos x_n \]

(1) 计算 $\lim\limits_{n \to \infty} x_n$。

(2) 计算 $\lim\limits_{n \to \infty} \frac{x_{n+1}}{x_n^2}$。

\end{problem}

\begin{solution}
	(1) 因为
	\[ \cos x_{n+1} - \cos x_n = x_{n+1} > 0 \]
	且 $0 < x_n < \frac{\pi}{2}$,因此 $0 < x_{n+1} < x_n$。故极限存在。

	设极限为 $a$,由 $\cos a - a = \cos a$,易得 $a = 0$。

	(2) 由于 $\cos x \sim 1 - \frac{x^2}{2}$,故
	\[ \lim_{n \to \infty} \frac{x_{n+1}}{x_n^2} = \lim_{n \to \infty} \frac{x_{n+1}}{2 - 2 \cos x_{n}} = \lim_{n \to \infty} \frac{x_{n+1}}{2 - 2 \cos x_{n+1} + 2x_{n+1}} = \frac{1}{2} \]
\end{solution}

\begin{problem}[000020]
设
\[ f_n(x) = \cos x + \cos^2 x + \cdots + \cos^n x \]

(1) 证明:对于每个 $n$,方程 $f_n(x) = 1$ 在 $[0, \frac{\pi}{3})$ 内有且仅有一个实根。

(2) 证明:$\lim\limits_{n \to \infty} x_n$ 存在,并求其值。
\end{problem}

\begin{solution}
	(1)设
	\[ g_n(x) =  - 1 + \sum_{i=1}^n x^i = \frac{2x - 1 - x^{n+1}}{1 - x}, \quad x \neq 1 \]
	因此 $f_n(x) = g_n(\cos x)$,由于 $\cos x$ 单调,即证 $g_n(x) = 0$ 在 $(\frac{1}{2}, 1]$ 内有且仅有一个实根。


	令 $h_n(x) = 2x - 1 - x^{n+1}$,求导
	\[ h'_n(x) = 2 - (n + 1)x^n, \quad h''_n(x) = - n(n+1) x^{n-1} \leqslant 0 \]
	可知 $h'_n(x)$ 单调递减,解得零点 $x_0 = \sqrt[n]{\frac{2}{n + 1}}$。

	当 $x \in (x_0, 1]$ 时,$h_n(x)$ 单调递减,知 $h_n(x) \geqslant h_n(1) = 0$。当 $x \in (x_0, 1)$ 时,$h_n(x)$ 单调递增。又 $h_n(\frac{1}{2}) <  h_n(1) = 1 < h_n(x_0)$,故存在唯一解。

	(2)设 $y_n = \cos x_n$。注意到
	\[ h_{n+1}(y_n) = 2 y_n - 1 - y_{n}^{n+1} > 2 y_n - 1 - y_{n}^n = h_n(y_n) = 0 \]
	故 $y_{n+1} < y_{n}$。因此 $y_n$ 单调递减且有界,故收敛。设极限为 $A$,显然 $A \in (\frac{1}{2}, 1)$,得到
	\[ 0 = 2A - 1 - \lim_{n \to \infty} x_n^{n+1} = 2A - 1 \]
	故 $A = \frac{1}{2}$,反推 $x_n \to \frac{\pi}{3}$。

	(2)其实可以更直接的定出答案。注意到
	\[ h_n\left(\frac{n + 1}{2n}\right) = \frac{1}{n} - \left(\frac{n + 1}{2n}\right)^{n+1} > \frac{1}{n} - \frac{1}{2^{n+1}} > 0 \]
	因此 $\frac{1}{2} < \cos x_n < \frac{n + 1}{2n}$。由夹逼准则知 $x_n \to \frac{\pi}{3}$。

\end{solution}

\begin{problem}[000021]
(1987 卷 I)设函数 $f(x)$ 在闭区间 $[0,1]$ 上可微,对于 $[0, 1]$ 上的每一个 $x$,函数 $f(x)$ 的值都在开区间 $(0, 1)$ 内,且 $f'(x) \neq 1$。证明:在 $(0, 1)$ 内有且仅有一个 $x$,使 $f(x) = x$。
\end{problem}

\begin{solution}
	设 $F(x) = f(x) - x$,注意到
	\[ F(0) = f(0) > 0, \quad F(1) = f(1) - 1 < 0 \]
	因此 $F(x)$ 至少存在一个实根。设存在两个不同的实根 $x_1, x_2$,由 Lagrange 中值定理知存在 $\xi \in (0, 1)$
	\[ f'(\xi) = \frac{f(x_1) - f(x_2)}{x_1 - x_2} = 1 \]
	故矛盾,因此实根唯一。
\end{solution}

\begin{problem}[000022]
设函数
\[ f(x) = x \int_{1}^{0} \ee^{-x^2t^2} \d t \]
则当 $0<a<x<b$ 时有:

A. $xf(x) > af(a)$,B. $bf(b) > x f(x)$,C. $xf(a) > af(x)$,D. $xf(b) > bf(x)$。
\end{problem}

\begin{solution}
	答案 D。首先积分换元
	\[ f(x) = \int_{1}^{0} \ee^{-(xt)^2} \d (xt) = -\int_{0}^{1} \ee^{-x^2} \d x  \]
	剩下的比较直观,就不写了。
\end{solution}

\begin{problem}[000022]
设函数
\[ f(x) = x \int_{1}^{0} \ee^{-x^2t^2} \d t \]
则当 $0<a<x<b$ 时有:

A. $xf(x) > af(a)$,B. $bf(b) > x f(x)$,C. $xf(a) > af(x)$,D. $xf(b) > bf(x)$。
\end{problem}

\begin{solution}
	答案 D。首先积分换元
	\[ f(x) = \int_{1}^{0} \ee^{-(xt)^2} \d (xt) = -\int_{0}^{1} \ee^{-x^2} \d x  \]
	剩下的比较直观,就不写了。
\end{solution}

\begin{problem}[000023]
设函数 $f$ 在 $[0, 1]$ 上二阶可导,且 $\int_{0}^{1} f(x) \d x = 0$,则

A. 当 $f'(x) < 0$ 时,$f(\frac{1}{2}) < 0$。
B. 当 $f''(x) < 0$ 时,$f(\frac{1}{2}) < 0$。

C. 当 $f'(x) > 0$ 时,$f(\frac{1}{2}) < 0$。
D. 当 $f''(x) > 0$ 时,$f(\frac{1}{2}) < 0$。
\end{problem}

\begin{solution}
	答案 D。考虑原函数 $F(x)$ 的 Lagrange 余项 Taylor 公式,令 $x_0 = \frac{1}{2}$,即存在 $\xi \in (0, 1)$ 使得
	\[ F(x) = F(x_0) + F'(x_0)(x-x_0) + \frac{F''(x_0)}{2}(x-x_0)^2 + \frac{F'''(\xi)}{6}(x-x_0)^3 \]
	注意到 $F(0) = F(1)$,带入即
	\[ 4 F'(x_0) + F'''(x_0) = 0 \]
	故 $f''(x) f(\frac{1}{2}) < 0$。
\end{solution}


\begin{problem}[000024]
设函数 $f$ 在 $[-2, 2]$ 上二阶可导,且 $|f| \leqslant 1$。又
\[ \frac{1}{2}[f'(0)]^2 + [f(0)]^3 > \frac{3}{2} \]
证明:存在 $\xi \in (-2,2)$,使得 $f''(\xi) + 3[f(\xi)]^2 = 0$。
\end{problem}

\begin{solution}
	构造
	\[ F(x) = \frac{1}{2}[f'(x)]^2 + [f(x)]^3,\quad F'(x) = f'(x)(f''(x) + 3f(x)) \]
	下面只需证存在 $\xi$ 使得 $F'(\xi) = 0$ 且 $f'(\xi) \neq 0$ 即可。注意到存在 $\eta_1 \in (0, 2)$ 使得
	\[ |f'(\eta_1)| = \frac{|f(2) - f(0)|}{2 - 0} \leqslant 1 \]
	故
	\[ F(\eta_1) \leqslant \frac{1}{2} + 1 = \frac{3}{2} \]
	同理存在 $\eta_2 \in (-2, 0)$ 使得 $F(\eta_2) \leqslant \frac{3}{2}$。又 $F(0) \geqslant \frac{3}{2}$,故存在最大值 $\xi \in (\eta_1, \eta_2)$,且由 Fermat 定理知 $F'(\xi) = 0$。此时 $F(\xi) > \frac{3}{2}$,故 $f'(\xi) \neq 0$。
\end{solution}


\begin{problem}[000025]
设
\[ f(x) = \int_{0}^{x} \ee^{t^2} \d x, \quad x > 0 \]

(1) 证明:对于任意的 $x$ 存在唯一的 $\theta_x \in (0, 1)$ 使得 $f(x) = x f'(x \cdot \theta_x)$。

(2)求 $\lim\limits_{x \to 0^+} \theta_x$。
\end{problem}

\begin{solution}
	(1)不妨将定义域延伸至 $f(0) = 0$。由中值定理知存在 $\xi \in (0, x)$ 使得
	\[ \frac{f(x) - f(0)}{x - 0} = \frac{f(x)}{x} = f'(\xi) \]
	又 $f'(x) = \ee^{x^2} > 0$,故 $\xi$ 唯一,即 $\theta_x = \frac{\xi}{x} \in (0, 1)$。

	(2)容易解得
	\[ \theta_x = \frac{1}{x} \sqrt{\ln \frac{f(x)}{x}} \]
	故极限为
	\[ \lim_{x \to 0^+} \theta_x = \lim_{x \to 0^+} \sqrt{\frac{f(x) - x}{x^3}} = \frac{\sqrt{3}}{3} \]
\end{solution}

\begin{problem}[000026]
已知 $f(x)$ 是二阶可导的正值函数,且 $f(0) = f'(0) = 1$ 并
\[ f(x) f''(x) \geqslant [f'(x)]^2 \]
那么

A. $f(2) \leqslant \ee^2 \leqslant \sqrt{f(1)f(3)}$。B. $\ee^2 \leqslant f(2) \leqslant \sqrt{f(1)f(3)}$。

C. $\sqrt{f(1)f(3)} \leqslant \ee^2 \leqslant f(2)$。D. $\sqrt{f(1)f(3)} \leqslant f(2) \leqslant \ee^2$。
\end{problem}

\begin{solution}
	答案是 B。注意到我们应该对 $\frac{f'}{f}$ 进行改造,构造
	\[ g(x) = \ln f(x) - x \]
	即
	\[ g'(x) = \frac{f'(x)}{f(x)} - 1, \quad g''(x) = \frac{f(x) f''(x) - [f'(x)]^2}{f^2(x)} \geqslant 0 \]
	故 $g'(x) \geqslant g'(0) = 0$,故 $g(x) \geqslant 0$,即 $f(x) \geqslant \ee^x$。注意到凹凸性,由 Jensen 不等式即可比大小。
\end{solution}


\section{积分}

\begin{problem}[000008]
求
\[ \int_{-\infty}^{+\infty} \ee^{-x^2}\d x \]
\end{problem}
\begin{solution}
	\[ \int_{-\infty}^{+\infty} \ee^{-x^2}\d x = \sqrt{\pi} \]
\end{solution}

\begin{problem}[000014]
求
\[ \int \frac{1}{(x^2+a^2)^2} \d x \]
\end{problem}
\begin{solution}
	令 $x = a \tan t$,则
	\[ x^2 + a^2 = a^2 \sec^2 t, \quad \d x = a \sec^2 t \d t \]
	有
	\[ LHS =  \frac{1}{a^3}\int \cos^2 t \d t = \frac{1}{2a^3} \left( \arctan \frac{x}{a} + \frac{ax}{x^2+a^2} \right) \]
\end{solution}


\begin{problem}[000015]
求
\[ \int \frac{\d x}{\sqrt{x^2 + a}} \d x \]
\end{problem}
\begin{solution}
	\[ LHS =  \ln |x + \sqrt{x^2+a}| + C \]
\end{solution}
