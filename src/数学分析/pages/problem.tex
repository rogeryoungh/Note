\chapter{习题}

\section{函数、极限、连续}

\begin{problem}[000001]
设 $a_1=1,a_k=k(a_{k-1}+1)$,试计算
\[ \lim_{n\to \infty}\prod_{k=1}^n\left(1+\frac{1}{a_k}\right)\]
\end{problem}
\begin{solution}
	先变形
	\[ \left(1+\frac{1}{a_k}\right)=\frac{a_{n+1}}{ka_n}\]
	累乘可以化简
	\[ \prod_{k=1}^n\left(1+\frac{1}{a_k}\right) = \frac{a_{n+1}}{(n+1)!}\]
	注意到
	\[ \frac{a_{n+1}}{(n+1)!}-\frac{a_{n}}{n!} = \frac{a_{n+1}-(n+1)a_n}{(n+1)!} = \frac{1}{n!}\]
	故
	\[ \lim_{n\to \infty}\prod_{k=1}^n\left(1+\frac{1}{a_k}\right) = \lim_{n\to \infty}\left(1+\frac{1}{2!}+\cdots+\frac{1}{n!}\right) = \ee\]
\end{solution}

\begin{problem}[000002]
设 $x_1 = 2, x_n + (x_n - 4)x_{n-1} = 3(n = 2, 3, \cdots)$,求 $\displaystyle\lim_{n\to \infty} x_n$。
\end{problem}

\begin{solution}
	显然是考不动点。

	考虑方程
	\[ x + (x-4)x - 3 = x^2 - 3x - 3 = 0 \]
	的解,取其中一解 $x_0 = \frac{3 + \sqrt{21}}{2}$。接下来考察单调性,设 $x_{n-1} \in [2, x_0)$,有
	\[ x_n - x_{n-1} = 4 - x_{n-1} - \frac{1}{x_n - 1} = -\frac{x_{n-1}^2 - 3x_{n-1} - 3}{x_{n-1} - 1} > 0 \]
	故序列 $\{x_n\}$ 单调递增且 $x_n \in [2, x_0)$。设极限为 $A$,解方程
	\[ A^2 - 3A - 3 = 0, \quad A = x_0 = \frac{3 + \sqrt{21}}{2} \]
\end{solution}

\begin{problem}[000003]
是否存在这样的函数,它在区间 $[0,1]$ 上每点取有限值,在此区间的任何点的任一邻域内无界。
\end{problem}
\begin{solution}
	构造
	\[ f(x) = \begin{cases}
			n, & x=\dfrac{m}{n},m,n\ \text{为互质整数} \\
			0, & x\ \text{为无理数}
		\end{cases} \]
\end{solution}

\begin{problem}[000004]
设 $f,g$ 是 $\mathbb{R}$ 上的实函数,且
\[ f(x+y)+f(x-y) = 2f(x)g(y)\]
在 $\mathbb{R}$ 上 $f(x)$ 不恒等于零但有界,试证:$|g(y)|\leqslant 1$
\end{problem}
\begin{solution}
	令 $M=\sup|f(x)|$,则有
	\[ 2M\geqslant |f(x+y)|+|f(x-y)| \geqslant |f(x+y)+f(x-y)| = 2|f(x)||g(y)| \]
	设存在 $y_0$ 使得 $|g(y_0)|=1+2\delta>1$。由上确界的定义知存在 $x_0$ 有
	\[ M \geqslant |f(x_0)| > \frac{M}{\delta+1}\]
	故
	\[ 2|f(x_0)||g(y_0)| > \frac{2(1+2\delta)M}{1+\delta} > 2M\]
	因此矛盾,故恒有 $|g(y)|\leqslant 1$。
\end{solution}


\begin{problem}[000005]
设 $f$ 是闭区间 $[a,b]$ 上的增函数(但不一定连续),如果 $f(a) \geqslant a,f(b) \leqslant b$,试证: $\exists x_0 \in [a,b]$,使得 $f(x_0) = x_0$。
\end{problem}
\begin{solution}
	设 $A=\{x \mid f(x) \geqslant x\}$,由题知 $a\in A$ 故 $A$ 非空。又 $f$ 定义在 $[a,b]$ 上,故 $A$ 有界。因此设 $x_0=\sup A\in [a,b]$ 是有意义的。又 $f(x)\in[a,b]$ 在定义域内,分类讨论如下

	1. 若 $y_0=f(x_0) > x_0$,由单调性知
	\[ f(y_0)=f(f(x_0)) \geqslant f(x_0) = y_0\]
	故 $y_0\in A$。这意味着 $\sup A \geqslant y_0 >x_0$,矛盾。

	2. 若 $y_0=f(x_0) < x_0$,由确界定义知 $\exists x_1\in A$ 使 $y_0<x_1\leqslant x_0$,由单调性知
	\[ f(x_1)\leqslant f(x_0)=y_0 <x_1\]
	这意味着 $x_1\notin A$,矛盾。

	故 $y_0=f(x_0)=x_0$,此时 $x_0 = \sup\{x \mid f(x) \geqslant x\}$。

	注意 $x_0$ 不一定在 $A$ 中,即 $f(x_0) \geqslant x_0$ 不一定成立。
\end{solution}


\begin{problem}[000006]
设 $f(x)$ 是定义在 $\mathbb{R}$ 上的函数且对任意 $x,y$有
\[ |xf(y)-yf(x)| \leqslant M|x|+M|y|\]
其中 $M > 0$。求证:存在常数 $a$ 使得对任意 $x$ 有 $|f(x)-ax| \leqslant M$
\end{problem}
\begin{solution}
	当 $x=0$ 时,有 $|f(0)|\leqslant M$。而当 $xy\ne 0$ 时,恒有
	\[ \left| f(x)-\frac{f(y)}{|y|}x \right| \leqslant M \left(1+\frac{|x|}{|y|}\right)\]
	若 $a$ 不存在,即对任意的 $a$ 存在 $x_0$ 使
	\[ |f(x_0)-ax_0|=M(1+2\delta)>M\]
	那么取 $a = \dfrac{f(y_0)}{|y_0|}$,当 $y_0=\dfrac{|x_0|}{\delta}$ 时,有
	\[ \left| f(x)-\frac{f(y_0)}{|y_0|}x \right| \leqslant M \left(1+ \frac{|x_0|}{|y_0|}\right)=M(1+\delta)\]
	因此矛盾,故存在 $a$。
\end{solution}

\begin{problem}[000007]
设 $\displaystyle\lim_{n\to\infty}a_n=A$,求证:$\displaystyle\lim_{n\to\infty}\frac{\sum a_n}{n}=A$。
\end{problem}

\begin{solution}
	即对于任给的 $\eps>0$,存在 $n>N_1$ 使得
	\[ |a_n-A|<\dfrac{\eps}{2}\]
	那么变形有
	\[ \left|\frac{\sum a_n}{n}-A\right| \leqslant \frac{\sum |a_n-A|}{n} = \frac{\sum_{k=1}^{N_1} |a_k-A|}{n} + \frac{\sum_{k=N_1+1}^{n} |a_k-A|}{n}\]
	注意到 $\sum_{k=1}^{N_1} |a_k-A|$ 已经为定值,从而存在 $n>N_2$ 使得
	\[ \frac{\sum_{k=1}^{N_1}|x_k-A|}{n}<\frac{\eps}{2}\]
	因此当 $n>\max\{N_1,N_2\}$ 时有
	\[ LHS < \frac{\eps}{2}+\frac{n-N_1}{n}\times \frac{\eps}{2} < \frac{\eps}{2}+\frac{\eps}{2} = \eps \]
\end{solution}

\begin{problem}[000017]
已知 $\lim\limits_{x \to +\infty} f(x)$ 存在,且
\[ f(x) = \frac{x^{1+x}}{(1+x)^x} - \frac{x}{\ee} + 2 \lim_{x \to \infty} f(x) \]
求 $f(x)$。
\end{problem}

\begin{solution}
	显然先取 $t = \frac{1}{x}$,设极限为 $A$,则
	\[ A = 2A + \lim_{t \to 0^+} \left( \frac{1}{t (t+1)^{\frac{1}{t}}} - \frac{1}{t \ee} \right) \]
	故取等价无穷小得
	\[ -A = \lim_{t \to 0^+} \frac{1 - \exp\left( \frac{\ln(t + 1)}{t} - 1\right)}{t \exp\frac{\ln(t + 1)}{t}} = \lim_{t \to 0^+} \frac{1 - \frac{\ln(t + 1)}{t}}{t \ee} = \frac{1}{2 \ee} \]
	因此
	\[ f(x) = \frac{x^{1+x}}{(1+x)^x} - \frac{x + 1}{\ee} \]
\end{solution}

\begin{problem}[000019]
设数列 $\{x_n\}$ 满足 $0 < x_n < \frac{\pi}{2}$,且
\[ \cos x_{n+1} - x_{n+1} = \cos x_n \]

(1) 计算 $\lim\limits_{n \to \infty} x_n$。

(2) 计算 $\lim\limits_{n \to \infty} \frac{x_{n+1}}{x_n^2}$。

\end{problem}

\begin{solution}
	(1) 因为
	\[ \cos x_{n+1} - \cos x_n = x_{n+1} > 0 \]
	且 $0 < x_n < \frac{\pi}{2}$,因此 $0 < x_{n+1} < x_n$。故极限存在。

	设极限为 $a$,由 $\cos a - a = \cos a$,易得 $a = 0$。

	(2) 由于 $\cos x \sim 1 - \frac{x^2}{2}$,故
	\[ \lim_{n \to \infty} \frac{x_{n+1}}{x_n^2} = \lim_{n \to \infty} \frac{x_{n+1}}{2 - 2 \cos x_{n}} = \lim_{n \to \infty} \frac{x_{n+1}}{2 - 2 \cos x_{n+1} + 2x_{n+1}} = \frac{1}{2} \]
\end{solution}

\begin{problem}[000030]
设
\[ x_{n+1} = 2 + \frac{1}{x_n}, \quad x_1 = 2 \]
求 $\lim\limits_{n \to \infty} x_n$。
\end{problem}

\begin{solution}
	设 $A = 2 + \frac{1}{A}$,取正解 $A = 1 + \sqrt{2}$。容易由数学归纳法证得 $x_n \in [2, \frac{5}{2}]$,注意到
	\[ |x_{n+1} - A| = \left|2 + \frac{1}{x_n} - A\right| = \frac{1}{Ax_n} |x_n - A| \leqslant \frac{1}{2A} |x_n - A| \]
	因此 $\lim\limits_{n \to \infty} x_n = A = 1 + \sqrt{2}$。

	也可以奇偶分开讨论单调性。
\end{solution}

\begin{problem}[000031]
设 $f(x)$ 在 $[0, 1]$ 上连续,且 $f(0) = f(1)$。证明:对于任意的 $n \in \mathbb{N}$,在 $[0, 1]$ 上至少存在一个 $\xi$ 使得
\[ f\left(\xi + \frac{1}{n}\right) = f(\xi) \]
\end{problem}

\begin{solution}
	设 $F(x) = f(x + \frac{1}{n}) - f(x)$,即证其在 $[0, 1 - \frac{1}{n}]$ 上有零点。注意到取一些点
	\[ \sum_{i=0}^{n - 1} F \left(\frac{i}{n}\right) = f(1) - f(0) = 0 \]
	则要么 $F$ 在这些点上值都为零,否则至少存在两个值异号,由介值定理知也存在零点。
\end{solution}

\begin{problem}[000059]
求极限
\[ \lim_{x \to +\infty} \left(\sqrt[x]{x} - 1\right)^{\frac{1}{\ln x}} \]
\end{problem}

\begin{solution}
	\[ \lim_{x \to +\infty} \left(\sqrt[x]{x} - 1\right)^{\frac{1}{\ln x}} = \frac{1}{\ee} \]
\end{solution}

\begin{problem}[000060]
求极限
\[ \lim_{x \to 0} \frac{\ln(\sin^ x + \ee^x) - 2x}{\ln(x^2 + \ee^{2x}) - 2x} \]
\end{problem}

\begin{solution}
	\[ \lim_{x \to 0} \frac{\ln(\sin^ x + \ee^x) - 2x}{\ln(x^2 + \ee^{2x}) - 2x} = 1 \]
\end{solution}

\section{一元微分学}

\begin{problem}[000009]
设 $f(z)$ 在 $[0,1]$ 上具有一阶连续导数,$f(0) = 0$,证明:存在$\xi \in [0,1]$,使得
\[ f'(\xi) = 2\int_0^1f(x) \d x \]
\end{problem}

\begin{solution}
	TODO。设上下界 $m, M$,中值定理。
\end{solution}

\begin{problem}[000010]
设 $f(x)$ 在 $[0, 1]$ 上连续,在 $(0, 1)$ 内可导,且 $f(0) = 0, f(1) = 1$,证明存在不同的 $\xi_1, \xi_2 \in (0, 1)$,使得
\[ \frac{1}{f'(\xi_1)} + \frac{1}{f'(\xi_2)} = 2 \]
\end{problem}

\begin{solution}
	由于连续,即存在 $x_0 \in (0, 1)$ 使得 $f(x_0) = \frac{1}{2}$。由中值定理,存在 $\xi_1 \in (0, x_0)$ 使得
	\[ f'(\xi_1) = \frac{f(x_0) - f(0)}{x_0 - 0} = \frac{1}{2x_0}
	\]
	同理,存在 $\xi_2 \in (x_0, 1)$ 使得
	\[ f'(\xi_2) = \frac{f(1) - f(x_0)}{1 - x_0} = \frac{1}{2(1 -x_0)} \]
	故
	\[ \frac{1}{f'(\xi_1)} + \frac{1}{f'(\xi_2)} = 2x_0 + 2(1 - x_0) = 2 \]

\end{solution}

\begin{problem}[000011]
设 $f(x)$ 在 $[0, 3]$ 上连续,在 $(0, 3)$ 内可导,且 $f(0) + f(1) + f(2) = 3, f(3) = 1$,证明存在 $\xi \in (0, 3)$,使得 $f'(\xi) = 0$。
\end{problem}

\begin{solution}
	TODO。设 $f(c) = f(3) = 1, c \in (0, 2)$。套中值定理。
\end{solution}

\begin{problem}[000012]
设 $f(x)$ 在 $[0, 1]$ 上连续,在 $(0, 1)$ 内可导,且
\[ f(1) = k \int_{0}^{\frac{1}{k}} x \ee^{1-x} f(x) \d x (k > 1) \]
证明至少存在一点 $\xi \in (0, 1)$,使得 $f'(\xi) = (1 - \xi^{-1})f(\xi)$。
\end{problem}

\begin{solution}
	TODO。构造 $F(x) = x \ee^{1-x} f(x)$。套中值定理。
\end{solution}

\begin{problem}[000013]
设 $f(x)$ 在 $[0, 1]$ 上连续,在 $(0, 1)$ 内二阶可导,过点 $A(0, f(0))$ 与 $B(1, f(1))$ 的直线与曲线 $y=f(x)$ 相交于点 $C(c, f(c))$,其中 $0 < c < 1$,证明存在 $\xi \in (0, 1)$,使得 $f''(\xi) = 0$。
\end{problem}

\begin{solution}
	TODO。构造 $F(x) = (1-x)f(0) + xf(1) - f(x)$,有 $F(0) = F(1) = F(c) = 0$。套中值定理。
\end{solution}

\begin{problem}[000016]
设 $f(x)$ 在 $[a, b]$ 上连续,在 $(a, b)$ 上导函数连续,且存在 $c \in (a, b)$ 使得 $f'(c) = 0$。证明:存在 $\xi \in (a, b)$,使得
\[ f'(\xi) - f(\xi) + f(a) = 0 \]
\end{problem}

\begin{solution}
	几何角度很直观,即 $\frac{f'(x) - f(a)}{f'(x)}$ 存在值为 $1$ 的时候。但是零点很恼人,严谨的说明是要费一番功夫的。

	积分构造
	\[ F(x) = \frac{f(x) - f(a)}{\ee^x}, \quad F'(x) = \frac{f'(x) - f(x) + f(a)}{\ee^x} = \frac{f'(x)}{\ee^x} - F(x) \]
	问题转化为存在 $\xi$ 使得 $F'(\xi) = 0$。

	易知 $F(a) = 0, F'(c) = -F(c)$。因此由 Lagrange 中值定理得,存在 $x_0 \in (a, c)$ 使得
	\[ F'(x_0) = \frac{F(c) - F(a)}{c - a} \]
	由于
	\[ F'(x_0) F'(c) = \frac{-F(c)^2}{c- a} < 0 \]
	因此由介值定理可知,必存在 $\xi \in (x_0, c)$ 使得 $F'(d) = 0$。
\end{solution}

\begin{problem}[000018]
(1)证明:当 $x \geqslant 1$ 时,有
\[ \frac{1}{x+1} < \ln\left(1 + \frac{1}{x}\right) < \frac{1}{x} \]
(2)证明:设 $f(x)$ 在 $[1, \infty)$ 连续可导,且
\[ f'(x) = \frac{1}{1 + f^2(x)} \left( \sqrt{\frac{1}{x}} - \sqrt{\ln\left(1 + \frac{1}{x}\right)} \right) \]
有 $\lim\limits_{x \to +\infty} f(x)$ 存在。
\end{problem}

\begin{solution}
	(1) 显然求导即证。

	(2) 显然 $f'(x) > 0$。由于 $1 + f^2(x) \geqslant 1$,因此
	\[ 0 < f'(x) < \frac{1}{\sqrt{x}} - \frac{1}{\sqrt{x+1}} < \frac{1}{2 x \sqrt{x}} \]
	故 $f(x)$ 在 $[1, \infty)$ 上单调递增,积分有
	\[ f(t) - f(1) = \int_{1}^{t} f'(x) \d x < \int_{1}^{t} \frac{1}{2 x \sqrt{x}} \d x = 1 - \frac{1}{\sqrt{t}} < 1 \]
	由于 $f$ 单调递增且有界,故极限存在。
\end{solution}

\begin{problem}[000020]
设
\[ f_n(x) = \cos x + \cos^2 x + \cdots + \cos^n x \]

(1) 证明:对于每个 $n$,方程 $f_n(x) = 1$ 在 $[0, \frac{\pi}{3})$ 内有且仅有一个实根。

(2) 证明:$\lim\limits_{n \to \infty} x_n$ 存在,并求其值。
\end{problem}

\begin{solution}
	(1)设
	\[ g_n(x) =  - 1 + \sum_{i=1}^n x^i = \frac{2x - 1 - x^{n+1}}{1 - x}, \quad x \neq 1 \]
	因此 $f_n(x) = g_n(\cos x)$,由于 $\cos x$ 单调,即证 $g_n(x) = 0$ 在 $(\frac{1}{2}, 1]$ 内有且仅有一个实根。


	令 $h_n(x) = 2x - 1 - x^{n+1}$,求导
	\[ h'_n(x) = 2 - (n + 1)x^n, \quad h''_n(x) = - n(n+1) x^{n-1} \leqslant 0 \]
	可知 $h'_n(x)$ 单调递减,解得零点 $x_0 = \sqrt[n]{\frac{2}{n + 1}}$。

	当 $x \in (x_0, 1]$ 时,$h_n(x)$ 单调递减,知 $h_n(x) \geqslant h_n(1) = 0$。当 $x \in (x_0, 1)$ 时,$h_n(x)$ 单调递增。又 $h_n(\frac{1}{2}) <  h_n(1) = 1 < h_n(x_0)$,故存在唯一解。

	(2)设 $y_n = \cos x_n$。注意到
	\[ h_{n+1}(y_n) = 2 y_n - 1 - y_{n}^{n+1} > 2 y_n - 1 - y_{n}^n = h_n(y_n) = 0 \]
	故 $y_{n+1} < y_{n}$。因此 $y_n$ 单调递减且有界,故收敛。设极限为 $A$,显然 $A \in (\frac{1}{2}, 1)$,得到
	\[ 0 = 2A - 1 - \lim_{n \to \infty} x_n^{n+1} = 2A - 1 \]
	故 $A = \frac{1}{2}$,反推 $x_n \to \frac{\pi}{3}$。

	(2)其实可以更直接的定出答案。注意到
	\[ h_n\left(\frac{n + 1}{2n}\right) = \frac{1}{n} - \left(\frac{n + 1}{2n}\right)^{n+1} > \frac{1}{n} - \frac{1}{2^{n+1}} > 0 \]
	因此 $\frac{1}{2} < \cos x_n < \frac{n + 1}{2n}$。由夹逼准则知 $x_n \to \frac{\pi}{3}$。

\end{solution}

\begin{problem}[000021]
(1987 卷 I)设函数 $f(x)$ 在闭区间 $[0,1]$ 上可微,对于 $[0, 1]$ 上的每一个 $x$,函数 $f(x)$ 的值都在开区间 $(0, 1)$ 内,且 $f'(x) \neq 1$。证明:在 $(0, 1)$ 内有且仅有一个 $x$,使 $f(x) = x$。
\end{problem}

\begin{solution}
	设 $F(x) = f(x) - x$,注意到
	\[ F(0) = f(0) > 0, \quad F(1) = f(1) - 1 < 0 \]
	因此 $F(x)$ 至少存在一个实根。设存在两个不同的实根 $x_1, x_2$,由 Lagrange 中值定理知存在 $\xi \in (0, 1)$
	\[ f'(\xi) = \frac{f(x_1) - f(x_2)}{x_1 - x_2} = 1 \]
	故矛盾,因此实根唯一。
\end{solution}


\begin{problem}[000023]
设函数 $f$ 在 $[0, 1]$ 上二阶可导,且 $\int_{0}^{1} f(x) \d x = 0$,则

A. 当 $f'(x) < 0$ 时,$f(\frac{1}{2}) < 0$。
B. 当 $f''(x) < 0$ 时,$f(\frac{1}{2}) < 0$。

C. 当 $f'(x) > 0$ 时,$f(\frac{1}{2}) < 0$。
D. 当 $f''(x) > 0$ 时,$f(\frac{1}{2}) < 0$。
\end{problem}

\begin{solution}
	答案 D。考虑原函数 $F(x)$ 的 Lagrange 余项 Taylor 公式,令 $x_0 = \frac{1}{2}$,即存在 $\xi \in (0, 1)$ 使得
	\[ F(x) = F(x_0) + F'(x_0)(x-x_0) + \frac{F''(x_0)}{2}(x-x_0)^2 + \frac{F'''(\xi)}{6}(x-x_0)^3 \]
	注意到 $F(0) = F(1)$,带入即
	\[ 4 F'(x_0) + F'''(x_0) = 0 \]
	故 $f''(x) f(\frac{1}{2}) < 0$。
\end{solution}


\begin{problem}[000024]
设函数 $f$ 在 $[-2, 2]$ 上二阶可导,且 $|f| \leqslant 1$。又
\[ \frac{1}{2}[f'(0)]^2 + [f(0)]^3 > \frac{3}{2} \]
证明:存在 $\xi \in (-2,2)$,使得 $f''(\xi) + 3[f(\xi)]^2 = 0$。
\end{problem}

\begin{solution}
	构造
	\[ F(x) = \frac{1}{2}[f'(x)]^2 + [f(x)]^3,\quad F'(x) = f'(x)(f''(x) + 3f(x)) \]
	下面只需证存在 $\xi$ 使得 $F'(\xi) = 0$ 且 $f'(\xi) \neq 0$ 即可。注意到存在 $\eta_1 \in (0, 2)$ 使得
	\[ |f'(\eta_1)| = \frac{|f(2) - f(0)|}{2 - 0} \leqslant 1 \]
	故
	\[ F(\eta_1) \leqslant \frac{1}{2} + 1 = \frac{3}{2} \]
	同理存在 $\eta_2 \in (-2, 0)$ 使得 $F(\eta_2) \leqslant \frac{3}{2}$。又 $F(0) \geqslant \frac{3}{2}$,故存在最大值 $\xi \in (\eta_1, \eta_2)$,且由 Fermat 定理知 $F'(\xi) = 0$。此时 $F(\xi) > \frac{3}{2}$,故 $f'(\xi) \neq 0$。
\end{solution}


\begin{problem}[000025]
设
\[ f(x) = \int_{0}^{x} \ee^{t^2} \d x, \quad x > 0 \]

(1) 证明:对于任意的 $x$ 存在唯一的 $\theta_x \in (0, 1)$ 使得 $f(x) = x f'(x \cdot \theta_x)$。

(2)求 $\lim\limits_{x \to 0^+} \theta_x$。
\end{problem}

\begin{solution}
	(1)不妨将定义域延伸至 $f(0) = 0$。由中值定理知存在 $\xi \in (0, x)$ 使得
	\[ \frac{f(x) - f(0)}{x - 0} = \frac{f(x)}{x} = f'(\xi) \]
	又 $f'(x) = \ee^{x^2} > 0$,故 $\xi$ 唯一,即 $\theta_x = \frac{\xi}{x} \in (0, 1)$。

	(2)容易解得
	\[ \theta_x = \frac{1}{x} \sqrt{\ln \frac{f(x)}{x}} \]
	故极限为
	\[ \lim_{x \to 0^+} \theta_x = \lim_{x \to 0^+} \sqrt{\frac{f(x) - x}{x^3}} = \frac{\sqrt{3}}{3} \]
\end{solution}

\begin{problem}[000026]
已知 $f(x)$ 是二阶可导的正值函数,且 $f(0) = f'(0) = 1$ 并
\[ f(x) f''(x) \geqslant [f'(x)]^2 \]
那么

A. $f(2) \leqslant \ee^2 \leqslant \sqrt{f(1)f(3)}$。B. $\ee^2 \leqslant f(2) \leqslant \sqrt{f(1)f(3)}$。

C. $\sqrt{f(1)f(3)} \leqslant \ee^2 \leqslant f(2)$。D. $\sqrt{f(1)f(3)} \leqslant f(2) \leqslant \ee^2$。
\end{problem}

\begin{solution}
	答案是 B。注意到我们应该对 $\frac{f'}{f}$ 进行改造,构造
	\[ g(x) = \ln f(x) - x \]
	即
	\[ g'(x) = \frac{f'(x)}{f(x)} - 1, \quad g''(x) = \frac{f(x) f''(x) - [f'(x)]^2}{f^2(x)} \geqslant 0 \]
	故 $g'(x) \geqslant g'(0) = 0$,故 $g(x) \geqslant 0$,即 $f(x) \geqslant \ee^x$。注意到凹凸性,由 Jensen 不等式即可比大小。
\end{solution}

\begin{problem}[000032]
(1988 卷 I)设函数 $f$ 在区间 $[a, b]$ 上连续,且在 $(a, b)$ 内有 $f'(x) > 0$。证明:在 $(a, b)$ 内存在唯一的 $\xi$,使得曲线 $y = f(x)$ 与两直线 $y = f(\xi), x = a$ 所围平面图形面积 $S_1$ 是曲线 $y = f(x)$ 与两直线 $y = f(\xi), x = b$ 所围平面图形 $S_2$ 的三倍。
\end{problem}

\begin{solution}
	考虑面积的函数
	\[  \begin{aligned}
			S_1(t) & = \int_{a}^{t} (f(t) - f(x)) \d x \\
			S_2(t) & = \int_{t}^{b} (f(x) - f(t)) \d x
		\end{aligned} \]
	设 $F(t) = S_1(t) - 3S_2(t)$。注意到
	\[ F(a) = -3S_2(a) < 0, \quad F(b) = S_1(b) > 0 \]
	故至少存在一个 $\xi \in (a, b)$ 使得 $F(\xi) = 0$。再注意到
	\[ \begin{aligned}
			F'(t)
			 & = \frac{\d}{\d t}\left( f(t)(3b - a - 2t) - \int_{a}^{t} f(x) \d x - 3 \int_{t}^{b} f(x) \d x \right) \\
			 & = f'(t) (3b - a - 2t) > 0
		\end{aligned} \]
	故单调,即零点唯一。
\end{solution}

\begin{problem}[000033]
如果 $f$ 在 $[a, b]$ 上导数连续,那么存在 $\xi \in (a, b)$ 满足
\[ \frac{af(b) - bf(a)}{a - b} = f(\xi) - \xi f'(\xi) \]
\end{problem}

\begin{solution}
	注意到
	\[ LHS = \frac{\frac{f(b)}{b} - \frac{f(a)}{a}}{\frac{1}{b} - \frac{1}{a}} \]
	取 $F(x) = \frac{f(x)}{x}, G(x) = \frac{1}{x}$,套 Cauchy 定理即可。
\end{solution}

\begin{problem}[000034]
如果 $f$ 在 $[a, b]$ 上导数连续且不是线型函数,那么存在 $\xi \in (a, b)$ 满足
\[ |f'(\xi)| \geqslant \left| \frac{f(b) - f(a)}{b - a} \right| \]
\end{problem}

\begin{solution}
	设
	\[ F(x) = f(x) - f(a) - \frac{f(b) - f(a)}{b - a}(x - a), \quad F'(x) = f'(x) - \frac{f(b) - f(a)}{b - a} \]
	由非线性性知存在 $F(c) \neq 0$,不妨设 $F(c) > 0$,存在两点 $\xi_1 \in (a, c), \xi_2 \in (c, b)$ 有
	\[ F'(\xi_1) = \frac{F(c) - F(a)}{c - a} > 0, \quad F'(\xi_2)  \]
	故有
	\[ f'(\xi_2) < \frac{f(b) - f(a)}{b - a} < f'(\xi_1) \]
	不难推出 $\xi_1, \xi_2$ 总有一个满足题意的。
\end{solution}

\begin{problem}[000035]
如果 $f$ 在 $[a, b]$ 上二阶可导且 $f(a) = f(b) = 0$,则对任意的 $x \in (a, b)$ 存在 $\xi_x \in (a, b)$ 满足
\[ f(x) = \frac{f''(\xi_x)}{2}(x - a)(x - b) \]
\end{problem}

\begin{solution}
	固定 $x$ 并定义
	\[ \lambda = \frac{2 f(x)}{(x - a)(x - b)} \]
	构造
	\[ F(u) = f(u) - \frac{\lambda}{2} (u - a)(u - b) \]
	注意到 $F(a) = F(x) = F(b) = 0$,套两次中值定理即得存在 $\xi_x$ 使得 $F''(\xi_x) = 0$。
\end{solution}

\begin{problem}[000036]
如果 $f$ 在 $[a, b]$ 上三阶可导且 $f(a) = f'(a) = f(b) = 0$,则对任意的 $x \in (a, b)$ 存在 $\xi_x \in (a, b)$ 满足
\[ f(x) = \frac{f'''(\xi_x)}{6}(x-a)^2(x-b) \]
\end{problem}

\begin{solution}
	TODO 仿照 000035 的计算方法,构造
	\[ F(u) = f(u) - \frac{\lambda}{6}(u - a)^2(u - b), \quad \frac{6f(x)}{(x - a)^2(x - b)} \]
	求导即证。
\end{solution}

\begin{problem}[000037]
如果 $f$ 在 $[a, b]$ 上二阶可导,则对任意的 $c \in (a, b)$ 存在 $\xi_c \in (a, b)$ 满足
\[ f''(\xi_c) = \frac{2 f(a)}{(a - b)(a - c)} + \frac{2 f(b)}{(b - c)(b - a)} + \frac{2f(c)}{(c- a)(c - b)} \]
\end{problem}

\begin{solution}
	TODO 仿照 000035 的计算方法,构造三个 $\lambda$ 和 $F(x)$,求导即证。
\end{solution}

\begin{problem}[000038]
设 $f$ 在区间 $[a, b]$ 上连续可导,且 $0 < a < b$,那么存在 $\xi, \eta \in (a, b)$ 满足
\[ f'(\eta) = (b^2 + ab + a^2 + 2)\frac{f'(\xi)}{3 \xi^2 + 2} \]
\end{problem}

\begin{solution}
	首先选取 $\eta$
	\[ \frac{f(b) - f(a)}{b - a} = f'(\eta) \]
	接下来即证
	\[ \frac{f(b) - f(a)}{b^3 - a^3 + 2(b - a)} = \frac{f'(\xi)}{3\xi^2 + 2} \]
	容易套用 Cauchy 中值定理。
\end{solution}

\begin{problem}[000039]
设 $f$ 在区间 $[0, 1]$ 上二阶可导且 $|f(0)| \leqslant 1, |f(1)| \leqslant 1$,同时 $|f''(x)| \leqslant 2$。求证 $|f'(x)| \leqslant 3$。
\end{problem}

\begin{solution}
	首先用两次 Taylor 公式,得到
	\[ f(1) - f(0) = f'(x) + \frac{f''(\xi)}{2}(1 - x)^2 - \frac{f''(\eta)}{2}x^2 \]
	因此
	\[ \begin{aligned}
			|f'(x)| & \leqslant \left| f(1) - f(0) - \frac{f''(\xi)}{2}(1 - x)^2 + \frac{f''(\eta)}{2}x^2 \right| \\
			        & \leqslant 2 + |1-x|^2 + |x|^2 \leqslant 3
		\end{aligned} \]
\end{solution}

\begin{problem}[000040]
设 $f$ 是 $[a,b]$ 上的连续可导函数,且 $0 < a < b$,那么存在 $\xi, \eta \in (a,b)$ 使得
\[ f'(\xi) = \frac{a+b}{2\eta}f'(\eta) \]
\end{problem}

\begin{proof}
	首先存在 $\xi \in (a,b)$ 使得
	\[ f'(\xi) = \frac{f(b) - f(a)}{b - a} \]
	构造
	\[ F(x) = f(x) - f(a) - \frac{f(b) - f(a)}{b^2 - a^2}(x^2 - a^2) \]
	注意到 $F(a) = F(b) = 0$,因此存在 $\eta \in (a, b)$ 使得
	\[ F'(\eta) = f'(\eta) - 2 \eta \frac{f(b) - f(a)}{b^2 - a^2} = f'(\eta) - \frac{2 \eta f'(\xi)}{b + a} = 0 \]
\end{proof}

\begin{problem}[000041]
设 $f$ 是 $[a,b]$ 上的连续可导函数,且 $f'(a) = f'(b) = 0$,那么存在 $\xi \in (a,b)$ 使得
\[ f(\xi) - f(a) = f'(\xi)(\xi - a) \]
\end{problem}

\begin{proof}
	首先构造
	\[ F(x) = \frac{f(x) - f(a)}{x - a}, \quad F'(x) = \frac{1}{x-a} \left(f'(x) - \frac{f(x) - f(a)}{x-a}\right), \quad x \in (a, b] \]
	可以令 $F(a) = f'(a) = 0$ 使其在 $x=a$ 处连续。注意到
	\[ F(b) = \frac{f(b) - f(a)}{b-a}, \quad F'(b) = -\frac{f(b) - f(a)}{(b-a)^2} \]
	故存在 $x_0 \in (a, b)$
	\[ F'(x_0) F'(b) = \frac{F(b) - F(a)}{b - a} F'(b) \leqslant 0 \]
	因此存在 $\xi \in (x_0, b)$ 使得 $F'(\xi) = 0$。
\end{proof}

\begin{problem}[000042]
设函数 $f$ 是 $[a,b]$ 上连续且二阶可导,且 $f'(a) = f'(b) = 0$,那么存在 $c \in (a,b)$ 使得
\[ |f''(c)| \geqslant \frac{4}{(b-a)^2} |f(b) - f(a)| \]
\end{problem}

\begin{proof}
	我们选取 $x_0$ 构造
	\[ F(x) = f(x) - f(x_0) - f'(x_0)(x-x_0) , \quad G(x) = (x-x_0)^2 \]
	连续使用两次微分中值定理得到
	\[ \frac{F(x)}{G(x)} = \frac{F(x) - F(x_0)}{G(x) - G(x_0)} = \frac{F'(\xi_1)}{G'(\xi_1)} = \frac{f'(\xi_1) - f'(x_0)}{2 (\xi_1 - x_0)} = \frac{f''(\xi_2)}{2} \]
	其中 $x_0 < \xi_2 < \xi_1 < x$。定 $x=\frac{a+b}{2}$,取 $x_0 = a$ 得到
	\[ f\left(\frac{a+b}{2}\right) = f(a) + \frac{f''(\eta_1)}{8} (b-a)^2, \quad \eta_1 \in \left(a, \frac{a+b}{2}\right) \]
	同理可取 $x_0 = b$ 得到
	\[ f\left(\frac{a+b}{2}\right) = f(b) + \frac{f''(\eta_2)}{8} (b-a)^2, \quad \eta_2 \in \left(\frac{a+b}{2}, b\right) \]
	从而有
	\[ \frac{4(f(b) - f(a))}{(b-a)^2} = \frac{f''(\eta_1) - f''(\eta_2)}{2} \]
	故
	\[ \frac{4}{(b-a)^2}|f(b) - f(a)| = \frac{|f''(\eta_1) - f''(\eta_2)|}{2} \leqslant \max\{f''(\eta_1), f''(\eta_2)\} \]
\end{proof}

\begin{problem}[000043]
设 $f$ 在区间 $[a, b]$ 上二阶可导且 $f(a) = f(b) = 0$,且二阶导数 $|f''(x)| \leqslant A$ 有界。则
\[ \left| f\left(\frac{a+b}{2}\right) \right| \leqslant \frac{A}{8}(b- a)^2 \]
\end{problem}

\begin{solution}
	用两次 Taylor 展开,得到
	\[ \frac{f(a) + f(b)}{2} = f\left(\frac{a+b}{2}\right) + \frac{f''(\xi) + f''(\eta)}{4}\frac{(b - a)^2}{4} \]
	故
	\[ \left| f\left(\frac{a+b}{2}\right) \right| \leqslant \frac{A}{8}(b- a)^2 \]
\end{solution}

\begin{problem}[000064]
(1995 卷 I)函数 $f(x)$ 和 $g(x)$ 在 $[a,b]$ 上存在二阶导数,并且 $g''(x) \neq 0$,$f(a) = f(b) = g(a) = g(b) = 0$,试证:

(1) 在开区间 $(a, b)$ 内 $g(x) \neq 0$。

(2) 在开区间 $(a, b)$ 内至少存在一点 $\xi$,使
\[ \frac{f(\xi)}{g(\xi)} = \frac{f''(\xi)}{g''(\xi)} \]
\end{problem}

\begin{solution}
	(1) Rolle 定理显然。

	(2) 构造
	\[ F(x) = f(x) g'(x) - f'(x) g(x) \]
	注意到 $F(a) = F(b) = 0$,而且
	\[ F'(x) = f(x) g''(x) - f''(x) g(x) \]
	因此存在 $\xi$ 使得
	\[ F'(\xi) = \frac{F(b) - F(a)}{b - a} = 0 \]
	原式成立。
\end{solution}

\section{一元积分学}

\begin{problem}[000008]
求
\[ \int_{-\infty}^{+\infty} \ee^{-x^2}\d x \]
\end{problem}
\begin{solution}
	\[ \int_{-\infty}^{+\infty} \ee^{-x^2}\d x = \sqrt{\pi} \]
\end{solution}

\begin{problem}[000014]
求
\[ \int \frac{1}{(x^2+a^2)^2} \d x \]
\end{problem}
\begin{solution}
	令 $x = a \tan t$,则
	\[ x^2 + a^2 = a^2 \sec^2 t, \quad \d x = a \sec^2 t \d t \]
	有
	\[ LHS =  \frac{1}{a^3}\int \cos^2 t \d t = \frac{1}{2a^3} \left( \arctan \frac{x}{a} + \frac{ax}{x^2+a^2} \right) \]
\end{solution}


\begin{problem}[000015]
求
\[ \int \frac{\d x}{\sqrt{x^2 + a}} \d x \]
\end{problem}
\begin{solution}
	\[ LHS =  \ln |x + \sqrt{x^2+a}| + C \]
\end{solution}


\begin{problem}[000022]
设函数
\[ f(x) = x \int_{1}^{0} \ee^{-x^2t^2} \d t \]
则当 $0<a<x<b$ 时有:

A. $xf(x) > af(a)$,B. $bf(b) > x f(x)$,C. $xf(a) > af(x)$,D. $xf(b) > bf(x)$。
\end{problem}

\begin{solution}
	答案 D。首先积分换元
	\[ f(x) = \int_{1}^{0} \ee^{-(xt)^2} \d (xt) = -\int_{0}^{1} \ee^{-x^2} \d x  \]
	剩下的比较直观,就不写了。
\end{solution}


\begin{problem}[000027]
已知 $a > 0$,则对于反常积分
\[ I = \int_{0}^{1} \frac{\ln x}{x^a} \d x \]
的敛散性的判别,下列正确的是

A. 当 $a > 1$ 时 $I$ 收敛。B. 当 $a < 1$ 时 $I$ 收敛。

C. 敛散性和 $a$ 的取值无关,必收敛。D. 敛散性与 $a$ 的取值无关,必发散。
\end{problem}
\begin{solution}
	当 $a < 1$ 时,注意到
	\[ f'(x) = \frac{1 - a \ln x}{x^{a+1}} \]
	故 $|f(x)| \leqslant |f(\ee^{1/a})|$,有界故收敛。当 $a \geqslant 1$ 时,考虑与 $x^{-a}$ 进行比较,注意到当 $0 < x < \frac{1}{\ee}$ 时有 $f(x) < -\frac{1}{x^a}$,故 $I$ 发散。
\end{solution}

\begin{problem}[000028]
设 $p$ 为常数,若反常积分
\[ I = \int_{0}^{1} \frac{\ln x}{x^p(1-x)^{1-p}} \d x \]
收敛,求 $p$ 的取值范围。
\end{problem}
\begin{solution}
	答案是 $(-1,1)$。考虑划分成两部分
	\[ I = \int_{0}^{\frac{1}{2}} \frac{\ln x}{x^p(1-x)^{1-p}} \d x + \int_{\frac{1}{2}}^{1} \frac{\ln x}{x^p(1-x)^{1-p}} \d x = I_1 + I_2 \]
	分别对 $x \to 0$ 和 $x \to 1$ 讨论即可,有点麻烦。
\end{solution}

\begin{problem}[000029]
已知
\[ b = \int_{1}^{+\infty} \left(\frac{2x^3 + ax + 1}{x(x+2)} - (2x-4)\right) \d x \]
其中 $a,b$ 为常数,求 $ab$。
\end{problem}
\begin{solution}
	答案是 $-4 \ln 3$。注意到
	\[ b = \int_{1}^{+\infty} \frac{(a+8)x+1}{x(x+2)} \d x= \int_{1}^{+\infty} \left(\frac{1}{2x} - \frac{2a+15}{2(x+2)}\right) \d x \]
	讨论一下,就知道 $a = -8, b = \frac{\ln 3}{2} $。
\end{solution}

\begin{problem}[000044]
任给 $\beta \geqslant 0$ 和 $b > a > 0$ 证明
\[ \left| \int_{a}^{b} \ee^{-\beta x} \frac{\sin x}{x} \d x \right| \leqslant \frac{2}{a} \]
\end{problem}

\begin{solution}
	用积分第二中值定理,得到
	\[ \int_{a}^{b} \frac{\ee^{-\beta x}}{x} \sin x \d x \leqslant \frac{\ee^{-a \beta}}{a} \int_{a}^{\xi} \frac{\ee^{-\beta x}}{x} \sin x \d x \]
	因此
	\[ \left| \int_{a}^{b} \ee^{-\beta x} \frac{\sin x}{x} \d x \right| \leqslant \frac{2}{a \ee^{a \beta}} \leqslant \frac{2}{a} \]
\end{solution}

\begin{problem}[000045]
设 $f$ 在 $[a, b]$ 上连续,且
\[ \int_{a}^{b} f(x) \d x = \int_{a}^{b} x f(x) = 0 \]
证明:存在 $x_1 \neq x_2$ 使得 $f(x_1) = f(x_2) = 0$。
\end{problem}

\begin{solution}
	显然 $f$ 至少存在一个零点,否则积分不可能为 $0$。假设 $x_0$ 是其唯一零点,不妨设左边小于 $0$ 右边大于 $0$。那么
	\[ \int_{a}^{b} (x - x_0) f(x) \d x = \int_{a}^{b} x f(x) \d x - x_0 \int_{a}^{b} f(x) \d x = 0 \]
	注意到 $(x - x_0) f(x) > 0$ 在 $x \neq x_0$ 成立,故左边积分不等于 $0$,矛盾。
\end{solution}

\begin{problem}[000046]
设 $f$ 在 $[a, b]$ 上可积且 $a > 0$,证明
\[ |f(0)| \leqslant \frac{1}{a} \int_{0}^{a} |f(x)| \d x + \int_{0}^{a} |f'(x)| \d x \]
\end{problem}

\begin{solution}
	有积分中值定理得到
	\[ \int_{0}^{a} f(x) \d x = a f(\xi) \]
	因此
	\[ f(0) = f(\xi) - \int_{0}^{\xi} f'(x) \d x = \frac{1}{a} \int_{0}^{a} f(x) \d x - \int_{0}^{\xi} f'(x) \d x \]
	从而
	\[ |f(0)| \leqslant \frac{1}{a} \int_{0}^{a} |f(x)| \d x + \int_{0}^{a} |f'(x)| \d x \]
\end{solution}

\begin{problem}[000047]
假设在 $f$ 在 $[0, 1]$ 上可导,且
\[ f(1) = 2 \int_{0}^{\frac{1}{2}} x f(x) \d x \]
证明存在 $\xi \in (0, 1)$ 使得
\[ f'(\xi) = -\frac{f(\xi)}{\xi} \]
\end{problem}

\begin{solution}
	构造 $F(x) = x f(x)$,由积分中值定理知存在 $\eta \in (0, \frac{1}{2})$ 使得
	\[ f(1) = 2\int_{0}^{\frac{1}{2}} F(x) \d x = F(\eta) \]
	又 $F(0) = 0, F(1) = f(1) = F(\eta)$,故存在 $\xi \in (\eta, 1)$ 使得
	\[ F'(\xi) = \frac{F(1) - F(\eta)}{1 - \eta} = 0 = f(\xi) + \xi f'(\xi) \]
\end{solution}

\begin{problem}[000048]
假设在 $f$ 在 $[0, 1]$ 上可导,那么存在 $c_x \in (a, x)$ 使得
\[ \int_{a}^{x} f(t) \d t = f(c_x) (x - a) \]
若 $f$ 在 $a$ 处可导且 $f'(a) \neq 0$ 则
\[ \lim_{x \to a+} \frac{c_x - a}{x -a} = \frac{1}{2} \]
\end{problem}

\begin{solution}
	由积分中值定理,$c_x$ 的存在是显然的。定义
	\[ I = \lim_{x \to a+} \frac{1}{(x-a)^2} \left( \int_{a}^{x} f(t) \d t - (x - a) f(a) \right) \]
	首先由 L'Hopital 法则得到
	\[ I = \lim_{x \to a+} \frac{f(x) - f(a)}{2(x - a)} = \frac{1}{2} f'(a) \]
	又
	\[ I = \lim_{x \to a+}\frac{f(c_x) - f(a)}{x - a} = f'(a) \frac{\partial c_x}{\partial x} \]
	当 $f'(a) \neq 0$ 时比较即知。
\end{solution}

\begin{problem}[000049]
假设 $f$ 在 $[a, b]$ 上存在二阶导数,那么存在 $\xi \in (a, b)$ 满足
\[ f''(\xi) = \frac{24}{(b-a)^3} \int_{a}^{b} \left(f(x) - f\left(\frac{a+b}{2}\right)\right) \d x \]
\end{problem}

\begin{solution}
	令 $x_0 = \frac{a+b}{2}$,由 Taylor 公式知存在 $\eta \in (a, x)$ 使得
	\[ f(x) - f(x_0) = f'(x_0)(x - x_0) + \frac{f''(\eta_x)}{2} (x - x_0)^2 \]
	积分得
	\[ \int_{a}^{b} (f(x) - f(x_0)) \d x = \frac{1}{2} \int_{a}^{b} f''(\eta_x) (x - x_0)^2 \d x  \]
	由积分中值定理,存在 $\xi$ 使
	\[ \frac{1}{2} \int_{a}^{b} f''(\eta_x) (x - x_0)^2 \d x = \frac{1}{2} f''(\xi) \int_{a}^{b} (x - x_0)^2 \d x = \frac{(b-a)^3}{24} f''(\xi)   \]
\end{solution}

\begin{problem}[000050]
假设 $f$ 在 $[0, 1]$ 上连续且 $f > 0$,则对任意正整数 $n$ 存在 $\xi_n$ 满足
\[ \frac{1}{n} \int_{0}^{1} f(x) \d x = \int_{0}^{\xi_n} f(x) \d x + \int_{1 - \xi_n}^{1} f(x) \d x \]
且
\[ \lim_{n \to +\infty} n \xi_n = \frac{1}{f(0) + f(1)} \int_{0}^{1} f(x) \d x \]
\end{problem}

\begin{solution}
	设 $F$ 为 $f$ 的原函数,构造
	\[ G(x) = F(x) - F(0) + F(1) - F(1 - x) \]
	注意到
	\[ G'(x) = f(x) + f(1 - x) > 0, \quad G(0) = 0, G(1) = 2F(1) \]
	由于 $G$ 连续且单调递增,因此存在 $\xi_n$ 使得 $\frac{1}{n}F(1) = G(\xi_n)$,且 $\xi_n$ 单调递减趋于 $0$。又由积分中值定理知存在 $\alpha_n \in (0, \xi_n), \beta_n \in (1-\xi_n, 1)$ 使得
	\[ G(\xi_n) = f(\alpha_n) \xi_n + f(\beta_n) \xi_n \]
	因此
	\[ \lim_{n \to +\infty} n \xi_n = \lim_{n \to +\infty} \frac{n G(\xi_n)}{f(\alpha_n) + f(\beta_n)} = \frac{F(1)}{f(0) + f(1)} \]
\end{solution}

\begin{problem}[000051]
如果 $f,g$ 在 $[a,b]$ 上连续且 $f, g > 0$,那么存在 $\xi \in (a,b)$ 满足
\[ \frac{f(\xi)}{\int_{a}^{\xi} f(x) \d x} - \frac{g(\xi)}{\int_{\xi}^{b} g(x) \d x} = 1 \]
\end{problem}

\begin{solution}
	设 $F, G$ 分别为 $f, g$ 的原函数,即证
	\[ \frac{F'(\xi)}{F(\xi)} - \frac{G'(\xi)}{G(b) - G(\xi)} = 1 \]
	构造
	\[ H(x) = \ee^{-x} F(x) (G(b) - G(x)) \]
	注意到 $H(a) = H(b) = 0$,因此存在 $\xi$ 使得
	\[ H'(\xi) = \frac{F(\xi) (G(b) - G(\xi))}{\ee^{\xi}} \left(\frac{F'(\xi)}{F(\xi)} - \frac{G'(\xi)}{G(b) - G(\xi)} - 1\right) = 0 \]
\end{solution}

\begin{problem}[000052]
如果 $f,g$ 在 $[a,b]$ 上连续且 $\varphi(x) \neq 0$,那么存在 $\xi \in (a,b)$ 满足
\[ g(\xi) \int_{a}^{b} f(x) \varphi(x) \d x = f(\xi) \int_{a}^{b} g(x) \varphi(x) \d x \]
\end{problem}

\begin{solution}
	构造
	\[ F(x) = \left(\int_{a}^{b} f(t) \varphi(t) \d t\right) \int_{a}^{x} g(t) \varphi(t) \d t - \left(\int_{a}^{b} g(t)\varphi(t) \d t \right) \int_{a}^{x} f(t) \varphi(t) \d t \]
	注意到
	\[ F'(x) =  \varphi(x)\left(g(x)\int_{a}^{b} f(t) \varphi(t) \d t - f(x) \int_{a}^{b} g(t) \varphi(t) \d t \right), \quad F(a) = F(b) = 0 \]
	故存在 $F'(\xi) = 0$。
\end{solution}

\begin{problem}[000053]
假设 $f$ 在 $[-1, 1]$ 且在 $x=0$ 处可导,$f(0) = 0, f'(0) \neq 0$。求极限
\[ I = \lim_{x \to 0+} \frac{\int_{0}^{x}(x^2 - t^2) f(t) \d t}{\int_{0}^{x} t f(x^2 - t^2) \d t} \]
\end{problem}

\begin{solution}
	容易发现
	\[ I = \lim_{x \to 0+} \frac{\int_{0}^{x} (x^2 - t^2) f(t) \d t}{\frac{1}{2} \int_{0}^{x^2} f(u) \d u} = \lim_{x \to 0+} \frac{2}{f(x^2)}\int_{0}^{x}f(t) \d t \]
	注意条件里没有导数连续,因此不能用 L'Hospital 法则,考虑用导数定义
	\[ I = \lim_{x \to 0+} \frac{2 \int_{0}^{x}f(t) \d t}{x^2} \frac{x^2}{f(x^2) - f(0)} = \lim_{x \to 0+} \frac{f(x)}{f'(0) x} = 1 \]
\end{solution}

\begin{problem}[000054]
假设 $f$ 在 $[0, 1]$ 上连续且 $f \geqslant 0$。若满足
\[ f^2(x) \leqslant 1 + 2 \int_{0}^{x} f(t) \d t \]
求证:$f(x) \leqslant 1 + x$。
\end{problem}

\begin{solution}
	令 $F$ 为 $f$ 的原函数,原方程即为 $F'(x)^2 \leqslant 1 + 2F(x)$。注意到
	\[ 0 \leqslant \frac{F'(x)}{\sqrt{1 + 2 F(x)}} = \left(\sqrt{2F+1}\right)' \leqslant 1 \]
	两侧积分得到
	\[ 1 \leqslant \sqrt{2F(x) + 1}  \leqslant 1 + x \]
	因此 $f = F' \leqslant \sqrt{2F(x) + 1} \leqslant x + 1$。
\end{solution}

\begin{problem}[000055]
假设 $f$ 在 $[0, 1]$ 上连续在 $(0, 1)$ 上可导,且 $f(0) = f(\frac{1}{4}) = 0$,且
\[ \int_{\frac{1}{4}}^{\frac{3}{4}} f(x) \d x = \frac{f(1)}{2} \]
证明:存在 $\xi \in (0, 1)$ 满足 $f''(\xi)  0$。
\end{problem}

\begin{solution}
	根据积分中值定理,知存在
	\[ \int_{\frac{1}{4}}^{\frac{3}{4}} f(x) \d x = \frac{f(\eta_1)}{2} = \frac{f(1)}{2} \]
	即 $f(\eta_1) = f(1)$,故存在 $f'(\eta_2) = 0$。又 $f(0) = f(\frac{1}{4})$ 知存在 $f'(\eta_3) = 0$,因此存在 $\xi \in (0, 1)$ 使得 $f''(\xi) = 0$。
\end{solution}

\begin{problem}[000056]
设 $f$ 在 $[a, b]$ 上导数存在且连续,又 $f(a) = 0$,求证:
\[ \int_{a}^{b} f^2(x) \d x \leqslant \frac{(b-a)^2}{2} \int_{a}^{b} (f'(x))^2 \d x \]
\end{problem}

\begin{solution}
	由 Cauchy 积分不等式得
	\[
		f^2(x)  = \left(\int_{a}^{x} f'(t) \d t\right)^2
		\leqslant (x-a) \int_{a}^{x} (f'(t))^2 \d t
	\]
	因此
	\[ \int_{a}^{b} f^2(x) \d x \leqslant \left(\int_{a}^{b} (x-a) \d x \right) \int_{a}^{b} (f'(x))^2 \d x = \frac{(b-a)^2}{2} \int_{a}^{b} (f'(x))^2 \d x \]
\end{solution}

\begin{problem}[000057]
设 $f$ 在 $[0, 1]$ 上导数存在且连续,又 $f(0) = 0$ 且 $0 \leqslant f' \leqslant 1$,求证:
\[ \int_{0}^{1} f^3(x) \d x \leqslant \left(\int_{0}^{1} f(x) \d x\right)^2 \]
\end{problem}

\begin{solution}
	设
	\[ G(x) =  \left(\int_{0}^{x} f(t) \d t\right)^2 - \int_{0}^{x} f^3(t) \d t \]
	求导得
	\[ G'(x) = f(x) \left(2 \int_{0}^{x} f(t) \d t - f^2(x)\right) \]
	令
	\[ H(x) = 2 \int_{0}^{x} f(t) \d t - f^2(x), \quad H'(x) = 2f(x)(1 - f'(x)) \geqslant 0 \]
	因此 $H(x) \geqslant H(0) = 0$,故 $G(x) \geqslant G(0) = 0$。
\end{solution}

\begin{problem}[000058]
设 $f$ 在 $[0, 1]$ 上二阶导数连续,$f(0) = f(1) = 0$,且在 $(0, 1)$ 上 $f > 0$,求证:
\[ \int_{0}^{1} \left|\frac{f''(x)}{f(x)}\right| \d x \geqslant 4 \]
\end{problem}

\begin{solution}
	不妨设 $f > 0$。设 $x = a$ 时取到最大值,对两侧端点取微分中值定理
	\[ \frac{f(a) - f(0)}{a - 0} = f'(\eta_1), \quad \frac{f(1) - f(a)}{1 - a} = f'(\eta_2) \]
	因此
	\[ \begin{aligned}
			\int_{0}^{1} \left|\frac{f''(x)}{f(x)}\right|
			 & \geqslant \frac{1}{f(a)} \int_{\eta_1}^{\eta_2} |f''(x)| \d x \\
			 & \geqslant \frac{|f'(\eta_1) - f'(\eta_2)|}{f(a)}              \\
			 & = \frac{1}{a(1-a)} \geqslant 4
		\end{aligned} \]
\end{solution}

\section{常微分方程}

\begin{problem}[000061]
(1991 卷 I)在上半平面求一条向上凹的曲线,其上任何一点 $P(x, y)$ 处的曲率等于此曲线在该点法线段 $PQ$ 长度的倒数($Q$ 是法线与 $x$ 轴的交点),且曲线在 $(1, 1)$ 处的切线与 $x$ 轴平行。
的值最小。
\end{problem}

\begin{solution}
	由题意得
	\[ \frac{y''}{(1+(y')^2)^{\frac{3}{2}}} = \frac{1}{y \sqrt{1 + (y')^2}} \]
	整理得
	\[ y'' y - (y')^2 = 1 \]
	构造 $p = y'$,得 $y'' = \frac{p \d p}{\d y}$,故
	\[ 1 + p^2 = yp \frac{\d p}{\d y} \]
	分离变量解得
	\[ y^2 - p^2 = 1 \]
	容易解得
	\[ y = \frac{\ee^{x-1} - \ee^{-(x-1)}}{2} \]
\end{solution}

\begin{problem}[000063]
(1993 卷 I)设物体 $A$ 从点 $(0,1)$ 出发,以速度大小为常数 $v$ 沿 $y$ 轴正向运动。物体 $B$ 从点 $(1,0)$ 与 $A$ 同时出发,其速度大小为 $2v$,方向始终指向 $A$,试建立物体 $B$ 的运动轨迹所满足的微分方程,并写出初始条
件。
\end{problem}

\begin{solution}
	首先物体 $A$ 的速度指向物体 $B$,即
	\[ \frac{\d y}{\d x} = \frac{1 + vt - y}{0 - x} \]
	同时速度是 $2v$,得到
	\[ \sqrt{(\d x)^2 + (\d y)^2} = 2 v \d t \]
	这里需要把变量 $t$ 消去,联立得
	\[ 2 x y'' + \sqrt{1 + (y')^2} = 0 \]
\end{solution}

\section{概率论与数理统计}

\begin{problem}[000062]
(1989 卷 I)设随机变量 $X$ 与 $Y$ 相互独立,且 $X$ 服从均值为 $1$、标准差为 $\sqrt{2}$ 的正态分布,而 $Y$ 服从标准正态分布,求随机变量 $Z = 2X - Y + 3$ 的概率密度函数。
\end{problem}

\begin{solution}
	由于 $Z$ 是 $X, Y$ 的线性组合,故也服从正态分布。分别计算 $E(Z), D(Z)$。
	\[ E(Z) = 2 E(X) - E(Y) + 3 = 5, \quad D(Z) = 4D(X) + D(Y) + 0 = 9 \]
	故 $Z \sim N(5, 9)$,得到
	\[ f_Z(z) = \frac{1}{3 \sqrt{2 \pi}} \exp\left(- \frac{(z-5)^2}{18 }\right) \]
\end{solution}
