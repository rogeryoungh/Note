\chapter{习题}

\section{函数、极限}

\begin{problem}[000001]
	设 $a_1=1,a_k=k(a_{k-1}+1)$,试计算
	$$\lim_{n\to \infty}\prod_{k=1}^n\left(1+\frac{1}{a_k}\right)$$
\end{problem}
\begin{solution}
	先变形
	$$\left(1+\frac{1}{a_k}\right)=\frac{a_{n+1}}{ka_n}$$
	累乘可以化简
	$$\prod_{k=1}^n\left(1+\frac{1}{a_k}\right) = \frac{a_{n+1}}{(n+1)!}$$
	注意到
	$$\frac{a_{n+1}}{(n+1)!}-\frac{a_{n}}{n!} = \frac{a_{n+1}-(n+1)a_n}{(n+1)!} = \frac{1}{n!}$$
	故
	$$\lim_{n\to \infty}\prod_{k=1}^n\left(1+\frac{1}{a_k}\right) = \lim_{n\to \infty}\left(1+\frac{1}{2!}+\cdots+\frac{1}{n!}\right) = \ee$$
\end{solution}

\begin{problem}[000002]
	设 $x_1 = 2, x_n + (x_n - 4)x_{n-1} = 3(n = 2, 3, \cdots)$,求 $\displaystyle\lim_{n\to \infty} x_n$。
\end{problem}

\begin{solution}
	显然是考不动点。

	考虑方程
	$$ x + (x-4)x - 3 = x^2 - 3x - 3 = 0 $$
	的解,取其中一解 $x_0 = \frac{3 + \sqrt{21}}{2}$。接下来考察单调性,设 $x_{n-1} \in [2, x_0)$,有
	$$ x_n - x_{n-1} = 4 - x_{n-1} - \frac{1}{x_n - 1} = -\frac{x_{n-1}^2 - 3x_{n-1} - 3}{x_{n-1} - 1} > 0$$
	故序列 $\{x_n\}$ 单调递增且 $x_n \in [2, x_0)$。设极限为 $A$,解方程
	$$
	A^2 - 3A - 3 = 0, \quad A = x_0 = \frac{3 + \sqrt{21}}{2}
	$$
\end{solution}

\begin{problem}[000003]
	是否存在这样的函数,它在区间 $[0,1]$ 上每点取有限值,在此区间的任何点的任一邻域内无界。
\end{problem}
\begin{solution}
	构造
	$$f(x) = \begin{cases}
		n,& x=\dfrac{m}{n},m,n\ \text{为互质整数}\\
		0,&x\ \text{为无理数}
	\end{cases}$$
\end{solution}

\begin{problem}[000004]
	设 $f,g$ 是 $\RR$ 上的实函数,且
	$$f(x+y)+f(x-y) = 2f(x)g(y)$$
	在 $\RR$ 上 $f(x)$ 不恒等于零但有界,试证:$|g(y)|\leqslant 1$
\end{problem}
\begin{solution}
	令 $M=\sup|f(x)|$,则有
	$$2M\geqslant |f(x+y)|+|f(x-y)| \geqslant |f(x+y)+f(x-y)| = 2|f(x)||g(y)|$$
	设存在 $y_0$ 使得 $|g(y_0)|=1+2\delta>1$。由上确界的定义知存在 $x_0$ 有
	$$M \geqslant |f(x_0)| > \frac{M}{\delta+1}$$
	故
	$$2|f(x_0)||g(y_0)| > \frac{2(1+2\delta)M}{1+\delta} > 2M$$
	因此矛盾,故恒有 $|g(y)|\leqslant 1$。
\end{solution}


\begin{problem}[000005]
	设 $f$ 是闭区间 $[a,b]$ 上的增函数(但不一定连续),如果 $f(a) \geqslant a,f(b) \leqslant b$,试证: $\exists x_0 \in [a,b]$,使得 $f(x_0) = x_0$。
\end{problem}
\begin{solution}
	设 $A=\{x \mid f(x) \geqslant x\}$,由题知 $a\in A$ 故 $A$ 非空。又 $f$ 定义在 $[a,b]$ 上,故 $A$ 有界。因此设 $x_0=\sup A\in [a,b]$ 是有意义的。又 $f(x)\in[a,b]$ 在定义域内,分类讨论如下

	1. 若 $y_0=f(x_0) > x_0$,由单调性知
	$$f(y_0)=f(f(x_0)) \geqslant f(x_0) = y_0$$
	故 $y_0\in A$。这意味着 $\sup A \geqslant y_0 >x_0$,矛盾。

	2. 若 $y_0=f(x_0) < x_0$,由确界定义知 $\exists x_1\in A$ 使 $y_0<x_1\leqslant x_0$,由单调性知
	$$f(x_1)\leqslant f(x_0)=y_0 <x_1$$
	这意味着 $x_1\notin A$,矛盾。

	故 $y_0=f(x_0)=x_0$,此时 $x_0 = \sup\{x \mid f(x) \geqslant x\}$。

	注意 $x_0$ 不一定在 $A$ 中,即 $f(x_0) \geqslant x_0$ 不一定成立。
\end{solution}


\begin{problem}[000006]
	设 $f(x)$ 是定义在 $\RR$ 上的函数且对任意 $x,y$有
	$$|xf(y)-yf(x)| \leqslant M|x|+M|y|$$
	其中 $M > 0$。求证:存在常数 $a$ 使得对任意 $x$ 有 $|f(x)-ax| \leqslant M$
\end{problem}
\begin{solution}
	当 $x=0$ 时,有 $|f(0)|\leqslant M$。而当 $xy\ne 0$ 时,恒有
	$$\left| f(x)-\frac{f(y)}{|y|}x \right| \leqslant M \left(1+\frac{|x|}{|y|}\right)$$
	若 $a$ 不存在,即对任意的 $a$ 存在 $x_0$ 使
	$$|f(x_0)-ax_0|=M(1+2\delta)>M$$
	那么取 $a = \dfrac{f(y_0)}{|y_0|}$,当 $y_0=\dfrac{|x_0|}{\delta}$ 时,有
	$$\left| f(x)-\frac{f(y_0)}{|y_0|}x \right| \leqslant M \left(1+ \frac{|x_0|}{|y_0|}\right)=M(1+\delta)$$
	因此矛盾,故存在 $a$。
\end{solution}

\begin{problem}[000007]
	设 $\displaystyle\lim_{n\to\infty}a_n=A$,求证:$\displaystyle\lim_{n\to\infty}\frac{\sum a_n}{n}=A$。
\end{problem}

\begin{solution}
	即对于任给的 $\eps>0$,存在 $n>N_1$ 使得
	$$|a_n-A|<\dfrac{\eps}{2}$$
	那么变形有
	$$\left|\frac{\sum a_n}{n}-A\right| \leqslant \frac{\sum |a_n-A|}{n} = \frac{\sum_{k=1}^{N_1} |a_k-A|}{n} + \frac{\sum_{k=N_1+1}^{n} |a_k-A|}{n}$$
	注意到 $\sum_{k=1}^{N_1} |a_k-A|$ 已经为定值,从而存在 $n>N_2$ 使得
	$$\frac{\sum_{k=1}^{N_1}|x_k-A|}{n}<\frac{\eps}{2}$$
	因此当 $n>\max\{N_1,N_2\}$ 时有
	$$LHS < \frac{\eps}{2}+\frac{n-N_1}{n}\times \frac{\eps}{2} < \frac{\eps}{2}+\frac{\eps}{2} = \eps$$
\end{solution}

\begin{problem}[000009]
	设 $f(z)$ 在 $[0,1]$ 上具有一阶连续导数,$f(0) = 0$,证明:存在$\xi \in [0,1]$,使得
	\[ f'(\xi) = 2\int_0^1f(x) \d x \]
\end{problem}

\begin{solution}
	TODO。设上下界 $m, M$,中值定理。
\end{solution}

\begin{problem}[000010]
	设 $f(x)$ 在 $[0, 1]$ 上连续,在 $(0, 1)$ 内可导,且 $f(0) = 0, f(1) = 1$,证明存在不同的 $\xi_1, \xi_2 \in (0, 1)$,使得
	\[ \frac{1}{f'(\xi_1)} + \frac{1}{f'(\xi_2)} = 2 \]
\end{problem}

\begin{solution}
	TODO。
\end{solution}

\begin{problem}[000011]
	设 $f(x)$ 在 $[0, 3]$ 上连续,在 $(0, 3)$ 内可导,且 $f(0) + f(1) + f(2) = 3, f(3) = 1$,证明存在 $\xi \in (0, 3)$,使得 $f'(\xi) = 0$。
\end{problem}

\begin{solution}
	TODO。设 $f(c) = f(3) = 1, c \in (0, 2)$。套中值定理。
\end{solution}

\begin{problem}[000012]
	设 $f(x)$ 在 $[0, 1]$ 上连续,在 $(0, 1)$ 内可导,且
	\[ f(1) = k \int_{0}^{\frac{1}{k}} x \ee^{1-x} f(x) \d x (k > 1) \]
	证明至少存在一点 $\xi \in (0, 1)$,使得 $f'(\xi) = (1 - \xi^{-1})f(\xi)$。
\end{problem}

\begin{solution}
	TODO。构造 $F(x) = x \ee^{1-x} f(x)$。套中值定理。
\end{solution}

\begin{problem}[000013]
	设 $f(x)$ 在 $[0, 1]$ 上连续,在 $(0, 1)$ 内二阶可导,过点 $A(0, f(0))$ 与 $B(1, f(1))$ 的直线与曲线 $y=f(x)$ 相交于点 $C(c, f(c))$,其中 $0 < c < 1$,证明存在 $\xi \in (0, 1)$,使得 $f''(\xi) = 0$。
\end{problem}

\begin{solution}
	TODO。构造 $F(x) = (1-x)f(0) + xf(1) - f(x)$,有 $F(0) = F(1) = F(c) = 0$。套中值定理。
\end{solution}

\section{积分}

\begin{problem}[000008]
	求
	\[ \int_{-\infty}^{+\infty} \ee^{-x^2}\d x \]
\end{problem}
\begin{solution}
	\[ \int_{-\infty}^{+\infty} \ee^{-x^2}\d x = \sqrt{\uppi} \]
\end{solution}

\begin{problem}[000014]
	求
	\[ \int \frac{1}{(x^2+a^2)^2} \d x \]
\end{problem}
\begin{solution}
	令 $x = a \tan t$,则
	\[ x^2 + a^2 = a^2 \sec^2 t, \quad \d x = a \sec^2 t \d t \]
	有
	\[ LHS =  \frac{1}{a^3}\int \cos^2 t \d t = \frac{1}{2a^3} \left( \arctan \frac{x}{a} + \frac{ax}{x^2+a^2} \right) \]
\end{solution}


\begin{problem}[000015]
	求
	\[ \int \frac{\d x}{\sqrt{x^2 + a}} \d x \]
\end{problem}
\begin{solution}
	\[ LHS =  \ln |x + \sqrt{x^2+a}| + C \]
\end{solution}
