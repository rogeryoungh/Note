\documentclass{purebook}
\usepackage{fixdif}

\makeindex[
  columns=2,
  program=truexindy,
  intoc=true,
  options=-M texindy -I xelatex -C utf8,
  title=索引,
  intoc
]

\title{高等代数笔记}
\author{\href{https://github.com/rogeryoungh}{rogeryoungh}}

\begin{document}

\newcommand{\ee}{\mathrm{e}}
\newcommand{\transpose}[1]{{#1}^\mathsf{T}}
\newcommand{\eps}{\varepsilon}
\newcommand{\vbf}[1]{\boldsymbol{#1}}

\newcommand{\rank}{\operatorname{rank}}
\newcommand{\diag}{\operatorname{diag}}
\renewcommand{\Im}{\operatorname{Im}}
\newcommand{\tr}{\operatorname{tr}}
\newcommand{\Hom}{\operatorname{Hom}}
\newcommand{\Ker}{\operatorname{Ker}}

\newcommand{\seq}[2]{#1_1,\cdots,#1_#2}

\frontmatter

\maketitle

\tableofcontents

\mainmatter

%% \newcommand{\ji}[2]{#1_1,\cdots,#1_#2}

\chapter{线性方程组的解法}

\section{矩阵消元法}

形如这样左端都是未知量 $x_n$ 的一次齐次式,右端是常数,
\[a_1x_1+a_2x_2+\cdots+a_nx_n=b\]
像这样的方程称为线性方程。每个未知量前面的数称为系数,右端的项称为常数项。

含 $n$ 个未知量的线性方程组称为 $n$ 元线性方程组,它的一般形式是
\begin{equation*}
	\left\{
	\begin{matrix}
		a_{11}x_1+a_{12}x_2+\cdots+a_{1n}x_n=b_1 \\
		a_{21}x_1+a_{22}x_2+\cdots+a_{2n}x_n=b_2 \\
		\cdots\qquad\cdots\qquad\cdots           \\
		a_{s1}x_1+a_{s2}x_2+\cdots+a_{sn}x_n=b_s
	\end{matrix}
	\right.
\end{equation*}
方程的个数 $s$ 与未知量的个数 $n$ 可以相等,也可以不等。

\begin{definition}[线性方程组的初等变换]
	线性方程组的初等变换有三种,分别为:

	\begin{enumerate}
		\item 把一个方程的倍数加到另一个方程上。
		\item 互换两个方程的位置。
		\item 用一个非零数乘某一个方程。
	\end{enumerate}
\end{definition}

对于线性方程组,若 $\ji{x}{n}$ 分别用数 $c_1,\cdots,c_n$ 代入后,每个方程都变成恒等式,那么称 $n$ 元有序组 $(c_1,\cdots,c_n)$ 是线性方程组的一个解。方程组所有解组成的集合称为这个线性方程组的解集,符合实际要求的解称为可行解。

通过初等变换能够使线性方程组变为阶梯形方程组,进一步可以变为简化阶梯形方程组,此种形式可以较方便的看出方程组的解。

\begin{theorem}
	初等变换不改变线性方程组的解。
\end{theorem}

可以把原线性方程组的系数和常数项按次序排成一张表,称为方程组的增广矩阵;而只列出系数的方程组称为系数矩阵。

\begin{definition}
	由 $sn$ 个数排成的 $s$ 行(横的)$n$ 列(纵的)表
	\begin{equation*}
		\left(
		\begin{matrix}
			a_{11} & a_{12} & \ldots & a_{1n} \\
			a_{21} & a_{22} & \ldots & a_{2n} \\
			\vdots & \vdots &        & \vdots \\a_{s1}&a_{s2}&\ldots&a_{sn}\\
		\end{matrix}
		\right)
	\end{equation*}
	称为一个 $s\times n$ 矩阵,记作 $A_{s\times n}$ 或 $A=(a_{ij})$,它的 $(i,j)$ 元也记作 $A(i;j)$。
\end{definition}

特殊的,如果矩阵 $A$ 的行数和列数相等皆为 $n$,则称它为 $n$ 级方阵或方阵。元素全为 $0$ 的矩阵称为零矩阵,记作 $0_{s\times n}$ 或 $0$。

\begin{definition}[初等行变换]\index{chudenghangbianhuan@初等行变换}
	矩阵的初等行变换有三种,分别为:

	\begin{enumerate}
		\item 把一行的倍数加到另一行上。
		\item 互换两行的位置。
		\item 用一个非零数乘某一行。
	\end{enumerate}
\end{definition}

矩阵经过初等行变换,可变成阶梯形矩阵,并可进一步化简成简化行阶梯形矩阵。

阶梯形矩阵的特点为 (1)元素全为 $0$ 的行(零行)在下方(如果有的话);(2)元素不全为 $0$ 的行(非零行),左起第一个不为 $0$ 的元素(主元),他们的列指标随着行指标递增而严格增大。

简化行阶梯形矩阵(行最简形矩阵)的特点为 (1)它是阶梯形矩阵;(2)每个非零行的主元都是 $1$;(3)每个主元所在的列的其余元素都是 $0$。

在解线性方程组时,可以通过一系列初等行变换,它的增广矩阵化为阶梯形矩阵,甚至继续化简为简化行阶梯形矩阵,都可简化求解过程。

\begin{theorem}
	任意矩阵都可以经过一系列初等行变换化为阶梯形矩阵,也可以变成简化行阶梯形矩阵。
\end{theorem}

\section{线性方程组的解的情况及其判别准则}


由于初等变换不改变线性方程组的解,其总可以化为阶梯形方程组。因此设阶梯形方程组有 $n$ 个未知量,它的增广矩阵 $J$ 有 $r$ 个非零行,$J$ 有 $n+1$ 列。

\begin{enumerate}
	\item 若阶梯形方程组中出现 $0=d$(其中 $d$ 为非零数)这种方程,即最后一个非零行的主元位于 $n+1$ 列,则阶梯形方程组无解。
	\item 最后一个非零行的主元不位于 $n+1$ 列。

	      \begin{enumerate}
		      \item $r=n$ 时,阶梯形方程恰有唯一解。
		      \item $r<n$ 时,有无穷多组解。
	      \end{enumerate}
\end{enumerate}

\begin{theorem}
	系数为有理数(实数、复数)的 $n$ 元线性方程组的解的情况只有三种可能:无解,有唯一解,有无穷多组解。
\end{theorem}

若一个线性方程组有解,则称它是相容的;否则称它是不相容的。

\section{数域}

\begin{definition}
	复数集的一个子集 $K$ 是一个数域,那么满足:

	\begin{enumerate}
		\item $0,1 \in K$;
		\item $a,b \in K \Rightarrow a \pm b,ab \in K$;
		\item $a,b \in K$,且 $b \ne 0 \Rightarrow \dfrac{a}{b} \in K$。
	\end{enumerate}
\end{definition}

其中,$\QQ,\RR,\CC$ 都是数域,但整数集 $\mathrm{Z}$ 不是数域。

有理数域是最小的数域。

\begin{theorem}
	任意数域都包含有理数域。
\end{theorem}


 % 线性方程组的解法
\chapter{数列极限}

\section{数列极限的概念}

\begin{definition}[数列极限的 $\eps - N$ 定义]
	设 $\{a_n\}$ 为数列,$A$ 为定数。若对任给的正数 $\eps$,总存在正整数 $N=N(\eps)$,使得当 $n>N$ 时有
	\[|a_n - A| < \eps\]
	则称数列 $\{a_n\}$ 收敛于 $A$,或称 $A$ 为数列 $\{a_n\}$ 的极限,记作
	\[\lim\limits_{n\to \infty} a_n = A \text{,或}\ a_n \to a(n \to \infty)\]
\end{definition}

若数列 $\{a_i\}$ 存在 $A \in \mathbb{R}$ 使得 $a_n \to A$ 成立,则称为收敛的。反之称为发散的,逻辑展开即:对任意 $A$ 都有 $a_n$ 不收敛至 $A$。

在使用 $\eps - N$ 语言时,$N(\eps)$ 的选取是非常有技巧的,需要多加练习才能感悟到。

特殊地,若 $\lim\limits_{n\to \infty} a_n = 0$,则称 $\{a_n\}$ 为无穷小数列。

\begin{definition}[无穷大数列]
	若数列 $\{a_n\}$ 满足:对任意正数 $M>0$,存在正整数 $N$,使得当 $n>N$ 时,

	(1) $a_n>M$,则称数列 $\{a_n\}$ 发散于正无穷大,记作 $\lim\limits_{n\to \infty} a_n = +\infty$,或 $a_n \to +\infty$。

	(2) 有 $a_n<M$,则称数列 $\{a_n\}$ 发散于负无穷大,记作 $\lim\limits_{n\to \infty} a_n = -\infty$,或 $a_n \to -\infty$。

	两者合称无穷大数列。
\end{definition}

\begin{example}
	证明数列 $a_n = \sin n$ 发散。
\end{example}

\begin{proof}
	不妨假设其极限为 $A$,任取 $\eps$ 存在 $N$ 使得当 $n > N$ 时有 $|\sin n - A| < \eps$。注意到
	\[ |2 \sin 1 \cos n| = |\sin(n+1) - \sin(n-1)| < 2 \eps \]
	可以得到 $\cos n \to 0$,又
	\[ |\sin 2n| = 2 |\sin n \cos n| < 2 |\cos n| < \frac{2\eps}{\sin 1} \]
	从而 $\sin n \to 0$。显然有矛盾
	\[ |\sin^2 2n + \cos^2 2n| < \frac{5 \eps^2}{\sin^2 1} < 1 \]
	故不存在极限,即发散。
\end{proof}

\section{收敛数列的性质}

\begin{theorem}[唯一性]
	若数列 $\{a_n\}$ 收敛,则它只有一个极限。
\end{theorem}

\begin{proof}
	如果数列 $\{a_n\}$ 同时以 $A,B$ 为极限,即任给 $\eps>0$,总存在 $N_1,N_2$,使得
	\[|a_n-A|<\eps,n>N_1;\quad |a_n-B|<\eps,n>N_2\]
	那么当 $n>\max\{N_1,N_2\}$ 时需要恒成立
	\[2\eps > |a_n-A|+|a_n-B| \geqslant |A-B|\]
	当 $A\ne B$ 时,对于 $2\eps <|A-B|$ 不恒成立,因此只能 $A=B$。
\end{proof}

\begin{theorem}[有界性]
	若数列 $\{a_n\}$ 收敛,则 $\{a_n\}$ 有界。
\end{theorem}

\begin{proof}
	不妨设 $\lim\limits_{n\to \infty} a_n = A$。令 $\eps = 1$,那么存在 $n>N$ 使得
	\[|a_n-A| \leqslant 1\]
	令
	\[M = \{|a_1|,\cdots,|a_N|,|A-1|,|A+1|\}\]
	那么对任意正整数 $n$,总有 $|a_n|\leqslant M$。
\end{proof}

\begin{theorem}[保序性]
	设 $\lim\limits_{n\to \infty} a_n = A,\lim\limits_{n\to \infty} b_n = B$,则有

	(1) 如果存在 $N$ 使得当 $n>N$ 时有 $a_n\geqslant b_n$ 恒成立,则 $A\geqslant B$。

	(2) 反之,如果 $A>B$,则存在 $N$ 使得当 $n>N_1$ 时 $a_n>b_n$ 恒成立。
\end{theorem}
\begin{proof}
	(1) 如果设 $B-A=2\delta>0$,那么存在 $N_2,N_3>N$
	\[|a_n-A|<\delta,n>N_2;\qquad |b_n-B|<\delta,n>N_3\]
	于是当 $n>\max\{N_2,N_3\}$ 时有
	\[a_n<A+\delta=B-\delta<b_n\]
	因此矛盾,故 $A\geqslant B$。

	(2) 设 $A-B=2\delta>0$,那么存在 $N_2,N_3$
	\[|a_n-A|<\delta,n>N_2;\qquad |b_n-B|<\delta,n>N_3\]
	于是存在 $N_1=\max\{N_2,N_3\}$,当 $n>N_1$ 时有
	\[a_n>A-\delta=B+\delta>b_n\]
\end{proof}

若 $b_n$ 是常数列,$A\ne 0$,我们还可得到推论:存在 $N$,使得当 $n>N$ 时,有
\[ \frac{1}{2}|A| < |a_n| < \frac{3}{2}|A| \]

\begin{theorem}[迫敛性,夹逼定理]
	设数列 $\{a_n\},\{b_n\},\{c_n\}$ 满足当 $n>N_0$ 有 $a_n\leqslant c_n\leqslant b_n$。若
	\[\lim_{n\to \infty}a_n = A = \lim_{n\to \infty}c_n\]
	则 $\lim\limits_{n\to \infty}b_n = A$。
\end{theorem}
\begin{proof}
	即对于任给的 $\eps>0$,存在 $N_1,N_2$,使得当 $n>N_1$ 有
	\[A-\eps<a_n<A+\eps\]
	当 $n>N_2$ 有
	\[A-\eps<c_n<A+\eps\]
	因此当 $n>\max\{N_0,N_1,N_2\}$ 时,有
	\[A-\eps < a_n \leqslant b_n \leqslant c_n < A+\eps\]
\end{proof}

\begin{example}
	如果 $\seq{a}{k} > 0$,那么有
	\[ \lim_{n \to \infty} \sqrt[n]{a_1^n + \cdots + a_k^n} = \max\{\seq{a}{k}\} \]
\end{example}

\begin{proof}
	不妨设 $a_1 = \max\{\seq{a}{k}\}$,那么有
	\[ a_1 < \sqrt[n]{a_1^n + \cdots + a_k^n} < \sqrt[n]{k a_1^n} \to a_1 \]
	由夹逼原理知原式成立。
\end{proof}

\begin{theorem}[四则运算]
	设 $\lim\limits_{n\to \infty} a_n = A,\lim\limits_{n\to \infty} b_n = B$,则有

	(1) $\{\alpha a_n+\beta b_n\}$ 收敛到 $\alpha A+\beta B$,其中 $\alpha,\beta$ 为常数。

	(2) $\{a_nb_n\}$ 收敛到 $AB$。

	(3) 当 $B\ne 0$ 时,$\{a_n/b_n\}$ 收敛到 $A/B$。
\end{theorem}
\begin{proof}
	(1) 任给 $\eps >0$,存在 $N_1,N_2$ 使得
	\[|a_n-A|<\frac{\eps}{2|\alpha|+1},n>N_1;\qquad |b_n-B|<\frac{\eps}{2|\beta|+1},n>N_2\]
	则当 $n>\max\{N_1,N_2\}$ 时有
	\[
		\begin{aligned}
			|(\alpha a_n+\beta b_n)-(\alpha A+\beta B)| & \leqslant |\alpha||a_n-A|+|\beta||b_n-B|                          \\
			                                            & < \frac{\eps|\alpha|}{2|\alpha|+1}+\frac{\eps|\beta|}{2|\beta|+1} \\
			                                            & < \frac{\eps}2+\frac{\eps}2=\eps
		\end{aligned}
	\]

	(2) 由收敛数列的有界性,存在 $M$ 使得 $|a_n|\leqslant M$,那么
	\[0\leqslant |a_nb_n-AB|=|(a_n-A)b_n+A(b_n-B)|\leqslant M|a_n-A|+|A||b_n-B|\]
	由迫敛性知 $\lim\limits_{n\to \infty}|a_nb_n-AB| =0$。

	(3) 由保号性的推论,存在 $N$ 使得当 $n>N$ 时有 $|b_n|>\dfrac{|B|}{2}$,那么
	\[0 \leqslant \left|\frac{1}{b_n}-\frac{1}{B}\right| = \frac{|b_n-B|}{|b_n||B|} \leqslant \frac{2}{|B|^2}{|b_n-B|}\]
	由迫敛性知 $\lim\limits_{n\to \infty}\left|\frac{1}{b_n}-\frac{1}{B}\right| =0$。
\end{proof}

\subsection{Stolz 定理}

Stolz 定理主要是用来处理 $\infty/\infty$ 型和 $0/0$ 型极限,可以认为是洛必达的替代。

\begin{theorem}
	对于任意的 $1 \leqslant k \leqslant n$,设 $b_k>0$ 且 $m \leqslant \dfrac{a_k}{b_k} \leqslant M$,则有
	\[m \leqslant \frac{\sum a_n}{\sum b_n} \leqslant M\]
\end{theorem}

\begin{theorem}[Stolz 定理一]
	设数列 $\{x_n\},\{y_n\}$,且 $\{y_n\}$ 严格单调地趋于 $+\infty$,如果
	\[\lim_{n\to \infty}\frac{x_n-x_{n-1}}{y_n-y_{n-1}}=A\]
	则
	\[\lim_{n\to \infty} \frac{x_n}{y_n} = A\]
\end{theorem}

\begin{proof}
	分类讨论 Todo……
\end{proof}

\begin{theorem}[Stolz 定理二]
	设数列 $\{y_n\}$ 严格单调地趋于 $0$,且数列 $\{x_n\}$ 也收敛到 $0$,那么如果
	\[\lim_{n\to \infty}\frac{x_n-x_{n-1}}{y_n-y_{n-1}}=A\]
	则
	\[\lim_{n\to \infty} \frac{x_n}{y_n} = A\]
\end{theorem}

\begin{proof}
	分类讨论 Todo……
\end{proof}


\section{数列收敛的判别法则}

我们需要一些更方便的判别法则。

\subsection{单调数列}

若数列 $\{a_n\}$ 各项满足关系式 $a_n \leqslant a_{n+1}(a_n \geqslant a_{n+1})$,则称 $\{a_n\}$ 为递增(递减)数列,统称为单调数列。

\begin{theorem}[单调有界定理]
	单调有界数列必有极限。
\end{theorem}

\begin{proof}
	不妨设 $\{a_i\}$ 为有上界的单调递增序列,由确界原理知存在上确界 $\beta$。按其定义,任给 $\eps > 0$ 都存在 $N$ 使得 $\beta - \eps < a_N < \beta$(否则 $\beta - \eps$ 就是新的上确界)。故 $|a_n - \beta| < \eps$,即收敛至 $\beta$。
\end{proof}

\subsection{子列}

设 $\{a_n\}$ 为数列,如果 $\{n_k\}$ 是一列严格递增的正整数,则数列 $\{a_{n_k}\}$ 称为数列 $\{a_n\}$ 的一个子列。子列显然有性质 $n_k \geqslant k$,归纳易证。

特殊的子列 $\{a_{2k}\}$ 和 $\{a_{2k-1}\}$ 分别称为偶子列与奇子列。数列本身也是其自己的子列。

\begin{theorem}[Weierstrass 致密性定理]
	任何有界数列必定有收敛的子列。
\end{theorem}

\begin{proof}
	不妨设数列包含无数个不同的 $a_n$,否则显然成立。假设数列有界,设其值域为 $[A_0, B_0]$。注意到我们可以对分区间为 $[A_0, \frac{A_0 + B_0}{2}]$ 和 $[\frac{A_0 + B_0}{2}, B_0]$,至少其中之一包含无穷多个 $a_n$,记为 $[A_1, B_1]$。我们可以不断划分,得到一闭缩区间套
	\[ [A_0, B_0] \supset [A_1, B_1] \subset \cdots   \]
	总是可以在区间中找到下标递增的项,即我们要求的子列。
\end{proof}

\begin{theorem}
	数列 $\{a_n\}$ 收敛的充要条件:$\{a_n\}$ 的任何子列都收敛。
\end{theorem}

\subsection{Cauchy 准则}

\begin{definition}
	设 $\{a_n\}$ 为数列,如果任给 $\eps>0$,均存在 $N(\eps)$ 使任取 $m,n>N(\eps)$ 有
	\[|a_m-a_n| < \eps\]
	则称 $\{a_n\}$ 为 Cauchy 数列或基本列。
\end{definition}

反之,若存在 $\eps > 0$ 使得任给 $N$ 都存在 $n, m > N$ 使得
\[ |a_n - a_m| \geqslant \eps \]
则称该数列为非 Cauchy 的。

\begin{theorem}
	Cauchy 数列必定是有界数列。
\end{theorem}
\begin{proof}
	取 $\eps=1$,则存在 $N$ 使得当 $m,n>N$ 时有
	\[|a_m-a_n| < 1\]
	令 $M = \max\{|a_k|+1 \mid 1 \leqslant k \leqslant N+1\}$,则当 $n\leqslant N$ 时显然有 $|a_n|\leqslant M$,而当 $n>N$ 时有
	\[|a_n| \leqslant |a_n-a_{N+1}| + |a_{N+1} < 1+ |a_{N+1}| \leqslant M\]
	这说明 $\{a_n\}$ 是有界数列。
\end{proof}

\begin{theorem}[Cauchy 收敛准则]
	$\{a_n\}$ 为 Cauchy 数列当且仅当它是收敛的。
\end{theorem}
\begin{proof}
	(1) 充分性:设 $\{a_n\}$ 收敛到 $A$,则任给 $\eps >0$ 存在 $N$,当 $n>N$ 时有
	\[|a_n-A|\leqslant \frac{\eps}{2}\]
	因此当 $m,n>N$ 时有
	\[|a_m-a_n| \leqslant |a_m-A| + |A-a_n| < \frac{\eps}{2}+\frac{\eps}{2}=\eps\]
	这说明 $a_n$ 为 Cauchy 数列。

	(2) 必要性:已证 Cauchy 列有界,则必存在收敛子列 $\{a_{u_k}\}$,因此任给 $\eps > 0$ 存在 $N_1$ 使得当 $u_i > N_1$ 时有
	\[ | a_{u_i} - A| < \frac{\eps}{2} \]
	又由定义知存在 $N_2$ 使得任取 $n, m > N_2$ 有
	\[ |a_n - a_m| < \frac{\eps}{2} \]
	因此取 $N = \max\{N_1, N_2\}$,取 $u_k > N$,则当 $n > N$ 时有
	\[ |a_n - A| \leqslant |a_n - a_{u_k}| + | a_{u_k} - A| \leqslant \eps \]
\end{proof}

\begin{theorem}[不动点原理]
	设递推数列 $a_{n+1} = f(a_n)$,假设 $a_n \subset (\alpha, \beta)$,若存在常数 $L \in (0, 1)$ 使得对任意 $x,y \in (\alpha, \beta)$ 有
	\[ |f(x) - f(y)| \leqslant L|x-y| \]
	则数列收敛。
\end{theorem}

\begin{proof}
	首先类似于等比
	\[ |a_{n+1} - a_n| \leqslant \cdots \leqslant L^{n-1}|a_2 - a_1| \]
	从而
	\[ |a_{n+k} - a_{n}| \leqslant \sum_{i=1}^{k} |a_{n + i} - a_{n + i - 1}| \leqslant \sum_{i=1}^{k} L^{n+i} |a_{2} - a_{1}| \leqslant \frac{L^{n-1}}{1-L}|a_2 - a_1|  \]
	由 Cauchy 收敛准则知数列收敛。
\end{proof}

\section{常见数列}

首先我们定义三个数列
\[ a_n = \left(1 + \frac{1}{n}\right)^n, \quad b_n = \left(1 + \frac{1}{n}\right)^{n+1}, e_n = \sum_{i=0}^{n}\frac{1}{k!} \]
我们通常定义 $a_n$ 的极限为 $\ee$。下证三者极限存在且相同。

其中 $e_n$ 的单调性是显然的。我们先证:
\[ a_n < a_{n+1},\quad b_n > b_{n+1} \]
首先
\[ \begin{aligned}
		a_n & = \sum_{k=0}^{n} \binom{n}{k} \frac{1}{n^k}                                                    \\
		    & = 1 + \sum_{k=1}^n \frac{1}{k!} \prod_{j=1}^{k-1} \left(1 - \frac{j}{n}\right)                 \\
		    & < 1 + \sum_{k=1}^n \frac{1}{k!} \prod_{j=1}^{k-1} \left(1 - \frac{j}{n+1}\right)               \\
		    & < 1 + \sum_{k=1}^{n+1} \frac{1}{k!} \prod_{j=1}^{k-1} \left(1 - \frac{j}{n+1}\right) = a_{n+1} \\
	\end{aligned} \]
再借助 Bernoulli 不等式
\[ \begin{aligned}
		\frac{b_{n-1}}{b_n} & = \frac{\left(1 + \frac{1}{n-1}\right)^n}{\left(1 + \frac{1}{n}\right)^{n+1}} \\
		                    & = \left(1+\frac{1}{n^2 - 1}\right)^n \frac{1}{1 + \frac{1}{n}}                \\
		                    & > \left(1+\frac{n}{n^2 - 1}\right) \frac{1}{1 + \frac{1}{n}}                  \\
		                    & = 1 + \frac{1}{(n+1)^2(n-1)} > 1
	\end{aligned} \]

注意到当 $n > 2$ 时
\[ a_n \leqslant 1 + \sum_{k=1}^n \frac{1}{k!} = e_n \leqslant 2 + \sum_{k=2}^n \frac{1}{k(k-1)} = 3 - \frac{1}{n} < 3 \]
故 $a_n$ 和 $b_n$ 单调有界,故必有极限。再注意到
\[ b_n = \left(1 + \frac{1}{n}\right) a_n \]
由极限的四则运算,故 $b_n$ 的极限也存在,且 $a_n$ 和 $b_n$ 收敛于同一个值。

我们定义 $a_n$ 的极限为 $\ee$,下证 $e_n \to \ee$。注意到固定 $u$ 有
\[ \begin{aligned}
		a_n & = 1 + \sum_{k=1}^n \frac{1}{k!} \prod_{j=1}^{k-1} \left(1 - \frac{j}{n}\right) \\
		    & > 1 + \sum_{k=1}^u \frac{1}{k!} \prod_{j=1}^{k-1} \left(1 - \frac{j}{n}\right)
	\end{aligned} \]
那么令 $n \to \infty$,有
\[ e_k = \sum_{k=0}^u \frac{1}{k!}\leqslant a_n \to \ee \]
故由夹逼定理知 $e_n \to \ee$。

 % 行列式
%% \newcommand{\mfrac}[2]{\frac{#1}{#2}}

\chapter{函数极限}

\section{函数极限的概念}

\begin{definition}
	设 $f$ 为定义在 $[a,+\infty)$ 上的函数,$A$ 为定数。若对任给的 $\eps>0$,存在正数 $M=M(\eps) \geqslant a$,使得当 $x>M$ 时,有
	\[ |f(x)-A| < \eps \]
	则称函数 $f$ 当 $x$ 趋于 $+\infty$ 时以 $A$ 为极限,记作
	\[ \lim_{x \to +\infty}f(x) = A\ \text{或}\ f(x) \to A(x \to +\infty) \]
\end{definition}

类似的有 $\lim\limits_{x \to -\infty}f(x)$ 和 $\lim\limits_{x \to \infty}f(x)$。不难证明
\[ \lim_{x \to \infty}f(x) = A \Leftrightarrow \lim_{x \to -\infty}f(x)=\lim_{x \to +\infty}f(x)=A \]

为了描述在某点处的极限,我们需要邻域的概念
\[ U(a;\delta) = (a-\delta,a+\delta) \]
和空心邻域的概念
\[ U^\circ (a;\delta) = (a-\delta,a) \cup (a, a+\delta) \]

\begin{definition}
	设函数 $f$ 在 $U^\circ(x_0;\delta')$ 内有定义,$A$ 为定数。若对任给的 $\eps>0$,存在正数 $\delta<\delta'$,使得当 $0<|x-x_0|<\delta$ 时,有 $|f(x)-A|<\eps$,则称函数 $f$ 当 $x$ 趋于 $x_0$ 时以 $A$ 为极限,记作
	\[ \lim_{x \to x_0}f(x) = A\ \text{或}\ f(x)\to A(x \to x_0) \]
\end{definition}

\begin{definition}
	设函数 $f$ 在 $U_+^\circ(x_0;\delta')$ 内有定义,$A$ 为定数。若对任给的 $\eps>0$,存在正数 $\delta<\delta'$,使得当 $x_0<x<x_0+\delta$ 时,有 $|f(x)-A|<\eps$,则称函数 $f$ 当 $x$ 趋于 $x_0^+$ 时以 $A$ 为极限,记作
	\[ \lim_{x \to x_0^+}f(x) = A\ \text{或}\ f(x)\to A(x \to x_0^+) \]
\end{definition}

类似的还有左极限 $\lim\limits_{x \to x_0^-}f(x)$,统称为单侧极限。又可记为
\[ f(x_0+0) = \lim_{x \to x_0^+}f(x)\ \text{与}\ f(x_0-0) = \lim_{x \to x_0^-}f(x) \]

同理还有
\[ \lim_{x \to x_0}f(x) = A \Leftrightarrow \lim_{x \to x_0^+}f(x)=\lim_{x \to x_0^-}f(x)=A \]

\section{函数极限的性质}

\begin{theorem}[唯一性]
	若极限 $\lim\limits_{x \to x_0}f(x)$ 存在,则此极限是唯一的。
\end{theorem}

\begin{theorem}[局部有界性]
	若极限 $\lim\limits_{x \to x_0}f(x)$ 存在,则 $f$ 在 $x_0$ 的某空心邻域 $U^\circ(x_0)$ 上有界。
\end{theorem}

\begin{theorem}[局部保序性]
	设 $\lim\limits_{x \to x_0}f(x)$ 与 $\lim\limits_{x \to x_0}g(x)$ 均存在。若存在正数 $N_0$,使得当 $n>N_0$ 时,有 $a_n\leqslant b_n$,则 $\lim\limits_{n\to \infty}a_n \leqslant \lim\limits_{n\to \infty}b_n$。
\end{theorem}

\begin{theorem}[夹逼定理]
	设 $\lim\limits_{x \to x_0}f(x) = \lim\limits_{x \to x_0}g(x) = A$,且在某 $U^\circ(x_0;\delta')$ 上有
	\[ f(x)\leqslant h(x) \leqslant g(x) \]
	则 $\lim\limits_{x \to x_0}h(x) = A$。
\end{theorem}

\begin{theorem}[四则运算法则]
	若 $\lim\limits_{x \to x_0}f(x)$ 与 $\lim\limits_{x \to x_0}g(x)$ 均存在,则
	\[ \lim_{x \to x_0}[f(x)\pm g(x)] = \lim_{x \to x_0}f(x) + \lim_{x \to x_0}g(x) \]
	\[ \lim_{x \to x_0}[f(x)g(x)] = \lim_{x \to x_0}f(x) \cdot \lim_{x \to x_0}g(x) \]
	若 $\lim\limits_{x \to x_0}g(x)\ne 0$,则
	\[ \lim_{x \to x_0}\frac{f(x)}{g(x)} = \frac{\lim\limits_{x \to x_0}f(x)}{\lim\limits_{x \to x_0}g(x)} \]
\end{theorem}

\section{函数极限存在的条件}

\begin{theorem}[海涅 Heine 定理]
	若 $f(x)$ 在 $U^\circ(x_0;\delta')$ 上有定义。$\lim\limits_{x \to x_0}f(x)$ 存在的充要条件是:任何含于 $U^\circ(x_0;\delta')$ 且以 $x_0$ 为极限的数列 $\{x_n\}$,极限 $\lim\limits_{x \to x_0}f(x_n)$ 都存在且相等。
\end{theorem}

即若对任何 $x_n\to x_0(n\to \infty)$ 有 $\lim\limits_{n\to \infty}f(x_n) = A$,则 $\lim\limits_{x \to x_0}f(x)=A$。

\begin{theorem}
	设 $f(x)$ 在点 $x_0$ 的某空心右邻域 $U_+^\circ(x_0)$ 有定义,则 $\lim\limits_{x \to x_0^+}f(x)=A$ 的充要条件是:对任何以 $x_0$ 为极限的递减数列 $\{x_n\}\subset U_+^\circ(x_0)$,有 $\lim\limits_{n\to \infty}f(x_n) = A$。
\end{theorem}

\begin{theorem}
	设 $f(x)$ 为定义在 $U_+^\circ(x_0)$ 上的单调有界函数,则右极限 $\lim\limits_{x \to x_0^+}f(x)=A$ 存在。
\end{theorem}

\begin{theorem}[Cauchy 准则]
	设 $f(x)$ 在 $U^\circ(x_0;\delta')$ 上有定义,则 $\lim\limits_{x \to x_0}f(x)$ 存在的充要条件是:任给 $\eps > 0$,存在正数 $\delta(<\delta')$,使得对任何 $x',x''\in U^\circ(x_0,\delta)$,有 $|f(x')-f(x'')|<\eps$。
\end{theorem}

\section{两个重要的极限}

\begin{proposition}
	\[ \lim_{x \to 0}\frac{\sin x}{x} = 1 \]
\end{proposition}

\begin{proof}
	首先注意到它是奇函数,只需讨论 $0 < x < \frac{\pi}{2}$。考虑圆上的角度为 $x$ 的弧的弧长和弦长,容易从几何角度得到
	\[ 0 < \sin x < x \]
	再考虑圆上的角度为 $x$ 的弧与三角形的面积,我们可以得到
	\[ \frac{1}{2}\cos x < \frac{1}{2} x < \frac{1}{2} \tan x \]
	即我们得到了
	\[ \cos x < \frac{\sin x}{x} < 1 \]
	利用夹逼定理,我们只需证明 $\cos x \to 1$,这点可以用后面的连续性证明。也可以直接
	\[ |\cos x - 1| = \left| 2 \sin^2 \frac{x}{2} \right| < \left|  \frac{x^2}{2} \right| \]
\end{proof}

第二个是把数列极限推广到函数

\begin{proposition}
	\[ \lim_{x \to \infty}\left(1+\frac{1}{x}\right)^x = \ee \]
\end{proposition}


\section{无穷小量与无穷大量}

\begin{definition}[无穷小量]
	设函数 $f$ 在某 $U^\circ(x_0)$ 上有定义,若 $\lim\limits_{x \to x_0}f(x)=0$,则称 $f$ 为当 $x \to x_0$ 时的无穷小量。
\end{definition}

\begin{definition}[有界量]
	设函数 $f$ 在某 $U^\circ(x_0)$ 上有界,则称 $f$ 为当 $x \to x_0$ 时的有界量。
\end{definition}

无穷小量收敛于 $0$ 的速度有快有慢。设当 $x \to x_0$ 时,$f$ 与 $g$ 均为无穷小量。

若 $\lim\limits_{x \to x_0}\mfrac{f(x)}{g(x)} = 0$,则称当 $x \to x_0$ 时 $f$ 为 $g$ 的高阶无穷小量,或称 $g$ 为 $f$ 的低阶无穷小量。

记作
\[ f(x)=o(g(x))(x \to x_0) \]
特别地,$f$ 为当 $x \to x_0$ 时的无穷小量记作
\[ f(x)=o(1)(x \to x_0) \]

若存在正数 $K$ 和 $L$,使得在某 $U^\circ(x_0)$ 上有
\[ K\leqslant \left|\frac{f(x)}{g(x)}\right| \leqslant L \]
则称 $f$ 与 $g$ 为当 $x \to x_0$ 时的同阶无穷小量。特别当
\[ \lim_{x \to x_0}\frac{f(x)}{g(x)} = c \ne 0 \]
时,$f$ 与 $g$ 必为同阶无穷小量。

若 $\lim\limits_{x \to x_0}\mfrac{f(x)}{g(x)} = 1$ 则称 $f$ 与 $g$ 是当 $x \to x_0$ 时的等价无穷小量。记作
\[ f(x) \sim g(x) (x \to x_0) \]

注意并不是任何两个无穷小量都可以进行这种阶的比较。例如 $x \to 0$ 时,$x\sin\dfrac{1}{x}$ 和 $x^2$ 都是无穷小量,但它们的比都不是有界量。

\begin{theorem}
	设函数 $f,g,h$ 在 $U^\circ(x_0)$ 上有定义,且有 $f(x) \sim g(x)(x \to x_0)$,则

	1.若 $\lim\limits_{x \to x_0}f(x)h(x) = A$,则 $\lim\limits_{x \to x_0}g(x)h(x) = A$。

	2.若 $\lim\limits_{x \to x_0}\frac{h(x)}{f(x)}=B$,则 $\lim\limits_{x \to x_0}\frac{h(x)}{g(x)}=B$
\end{theorem}

\begin{definition}[无穷大量]
	设函数 $f$ 在某 $U^\circ(x_0)$ 上有定义,若对任给的 $G>0$,存在 $\delta>0$,使得当 $x\in U^\circ(x_0;\delta)\subset U^\circ(x_0)$ 时,有 $|f(x)|>G$,则称函数 $f$ 当 $x \to x_0$ 时有非正常极限 $\infty$,记作 $\lim\limits_{x \to x_0}f(x) = \infty$。
\end{definition}

\section{常见等价无穷小}

实际上这些等价无穷小就是 Talor 展开。

\[
	\begin{aligned}
		\frac{1}{1-x} & = \sum_{k=0}^\infty x^n,(-1,1)                                                                                                                  \\
		              & = 1 + x + x^2 + x^3 + x^4 + x^5 + x^6 + O(x^7)                                                                                                  \\
		\ln(1+x)      & = \sum_{k=0}^\infty\frac{(-1)^k}{k+1}x^{k+1},(-1,1]                                                                                             \\
		              & = x - \frac{x^2}{2} + \frac{x^3}{3} - \frac{x^4}{4} + \frac{x^5}{5} - \frac{x^6}{6} + \frac{x^7}{7} + O(x^8)                                    \\
		\sin x        & = \sum_{k=0}^\infty \frac{(-1)^k}{(2k+1)!}x^{2k+1},\mathbb{R}                                                                                   \\
		              & = x - \frac{x^3}{6} + \frac{x^5}{120} - \frac{x^7}{5040} + \frac{x^9}{362880}+ O(x^{11})                                                        \\
		\cos x        & = \sum_{k=0}^\infty \frac{(-1)^k}{(2k)!}x^{2k},\mathbb{R}                                                                                       \\
		              & = 1 - \frac{x^2}{2} + \frac{x^4}{24} - \frac{x^6}{720} + \frac{x^8}{40320} + \frac{x^{10}}{3628800} + O(x^{12})                                 \\
		\ee^x         & = \sum_{k=0}^\infty\frac{1}{k!}x^k,\mathbb{R}                                                                                                   \\
		              & = 1 + x + \frac{x^2}{2} + \frac{x^3}{6} + \frac{x^4}{24} + \frac{x^5}{120} + \frac{x^6}{720} + \frac{x^7}{5040} + \frac{x^8}{40320} + O(x^{10}) \\
		\tan x        & = \sum_{k=1}^\infty \frac{(-4)^k(1-4^k)B_{2k}}{(2k)!}x^{2k-1},(-\frac{\pi}{2},\frac{\pi}{2})                                                    \\
		              & = x + \frac{x^3}{3} + \frac{2x^5}{15} + \frac{17x^{7}}{315} + \frac{67x^9}{2835} + O(x^{11})                                                    \\
		\sqrt{x+1}    & = 1 + \sum_{k=1}^\infty \left(-\frac{1}{2}\right)^k (2k-1)!!x^k,(-1, +\infty)                                                                   \\
		              & = 1 + \frac{x}{2}-\frac{x^2}{8} +\frac{x^3}{16}-\frac{5 x^4}{128} +\frac{7 x^5}{256}   - \frac{21x^6}{1024}   +  O(x^{7})                       \\
		\arcsin x     & = x + \frac{x^3}{6} + \frac{3x^5}{40} + \frac{5 x^7}{112} + \frac{35x^9}{1152}                                                + O(x^{11})       \\
		\arctan x     & = x - \frac{x^3}{3} + \frac{x^5}{5} - \frac{x^7}{7}                                                     + \frac{x^9}{9}          + O(x^{11})
	\end{aligned}
\]

% \begin{equation*}
%     \begin{aligned}
%         \frac{x}{1-x} & = x + x^2 + x^3 + x^4 + x^5                                                      & + O(x^6)    \\
%         \ln(1+x)      & = x - \frac{x^2}{2} + \frac{x^3}{3} - \frac{x^4}{4} + \frac{x^5}{5}              & + O(x^6)    \\
%         \sin x        & = x - \frac{x^3}{6} + \frac{x^5}{120}                                            & + O(x^{7})  \\
%         1- \cos x     & = \frac{x^2}{2} - \frac{x^4}{24}                                                 & + O(x^{6})  \\
%         \ee^x-1       & = x + \frac{x^2}{2} + \frac{x^3}{6} + \frac{x^4}{24} + \frac{x^5}{120}           & +  O(x^{6}) \\
%         \tan x        & = x + \frac{x^3}{3} + \frac{2x^5}{15}                                            & + O(x^{7})  \\
%         \sqrt{x+1}-1  & = \frac{x}{2}-\frac{x^2}{8} +\frac{x^3}{16}-\frac{5 x^4}{128} +\frac{7 x^5}{256} & +  O(x^{6}) \\
%         \arcsin x     & = x + \frac{x^3}{6} + \frac{3x^5}{40}                                            & + O(x^7)    \\
%         \arctan x     & = x - \frac{x^3}{3} + \frac{x^5}{5}                                              & + O(x^7)
%     \end{aligned}
% \end{equation*}

\section{函数的连续性}

\begin{definition}[连续性]
	设函数 $f$ 在某 $U(x_0)$ 上有定义。若
	\[ \lim_{x\to x_0}f(x) = f(x_0) \]
	则称 $f$ 在点 $x_0$ 连续。
\end{definition}

记 $\Delta x = x-x_0$,称为自变量 $x$ 在点 $x_0$ 的增量或改变量。设 $y_0=f(x_0)$,相应的函数 $y$ 在点 $x_0$ 的增量记为
\[ \Delta y = f(x)-f(x) = f(x+\Delta)-f(x_0) = y-y_0 \]

连续性的 $\eps-\delta$ 形式定义:若对任给的 $\eps>0$,存在 $\delta>0$,使得当 $|x-x_0|<\delta$ 时,有 $|f(x)-f(x_0)|<\eps$,则称函数 $f$ 在点 $x_0$ 连续。

或者进一步表示为
\[ \lim_{x\to x_0}f(x) = f\left(\lim_{x\to x_0}x\right) \]

\begin{definition}
	设函数 $f$ 在某 $U_+(x_0)$ 上有定义。若
	\[ \lim_{x\to x_0^+}f(x) = f(x_0) \]
	则称 $f$ 在点 $x_0$ 右连续。同理左连续。
\end{definition}

因此函数 $f$ 在点 $x_0$ 连续的充要条件是:$f$ 在点 $x_0$ 既是左连续,又是右连续。

\begin{definition}[间断点]
	设函数 $f$ 在某 $U^\circ(x_0)$ 上有定义。若 $f$ 在点 $x_0$ 无定义,或 $f$ 在点 $x_0$ 有定义而不连续,则称点 $x_0$ 为函数 $f$ 的间断点或不连续点。
\end{definition}

若 $\lim\limits_{x\to x_0}f(x)=A$,而 $f$ 在点 $x_0$ 无定义,或有定义但 $f(x_0)\ne A$,则称点 $x_0$ 为 $f$ 的可去间断点。

若函数 $f$ 在点 $x_0$ 的左、右极限都存在,但 $\lim\limits_{x\to x_0^+}f(x) \ne \lim_{x\to x_0^-}f(x)$,则称点 $x_0$ 为函数 $f$ 的跳跃间断点。

可去间断点与跳跃间断点统称为第一类间断点,所有其他形式的间断点统称为第二类间断点。

若函数 $f$ 在区间 $I$ 上的每一点都连续,则称 $f$ 为 $I$ 上的连续函数。对于闭区间或半开区间的端点,函数在这些点上连续是指左连续或右连续。

\subsection{连续函数的性质}

\begin{theorem}[局部有界性]
	若函数 $f$ 在点 $x_0$ 连续,则 $f$ 在某 $U(x_0)$ 上有界。
\end{theorem}

\begin{theorem}[局部保号性]
	若函数 $f$ 在点 $x_0$ 连续,且 $f(x_0)>0$,则对任何正数 $r<f(x_0)$,存在某 $U(x_0)$,使得对一切 $x\in U(x_0)$,有 $f(x)>r$。
\end{theorem}

\begin{theorem}[四则运算]
	若函数 $f,g$ 在点 $x_0$ 连续,则 $f\pm g,f\cdot g,f/g$ 也都在点 $x_0$ 连续。
\end{theorem}

\begin{theorem}
	若函数 $f$ 在点 $x_0$ 连续,$g$ 在点 $u_0$ 连续,$u_0=f(x_0)$,则复合函数 $g\circ f$ 在 $x_0$ 连续。
\end{theorem}

\begin{definition}
	设 $f$ 为定义在数集 $D$ 上的函数。若存在 $x_0\in D$,使得对一切 $x\in D$,有 $f(x_0)\ge f(x)$,则称 $f$ 在 $D$ 上有最大值,并称 $f(x_0)$ 为 $f$ 在 $D$ 上的最大值。
\end{definition}

\begin{theorem}[最大、最小值定理]
	若函数 $f$ 在闭区间 $[a,b]$ 上连续,则 $f$ 在闭区间 $[a,b]$ 上有最大值与最小值。
\end{theorem}

\begin{theorem}[介值定理]
	若函数 $f$ 在闭区间 $[a,b]$ 上连续,且 $f(a)\ne f(b)$。若 $\mu$ 为介于 $f(a)$ 和 $f(b)$ 之间的任何实数。则至少存在一点 $x_0\in (a,b)$ 使得 $f(x_0)=\mu$。
\end{theorem}

\begin{theorem}
	若函数 $f$ 在 $[a,b]$ 上严格单调并连续,则反函数 $f^{-1}$ 在其定义域 $[\min\{f(a),f(b)\},\max\{f(a),f(b)\}]$ 上连续。
\end{theorem}

\begin{definition}
	设 $f$ 是定义在区间 $I$ 上的函数。若对任给的 $\eps>0$ 存在 $\delta=\delta(\eps)>0$ 使得对任何 $x',x''\in I$,只要 $|x'-x''|<\delta$ 就有
	\[ |f(x')-f(x'')|<\eps \]
	就称函数 $f$ 在区间 $I$ 上一致连续。
\end{definition}

\begin{theorem}[一致连续性]
	若函数 $f$ 在闭区间 $[a,b]$ 上连续,则 $f$ 在 $[a,b]$ 上一致连续。
\end{theorem}

\subsection{初等函数的连续性}

\begin{theorem}
	设 $p>0$,$a,b$ 为任意两个实数,则有
	\[ p^a\cdot p^b = p^{a+b},(p^a)^b=p^{ab} \]
\end{theorem}

\begin{theorem}
	指数函数 $a^x(a>0)$ 在 $\mathbb{R}$ 上是连续的。
\end{theorem}


 % 线性方程组的解系
\chapter{矩阵的运算}

\section{矩阵的运算}

数域上 $K$ 两个矩阵的行数、列数都相等,且所有元素对应相等,那么称两个矩阵相等。

\begin{definition}
	设数域 $K$ 上的 $s \times n$ 矩阵 $A=(a_{ij}),B=(b_{ij})$,令 $A,B$ 的和为
	\[A + B \coloneqq  (a_{ij}+b_{ij})_{s \times n}\]
\end{definition}

\begin{definition}
	设数域 $K$ 上的 $s \times n$ 矩阵 $A=(a_{ij})$,令 $k\in K$ 与 $A$ 的数量乘积为
	\[kA \coloneqq  (ka_{ij})_{s \times n}\]
\end{definition}

不难验证,矩阵的加法和数量乘法满足类似于 $n$ 维向量的 8 条运算法则。

同样定义矩阵的减法
\[A - B \coloneqq  A + (-B)\]

\begin{definition}
	设 $A=(a_{ij})_{s \times n},B=(a_{ij})_{n \times m}$,令 $A,B$ 的乘积为
	\[AB = \left(\sum_{k=1}^na_{ik}b_{kj}\right)_{s \times m}\]
\end{definition}

同样,若 $AB$ 和 $BA$ 几乎都不相等,甚至不一定能够运算。

\begin{theorem}
	设 $A = (a_{ij})_{s \times n},  B = (b_{ij})_{n \times m},C = (c_{ij})_{m \times r}$,则
	\[(AB)C = A(BC)\]
\end{theorem}

注意到,若 $A,B\ne 0$,有可能 $BA = 0$。因此 $BA = 0$ 不能推出 $B=0$ 或 $A=0$。

\begin{definition}[零因子]
	对于矩阵 $A$,若存在矩阵 $B\ne 0$ 使得 $AB = 0$,那么称 $A$ 是一个左零因子。如果存在一个矩阵 $C\ne 0$ 使得 $CA = 0$,那么称 $A$ 是一个右零因子。左零因子和右零因子称为零因子。 
\end{definition}

特殊的,零矩阵是零因子,称为平凡的零因子。

\begin{theorem}
	矩阵的乘法有分配律(左分配律、右分配律)
	\[A(B+C) = AB+AC\]
	\[(B+C)D = BD + CD\]
\end{theorem}

矩阵的乘法不适合消去律,从 $AC = BC$ 且 $C\ne 0$ 不能推出 $A=B$。

主对角线上元素都是 $1$,其余元素都是 $0$ 的 $n$ 级矩阵称为 $n$ 级单位矩阵,记作 $E_n$ 或者简记作 $E$(有些书上记作 $I$)。主对角线上元素是同一个数 $k$,其余元素全为 $0$ 的 $n$ 级矩阵称为数量矩阵,可以记作 $kE$。

因此有
\[E_s A_{s \times n} = A_{s \times n}, A_{s \times n} E_n= A_{s \times n}\]
若 $A$ 是 $n$ 级矩阵,则
\[EA = AE = A\]
矩阵的乘法与数量乘法满足下述关系式
\[k(AB) = (kA)B = A(kB)\]

数量矩阵还有
\[ \begin{aligned}
		kE + lE = (k+l)E \\
		k(lE) = (kl)E    \\
		(kE)(lE) = (kl)E
	\end{aligned} \]

矩阵的乘法虽不满足交换律,但若对具体的两个矩阵 $A$ 与 $B$,也有可能 $AB = BA$,那么称 $A$ 与 $B$ 可交换。比如数量矩阵与任一同级矩阵可交换
\[(kE)A = A(kE) = kA\]

\begin{definition}
	定义 $n$ 级矩阵 $A$ 的非负整数次幂为
	\[ A^n \coloneqq  A \cdots A, \quad A^{n+1} \coloneqq A A^{n} \]
\end{definition}


\begin{theorem}
	\[ \transpose{(A+B)} = \transpose{A} + \transpose{B}, \quad \transpose{(kA)} = k\transpose{A}, \quad \transpose{(AB)} = \transpose{B}\transpose{A} \]
\end{theorem}

如果把 $n$ 元线性方程组的系数矩阵记作 $A$,称常数项组成的列向量为 $\vbf{\beta}$,未知量 $\seq{x}{n}$ 组成的列向量为 $X$,那么 $n$ 元线性方程组可以写成
\[AX = \vbf{\beta}\]
于是列向量 $\vbf{\eta}$ 是方程组的 $AX = \vbf{\beta}$ 的解当且仅当 $A \vbf{\eta} = \vbf{\beta}$。

\section{特殊矩阵}

\begin{definition}
	主对角线以外的元素全为 $0$ 的方阵称为对角矩阵,简记作
	\[\diag\{\seq{d}{n}\}\]
\end{definition}

\begin{definition}
	只有一个元素是 $1$,其他元素全为 $0$ 的矩阵称为基本矩阵。$(i,j)$ 元为 $1$ 的基本矩阵记作 $E_{ij}$。
\end{definition}

\begin{definition}
	主对角线下(上)方的元素全为 $0$ 的方阵称为上(下)三角矩阵。
\end{definition}

显然 $A=(a_{ij})$ 为上三角矩阵的 充分必要条件是
\[a_{ij}=0,\text{当}\ i>j\]
同样,上三角矩阵也可表述为
\[A = \sum_{i=1}^n\sum_{j=i}^na_{ij}E_{ij}\]

\begin{definition}
	由单位矩阵经过一次初等行(列)变换得到的矩阵称为初等矩阵。
\end{definition}

初等矩阵有且只有三种类型:$P(j,i(k)),P(i,j),P(i(c))$,其中 $c\ne 0$。

1. 用 $P(j,i(k))$ 左乘 $A$,即把 $A$ 的第 $i$ 行的 $k$ 倍加到第 $j$ 行上。

2. 用 $P(j,i(k))$ 右乘 $A$,即把 $A$ 的第 $j$ 列的 $k$ 倍加到第 $i$ 列上。

类似的,用 $P(i,j)$ 左(右)乘 $A$,就相当于把 $A$ 的第 $i$ 行(列)与第 $j$ 行(列)互换。

用 $P(i(c))$ 左(右)乘 $A$,就相当于用 $c$ 乘 $A$ 的第 $i$ 行(列)。

\begin{definition}
	一个矩阵 $A$ 如果满足 $\transpose{A} = A$,那么称 $A$ 是对称矩阵。
\end{definition}

\begin{definition}
	如果一个矩阵 $A$ 如果满足 $\transpose{A} = -A$,那么称 $A$ 是斜对称矩阵。
\end{definition}

\section{矩阵乘积的秩与行列式}

\begin{theorem}
	设 $A = (a_{ij})_{s \times n},B = (b_{ij})_{n \times m}$,则
	\[\rank(AB) \leqslant \min\{\rank(A),\rank(B)\}\]
\end{theorem}

\begin{theorem}
	设 $A = (a_{ij})_{s \times n},B = (b_{ij})_{n \times m}$,则
	\[|AB| = |A||B| = |B||A| = |BA|\]
\end{theorem}

\begin{theorem}[Binet $-$ Cauchy 公式]
	设 $A = (a_{ij})_{s \times n},B = (b_{ij})_{n \times m}$,则
	
	1. 如果 $s>n$,那么 $|AB| = 0$;
	
	2. 如果 $s\leqslant n$,那么  $|AB|$ 等于 $A$ 的所有 $s$ 阶子式的与相应 $s$ 阶子式的乘积之和,即
	\[|AB|=\sum_{1 \leqslant v_1<\cdots< v_s \leqslant n}A \dbinom{1,\cdots,s}{\seq{v}{s}} \cdot B \dbinom{\seq{v}{s}}{1,\cdots,s}\]
\end{theorem}

\begin{theorem}
	\[ \rank(\transpose{A}A) = \rank(A \transpose{A}) = \rank(A) \]
\end{theorem}

\begin{proof}
	试证 $A\vbf{X} = \vbf{0}$ 与 $(\transpose{A}A) \vbf{X} = \vbf{0}$ 同解。
	
	假设 $\vbf{\eta}$ 是 $A \vbf{X} = \vbf{0}$ 的一个解,显然
	\[ (\transpose{A}A) \vbf{X} = \transpose{A} (A \vbf{\eta}) = \transpose{A} \vbf{0} = \vbf{0} \]
	反之,设 $\vbf{\eta}$ 是 $(\transpose{A}A) \vbf{X} = \vbf{0}$ 的一个解,注意到
	\[ \transpose{(A \vbf{\eta})}(A \vbf{\eta}) = \transpose{\vbf{\eta}} ((\transpose{A}A) \vbf{\eta}) = 0 \]
	即向量 $A \vbf{\eta}$ 的长度为 $0$,故 $A \vbf{\eta} = \vbf{0}$。
\end{proof}

\begin{example}
	如果数域上的 $n$ 级矩阵 $A$ 满足 $A \transpose{A} = E$,且 $|A| = -1$,求证:$|E+A| = 0$。
\end{example}

\begin{solution}
	注意到
	\[ |E+A| = |A\transpose{A} + AE| = -|\transpose{A} + E| = -|A+E| \]
	故 $|E+A| = 0$。
\end{solution}

\begin{example}
	如果 $m \times n$ 的矩阵 $A$ 和 $n \times s$ 的矩阵 $B$ 满足 $AB = O$,求证:$\rank(A) + \rank(B) \leqslant n$。
\end{example}

\begin{solution}
	注意到 $B$ 的每个列向量都是 $A \vbf{\beta}_i = \vbf{0}$ 的解。因此有 $\rank(B) \leqslant n - \rank(A)$。
\end{solution}

\section{可逆矩阵}

\begin{definition}
	对于数域 $K$ 上的矩阵 $A$,如果存在数域 $K$ 上的矩阵 $B$,使得
	\[AB = BA = E\]
	那么称 $A$ 是可逆矩阵(或非奇异矩阵),$B$ 称为 $A$ 的逆矩阵,记作 $A^{-1}$。
\end{definition}

易知可逆矩阵一定是方阵,$n$ 级矩阵 $A$ 可逆的充分必要条件是
\[|A| \ne 0\]

\begin{definition}
	设矩阵 $A = (a_{ij})$,那么 $A$ 的伴随矩阵为
	\[A^*=(A_{ij})\]
\end{definition}

有
\[AA^* = |A|E\]

\begin{theorem}
	数域 $K$ 上 $n$ 级矩阵 $A$ 可逆的充分必要条件是 $|A| \ne 0$。当 $A$ 可逆时,
	\[A^{-1} = \frac{A^*}{|A|}\]
\end{theorem}

易得,可逆矩阵有如下性质

1. $(A^{-1})^{-1} = A$

2. $(AB)^{-1} = B^{-1}A^{-1}$

可逆矩阵能够通过初等行变换变成第简化行阶梯形矩阵一定是单位矩阵。

\begin{theorem}
	矩阵 $A$ 可逆的充分必要条件是它可以表示成一些初等矩阵的乘积。
\end{theorem}

用一个可逆矩阵左(右)乘一个矩阵 $A$,不改变 $A$ 的秩。这里给出了求可逆矩阵第逆矩阵的又一种方法,称为初等变换法。

\begin{example}
	设 $A, B$ 分别是 $n \times m, m \times n$ 的矩阵,如果已知 $E_n - AB$ 可逆,求 $(E_m - BA)^{-1}$。
\end{example}

\begin{solution}
	设存在 $m$ 级矩阵 $X$ 使得
	\[ (E_m - BA)(E_m + X) = E_m \]
	即
	\[ X - BAX = BA \]
	直觉指引我们构造 $X = BYA$,带入有
	\[ B(E_m - AB)YA = BA \]
	因此可以令 $Y = (E_m - AB)^{-1}$,故
	\[ (E_m - BA)^{-1} = E_m + B(E_n - AB)^{-1}A \]
\end{solution}

\begin{example}
	设方阵 $A, B$ 满足 $|A| + |B| = 0$,且 $A, B$ 是对合矩阵,即 $A^2 = B^2 = E$。证明:那么 $A+B, E + AB$ 都不可逆。
\end{example}

\begin{solution}
	可以得到 $|A| = \pm 1$,不妨设 $|A| = 1, |B| = -1$,则
	\[ \begin{aligned}
			|A + B| & = |A(A+B)| = |E+AB|    \\
			        & = -|(A+B)B| = -|AB+E|
		\end{aligned} \]
	故 $|A+B| = |AB+E| = 0$。
\end{solution}

\begin{example}
	如果 $A^3 = 0$,求 $(E - A)^{-1}$。
\end{example}

\begin{solution}
	注意到
	\[ (E - A)(E + A + A^2) = E - A^3 = E \]
	因此
	\[ (E - A)^{-1} = E + A + A^2 \]
\end{solution}

\begin{example}
	设 $n$ 级矩阵 $A, B$ 满足 $A + B = AB$,求证:$E - A, E - B$ 都可逆,且 $AB = BA$。
\end{example}

\begin{solution}
	注意到
	\[ (E-A)(E-B) = E - A - B + AB = E \]
	因此
	\[ E = (E-B)(E-A) = E + BA - AB \]
	得到 $AB = BA$。
\end{solution}

\section{矩阵的分块}

由矩阵 $A$ 的若干行、若干列的交叉位置元素按原来顺序排成的矩阵称为 $A$ 的一个子矩阵。若把分为若干组,列也分成若干组,从而 $A$ 被分成若干个子矩阵,把 $A$ 看成是由这些子矩阵组成的,这称为矩阵的分块,这种由子矩阵组成的矩阵称为分块矩阵。

\begin{theorem}
	设 $A = \left(\begin{matrix} B & C\\ 0 & D \end{matrix}\right)$,其中 $B,C,D$ 都是方阵,那么
	\[A^{-1} = \left(\begin{matrix} B^{-1} & -B^{-1}CD^{-1}\\ 0 & D^{-1} \end{matrix}\right)\]
\end{theorem}

\section{正交矩阵}

\begin{definition}
	实数域上的 $n$ 级矩阵 $A$ 如果满足
	\[A\transpose{A}=E\]
	那么称 $A$ 是正交矩阵。
\end{definition}

那么其具有如下性质

1. 若 $A$ 和  $B$ 都是 $n$ 级正交矩阵,则 $AB$ 也是正交矩阵。

2. 若 $A$ 是正交矩阵,则 $A^{-1}$ (即 $\transpose{A}$)也是正交矩阵。

3. 若 $A$ 是正交矩阵,则 $|A|=\pm 1$


引用 Kronecker 记号 $\delta_{ij}$,它的含义是
\[\delta_{ij}=\begin{cases}
		1,\quad \text{当}\ i = j \\
		0,\quad \text{当}\ i \ne j
	\end{cases}\]

\begin{theorem}
	设实数域上 $n$ 级矩阵 $A$ 的行向量组为 $\seq{\gamma}{n}$,列向量组为 $\seq{\alpha}{n}$,则
	\[\gamma_i\gamma_j'=\delta_{ij},\alpha_i'\alpha_j=\delta_{ij},1 \leqslant i,j \leqslant n\]
\end{theorem}

\begin{definition}
	在 $\mathbb{R}^n$ 中,任给 $\alpha = (\seq{a}{n}),\vbf{\beta}=(\seq{b}{n})$,规定
	\[(\alpha,\vbf{\beta}) \coloneqq  \sum a_nb_n = \alpha\transpose{\vbf{\beta}}\]
	这个二元实值函数 $(\alpha,\vbf{\beta})$ 称为 $\mathbb{R}^n$ 的一个内积(通常称为标准内积)。
\end{definition}

可以验证 $\mathbb{R}^n$ 的标准内积有下列性质:

1. 对称性 $(\alpha,\vbf{\beta}) = (\vbf{\beta},\alpha)$。

2. 线性性之一 $(\alpha+\gamma,\vbf{\beta}) = (\alpha,\vbf{\beta}) + (\gamma,\vbf{\beta})$。

3. 线性性之二 $(k\alpha,\vbf{\beta}) = k(\alpha,\vbf{\beta})$。

4. 正定性 $(\alpha,\alpha) \geqslant 0$,当且仅当 $\alpha=\vbf{0}$ 时等号成立。

可以验证
\[(k_1\alpha_1+k_2\alpha_2, \vbf{\beta}) = k_1(\alpha_1,\vbf{\beta}) + k_2(\alpha_2,\vbf{\beta})\]
\[(\alpha, k_1\vbf{\beta}_1+k_2\vbf{\beta}_2) = k_1(\alpha,\vbf{\beta}_1) + k_2(\alpha,\vbf{\beta}_2)\]

$n$ 维向量空间 $\mathbb{R}^n$ 有了标准内积后,就称 $\mathbb{R}^n$ 为一个欧几里得空间。其中,向量 $\alpha$ 的长度 $|\alpha|$ 规定为
\[|\alpha| \coloneqq  \sqrt{(\alpha,\alpha)}\]

不难验证
\[|k\alpha| = |k||\alpha|\]

长度为 $1$ 的向量称为单位向量,把非零向量 $\alpha$ 除以 $|\alpha|$ 称为把 $\alpha$ 单位化。

在欧几里得空间 $\mathbb{R}^n$ 中,如果 $(\alpha,\vbf{\beta})=0$,那么称 $\alpha$ 与 $\vbf{\beta}$ 是正交的,记作 $\alpha \bot \vbf{\beta}$。由非零向量组成的向量组如果其中每两个不同的向量都正交,那么称它们为正交向量组。如果其每个向量都是单位向量,那么称它为正交单位向量组。

特殊的,零向量与任何向量正交,仅由一个非零向量组成的向量组也是正交向量组。

不难验证,欧几里得空间 $\mathbb{R}^n$ 中 $n$ 个向量组成的正交向量组一定是 $\mathbb{R}^n$ 的一个基,称它为正交基。如果其每个向量都是单位向量,那么称它为 $\mathbb{R}^n$ 的一个标准正交基。

\begin{theorem}
	实数域上的 $n$ 级矩阵 $A$ 是正交矩阵的充分必要条件为:$A$ 的行(列)向量组是欧几里得空间 $\mathbb{R}^n$ 的一个标准正交基。
\end{theorem}

\begin{theorem}
	设 $\seq{\alpha}{s}$ 是欧几里得空间 $\mathbb{R}^n$ 中一个线性无关的向量组,令
	\[
		\begin{aligned}
			\vbf{\beta}_1 & =\alpha_1                                                                                           \\
			\vbf{\beta}_2 & =\alpha_2 - \frac{\alpha_2,\vbf{\beta}_1}{\vbf{\beta}_1,\vbf{\beta}_1}\vbf{\beta}_1                 \\
			              & \cdots                                                                                              \\
			\vbf{\beta}_s & = \alpha_s-\sum_{j=1}^{s-1}\frac{\alpha_s,\vbf{\beta}_j}{\vbf{\beta}_j,\vbf{\beta}_j}\vbf{\beta}_j
		\end{aligned}
	\]
	则 $\seq{\vbf{\beta}}{s}$ 是正交向量组,并且与 $\seq{\alpha}{s}$ 等价。
\end{theorem}

这给出了在欧几里得空间 $\mathbb{R}^n$ 中从一个线性无关的向量组 $\seq{\alpha}{s}$ 出发,够造出与它等价的一个正交向量组的方法,这种方法称为 施密特 Schmidt 正交化过程,只要再将 $\seq{\vbf{\beta}}{s}$ 中每个向量单位化,则其就是与原向量组等价的正交单位向量组,就是 $\mathbb{R}^n$ 的一个标准正交基。

\section{\texorpdfstring{$K^n$ 到 $K^s$ 的线性映射}{Kn 到 Ks 的线性映射}}

映射视作熟知的。设 $f$ 是集合 $S$ 到集合 $S'$,$S$ 所有元素在 $f$ 下的象组成的集合叫做 $f$ 的值域或者 $f$ 的象,记作 $f(S)$ 或 $\Im f$。

\begin{definition}
	数域 $K$ 上的向量空间 $K^n$ 到 $K^s$ 的一个映射 $\sigma$ 如果保持加法和数量乘法,即 $\forall \alpha,\vbf{\beta} \in K^n,k\in K$ 有
	\[\sigma (\alpha+\vbf{\beta}) = \sigma(\alpha) + \sigma(\vbf{\beta})\]
	\[\sigma(k\alpha) = k\sigma(\alpha)\]
	那么称 $\sigma$ 是 $K^n$ 到 $K^s$ 上的一个线性映射。
\end{definition}

设 $A$ 是数域 $K$ 上 $s \times n$ 矩阵,令
\[
	\begin{aligned}
		\vbf{A} : K^n & \rightarrow K^s  \\
		\alpha        & \mapsto A\alpha
	\end{aligned}
\]


\begin{definition}
	设 $\sigma$ 是 $K^n$ 到 $K^s$ 的一个映射,$K^n$ 的一个子集
	\[ \{ \alpha \in K^n \mid \sigma(\alpha) = \vbf{0} \} \]
	称为映射 $\sigma$ 的核,记作 $\Ker \sigma$
\end{definition}

不难验证,如果 $\sigma$ 是 $K^n$ 到 $K^s$ 的一个线性映射,那么 $\Ker \sigma$ 是 $K^n$ 的一个子空间。

\begin{theorem}
	设数域上 $K$ 上齐次线性方程组 $AX = \vbf{0}$ 的解空间是 $W$,$A$ 对应的线性映射为 $\vbf{A}$ 则
	\[ \Ker \vbf{A} = W \]
\end{theorem}

因此
\[ \dim \Ker \vbf{A} + \dim \Im \vbf{A} = \dim K^n \]
 % 矩阵的运算
%% \newcommand{\mfrac}[2]{\frac{#1}{#2}}

\chapter{导数和微分}

\section{导数的定义}

\begin{definition}
	设函数 $y=f(x)$ 在点 $x_0$ 的某邻域有定义,若极限
	$$\lim_{x\to x_0}\frac{f(x)-f(x_0)}{x-x_0}$$
	存在,则称函数 $f$ 在点 $x_0$ 可导,并称该极限为函数 $f$ 在点 $x_0$ 的导数,记作 $f'(x_0)$。
\end{definition}

\begin{theorem}
	若函数 $f$ 在点 $x_0$ 可导,则 $f$ 在点 $x_0$ 连续。
\end{theorem}

\begin{definition}
	设函数 $y=f(x)$ 在点 $x_0$ 的某右邻域 $[x_0,x_0+\delta)$ 上有定义,若右极限
	$$\lim_{\Delta x\to 0^+}\frac{\Delta y}{\Delta x} = \lim_{\Delta x\to 0^+}\frac{f(x_0+\Delta x)-f(x_0)}{\Delta x},(0<\Delta x<\delta)$$
	存在,则称该极限值为 $f$ 在点 $x_0$ 的右导数,记作 $f_+'(x)$。同理有左导数。
\end{definition}

左导数和右导数统称为单侧导数。

\begin{theorem}
	若函数 $y=f(x)$ 在点 $x_0$ 的某邻域上有定义,则 $f'(x_0)$ 存在的充要条件是 $f_-'(x)$ 与 $f_+'(x)$ 都存在且相等。
\end{theorem}

若函数 $f$ 在区间 $I$ 上每一点都可导(对区间端点,仅考虑相应的单侧导数),则称 $f$ 为 $I$ 上的可导函数。此时对每一个 $x\in I$,都有 $f$ 的一个导数 $f'(x)$ (或单侧导数)与之对应。

这样就定义了一个在 $I$ 上的函数,称为导函数,简称为导数。记作 $f',y',\dfrac{\d y}{\d x}$,即

$$f'(x) = \lim_{\Delta x \to 0}\frac{f(x+\Delta)-f(x)}{\Delta},x\in I$$

有时 $f'(x_0)$ 也可写作 $y'\mid_{x=x_0}$ 或 $\mfrac{\d y}{\d x}\mid_{x=x_0}$。

曲线 $y = f(x)$ 在点 $(x_0,y_0)$ 的切线方程是
$$y-y_0 = f'(x_0)(x-x_0)$$

\begin{definition}
	若函数 $f$ 在点 $x_0$ 的某邻域 $U(x_0)$ 上对一切 $x\in U(x_0)$ 有
	$$f(x_0) \geqslant f(x)$$
	则称 $f$ 在点 $x_0$ 取得极大值,称点 $x_0$ 为极大值点。同理有极小值点。
\end{definition}

极大值、极小值统称为极值,极大值点、极小值点统称为极值点。

\begin{theorem}[费马定理]
	设函数 $f$ 在点 $x_0$ 的某邻域上有定义,且在点 $x_0$ 可导。若点 $x_0$ 为极值点,则必有 $f'(x_0)=0$。
\end{theorem}

\section{求导法则}

\begin{theorem}
	若函数 $u(x)$ 和 $v(x)$ 在点 $x_0$ 可导,则函数 $f(x)=u(x)\pm v(x)$ 在点 $x_0$ 也可导,且
	$$f'(x_0) = u'(x_0)\pm v'(x_0)$$
	函数 $f(x)=u(x)v(x)$ 在点 $x_0$ 也可导,且
	$$f'(x_0) = u'(x_0)v'(x_0)$$
	若 $v(x)\ne 0$,则函数 $f(x)=\dfrac{u(x)}{v(x)}$ 在点 $x_0$ 也可导,且
	$$f'(x_0) = \frac{u'(x_0)v(x_0)-u(x_0)v'(x_0)}{v(x_0)^2}$$
\end{theorem}

\begin{theorem}
	设 $y=f(x)$ 为 $x=\phi(x)$ 的反函数,若 $\phi(y)$ 在点 $y_0$ 的某邻域上连续、严格单调且 $\phi'(y_0)\ne0$,则 $f(x)$ 在点 $x_0=\phi(y_0)$ 可导,且
	$$f'(x_0)=\frac{1}{\phi'(y_0)}$$
\end{theorem}

\begin{theorem}
	设 $u=\phi(x)$ 在点 $x_0$ 可导,$y=f(u)$ 在点 $u_0=\phi(x_0)$ 可导,则复合函数 $f\circ \phi$ 在点 $x_0$ 可导,且
	$$(f\circ \phi)'(x_0) = f'(u_0)\phi'(x_0) = f'(\phi(x_0))\phi'(x_0)$$
\end{theorem}

\subsection{基本求导法则}

1. $(u\pm v)' = u'\pm v'$

2. $(uv)' = u'v+uv'$

3. $\left(\dfrac{u}{v}\right) = \dfrac{u'v-uv'}{v^2}$

4. $(u\pm v)' = u'\pm v'$

\subsection{基本初等函数导数公式}

1. $(c)' = 0$($c$ 为常数)

2. $(x^a)' = ax^{a-1}$($a$ 为任意实数)

3. $(\sin x)' = \cos x,(\cos x)'=-\sin x,(\tan x)'=\sec^2x$

4. $(\cot x)' = -\csc^2 x,(\sec x)'=\sec x \tan x,(\csc x)'=-\csc x \cot x$

5. $(\arcsin x)'=\dfrac{1}{\sqrt{1-x^2}},(\arccos x)' = -\dfrac{1}{\sqrt{1-x^2}},(\arctan x)'=\dfrac{1}{1+x^2}$

6. $(a^x)'=a^x \ln a(a>0\text{且}a\ne 1)$

7. $(\log_a|x|)'=\dfrac{1}{x\ln a}(a>0\text{且}a\ne 1)$

\section{单调性与导数}

\begin{theorem}
	设 $f$ 在区间 $I$ 上可导,则 $f(x)$ 在 $I$ 上递增(减) 的充要条件时
	$$f'(x) \geqslant 0(\leqslant 0)$$
\end{theorem}

\begin{theorem}[介值定理]
	设 $f$ 为 $[a,b]$ 上的连续函数,$\mu$ 时严格介于 $f(a)$ 和 $f(b)$ 之间的数,则存在 $\xi\in (a,b)$,使得 $f(\xi)=\mu$。
\end{theorem} % 矩阵的相抵与相似
\chapter{二次型 · 矩阵的合同}

\section{二次型及其标准型}

\begin{definition}
    数域 $K$ 上的一个 $n$ 元二次型型是系数在 $K$ 中的 $n$ 个变量的齐次多项式,它的一般形式是
    \[f(\ji{x}{n}) = \sum_{i=1}^n\sum_{j=1}^na_{ij}x_ix_j\]
    其中 $a_{ij} = a_{ji}$。
\end{definition}

把二次型的系数按原来顺序排成一个 $n$ 级矩阵 $A$,则称 $A$ 是二次型 $f(\ji{x}{n})$ 的矩阵,它是对称矩阵。

再令 $X = \transpose{(\ji{x}{n})}$,则二次型可以写作
\[f(\ji{x}{n}) = \transpose{X}AX\]

令 $Y = \transpose{(y_1,\cdots,y_n)}$,设 $C$ 是数域 $K$ 上的 $n$ 级可逆矩阵,则关系式
\[X = CY\]
称为变量 $\ji{x}{n}$ 到变量 $y_1,\cdots,y_n$ 的一个非退化线性替换。

\begin{definition}
    数域上两个 $n$ 元二次型 $\transpose{X}AX$ 与 $\transpose{Y}AY$,如果存在一个非退化线性替换 $X = CY$,把 $\transpose{X}AX$ 变成 $\transpose{Y}BY$ 那么称二次型 $\transpose{X}AX$ 与 $\transpose{Y}BY$ 等价,记作 $\transpose{X}AX \cong \transpose{Y}BY$。
\end{definition}

\begin{definition}
    数域 $K$ 上两个 $n$ 级矩阵 $A$ 与 $B$,如果存在 $K$ 上的一个 $n$ 级可逆矩阵 $C$,使得
    \[\transpose{C}AC = B\]
    那么称 $A$ 与 $B$ 合同,记作 $A\simeq B$。
\end{definition}

如果二次型 $\transpose{X} A X$ 等价于一个只含平方项的二次型,那么这个只含平方项的二次型称为 $\transpose{X}AX$ 的一个标准形。

如果 $T$ 是正交矩阵,那么变量的替换 $X = TY$ 称为正交替换。

\begin{theorem}
    设 $A,B$ 都是数域 $K$ 上 $n$ 级矩阵,则 $A$ 合同于 $B$ 当且仅当 $A$ 经过一系列成对初等行、列变换可以变成 $B$,此时对 $I$ 只作其中的初等列变换得到的可逆矩阵 $C$,就使得 $\transpose{C}AC=B$。
\end{theorem}

\begin{theorem}
    数域 $K$ 上的任一对称矩阵都合同于一个对角矩阵。
\end{theorem}

这意味着数域 $K$ 上任一 $n$ 元二次型都等价于一个只含平方项的二次型。

二次型 $\transpose{X}AX$ 的矩阵 $A$ 的秩就称为二次型 $\transpose{X}AX$ 的秩。

\section{实二次型的规范形}

实数域上的二次型简称为实二次型,$n$ 元实二次型 $\transpose{X}AX$ 经过一个适当的非退化线性替换 $X = CY$ 可以化成下述形式的标准形
\[d_1y_1^2+\cdots+d_py_p^2-d_{p+1}y_{p+1}^2-\cdots-d_ry_r^2\]
其中 $d_i>0,i=1,\cdots,r$。再做一次非退化线性替换可以变成
\[z_1^2+\cdots+z_p^2-z_{p+1}^2-\cdots-z_r^2\]

\begin{theorem}
    $n$ 元实二次型 $\transpose{X}AX$ 的规范形是唯一的。
\end{theorem}

\begin{definition}
    在实二次型 $\transpose{X}AX$ 的规范形中,系数为 $+1$ 的平方项个数 $p$ 称为 $\transpose{X}AX$ 的正惯性指数,系数为 $-1$ 的平方项个数 $r-p$ 称为 $\transpose{X}AX$ 的负惯性指数;正惯性指数减去负惯性指数所得的差 $2p-r$ 称为 $\transpose{X}AX$ 的符号差。
\end{definition}

任一 $n$ 级实对称矩阵合同于对角矩阵 $\diag\{1,\cdots,1,-1,\cdots,-1,0,\cdots,0\}$,其中 $0$ 的个数等于 $\transpose{X}AX$ 的正惯性指数,$-1$ 的个数等于 $\transpose{X}AX$ 的负惯性指数(也分别称作 $A$ 的惯性指数),这个对角举着称为 $A$ 的合同规范形。

现讨论复数域上的二次型,简称为复二次型。设 $n$ 元复二次型 $\transpose{X}AX$ 经过一个适当的非退化线性替换 $X = CY$ 变成下述形式的标准形
\[d_1y_1^2+\cdots+d_ry_r^2\]
其中 $d_i\ne 0,i=1,\cdots,n$,$r$ 是这个二次型的秩。再做一个非退化线性替换可得
\[z_1^2+\cdots+z_r^2\]
把这个标准形叫做复二次型 $\transpose{X}AX$ 的规范形,显然其完全由其秩决定,故只有一种形式。

\section{正定二次型与正定矩阵}

\begin{definition}
    $n$ 元实二次型 $\transpose{X}AX$ 称为正定的,如果对于 $\RR^n$ 中任一非零列向量 $\alpha$,都由 $\transpose{\alpha}A\alpha>0$。
\end{definition}

\begin{theorem}
    $n$ 元实二次型 $\transpose{X}AX$ 是正定的当且仅当它的正惯性系数等于 $n$。
\end{theorem}

\begin{definition}
    实对称矩阵 $A$ 称为正定的,如果实二次型 $\transpose{X}AX$ 是正定的 。
\end{definition}

正定的实对称矩阵简称为正定矩阵。

\begin{theorem}
    实对称矩阵 $A$ 是正定的充分必要条件是 $A$ 的说有顺序主子式全大于 $0$。
\end{theorem}

\begin{definition}
    实对称矩阵 $A$ 称为半正定(负定,半负定,不定)的,如果实二次型对于 $\RR^n$ 中任一非零列向量 $\alpha$,都有
    \[\transpose{\alpha}A\alpha \geqslant 0 \quad (\transpose{\alpha}A\alpha<0,\transpose{\alpha}A\alpha\leqslant 0)\]
    如果 $\transpose{X}AX$ 既不是半正定的,又不是半负定的,那么称它是不定的。
\end{definition}

\begin{definition}
    实对称矩阵 $A$ 称为半正定(负定,半负定,不定)的,如果实二次型 $\transpose{X}AX$ 是半正定(负定,半负定,不定)的。
\end{definition}

\begin{theorem}
    $n$ 级实对称矩阵 $A$ 是半正定的,当且仅当 $A$ 的所有主子式全非负。
\end{theorem}

\begin{theorem}
    实对称矩阵 $A$ 负定的充分必要条件是:它的奇数阶顺序主子式全小于 $0$,偶数阶顺序主子式全大于 $0$。
\end{theorem}

\begin{theorem}
    设二元实值函数 $F(x,y)$ 有一个稳定点 $\alpha=(x_0,y_0)$ (即 $F(x,y)$ 在 $(x_0,y_0)$ 处的一阶偏导数全为 $0$)。设 $F(x,y)$ 在 $(x_0,y_0)$ 的一个邻域内有 3 阶连续偏导数。令
    \[H = \left(\begin{matrix}
        F_{xx}''(x_0,y_0) & F_{xy}''(x_0,y_0)\\
        F_{yx}''(x_0,y_0) & F_{yy}''(x_0,y_0)
    \end{matrix}\right)\]
    称 $H$ 是 $F(x,y)$ 在 $(x_0,y_0)$ 处的黑塞(Hesse)矩阵。如果 $H$ 是正定的,那么 $F(x,y)$ 在 $(x_0,y_0)$ 处达到极小值。如果 $H$ 是负定的,那么 $F(x,y)$ 在 $(x_0,y_0)$ 处达到极大值。
\end{theorem}

其可推广到 $n$ 元函数的情形:设 $F(\ji{x}{n})$ 有一个稳定点 $\alpha = (\ji{a}{n})$,设 $F(\ji{x}{n})$ 在 $\alpha$ 的一个邻域内有 3 阶连续偏导数,令
\[H = (F_{x_ix_j}''(\alpha))\]
称 $H$ 是 $F(\ji{x}{n})$ 在 $\alpha$ 处的黑塞矩阵。如果 $H$ 是正定的,那么 $F$ 在 $(x_0,y_0)$ 处达到极小值。如果 $H$ 是负定的,那么 $F$ 在 $(x_0,y_0)$ 处达到极大值。 % 二次型 · 矩阵的合同
\chapter{常微分方程}

如果知道函数及其导数、微分组成的关系式,得到的便是微分方程。

\section{基本概念}

自变量只有一个的方程是\textbf{常微分方程},两个以上则称\textbf{偏微分方程},其中未知函数最高阶导数的阶数称为微分方程的阶数。一般的 $n$ 阶常微分方程具有形式
\[ F\left(x, y, \frac{\d y}{\d x}, \cdots, \frac{\d^n y}{\d x^n} \right) = 0 \]
如方程的左端为 $y$ 及各阶导数的一次有理整式,则称为 $n$ 阶线性微分方程,否则称为非线性微分方程。一般的 $n$ 阶线性微分方程有形式
\[ \frac{\d^n y}{\d x^n} + a_{n-1}(x) \frac{\d^{n-1} y}{\d x^{n-1}} + \cdots + a_1(x) \frac{\d y}{\d x} + a_0(x) y = f(x) \]
这里 $a_i(x), f(x)$ 是关于 $x$ 的已知函数。

如果函数 $y = \varphi(x)$ 带入方程后,能使其变为恒等式,则称函数 $y=\varphi(x)$ 为方程的解。如果隐函数 $\Phi(x, y) = 0$ 是其解,则称隐式解,也可以不加区分的称为解。

含有 $n$ 个独立的任意常数 $C_1, \cdots, C_n$ 的解
\[ y = \varphi(x, C_1, \cdots, C_n) \]
称为 $n$ 阶方程的通解,其中独立性指 $\varphi$ 及各阶偏导数关于 $n$ 个常数 $C_1, \cdots, C_n$ 的 Jacobi 行列式不为 $0$。为了确定微分方程的一个特解,需要给出定解条件。常见的定解条件是初值条件和边值条件,其中初值条件指:当 $x = x_0$ 时,有
\[ y=y_0, y' = y_0^{(1)}, \cdots, y^{(n-1)} = y_0^{(n-1)} \]
满足初值条件的解称为微分方程的特解。

\section{变量分离微分方程}

形如
\[ \frac{\d y}{\d x} = f(x) g(y) \]
的方程,称为变量分离方程,假如 $g(y) \neq 0$ 我们可以很容易的分离它
\[ \frac{\d y}{g(y)} = f(x) \d x \]
两边同时积分即可。另外,不要忘记当 $g(y) = 0$ 时也有解 $y=y_0$。

有两种常见的变式:

(1)形如
\[ \frac{\d y}{\d x} = g\left(\frac{y}{x}\right) \]
的方程,记作齐次微分方程。做变量代换 $u = \frac{y}{x}$,有
\[ \frac{\d u}{\d x} = \frac{1}{x} \left(\frac{\d y}{\d x} -\frac{y}{x} \right) = \frac{g(u) - u}{x} \]
就变成变量分离的了。

(2)形如
\[ \frac{\d y}{\d x} = g \left(\frac{a_1 x + b_1 y + c_1}{a_2 x + b_2 y + c_2}\right) \]
分为三种情况讨论:
\begin{enumerate}
	\item 如果
	      \[ \frac{a_1}{a_2} = \frac{b_1}{b_2} = \frac{c_1}{c_2} = k \]
	      则比较显然。
	\item 如果
	      \[ \frac{a_1}{a_2} = \frac{b_1}{b_2} = k \neq \frac{c_1}{c_2} \]
	      令 $u = a_2 x + b_2 y$,此时有
	      \[ \frac{\d u}{\d x} = g \left(a_2 + b_2 \frac{k u + c_1}{u + c_2}\right) \]
	      是变量分离方程。
	\item 对于剩余的情况,把分子分母看成两条不相交的直线,尝试平移到原点。设交点为 $(x, y) = (x_0, y_0)$,有
	      \[ \frac{\d y}{\d x} = \frac{\d (y - y_0)}{\d (x - x_0)} = g \left(\frac{a_1 (x - x_0) + b_1 (y - y_0)}{a_2(x - x_0) + b_2 (y - y_0)} \right) \]
	      也变成了齐次形式。
\end{enumerate}

\section{一阶线性微分方程}

一阶线性常微分方程的一般形式是
\[ \frac{\d y}{\d x} = p(x) y + q(x) \]
其中 $p(x), q(x)$ 在给定区间上连续的函数。当 $q(x) = 0$ 时,称为一阶线性齐次常微分方程;否则称一阶线性非齐次常微分方程。

以下用大写字母简记为积分,方便记忆。

\paragraph{一阶线性其次常微分方程} 即
\[ \frac{\d y}{\d x} = p(x) y \]
那么我们可以分离变量
\[ \frac{\d y}{y} = p(x) \d x \]
两边积分得到
\[ \ln |y| = \int_{x_0}^{x} p(t) \d t + C_1 \]
其中 $C_1$ 是任意常数;若引入任意非零常数 $C_2 = \pm \ee^{C_1}$,就得到
\[ y(x) = C_2 \ee^{\int_{x_0}^{x} p(t) \d t } = C_2 \ee^{P(x)} \]
显然 $y = 0$ 也是一解,故 $C_2$ 是可以取任意常数的。

这样对于给定初值的问题
\[ \frac{\d y}{\d x} = p(x) y, \quad y(x_0) = y_0 \]
的通解为
\[ y(x) = y_0 \ee^{ \int_{x_0}^{x} p(t) \d t } = y_0 \ee^{P(x)} \]

除此之外,我们有以下性质:

\begin{itemize}
	\item 方程的解要么恒为零,要么恒不为零。
	\item 方程任何有限个解的线性组合仍是解,所有解构成一个一维线性空间。
\end{itemize}

\paragraph{一阶线性非齐次常微分方程} 一阶线性非齐次常微分方程为
\[ \frac{\d y}{\d x} = p(x) y + q(x) \]
考虑套用之前的公式,设
\[ u(x) = y \ee^{-\int_{x_0}^{x} p(t) \d t} = y \ee^{-P(x)} \]
注意到
\[ \frac{\d u}{\d x} \cdot \ee^{\int_{x_0}^{x} p(t) \d t} = y' - p(x) y = q(x) \]
可以得到 $y(x)$ 的解
\[ y(x) = \ee^{\int_{x_0}^{x} p(t) \d t} \left(  \int \ee^{-\int p(x) \d x} q(x) \d x + C \right)
	= \ee^{P(x)} \left( \int \ee^{-P(x)} q(x) + C \right)  \]

可以发现一个有趣的性质,非齐次方程的解可以由齐次形式下的解的“常数变易”得到。因此先考虑解齐次形式,再变易常数为函数 $c(x)$,这种解法记作常数变易法。

\paragraph{Bernoulli 微分方程} 形如
\[ \frac{\d y}{\d x} = p(x) y + q(x) y^n, \quad n \neq 0,1 \]
的方程称为 Bernoulli 微分方程。设 $y \neq 0$,得到
\[ y^{-n} \frac{\d y}{\d x} = \frac{\d (y^{1-n})}{(1-n)\d x} = y^{1-n}p(x) + q(x) \]
换元 $u = y^{1-n}$ 即可。

\section{恰当微分方程}

有时可以考虑答案是全微分的形式。设
\[ f(x, y) \d x + g(x, y) \d y = 0 \]
并假设 $f, g$ 连续。设其左端恰好是某个二元函数 $u(x, y)$ 的全微分
\[ f(x, y) \d x + g(x, y) \d y = \d u(x, y) = \frac{\partial u}{\partial x} \d x + \frac{\partial u}{\partial y} \d y \]
则称其为恰当微分方程。容易验证其通解为 $u(x, y) = C$。

注意到
\[ \frac{\partial f}{\partial y} = \frac{\partial^2 u}{\partial y \partial x} = \frac{\partial^2 u}{\partial x \partial y} = \frac{\partial g}{\partial x} \]
故其是该方程为恰当微分方程的充要条件。

再考虑如何求出 $u$,先以 $y$ 为参数
\[ u = \int f(x, y) \d x + \varphi(y) \]
回带得到
\[ \frac{\partial u}{\partial y} = g(x, y) = \frac{\d \varphi(y)}{\d y} + \frac{\partial}{\partial y}\int f(x, y) \d x \]
总之有解
\[ u = \int f(x, y) \d x + \int \left( g(x, y) - \frac{\partial}{\partial y}\int f(x, y) \d x  \right) \d y \]

这个形式过于复杂,一般情况下多瞪眼可能更快。下面是一些常见的表

\[ \begin{aligned}
	y \d x + x \d y &= \d x y \\
	\frac{y \d x - x \d y}{y^2} &= \d \frac{x}{y} \\
	\frac{y \d x - x \d y}{xy} &= \d \ln\left|\frac{x}{y}\right| \\
	\frac{y \d x - x \d y}{x^2 + y^2} &= \d \arctan \frac{x}{y} \\
	\frac{y \d x - x \d y}{x^2 - y^2} &= \d \ln\left| \frac{x-y}{x+y}\right| \\
\end{aligned} \]

假如一个方程乘以 $\mu(x, y)$ 就成为了恰当微分方程,则称 $\mu$ 是原方程的积分因子。假若方程有解存在,则必有积分因子存在,而且不是唯一的。因此之前的很多方法都可以改写成积分因子法,但从零瞪出一个因子非常有技巧性,实战中不实用。

\section{齐次线性微分方程}

一般的 $n$ 阶线性微分方程有形式
\[ \frac{\d^n y}{\d x^n} + a_{n-1}(x) \frac{\d^{n-1} y}{\d x^{n-1}} + \cdots + a_1(x) \frac{\d y}{\d x} + a_0(x) y = f(x) \]

倘若 $f(x) \equiv 0$ 则是齐次线性微分方程,否则为非齐次。对其分析过程比较复杂,这里直接给出结论:方程的线性无关解最大个数为 $n$,所有解构成一个 $n$ 维线性空间。其非齐次的通解可以将各个系数用常数变易法待定,计算即可。

注意到
\[ \ee^{i x} = \cos x + i \sin x \]
反推得到
\[ \sin x = \frac{1}{2}(\ee^{i x} + \ee^{-i x}), \quad \cos x = \frac{1}{2} (\ee^{ix} - \ee^{-ix}) \]
因此我们在计算时完全可以将指数和三角统一起来,叙述过程时再改成三角。

\paragraph{常系数}

设所有系数都是常数,即方程为
\[ L[y] \equiv \frac{\d^n y}{\d x^n} + a_{n-1} \frac{\d^{n-1} y}{\d x^{n-1}} + \cdots + a_1 \frac{\d y}{\d x} + a_0 y = 0 \]
考虑代入
\[ L[\ee^{\lambda x}] = (\lambda^n + a_{n-1} \lambda^{n-1} + \cdots + a_0) \ee^{\lambda x} = F(\lambda) \ee^{\lambda x} \]
其中方程 $F(\lambda)$ 是关于 $\lambda$ 的 $n$ 次多项式,我们求解 $F(\lambda) = 0$ 即可,因此又称为特征方程,其根称为特征根。假如所有的根都不是重根,那么可以直接得到 $n$ 个线性无关的解 $y_i = \ee^{\lambda_i x}$。

假若方程存在 $k$ 重根 $\mu$,即特征方程存在因子 $(\lambda - \mu)^k$,考虑平移代换 $z = y \ee^{\mu x}$,继续代入 $y = \ee^{\lambda x}$ 有
\[ L[y \ee^{\mu x}] = L[\ee^{(\lambda + \mu) x}] = F(\lambda + \mu) \ee^{(\lambda + \mu) x} \]
显然 $F(\lambda + \mu)$ 含有 $k$ 重零根,对比系数可以得到方程 $L[y] = 0$ 的 $0 \sim k-1$ 项系数为 $0$,即方程形如
\[ L[z] = \frac{\d^n y}{\d x^n} + b_{n-1} \frac{\d^{n-1} y}{\d x^{n-1}} + \cdots + b_{k} \frac{\d^k y}{\d x^k} = 0 \]
可以观察出其 $k$ 个线性无关的解 $y = 1, x, x^2, \cdots, x^{k-1}$,故原方程的 $k$ 个解为
\[ z = \ee^{\mu x}, x \ee^{\mu x}, \cdots, x^{k-1} \ee^{\mu x} \]
 % 多项式环
\chapter{概率论}

考研中遇到的概率论很少,暂放于此。

\section{随机事件与概率}

假定某个试验有有限个可能的结果 $e_1, \cdots, e_N$,其出现机会是等可能的。假定事件 $E$ 包含了其中的 $M$ 个结果,则称事件 $E$ 的概率为
\[ P(E) = M / N \]
这是古典概型。另一些时候,我们把结果扩展到无限的情况。我们把度量(可以通俗的理解为面积)相同的事件称为等可能的,这是几何概型。

称集合 $\Omega$ 随机事件的样本空间,其元素 $\omega$ 称为基本事件。事件 $A$ 是 $\Omega$ 的一个子集,赋给每个事件一个实数值 $P(A)$,称为概率。其满足要求:

\begin{itemize}
	\item 非负性:$P(A) \geqslant 0$。
	\item 规范性:$P(\Omega) = 1, P(\varnothing) = 0$。
	\item 加法公理。
\end{itemize}

若两个事件不能在同一次试验中同时发生,则称为互斥的。若一些事件中任意两个都互斥,则称为两两互斥。可以进一步导出对立事件的概念。可以把集合的交并补也引入。

\begin{theorem}[加法公理]
	若干个互斥事件之和的概率,等于各事件的概率之和:
	\[ P\left(\bigcup A_i\right) = \sum P(A_i) \]
\end{theorem}

\begin{note}
	这条公理其实是在古典定义、统计定义下是可证明的,但是为什么是公理呢?因为我们的确可以建立一种新的概率理论,在其中加法公理不成立。类似于平行公设。
\end{note}

\begin{definition}
	设 $A,B$ 是两个事件且 $P(A) > 0$,我们称在已知 $A$ 发生的条件下事件 $B$ 发生的概率为条件概率,记作
	\[ P(B | A) = \frac{P(AB)}{P(A)} \]
\end{definition}

假如 $P(B \mid A) > P(B)$,我们可以说事件 $A$ 促进了事件 $B$ 的发生。反之 $P(B \mid A) = P(A)$,则 $B$ 的发生对 $A$ 无影响。

形式的说:

\begin{definition}[条件概率]
	若两事件 $A, B$ 满足
	\[ P(A B) = P(A) P(B) \]
	则称 $A, B$ 独立。
\end{definition}

变换形式是为了避免 $0$ 的讨论。

设一列事件 $A_1, A_2, \cdots$,假如从中取出任意有限个都成立
\[ P(A_{i1} \cdots A_{im}) = P(A_{i1}) \cdots P(A_{im}) \]
那么称事件 $A_1, A_2, \cdots$ 相互独立。注意与两两独立的区别。


\begin{theorem}[全概率公式]
	设 $B_1, \cdots$ 为一列事件,他们两两互斥且每次试验至少发生一个。有时称这种性质为“完备事件群”。那么对任意事件 $A$ 有
	\[ P(A) = P(B_1)P(A \mid B_1) + P(B_2)P(A \mid B_2) + \cdots \]
\end{theorem}


\begin{theorem}[全概率公式]
	贝叶斯公式:对 $n$ 个两两不相容事件 $A_1, \cdots, A_n$,则对事件 $B$ 有
	\[ P\left(A_j | B\right) = \frac{P(A_j)P(B | A_j)}{\sum\limits_{i=1}^n P(A_i)P(B | A_i) } \]
\end{theorem}

\begin{note}
	若 $P(AB) = 0$,不意味着 $AB = \varnothing$。比如 $[0,1]\cap [1,2] = \{1\}$,但概率是 $0$。
\end{note}

\section{一维随机变量及其分布}

随机变量就是“其值会随机而定”的变量,是一个实值单值函数。设随机实验的样本空间为 $\Omega$,若对于任意 $\omega \in \Omega$ 都有唯一实数 $X(\omega)$ 与其对应,且对任意实数 $x$,$\{\omega \mid X(\omega) \leqslant x, \omega \in \Omega\}$ 是随机时间,则称定义在 $\Omega$ 上的实值单值函数为随机变量。

\paragraph{离散型随机变量}

研究一个随机变量,不只是查看其能取哪些值,更要看其取各种值的概率如何。

\begin{definition}
	设 $X$ 是离散型随机变量,其全部可能值为 $\{a_1, \cdots\}$,则称
	\[ p_i = P\{X = a_i\} \]
	为其概率函数。称
	\[ F(x) = P\{X \leqslant x\} = \sum_{a_i \leqslant x} p_i \]
	是其分布函数。记为 $X \sim F(x)$,称 $X$ 服从 $F(x)$ 分布。
\end{definition}

其具有以下的性质:

\begin{itemize}
	\item $F(x)$ 对 $x$ 单调不减。
	\item $F(x)$ 是 $x$ 的右连续函数。
	\item $F(- \infty) = \lim\limits_{i \to -\infty} F(x) = 0$,$F(+\infty) = \lim\limits_{x \to + \infty} F(x) = 1$。
	\item $P\{X \leqslant a\} = F(a)$,$P\{X < a\} = F(a - 0)$,$P\{X = a\} = F(a) - F(a - 0)$。
\end{itemize}

如果随机变量只取有限的可列值,则称为离散型随机变量,可以写分布列。

\paragraph{连续型随机变量}

如果随机变量的分布函数是 $F(x)$,则 $f(x) = F'(x)$ 是其的概率密度函数。

其具有以下的性质:

\begin{itemize}
	\item $f(x) \geqslant 1$。
	\item 对任何常数 $a, b$ 有
	      \[ P\{a \leqslant X \leqslant b\} = F(b) - F(a) = \int_{a}^{b} f(x) \d x, \qquad \int_{-\infty}^{\infty} f(x) \d x = 1 \]
\end{itemize}

\subsection{常见随机分布}

\paragraph{二项分布}
如果 $X$ 的概率分布为
\[ P\{X = k\} = \binom{n}{k} p^k(1 - p)^{n-k}. \quad k = 0, 1, \cdots, n \]
则称 $X$ 服从参数为 $(n, p)$ 的二项分布,记为 $X \sim B(n, p)$。特别的,当 $n=1$ 时称为二项分布。二项分布也是 $n$ 重伯努利实验中事件 $A$ 发生的次数,其中 $P(A) = p$。

\paragraph{泊松分布}
如果 $X$ 的概率分布为
\[ P\{X = k\} = \frac{\lambda^k}{k!} \ee^{-\lambda} , \quad k = 0, 1, \cdots \]
则记为 $X \sim P(\lambda)$。

\paragraph{几何分布}
如果 $X$ 的概率分布为
\[ P\{X=k\} = (1 - p)^{k-1}p, \quad k = 1, 2, \cdots \]
则记为 $X \sim G(p)$。

\paragraph{超几何分布}
如果 $X$ 的概率分布为
\[ P\{X = k\} = \frac{\binom{M}{k} \binom{N - M}{n - k}}{\binom{N}{n}}, \quad \max(0, n - N + M) \leqslant k \leqslant \min(M, n) \]
则记为 $X \sim H(n, N, M)$。设有 $N$ 个产品组成的整体,其中有 $M$ 个不合格产品,从中取出 $n$ 个,次品数为 $k$,这就是组合意义。

\paragraph{均匀分布}
如果 $X$ 的概率密度函数为
\[ f(x) = \frac{1}{b - a}, \quad a < x < b \]
则记为 $X \sim U(a, b)$。

\paragraph{指数分布}
如果 $X$ 的概率密度函数为
\[ f(x) = \lambda \ee^{-\lambda x}, \quad x > 0 \]
则记为 $X \sim E(\lambda)$。

\paragraph{正态分布}
如果 $X$ 的概率密度为
\[ f(x) = \frac{1}{\sqrt{2 \pi}  \sigma} \exp \left(- \frac{(x - \mu)^2}{2\sigma^2}  \right) \]
则记为 $X \sim N(\mu, \sigma^2)$。

\section{多维随机变量}

设 $X = (X_1, \cdots, X_N)$,其中每个分量都是一维随机变量,则 $X$ 是一个 $n$ 维随机变量。

\paragraph{离散型}

假如每个分量是离散型的,那么称 $X$ 是 $n$ 维离散型随机变量。

\begin{definition}
	以 $\{a_{i1}, \cdots\}$ 记 $X_i$ 的全部可能值,则事件的概率
	\[ p(j_1, \cdots, j_n) = P\{X_1 = a_{1j_1}, \cdots, X_n = a_{n j_n}\} \]
	为随机变量的概率函数。显然应该满足条件
	\[ p(j_1, \cdots) \geqslant 0, \quad \sum_{j_n} \cdots \sum_{j_1} p(j_1, \cdots)  =1 \]
\end{definition}

\paragraph{连续型}
连续型随机变量不能简单的定义为各分量都是“一维连续型随机变量的那种”。考虑 $X_1 = X_2$,则 $(X_1, X_2)$ 仅在对角线处有值,故不可能存在概率密度函数。

\begin{definition}
	设 $f(x_1, \cdots, x_n)$ 是定义在 $\mathbb{R}^n$ 上的非负函数,使得对 $\mathbb{R}^n$ 中的任何集合 $A$,有
	\[ P\{X \in A\} = \int_A \cdots \int f(x_1, \cdots, x_n) \d x_1 \cdots \d x_n \]
	则称 $f$ 是 $X$ 的概率密度函数。显然当 $P\{ X \in \mathbb{R}^n\} = 1$。
\end{definition}

也可以定义分布函数
\[ F(x_1, \cdots, x_n) = P\{X_1 \leqslant x_1, \cdots, X_n \leqslant x_n \} \]

对于二维 $(X, Y)$,若 $f$ 在点 $(x, y)$ 处连续,则
\[ \frac{\partial^2 F(x, y)}{\partial x \partial y} = f(x, y) \]
反之若 $F(x, y)$ 连续可导,则此式是其概率密度。

\paragraph{边缘分布}

对任意的 $n$ 个实数,$x_1, x_2, \cdots, x_n$,称 $n$ 元函数
\[ F(x_1, \cdots, x_n) = P\{X_1 \leqslant x_1, X_2 \leqslant x_2, \cdots. X_n \leqslant x_n\} \]
为多维随机变量 $(X_1, \cdots, X_n)$ 的联合分布函数,记为 $(X_1, \cdots, X_n) \sim F(x_1, \cdots, x_n)$。

\begin{itemize}
	\item $F(x, y)$ 是 $x, y$ 的单调不减函数。
	\item $F(x, y)$ 是 $x, y$ 的右连续函数。
	\item $F(-\infty, y) = F(x, -\infty) = F(-\infty, -\infty) = 0$,$F(+\infty, +\infty) = 1$。
	\item 对任意 $x_1 < x_2, y_1 < y_2$,有
	      \[ P\{x_1 < X \leqslant x_2, y_1 Y \leqslant y_2\} = F(x_2, y_2) - F(x_2, y_1) - F(x_1, y_2) + F(x_1, y_1) \geqslant 0 \]
\end{itemize}

设其联合分布函数为 $F(x,y)$,定义边缘分布函数
\[ F_X(x) = P\{X \leqslant x\} = F(x, +\infty) \]
同理,有 $F_Y(y) = F(+\infty, y)$。


\subsection{常见的二维分布}

\paragraph{二维均匀分布} 称 $(X,Y)$ 在平面有界区域 $D$ 上服从均匀分布,如果 $(X, Y)$ 的概率密度为
\[ f(x, y) = \frac{1}{S_D}, \quad (x, y) \in D \]

\paragraph{二维正态分布} TODO。

\subsection{二维随机变量的独立性}

\paragraph{条件概率分布}

对于二维离散型随机向量 $(X, Y)$,其联合概率分布为
\[ p_{i,j} = P\{X= x_i, Y = y_j\} \]
记为 $(X,Y) \sim p_{i,j}$。依条件概率的定义,有
\[ P\{ X = x_i \mid Y = y_i \} = \frac{P\{X = x_i, Y = y_j\}}{P\{Y=y_j\}} = \frac{p_{i, j}}{p_{, j}} \]

设二维连续型随机向量 $(X, Y)$,其概率密度是 $f(x, y)$。假设 $y \in [a, b]$,依条件概率的定义,有
\[ P\{X \leqslant c \mid a \leqslant Y \leqslant b\} = \frac{P\{X \leqslant c , a \leqslant Y \leqslant b\}}{P\{a \leqslant y \leqslant b \}} = \frac{\int_{-\infty}^c \int_{a}^b f(x, y) \d y \d x }{\int_a^b f_Y(y) \d y }\]
对 $c$ 求导,即得条件密度函数
\[ f_{X \mid Y}(x \mid a \leqslant y \leqslant b) = \frac{\int_{a}^b f(x, y) \d y}{\int_{a}^b f_{Y}(y) \d t} \]
考虑极限,$a, b$ 收敛于 $y$ 处,有
\[ f_{X \mid Y}(x \mid y) = \frac{f(x, y)}{f_{Y}(y)} \]

\paragraph{独立性}

与一维情况类似,我们可以推广为 $n$ 维。

\begin{definition}
	设 $n$ 维随机向量 $(X_1, \cdots, X_n)$ 的联合密度函数为 $f(x_1, \cdots, x_n)$,而 $f_i$ 是 $X_i$ 的边缘密度函数,若
	\[ f(x_1, \cdots, x_n) = f_1(x_1) \cdots f_n(x_n) \]
	则称随机变量 $X_1, \cdots, X_n$ 相互独立。
\end{definition}

\paragraph{随机变量函数的概率分布}

离散的情况是容易的,实在不行手算。

设连续型随机变量 $X$ 有密度函数 $f(x)$,对于函数 $g$,考虑 $Y=g(X)$ 下的情况。假设 $g$ 严格上升,设 $h = g^{-1}$,有
\[ P\{Y \leqslant y\} = P\{X \leqslant h(y)\} = \int_{-\infty}^{h(y)} f(t) \d t \]
则 $y$ 的密度函数为
\[ l(y) = f(h(y)) h'(y) \]
同理,对严格递减也可以进行讨论,得到结论
\[ l(y) = f(h(y)) |h'(y)| \]

例如 $Y = aX+b$,则 $Y$ 的密度函数为
\[ l(y) = \frac{1}{|a|}f\left(\frac{y-b}{a}\right) \]

现考虑二元情况 $(X_1, X_2)$ 的密度函数为 $f(x_1, x_2)$,并有变量
\[ Y_1 = g_1(X_1, X_2), \quad Y_2 = g_2(X_1, X_2) \]
假定存在逆变换
\[ X_1 = h_1(Y_1, Y_2), \quad X_2 = h_2(Y_1, Y_2) \]
此时 Jacobi 行列式
\[ J(y_1, y_2) = \left|\begin{matrix}
		\partial h_1 / \partial y_1 & \partial h_1 / \partial y_2 \\
		\partial h_2 / \partial y_1 & \partial h_2 / \partial y_2
	\end{matrix}\right| \]
不为 $0$。在 $(Y_1, Y_2)$ 的平面上任取一个区域 $A$,对应于 $(X_1, X_2)$ 上的区域是 $B$。即
\[ \begin{aligned}
		P\{(Y_1, Y_2) \in A\} & = P\{(X_1, X_2) \in B\}                                                      \\
		                      & = \iint_{B} f(x_1, x_2) \d x_1 \d x_2                                        \\
		                      & = \iint_{A} f(h_1(y_1, y_2), h_2(y_1, y_2)) \cdot |J(y_1, y_2) \d y_1 \d y_2
	\end{aligned} \]
因此 $(Y_1, Y_2)$ 的概率密度为
\[ l(y_1, y_2) = f(h_1(y_1, y_2), h_2(y_1, y_2)) | J(y_1, y_2)| \]

比较常见的例子是线性变换
\[ Y_1 = a_{11} X_1 + a_{12} X_2, \quad Y_2 = a_{21} X_1 + a_{22} X_2 \]
当其行列式不为 $0$ 时,设其存在逆变换
\[ X_1 = b_{11} Y_1 + b_{12} Y_2, \quad X_2 = b_{21} Y_1 + b_{22} Y_2 \]
可得
\[ l(y_1, y_2) = f(b_{11} y_1 + b_{12} y_2, b_{21} y_1 + b_{22} y_2) | b_{11} b_{22} - b_{12} b_{21} | \]

\section{随机变量的数字特征}

设 $X$ 是随机变量,其分布列为 $p_i = P\{X = x_i\}$,记
\[ E(X) = \sum_{i=1}^\infty x_i p_i \]
为随机变量 $X$ 的数学期望。若 $X$ 是连续型随机变量,则记
\[ E(X) = \int_{-\infty}^{+\infty} x f(x) \d x \]
为其期望。

其拥有线性性。比如设 $X, Y$ 相互独立,有
\[ E(X Y) = E(X) E(y), \quad E(X \pm Y) = E(X) \pm E(Y) \]

我们记 $E[(X - E(X))^2]$ 为 $X$ 的方差,有
\[ D(X) = E[(X - E(X))^2] = E(X^2) - (E(X))^2 \]
称 $\sqrt{D(X)}$ 为 $X$ 的标准差,或者均方差,记为 $\sigma(X)$。

\begin{theorem}[切比雪夫不等式]
	如果随机变量 $X$ 的期望 $E(X)$ 和方差 $D(X)$ 存在,则对任意 $\eps > 0$ 有
	\[ P\{|X - E(X)| < \eps\} \geqslant 1 - \frac{D(X)}{\eps^2} \]
\end{theorem}

我们定义 $(X, Y)$ 的协方差为
\[ \Cov(X, Y) = E[(X - E(X)(Y - E(Y)))] = E(XY) - E(X) E(Y) \]

称 $\rho_{XY} = \frac{\Cov(X, Y)}{\sqrt{D(X) D(Y)}}$ 为 $X, Y$ 的相关系数。

\section{大数定律与中心极限定理}

设随机变量 $X$ 与随机变量序列 $\{X_n\}$,如果对任意的 $\eps > 0$ 有
\[ \lim_{n \to \infty} P\{|X_n - X| < \eps \} =1 \]
则称随机变量序列 $\{X_n\}$ 依概率收敛于随机变量 $X$,记为
\[ \lim_{n \to \infty} X_n = X(P), \quad \text{或}\ X_n \stackrel{P}{\longrightarrow} X(n \to \infty) \]

\begin{theorem}[切比雪夫大数定律]
	设 $\{X_n\}$ 是相互独立的随机变量序列,如果方差 $D(X)$ 存在且有一致有上界,则 $\{X_n\}$ 服从大数定律
	\[ \frac{1}{n} \sum_{i=1}^n X_i \stackrel{P}{\longrightarrow} \frac{1}{n} \sum_{i=1}^n E(X_i) \]
\end{theorem}

\begin{theorem}[伯努利大数定律]
	假设 $\mu_n$ 是 $n$ 重伯努利实验中时间 $A$ 发生的次数,在每次实验中 $A$ 发生的概率为 $p(0 < p < 1)$,则
	\[ \frac{\mu_n}{n} \stackrel{P}{\longrightarrow} p \]
\end{theorem}

\begin{theorem}[辛钦大数定律]
	设 $\{X_n\}$ 是独立同分布的随机变量序列,如果 $E(X_i) = \mu$ 存在,则
	\[ \frac{1}{n} \sum_{i=1}^n X_i \stackrel{P}{\longrightarrow} \mu \]
\end{theorem}

\begin{theorem}[列维 - 林德伯格定理]
	假设 $\{X_n\}$ 是独立同分布的随机变量序列,如果
	\[ E(X_i) = \mu, D(X_i) = \sigma^2 > 0 \]
	存在,则对任意的实数 $x$ 有
	\[ \lim_{n \to \infty} P\left\{ \frac{\sum_{i=1}^n X_i - n \mu}{\sigma \sqrt{n}} \leqslant x \right\} = \frac{1}{\sqrt{2 \pi}} \int_{-\infty}^x \exp\left(-\frac{t^2}{2}\right) \d t = \Phi(x)  \]
\end{theorem}

\begin{theorem}[棣莫弗 - 拉普拉斯定理]
	设随机变量 $Y_n \sim B(n, p)$,其中 $0 < p < 1$ 且 $n > 1$,则对任意的实数 $x$,有
	\[ \lim_{n \to \infty} P\left\{ \frac{Y_n - np}{\sqrt{np(1 - p)}} \leqslant x \right\} = \frac{1}{\sqrt{2 \pi}} \int_{-\infty}^x \exp\left(-\frac{t^2}{2}\right) \d t = \Phi(x)  \]
\end{theorem}

\section{数理统计}

研究对象的全体称为总体,组成总体的每一个元素称为个体。我们把总体和 $X$ 等同起来,所谓总体的分布就是指 $X$ 的分布。

$n$ 个相互独立且与总体 $X$ 具有相同概率分布的随机变量 $X_1, \cdots, X_n$ 所组成的总体为 $(X_1, \cdots, X_n)$ 称为来自总体 $X$ 容量为 $n$ 的一个简单随机样本,简称样本。一次抽样结果的 $n$ 个具体数值称为 $X_1, \cdots, X_n$ 的一个观测值(样本值)。

假设总体 $X$ 的分布函数为 $F$,则 $(X_1, \cdots, X_n)$ 的分布函数为
\[ F(x_1, \cdots, x_n) = \prod_{i=1}^n F(x_i) \]

设 $X_1, \cdots, X_n$ 为来自总体 $X$ 的一个样本,$g$ 为仅与 $x$ 有关的 $n$ 元函数,则称 $g$ 为样本的一个统计量。若 $(x_1, \cdots, x_n)$ 为样本值,则 $g(x_1, \cdots, x_n)$ 为观测值。

样本均值
\[ \overline{X} = \frac{1}{n} \sum_{i=1}^n X_i \]
样本方差
\[ S^2 = \frac{1}{n-1} \sum_{i=1}^n (X_i - \overline{X})^2 \]
样本 $k$ 阶(原点)矩
\[ A_k = \frac{1}{n} \sum_{i=1}^n X_i^k \]
样本 $k$ 阶中心矩
\[ B_k = \frac{1}{n} \sum_{i=1}^n (X_i - \overline{X})^k \]

将 $n$ 个观测量从小到大的顺序排列,记随机变量 $X_{(k)}$ 为第 $k$ 顺序统计量。

常用统计量:

\[
	\begin{aligned}
		E(X_i)          & = E(X)             \\
		D(X_i)          & = D(X)             \\
		E(\overline{X}) & = E(X)             \\
		D(\overline{X}) & = \frac{1}{n} D(X) \\
		E(S^2)          & = D(X)
	\end{aligned}
\]

\subsection{三大分布}

\newcommand{\calX}{\mathcal{X}}

\paragraph{$\calX^2$ 分布}

若随机变量 $X_1, \cdots, X_n$ 相互独立且都服从标准正态分布,则随机变量 $X = \sum X_i^2$ 服从自由度为 $n$ 的 $\calX^2$ 分布,记为 $X \sim \calX^2(n)$。

对于给定的 $\alpha(0 < \alpha < 1)$,称满足
\[ P\left\{\calX^2 > \calX_\alpha^2(n)\right\} = \int_{\calX_\alpha^2(n)}^n f(x) \d x = \alpha  \]
的 $\calX_\alpha^2(n)$ 为 $\calX^2(n)$ 分布的上 $\alpha$ 分位点。

\paragraph{$t$ 分布}

设随机变量 $X \sim N(0, 1), Y \sim \calX^2(n)$,$X$ 与 $Y$ 互相独立,则随机变量 $t = \frac{X}{\sqrt{Y / n}}$ 服从自由度为 $n$ 的 $t$ 分布,记为 $t \sim t(n)$.

\paragraph{$F$ 分布}

设随机变量 $X \sim \calX^2(n_1), y \sim \calX^2(n_2)$,且 $X$ 与 $Y$ 相互独立,则 $F = \frac{X / n_1}{Y / n_2}$ 服从自由度为 $(n_1, n_2)$ 的 $F$ 分布,记为 $F \sim F(n_1, n_2)$。

\subsection{参数的点估计}

设总体 $X$ 的分布函数为 $F(x; \theta)$,其中 $\theta$ 是一个未知参数,$X_1, \cdots, X_n$ 是取自总体 $X$ 的一个样本。由样本构造一个适当的统计量 $\hat{\theta}(X_1, \cdots, X_n)$ 作为参数 $\theta$ 的估计,则称 $\hat{\theta}$ 为其估计量。

如果 $x_1, \cdots, x_n$ 是样本的一个观察值,将其带入估计量得值 $\hat{\theta}(x_1, \cdots, x_n)$ 并以此值作为未知参数的近似值,则称为 $\theta$ 的估计值。

\paragraph{矩估计法}

设总体 $X$ 分布有 $n$ 个样本,有 $k$ 个未知参数。若 $X$ 的原点矩存在,我们令样本矩等于总体矩
\[ \frac{1}{n} \sum_{i=1}^{N} X_i^l = E(X^l), \quad l = 1, \cdots, k \]
这是包含 $k$ 个参数的 $k$ 个方程,由此解得矩估计量和矩估计值。

一般约定:用矩法方程求总体未知参数的估计量时,从低阶开始。

\paragraph{最大似然估计法}

最大似然原理:对未知参数 $\theta$ 进行估计时,在该参数可能的取值范围 $I$ 内选取,用使”样本获得观测值 $x_1, \cdots, x_n$ 的概率最大的参数值 $\hat{\theta}$ 作为 $\theta$ 的估计。

假设 $X$ 是离散型随机变量,其概率分布为 $P\{X = x\} = p(x; \theta)$,那么求其取值概率
\[ L(x_1, \cdots, x_n;\theta) = P\{X_1 = x_1, \cdots, X_n = x_n\} = \prod_{i=1}^n P\{X_i=x_i\} = \prod_{i=1}^n p(x_i; \theta) \]
称为样本的似然函数。若存在 $\hat{theta}$ 使得 $L$ 取到最大值,则称 $\hat{theta}$ 为最大似然估计值,对应的统计量是 $\theta$ 的最大似然估计量。

同理,连续型随机变量也有
\[ L(x_1, \cdots, x_n; \theta) = \prod_{i=1}^n f(x_i; \theta) \]



 % 线性空间
\chapter{线性映射}

\section{线性映射及其运算}

\begin{definition}[线性映射]\index{xianxingyingshe@线性映射}
	设域 $F$ 上两个线性空间 $V,V'$,若映射 $\vbf{A} : V \to V'$ 保持加法运算和纯量乘法运算,即
	\[ \vbf{A}(\alpha + \beta) = \vbf{A}(\alpha) + \vbf{A}(\beta), \quad \vbf{A}(k\alpha) = k \vbf{A}(\alpha) \]
	则称 $\vbf{A}$ 是 $V$ 到 $V'$ 的一个线性映射。
\end{definition}

线性空间 $V$ 到自身的线性映射通常称为 $V$ 上的线性变换,域 $F$ 上线性空间 $V$ 到 $F$ 的线性映射称为 $V$ 上的线性函数。\index{xianxingbianhuan@线性变换}

$\vbf{A}(\alpha)$ 也可写成 $\vbf{A}\alpha$。

特殊的,从 $V$ 映射到 $V'$ 的零向量是零映射,记作 $\vbf{0}$;$V$ 上映射到自身的变换叫做恒等变换,记作 $\vbf{I}$;映射 $\alpha$ 到 $k\alpha$ 的变换叫由 $k$ 决定的数乘变换,记作 $\vbf{k}$。

定积分 $\displaystyle \vbf{J}(f(x)) = \int_a^b f(x) \d x$ 也是 $C[a,b]$ 到 $\mathbb{R}$ 的一个线性映射。

\paragraph{线性映射的存在性}

\begin{theorem}
	设域 $F$ 上的线性空间 $V,V'$,且 $V$ 是有限维的。从 $V$ 中取一个基 $\seq{\alpha}{n}$,从 $V'$ 任意取定 $n$ 个向量 $\seq{\gamma}{n}$(可以有相同),令
	\[ \vbf{A} : V \to V', \alpha = \sum_{i=1}^n \alpha_i\alpha_i \mapsto \sum_{i=1}^n \alpha_i\gamma_i \]
	则 $\vbf{A}$ 是 $V$ 到 $V'$ 的一个线性映射,且 $\vbf{A}(\alpha_i) = \gamma_i$。
\end{theorem}

设域 $F$ 上的线性空间 $V$ 有两个子空间 $U,W$,且 $V = U \oplus W$。任取 $\alpha \in V$,设 $\alpha = \alpha_1 + \alpha_2, \alpha_1 \in U, \alpha_2\in W$。令
\[ \vbf{P}_U : V \to V, \alpha \mapsto \alpha_1 \]
则 $\vbf{P}_U$ 是 $V$ 上的一个线性变换。称 $\vbf{P}_U$ 是平行于 $W$ 在 $U$ 上的投影,它满足
\[ \vbf{P}_U(\alpha) =
	\begin{cases}
		\alpha, & \alpha \in U  \\
		0 ,	 & \alpha \in W
	\end{cases} \]
可以证明,满足该式的投影变换是唯一的。

类似的,定义 $\vbf{P}_W(\alpha) = \alpha_2$,称它为平行于 $U$ 在 $W$ 的投影。

\begin{definition}[幂等变换]\index{midengibanhuan@幂等变换}
	线性空间 $V$ 上的线性变换 $\vbf{A}$ 如果满足 $\vbf{A}^2 = \vbf{A}$,则称 $\vbf{A}$ 是幂等变换。
\end{definition}

\begin{definition}
	线性空间 $V$ 上的两个线性变换 $\vbf{A}, \vbf{B}$ 如果满足 $\vbf{A}\vbf{B} = \vbf{B}\vbf{A} = 0$,则称 $\vbf{A}$ 与 $\vbf{B}$ 正交。
\end{definition}

任取 $\alpha\in V$,设 $\alpha = \alpha_1 + \alpha_2, \alpha_1 \in U, \alpha_2 \in W$,则
\[
	\begin{aligned}
		\vbf{P}_U^2(\alpha) = \vbf{P}_U(\vbf{P}_U(\alpha)) = \alpha_1 = \vbf{P}_U(\alpha) \\
		\vbf{P}_W^2(\alpha) = \vbf{P}_W(\vbf{P}_W(\alpha)) = \alpha_1 = \vbf{P}_W(\alpha)
	\end{aligned}
\]
且
\[ \vbf{P}_U\vbf{P}_W(\alpha) = \vbf{P}_U(\alpha_2) = 0 = \vbf{P}_W(\alpha_1) = \vbf{P}_W\vbf{P}_U(\alpha) \]
由此得出
\[ \vbf{P}_U^2 = \vbf{P}_U, \vbf{P}_W^2 = \vbf{P}_W, \vbf{P}_U \]

\paragraph{线性映射的运算}

设域 $F$ 上的线性空间 $V,V'$,把所有 $V \to V'$ 的线性映射所组成的集合记作 $\Hom(V,V')$,同样有 $\Hom{V,V}$。

设域 $F$ 上的线性空间 $V,U,W$,其中 $\vbf{A} \in \Hom(V,U), \vbf{B} \in \Hom(U,W)$。线性映射作为映射,有映射的乘法 $\vbf{B}\vbf{A}$。

若 $\vbf{A}\in \Hom(V,V')$ 可逆,则 $\vbf{A}$ 是 $V \to V'$ 的一同构映射,从而 $\vbf{A}^{-1}$ 是 $V' \to V$ 的同步映射。于是 $\vbf{A}^{-1} \in \Hom(V',V)$。

设 $\vbf{A},\vbf{B} \in \Hom (V,V')$,由于陪域 $V'$ 是线性空间,因此可以定义加法和纯量乘法如下
\[ (\vbf{A} + \vbf{B}) \alpha \coloneqq \vbf{A}\alpha + \vbf{B}\alpha, (k\vbf{A})\alpha \coloneqq k(\vbf{A} \alpha) \]
显然其运算结果都是线性映射,称 $\vbf{A} + \vbf{B}$ 是 $\vbf{A}$ 与 $\vbf{B}$ 的和,$k\vbf{A}$ 是 $k$ 与 $\vbf{A}$ 的纯量乘积。

不难验证,$\Hom(V,V')$ 是域 $F$ 上的线性空间。特别的,$\Hom(V,V)$ 是一个有单位元的环,还可证明其上变换的乘法与纯量加法满足
\[ k(\vbf{A}\vbf{B}) = (k\vbf{A})\vbf{B} = \vbf{A}(k\vbf{B}) \]

于是有

\begin{definition}[代数]\index{daishu@代数}
	设域 $F$ 上的线性空间 $A$ 对于其上的加法和纯量乘法是一个有单位元的交换环,且
	\[ k(\alpha \beta) = (k\alpha)\beta = \alpha(k\beta), \forall k \in F,\alpha,\beta \in A \]
	那么称 $A$ 是一个(结合)代数,把线性空间 $A$ 的维数称为代数 $A$ 的维数。
\end{definition}

容易看出,$\Hom(V,V)$ 是域 $F$ 上的代数,$M_n(F)$ 也是域 $F$ 上的代数。

因此可以在 $\Hom(V,V)$ 上定义 $\vbf{A}$ 的正整数幂
\[ \vbf{A}^m \coloneqq \vbf{A}^m \cdot \vbf{A}, \vbf{A}^0 = \vbf{I} \]
容易验证
\[ \vbf{A}^m \cdot \vbf{A}^n = \vbf{A}^{m+n}, (\vbf{A}^m)^n = \vbf{A}^{mn} \]
设 $f(x) = a_0 + a_1 x + \cdots + a_mx^m \in F[x]$,用 $\vbf{A}$ 代入得
\[ f(\vbf{A}) = a_0\vbf{I} + a_1\vbf{A} + \cdots + a_m \vbf{A}^m \]
显然 $f(\vbf{A}) \in \Hom(V,V)$,称 $f(\vbf{A})$ 是 $\vbf{A}$ 的多项式。

把 $\vbf{A}$ 的所有多项式的全体记作 $F[\vbf{A}]$,不难发现其是 $\Hom(V,V)$ 的一个子环,且 $F[\vbf{A}]$ 是交换环。

$F[\vbf{A}]$ 中所有数乘变换组成的集合是 $F[\vbf{A}]$ 是 $F[\vbf{A}]$ 的一个子环,且域 $F$ 到这个子环之间存在双射
\[ \tau : k \mapsto \vbf{k} \]
双射 $\tau$ 保持加法与乘法运算,因此 $F[\vbf{A}]$ 可以看作 $F$ 的一个扩环。于是 $F$ 上一元多项式中的不定元 $x$ 可以用 $F[\vbf{A}]$ 中任意元素代入,从而在 $F[x]$ 中关于加法和乘法的等式在 $F[\vbf{A}]$ 中也成立。

更多的,我们还有镜面反射变换
\[ \vbf{R}_U = \vbf{I} - 2\vbf{P}_W \]
平移变换
\[ \vbf{T_a} : f(x) \mapsto f(x+a) \]
利用 Taylor 公式不难发现
\[
	\begin{aligned}
		f(x+a) & = f(x) + a\vbf{A} + \frac{a^2}{2!}f''(x) + \cdots + \frac{a^{n-1}}{(n-1)!} f^{(n-1)}(x)							\\
			   & = \left( \vbf{I} + a\vbf{D} + \frac{a^2}{2!}\vbf{D} + \cdots + \frac{a^{n-1}}{(n-1)!} \vbf{D}^{(n-1)} \right)f(x)
	\end{aligned}
\]
即平移 $\vbf{T}_a$ 是导数 $\vbf{D}$ 的一个多项式。

\section{线性映射的核与象}

\begin{definition}[核] \index{he@核}
	设域 $F$ 上的线性空间 $V,V'$,其中 $\vbf{A} : V \to V'$,令 $V'$ 的零向量在 $\vbf{A}$ 下的原象集称为 $\vbf{A}$ 的和,记作 $\Ker \vbf{A}$,即
	\[ \Ker \vbf{A} \coloneqq \{ \alpha \in V \mid \vbf{A}\alpha = 0 \} \]
	$\vbf{A}$ 的象(值域)记作 $\Im A$ 或 $\vbf{A} V$。
\end{definition}

\begin{proposition}
	设域 $F$ 上线性空间 $V,V'$ 的线性映射 $\vbf{A} : V \to V'$,则
	
	(1) $\vbf{A}$ 是单射当且仅当 $\Ker \vbf{A} = 0$。
	
	(2) $\vbf{A}$ 是满射当且仅当 $\Im \vbf{A} = V'$。
\end{proposition}

\begin{theorem}
	设域 $F$ 上线性空间 $V,V'$ 的线性映射 $\vbf{A} : V \to V'$,则
	\[ V / \Ker \vbf{A} \cong \Im \vbf{A} \]
\end{theorem}

\begin{theorem}
	设域 $F$ 上线性空间 $V,V'$,且 $V$ 是有限维的。设线性映射 $\vbf{A} : V \to V'$,则 $\Ker \vbf{A}$ 和 $\Im \vbf{A}$ 都是有限维的,且
	\[ \dim(\Ker \vbf{A}) + \dim(\Im \vbf{A}) = \dim(V) \]
\end{theorem}

当 $V$ 是有限维时,线性映射 $\vbf{A} : V \to V'$,则 $\vbf{A}$ 的核的维数也称为 $\vbf{A}$ 的测度。$\vbf{A}$ 的象 $\Im \vbf{A}$ 的维数称为 $\vbf{A}$ 的秩,记作 $\rank(\vbf{A})$。

\section{线性映射和线性变换的矩阵表示}

设域 $F$ 上的有限维线性空间 $V,V'$,其上的线性映射 $\vbf{A} : V \to V'$。设 $\dim V = n,\dim V' = s$。我们知道 $\vbf{A}$ 被其在 $V$ 上的一个基所确定,不妨取 $V$ 上的一个基 $\seq{\alpha}{n}$,$\vbf{A}$ 完全被 $\seq{\vbf{A} \alpha}{n}$ 决定。由于 $\vbf{A} \alpha_i \in V'$,因此在 $V'$ 中取一个基 $\seq{\eta}{s}$,$\vbf{A} \alpha_i$ 被它在基 $\seq{\eta}{s}$ 下的坐标所决定,形式的记作
\[ 
	\left( \vbf{A} \alpha_1 , \vbf{A} \alpha_2 , \cdots , \vbf{A} \alpha_n \right) = 
	\left( \eta_1 , \eta_2 , \cdots , \eta_n \right)
	\left( \begin{matrix}
			a_{11} & a_{11} & \cdots & a_{1n} \\
			a_{21} & a_{22} & \cdots & a_{2n} \\
			\vdots & \vdots &		& \vdots \\
			a_{s1} & a_{s1} & \cdots & a_{sn}
		\end{matrix} \right)
\]
把 $s\times n$ 矩阵记作 $A$,它的第 $j$ 列就是 $\vbf{A} \alpha_j$ 在 $\seq{\eta}{s}$ 下的坐标。称 $A$ 是线性映射 $\vbf{A}$ 在 $V$ 的基 $\seq{\alpha}{n}$ 和 $V'$ 的基 $\seq{\eta}{s}$ 下的矩阵。于是线性映射 $\vbf{A}$ 有了矩阵表示。

通常把 $(\seq{\vbf{A} \alpha}{n})$ 记成 $\vbf{A}(\seq{\alpha}{n})$ 于是上式可以记成
\[ \vbf{A}(\seq{\alpha}{n}) = (\seq{\alpha}{n}) A\]

对于 $V$ 上的线性变换 $\vbf{A}$,由于 $\vbf{A} \alpha_i \in V$,因此 $\vbf{A} \alpha_i$ 可以用 $V$ 的基 $\seq{\alpha}{n}$ 线性表出,于是有
\[ 
	\left( \vbf{A} \alpha_1 , \vbf{A} \alpha_2 , \cdots , \vbf{A} \alpha_n \right) = 
	\left( \alpha_1 , \alpha_2 , \cdots , \alpha_n \right)
	\left( \begin{matrix}
			a_{11} & a_{11} & \cdots & a_{1n} \\
			a_{21} & a_{22} & \cdots & a_{2n} \\
			\vdots & \vdots &		& \vdots \\
			a_{n1} & a_{n1} & \cdots & a_{nn}
		\end{matrix} \right)
\]
把右端的 $n$ 级矩阵记作 $A$,它的第 $j$ 列就是 $\vbf{A} \alpha_j$ 在 $\seq{\alpha}{n}$ 下的坐标。称 $A$ 是线性映射 $\vbf{A}$ 在 $V$ 的基 $\seq{\alpha}{n}$ 下的矩阵。于是线性映射 $\vbf{A}$ 有了矩阵表示。
\[ \vbf{A}(\seq{\alpha}{n}) = (\seq{\alpha}{n}) A\]

设域 $F$ 上 $n$ 维线性空间 $V$ 上的幂等变换,有
\[ V = \Im \vbf{A} \oplus \Ker \vbf{A} \]
且 $\vbf{A}$ 是平行于 $\Ker \vbf{A}$ 在 $\Im \vbf{A}$ 上的投影。在 $\Im \vbf{A}$ 中取一个基 $\seq{\alpha}{r}$;在 $\Ker \vbf{A}$ 中取一个基 $\seq{\beta}{{n-r}}$,则 $\seq{\alpha}{r},\seq{\beta}{{n-r}}$ 是 $V$ 的一个基,由于 $\vbf{A}$ 是平行于 $\Ker \vbf{A}$ 在 $\Im A$ 上的投影,因此
\[ \vbf{A} \alpha_i = \alpha_i, \vbf{A} \beta_j = 0 \]
从而幂等变换 $\vbf{A}$ 在基 $\seq{\alpha}{r},\seq{\beta}{{n-r}}$ 下的矩阵 $\vbf{A}$ 为
\[ A = \left(\begin{matrix}
			E_r & 0  \\
			0   & 0
		\end{matrix}\right) \]
其中 $r = \rank(A) = \dim(\Im \vbf{A}) = \rank(\vbf{A})$。

\paragraph{$\Hom(V,V')$ 与 $M_{s \times n}(F)$ 的关系,$\Hom(V,V)$ 与 $M_n(F)$ 的关系}

\begin{theorem}
	设域 $F$ 上的 $n$ 维线性空间 $V$ 和 $s$ 维的线性空间 $V'$,则线性映射 $\vbf{A} : V \to V'$ 与它在 $V$ 的一个基和 $V'$ 的一个基下的矩阵 $A$ 的对应 $\sigma$ 是线性空间 $\Hom(V,V') \to M_{s \times n}(F)$ 的同构映射,从而
	\[ \Hom(V,V') \cong M_{s \times n}(F) \]
	\[ \dim(\Hom(V,V')) = sn = (\dim V)(\dim V') \]
\end{theorem}

特别的有
\[ \Hom(V,V) \cong M_{n}(F) \]
\[ \dim(\Hom(V,V)) = sn = (\dim V)^2 \]

注意到 $\Hom(V,V)$ 与 $M_n(F)$ 都是域 $F$ 上的代数,它们都有加法、纯量乘法、乘法运算,可以证明 $\sigma$ 映射保持乘法运算
\[ \sigma(\vbf{A} \vbf{B}) = AB = \sigma(\vbf{A})\sigma(\vbf{B}) \]
因此 $\sigma$ 是 $\Hom(V,V) \to M_n(F)$ 的同构映射。

\begin{definition}
	设域 $F$ 上的代数 $F,F'$,如果存在双射 $\sigma : M \to M'$,使得 $\sigma$ 既是线性空间 $M \to M'$ 的同构映射,又是环 $M \to M'$ 的同构映射,那么称代数 $M$ 与 $M'$ 是同构的,并且称 $\sigma$ 是代数 $M \to M'$ 的一个同构映射。
\end{definition}

\begin{theorem}
	设域 $F$ 上 $n$ 维线性空间 $V$ 上的线性变换 $\vbf{A}$,$\vbf{A}$ 与它在 $V$  的一个基下的矩阵 $A$ 的对应是代数 $\Hom(V,V) \to M_n(F)$ 的同构映射,从而代数 $\Hom(V,V)$ 与 $M_n(F)$ 是同构的。
\end{theorem}

\paragraph{向量在线性映射(或线性变换)下的象的坐标}。

设域 $F$ 上的 $n$ 维线性空间和 $s$ 维线性空间,线性映射 $\vbf{A} : V \to V'$ 在 $V$ 的一个基 $\seq{\alpha}{n}$ 和 $V'$ 的一个基 $\seq{\eta}{s}$ 下的矩阵为 $A$。设 $V$ 中向量 $\alpha$ 在基 $\seq{\alpha}{n}$ 下的坐标为 $X$,有
\[ \vbf{A} \alpha = (\seq{\eta}{n})AX \]
则 $\vbf{A} \alpha$ 在基 $\seq{\eta}{s}$ 下的坐标为 $AX$。

特别的,设线性空间 $V$ 中的一组基 $\seq{\alpha}{n}$,其上的线性变换 $\vbf{A}$ 在此基下的矩阵为 $A$。把向量 $\alpha$ 在此基下的坐标记作 $X$,有
\[ \vbf{A} \alpha = (\seq{\alpha}{n})AX \]
即向量 $\vbf{A}$ 在此基下的坐标是 $AX$。

若向量 $\gamma$ 在基 $\seq{\alpha}{n}$ 下的坐标为 $Y$,则
\[ \vbf{A}\alpha = \gamma \Leftrightarrow AX = Y \]

\paragraph{线性变换在不同基下的矩阵之间的关系}

\begin{theorem}
	设域 $F$ 上的 $n$ 维线性空间,线性变换 $\vbf{A}$ 在基 $\seq{\alpha}{n}$ 下的矩阵为 $A$,在基 $\seq{\eta}{s}$ 下的矩阵为 $B$,从基 $\seq{\alpha}{n}$ 到基 $\seq{\eta}{s}$ 的过度矩阵为 $S$,则
	\[ B = S^{-1}AS \]
\end{theorem}

即 同一个线性变换 $\vbf{A}$ 在 $V$ 的不同基下的矩阵是相似的。

由于行列式、秩、迹都是相似关系下的不变量,因此我们把 $\vbf{A}$ 在 $V$ 的基下的矩阵 $A$ 的行列式、秩、迹分别叫做线性变换 $\vbf{A}$ 的行列式、秩、迹,依次记作 $\det(\vbf{A}),\rank(\vbf{A}),\tr(\vbf{A})$。

\section{线性变换的特征值和特征向量,线性变换可对角化的条件}

\begin{definition}
	设域 $F$ 上的线性空间 $V$ 上的一个线性变换 $\vbf{A}$,若 $V$ 中存在一个非零向量 $\xi$,存在 $\lambda_0 \in F$ 使得
	\[ \vbf{A} \xi = \lambda_0\xi \]
	则称 $\lambda_0$ 是线性变换 $\vbf{A}$ 的一个特征值,称 $\xi$ 是 $\vbf{A}$ 的属于特征值 $\lambda_0$ 的一个特征向量。
\end{definition}

对于几何空间 $V$,那么 $\vbf{A}$ 对 $\xi$ 的作用是把 $\xi$ 拉伸或压缩 $\lambda_0$ 倍。

\begin{theorem}
	设域 $F$ 上 $n$ 为线性空间 $V$ 上的线性变换 $\vbf{A}$ 可对角化当且仅当 $\vbf{A}$ 有 $n$ 个线性无关的特征向量 $\seq{\xi}{n}$,此时 $\vbf{A}$ 在 $\seq{\xi}{n}$ 下的矩阵为
	\[ \diag\{\seq{\lambda}{n}\} \]
	其中 $\lambda_i$ 是 $\xi_i$ 所属的特征值。该矩阵称为 $\vbf{A}$ 的标准型。除了主对角线上元素的排列次序外,$\vbf{A}$ 的标准型是由 $\vbf{A}$ 唯一决定的。
\end{theorem}

设域 $F$ 上线性空间 $V$ 上的线性变换 $\vbf{A}$,$\lambda_0$ 是 $\vbf{A}$ 的一个特征值,令
\[ V_{\lambda_0} \coloneqq \{ \alpha \in V \mid \vbf{A}\alpha = \lambda_0\alpha \} \]
易验证 $V_{\lambda_0}$ 是 $V$ 的一个子空间,称 $V_{\lambda_0}$ 是数域特征值 $\lambda_0$ 的特征子空间。且
\[ V_{\lambda_0} = \Ker(\lambda_0\vbf{I} - \vbf{A}) \]

\begin{proposition}
	域 $F$ 上 $n$ 维线性空间 $V$ 上的线性变换 $\vbf{A}$ 可对角化当且仅当下式成立
	\[ V = V_{\lambda_1} \oplus \cdots \oplus V_{\lambda_s} \]
	其中 $\seq{\lambda}{s}$ 是 $\vbf{A}$ 全部的不同的特征值。
\end{proposition}

由于 $n$ 维线性空间 $V$ 上线性变换 $\vbf{A}$ 在 $V$ 的不同基下的矩阵是相似的,而相似的矩阵有相等的多项式,因此我们把 $\vbf{A}$ 在 $V$ 的一个基下的矩阵 $A$ 的特征多项式称为线性变换 $\vbf{A}$ 的特征多项式。设 $\lambda_i$ 是 $\vbf{A}$ 的一个特征值,把 $\lambda_i$ 作为特征多项式的根的重数叫做 $\lambda_i$ 的代数重数,把 $\vbf{A}$ 的属于特征值 $\lambda_i$ 的特征子空间 $V_{\lambda_i}$ 的维数叫做 $\lambda_i$ 的几何重数。\index{daishuchongshu@代数重数} \index{jihechognshu@几何重数}

因此 $\vbf{A}$ 可标准化当且仅当 $\vbf{A}$ 的标准型为对角矩阵,其主对角线上的元素是 $\vbf{A}$ 的全部特征值,且每个特征值 $\lambda_i$ 出现的次数等于它的几何重数。即当 $\vbf{A}$ 可对角化时,$\vbf{A}$ 的特征多项式为
\[ (\lambda - \lambda_1)^{r_1}\cdots(\lambda - \lambda_s)^{r_s} \]

\begin{theorem}
	域 $F$ 上 $n$ 维线性空间 $V$ 上的线性变换 $\vbf{A}$ 可对角化当且仅当 $\vbf{A}$ 的特征多项式在 $F[\lambda]$ 中可分解成
	\[ (\lambda - \lambda_1)^{l_1} \cdots (\lambda - \lambda_s)^{l_s} \]
	其中 $\seq{\lambda}{s}$ 两两不等,其诶 $\vbf{A}$ 的每个特征值 $\lambda_i$ 的几何重数等于它的代数重数。
\end{theorem}

\section{线性变换的不变子空间}

\begin{definition}[不变子空间] \index{bubianzikongjian@不变子空间}
	设域 $F$ 上的线性空间 $V$ 上的线性变换 $\vbf{A}$,若 $V$ 的子空间 $W$ 如果满足对任意 $\alpha \in W$ 都有 $\vbf{A} \alpha \in W$ 那么称 $W$ 是 $\vbf{A}$ 的不变子空间,简称为 $\vbf{A}$- 子空间。
\end{definition}

设线性空间 $V$ 上的可交换线性变换 $\vbf{A},\vbf{B}$,容易验证 $\Ker \vbf{B},\Im \vbf{B},\vbf{B}$ 的特征子空间都是 $\vbf{A}$- 子空间

\begin{theorem}
	设域 $F$ 上 $n$ 维线性空间 $V$ 上的线性变换 $\vbf{A}$,则 $\vbf{A}$ 在 $V$ 的一个基下的矩阵为分块对角矩阵当且仅当 $V$ 能被分解为 $\vbf{A}$ 的非平凡不变子空间的直和:$V = W_1 \oplus \cdots \oplus W_s$,并且 $A_i$ 是 $\vbf{A} \mid W_i$ 在 $W_i$ 下的矩阵。
\end{theorem}

\begin{theorem}
	设域 $F$ 上的线性空间 $V$ 上的线性变换 $\vbf{A}$,若在 $K[x]$ 中有
	\[ f(x) = f_1(x) \cdots f_s(x) \]
	且 $f_1(x),\cdots,f_s(x)$ 两两互素,则
	\[ \Ker f(\vbf{A}) = \Ker f_1(\vbf{A}) \oplus \cdots \oplus \Ker f_s(\vbf{A}) \]
\end{theorem}

\begin{definition}[零化多项式] \index{linghuaduoxiangshi@零化多项式}
	设域 $F$ 上线性空间 $V$ 上的线性变换 $\vbf{A}$,若 $F$ 上的一元多项式 $f(\vbf{A}) = \vbf{0}$,那么称 $f(x)$ 为 $\vbf{A}$ 的一个零化多项式。
\end{definition}

设 $\dim V = 0$,则 $\dim(\Hom(V,V)) = n^2$,从而
\[ \vbf{I}, \vbf{A}, \vbf{A}^2, \cdots, \vbf{A}^{n^2} \]
一定线性相关,于是存在一组不全为 $0$ 的元素 $k_0,\cdots,k_{n^2}$ 使得
\[ k_0\vbf{I} + k_1\vbf{A} + k_2\vbf{A}^2 + \cdots + k_{n^2}\vbf{A}^{n^2} = \vbf{0} \]
令 $f(x) = k_0 + k_1x + k_2x^2 + \cdots + k_{n^2}x^{n^2}$,则有 $f(\vbf{A}) = \vbf{0}$,于是 $f(x)$ 是 $\vbf{A}$ 的一个零化多项式。

\begin{definition}
	设 $F$ 上的 $n$ 级矩阵 $A$,若 $f(x) \in F[x]$ 使得 $f(A) = 0$,那么称 $f(x)$ 是 $A$ 的一个零化多项式。
\end{definition}

\begin{theorem}[Hamilton - Cayley 定理]
	设域 $F$ 上的 $n$ 级矩阵,则 $A$ 的特征多项式 $f(\lambda)$ 是 $A$ 的一个零化多项式。
\end{theorem}

利用该定理可以把 $V$ 分解为 $\vbf{A}$ 的非平凡不变子空间的直和:设 $\vbf{A}$ 的特征多项式 $f(\lambda)$ 在 $F[\lambda]$ 中分解为
\[ f(\lambda) = p_1^{r_1}(\lambda) \cdots p_s^{r_s}(\lambda) \]
其中 $p_i(\lambda)$ 是 $F$ 上两两不等的首一不可约多项式,则
\[ V = \Ker(p_1^{r_1}(\vbf{A})) \oplus \cdots \oplus \Ker(p_s^{r_s}(\vbf{A})) \]

如果 $f(\lambda)$ 可以分解为
\[ f(\lambda) = (\lambda - \lambda_1)^{r_1} \cdots (\lambda - \lambda_s)^{r_s} \]
其中 $\lambda_i$ 是 $F$ 中两两不等的元素,则
\[ V = \Ker((\vbf{A} - \lambda_1\vbf{I})^{r_1}) \oplus \cdots \oplus \Ker((\vbf{A} - \lambda_s\vbf{I})^{r_s}) \]
其中 $\Ker((\vbf{A} - \lambda_j\vbf{I})^{r_j})$ 称为 $\vbf{A}$ 的根子空间。

\section{*线性变换和矩阵的最小多项式}

\begin{definition}[最小多项式] \index{zuixiaoduoxiangshi@最小多项式}
	设域 $F$ 上线性空间 $V$ 的一个线性变换 $\vbf{A}$ 的所有非零的零化多项式中,次数最低的首项系数为 $1$ 的多项式称为 $\vbf{A}$ 的最小多项式。
\end{definition}

\section{幂零变换的 Jordan 标准型}

\begin{definition}
	若 $\eta \in W$,且存在一个正整数 $t$ 使得 $\vbf{B}^{t-1}\eta \ne 0, \vbf{B}^t\eta = 0$,则称子空间 $\langle \vbf{B}^{t-1}\eta,\vbf{B}\eta,\eta \rangle$ 是由 $\eta$ 生成的 $\vbf{B}$- 强循环子空间。 
\end{definition}

\begin{theorem}
	设域 $F$ 上 $r$ 维线性空间 $W$ 上的幂零变换 $\vbf{B}$,其幂零指数为 $l$,则 $W$ 能分解成 $\dim W_0$ 个 $\vbf{B}$- 强循环子空间的直和,其中 $W_0$ 是 $\vbf{B}$ 的属于特征值 $0$ 的特征子空间。
\end{theorem}

\begin{theorem}
	设域 $F$ 上 $r$ 维线性空间 $W$ 上的幂零变换,其幂零指数为 $l$,则 $W$ 中存在一个基使得 $\vbf{B}$ 在此基下的矩阵 $B$ 为一个 Jordan 形矩阵,其中每个 Jordan 块的主对角元都是 $0$,且级数不超过 $l$;Jordan 块的总数等于 $\dim(\Ker \vbf{B}) = r - \rank(\vbf{B})$;$t$ 级 Jordan 块的个数 $N(t)$ 为
	\[ N(t) = \rank(\vbf{B}^{t+1}) + \rank(\vbf{B}^{t-1}) - 2 \rank(\vbf{B}^t) \]
	把 $B$ 称为 $\vbf{B}$ 的 Jordan 标准型。除了 Jordan 块的排列次序外,$\vbf{B}$ 的 Jordan 标准型是唯一的。
\end{theorem}

\section{线性变换的有理标准型}

\section{线性函数与对偶空间}

\begin{definition}[线性函数]\index{xianxinghanshu@线性函数}
	设域 $F$ 上的线性空间 $V$,映射 $f : V \to F$ 若对任意的 $k \in F, \alpha,\beta \in V$ 满足
	\[
		\begin{aligned}
			f(\alpha+\beta) & = f(\alpha) + f(\beta) \\
			f(k\alpha)	  & = kf(\alpha)
		\end{aligned}
	\]
	那么称 $f$ 是 $V$ 上的线性函数。
\end{definition}

例如,令
\[ \tr : M_n(F) \to F, A = (a_{ij}) \mapsto \sum_{i=1}^{n} a_{ii} \]
容易验证 $\tr$ 是 $M_n(F)$ 上的线性函数,称它为迹函数。

设 $V$ 是域 $F$ 上的线性空间,由于 $V$ 上的线性函数可看成 $V \to F$ 的映射,因此可以把 $V$ 上所有线性函数的全集记作 $\Hom(V,F)$。它是域 $F$ 上的一个线性空间,称它为 $V$ 上的线性函数空间。

容易看出 $\Hom(V,f) \cong V$,任取 $f \in \Hom(V,F)$ 由于 $f$ 完全被它在 $V$ 上的一个基 $\seq{\alpha}{n}$ 决定,因此对应法则
\[ \sigma : \Hom(V,F) \to F^n, \quad f \mapsto (f(\alpha_1),\cdots,f(\alpha_n)) \]
是一个映射,显然 $\sigma$ 是满射、单射,且保持加法和纯量运算,因此 $\sigma$ 是 $\Hom(V,F) \to F^n$ 上的同构映射,从而 $\sigma^{-1}$ 是 $\Hom(V,F) \to F^n$ 的同构映射。在 $F^n$ 中取标准基 $\seq{\vbf{\eps}}{n}$ 则 $\sigma^{-1}(\vbf{\eps}_1),\cdots,\sigma^{-1}(\vbf{\eps}_1)$ 是 $\Hom(V,F)$ 的一个基。

记 $f_i = \sigma^{_1}(\vbf{\eps}_i)$,则 $\sigma(f_i) = \vbf{\eps}$。于是有 $f_i(\alpha_j) = \delta_{ij}$。

$\Hom(V,F)$ 的这个基 $\seq{f}{n}$ 称为 $V$ 的基 $\seq{\alpha}{n}$ 的对偶基,把 $\Hom(V,F)$ 称为 $V$ 的对偶空间,记作 $V^*$。

\begin{theorem}
	设域 $F$ 上 $n$ 维线性空间 $V$,在 $V$ 中取两个基:$\seq{\alpha}{n}$ 与 $\seq{\beta}{n}$;$V^*$ 中相对应的对偶基分别为 $\seq{f}{n}$ 与 $\seq{g}{n}$。如果 $V$ 中基 $\seq{\alpha}{n}$ 到基 $\seq{\beta}{n}$ 的过渡矩阵是 $A$,那么 $V^*$ 中基 $\seq{f}{n}$ 到基 $\seq{g}{n}$ 的过渡矩阵 $B$ 为
	\[ B = \transpose{(A^{-1})} \]
\end{theorem}

设域 $F$ 上 $n$ 维线性空间 $V$,取 $V$ 一个基 $\seq{\alpha}{n}$,在 $V^*$ 中取相应的对偶基 $\seq{f}{n}$。映射
\[ \sigma : V \to V^*, \quad \alpha = \sum_{i=1}^n x_i\alpha_i \mapsto \sum_{i=1}^n x_if_i \]
是一个同构映射,把 $\alpha$ 在 $\sigma$ 下的像记作 $f_\alpha$ 或 $\alpha^*$。对 $V$ 中任意向量 $\beta = \sum_{i=1}^n y_i\alpha_i$,由于 $f_i(\beta) = y_i$,因此有
\[ f_\alpha(\beta) = \left( \sum_{i=1}^n x_if_i \right) = \sum_{i=1}^n x_if_i(\beta) = \sum_{i=1}^n x_iy_i \]
这表明 $\alpha$ 在 $\sigma$ 下的像 $f_\alpha$ 在 $\beta$ 处的函数值等于 $\alpha$ 与 $\beta$ 的坐标的对应分量乘积之和。

进一步地,我们可以考虑对偶空间 $V^*$ 的对偶空间 $(V^*)^*$,简记为 $V^{**}$,称 $V^{**}$ 是 $V$ 的双重对偶空间,有
\[ V \cong V^{*} \cong V^{**} \]

设 $\seq{\alpha}{n}$ 是 $V$ 的一个基,$V^*$ 中相应的对偶基为 $\seq{f}{n}$。设同构映射 
\[ \sigma : V \to V^*, \quad \alpha \mapsto f_\alpha \]
同理,有同构映射
\[ \tau : V^* \to V^{**}, \quad f_\alpha \mapsto \alpha^{**} \]
任取 $f \in V$,有
\[ f_\alpha = \sum_{i=1}^n x_if_i, \quad f = \sum_{i=1}^n f(\alpha_i) f_i \]
从而
\[ \alpha^{**}(f) = \sum_{i=1}^n x_if(\alpha_i) = f\left( \sum_{i=1}^n x_i\alpha_i \right) = f(\alpha) \]
由于  $(\tau\sigma)\alpha = \tau(\sigma\alpha) = \tau(f_\alpha) = \alpha^{**}$,因此 $\alpha^{**}(f)$ 的值不依赖于 $V$ 中基的选择。我们称这种不依赖于基的选择的同构映射为标准同构或自然同构。即 $\tau\sigma : V \to V^{**}$ 是自然同构。\index{zirantonggou@自然同构}

于是 $V$ 与 $V^{*}$ 互为对偶空间。
 % 线性映射
\chapter{具有度量的线性空间}

\section{双线性函数}

\begin{definition}[双线性函数]\index{shuangxianxinghanshu@双线性函数}
    设域 $F$ 上的线性空间 $V$,映射 $f : V \times V \to F$ 如果对任意的 $k_1,k_2 \in F$ 和任意的 $\beta_1,\alpha_2,\beta_1,\beta_2,\alpha,\beta \in V$ 有
    
    (1) $f(k_1\alpha_1 + k_2\alpha_2,\beta) = k_1 f(\alpha_1,\beta) + k_2f(\alpha_2,\beta)$
    
    (2) $f(\alpha,k_1\beta_1 + k_2\beta_2) = k_1 f(\alpha,\beta_1) + k_2f(\alpha,\beta_2)$
    
    那么称 $f$ 是 $V$ 上一个双线性函数,$f$ 也写成 $f(\alpha,\beta)$。
\end{definition}

即当 $\beta$ 固定时,映射 $\alpha \mapsto f(\alpha,\beta)$ 是 $V$ 上的一个线性函数,记作 $\beta_R$。

即当 $\alpha$ 固定时,映射 $\beta \mapsto f(\alpha,\beta)$ 是 $V$ 上的一个线性函数,记作 $\alpha_L$。

设域 $F$ 上 $n$ 维线性空间 $V$ 中取一个基 $\ji{\alpha}{n}$,设 $V$ 中向量 $\alpha,\beta$ 在此基下的坐标为
\[ X = \transpose{(\ji{x}{n})}, \quad Y = \transpose{(\ji{y}{n})} \]
设 $f$ 是 $V$ 上的一个双线性函数,则
\[ f(\alpha,\beta) = f\left( \sum_{i=1}^n x_i\alpha_i,\sum_{j=1}^ny_j\alpha_j \right) = \sum_{i=1}^n\sum_{j=1}^n x_iy_jf(\alpha_i,\alpha_j) \]
令
\[ A = \left(\begin{matrix}
            f(\alpha_1,\alpha_1) & f(\alpha_1,\alpha_2) & \cdots & f(\alpha_1,\alpha_n)  \\
            f(\alpha_2,\alpha_1) & f(\alpha_2,\alpha_2) & \cdots & f(\alpha_2,\alpha_n)  \\
            \vdots               & \vdots               &        & \vdots                \\
            f(\alpha_n,\alpha_1) & f(\alpha_n,\alpha_2) & \cdots & f(\alpha_n,\alpha_n)
        \end{matrix}\right) \]

称 $A$ 是双线性函数 $f$ 在基 $\ji{\alpha}{n}$ 下的度量矩阵,它是由 $f$ 及基 $\ji{\alpha}{n}$ 唯一决定的。于是有
\[ f(\alpha,\beta) = \transpose{X}AY \]
反之,任给域 $F$ 上一个 $n$ 级矩阵 $A = (a_{ij})$,定义映射 $f : V \times V \to F$ 如下
\[ f(\alpha,\beta) = \transpose{X}AY = \sum_{i=1}^n\sum_{j=1}^na_{ij}x_iy_j \]
则 $f$ 是 $V$ 上的一个双线性函数,且 $f$ 在基 $\ji{\alpha}{n}$ 下的度量矩阵为 $A$。

因此若 $\transpose{X}AY = \transpose{X}BY$,则 $A = B$。

称表达式 $\transpose{X}AY$ 为 $X$ 与 $Y$ 的双线性形。

\begin{theorem}
    设域 $F$ 上 $n$ 维线性空间 $V$ 上的一个双线性函数,取 $V$ 中两个基 $\ji{\alpha}{n}$ 与 $\ji{\beta}{n}$,设
    \[ (\ji{\beta}{n}) = (\ji{\alpha}{n})P \]
    且 $f$ 在这两个基的度量矩阵分别为 $A,B$,则
    \[ B = \transpose{P}AP \]
\end{theorem}

即 $f$ 在 $V$ 的不同基下的度量矩阵是合同的,他们有相同的秩,于是称度量矩阵的秩为 $f$ 的矩阵秩,记作 $\rank_m f$。

设域 $F$ 上线性空间 $V$ 上双线性函数 $f$,则称 $V^*$ 的子空间
\[ \langle \alpha_L,\beta_R \mid \alpha,\beta \in V \rangle \]
称为 $f$ 的秩空间,把 $f$ 的秩空间的维数称为 $f$ 的秩,记作 $\rank f$。

可以证明,$f$ 的矩阵秩不超过 $f$ 的秩。

\begin{definition}
    设域 $F$ 上线性空间 $V$ 上的双线性函数 $f$,则 $V$ 的子集
    \[ \{ \alpha \in V \mid f(\alpha,\beta) = 0 \} \]
    称为 $f$ 在 $V$ 中的做根,记作 $\rad_{L} V$,$V$ 的另一个子集
    \[ \{ \beta \in V \mid f(\alpha,\beta) = 0 \} \]
    称为 $f$ 在 $V$ 中的右根,记作 $\rad_{R} V$。
\end{definition}

容易验证,$f$ 在 $V$ 中的左根和右根都是 $V$ 的子空间。

\begin{definition}
    如果 $V$ 上双线性函数 $f$ 的做根和右根都是零子空间,那么称 $f$ 是非退化的。
\end{definition}

\begin{theorem}
    域 $F$ 上 $n$ 维线性空间 $V$ 上的双线性函数 $f$ 是非退化的,当且仅当 $f$ 在 $V$ 的一个基下的度量矩阵是满秩矩阵。
\end{theorem}

\begin{definition}
    设域 $F$ 上线性空间 $V$ 上的一个双线性函数,如果
    \[ f(\alpha,\beta) = f(\beta,\alpha) \]
    那么称 $f$ 是对称的,如果
    \[ f(\alpha,\beta) = -f(\beta,\alpha) \]
    那么称 $f$ 是反对称的(斜对称的)。
\end{definition}

\begin{theorem}
    设特征不为 $2$ 的域 $F$ 上 $n$ 维线性空间 $V$ 上的对称双线性函数 $f$,则 $V$ 中存在一个基使得 $f$ 在此基下的度量矩阵为对角矩阵,
\end{theorem}

\begin{theorem}
    设特征不为 $2$ 的域 $F$ 上 $n$ 维线性空间 $V$ 上的反对称双线性函数 $f$,则存在 $V$ 的一个基,把它记成 $\delta_{1},\delta_{-1},\cdots,\delta_{r},\delta_{-r},\eta_1,\cdots,\eta_s$(其中 $0 \leqslant 2r \leqslant n, s = n-2r$),使得 $f$ 在这个基下的度量矩阵具有形式
    \[ \diag\left\{
        \left(\begin{matrix} 0 & 1 \\ -1 & 0 \end{matrix}\right),
        \cdots,
        \left(\begin{matrix} 0 & 1 \\ -1 & 0 \end{matrix}\right),
        0,\cdots,0
        \right\} \]
\end{theorem}

\begin{definition}
    设域 $F$ 上的线性空间 $V$,映射 $q : V \to F$ 称为 $V$ 上的二次函数,如果存在 $V$ 上的一个对称双线性函数 $f$,使得
    \[ q(\alpha) = f(\alpha,\alpha) \]
\end{definition}

显然,对于一个对称双线性函数 $f$ 就有唯一的一个二次函数 $q$。

\begin{theorem}
    设特征不为 $2$ 的域 $F$ 上的线性空间 $V$,$q$ 是 $V$ 上的一个二次函数,则存在 $V$ 上唯一的对称双线性函数 $f$ 使得
    \[ f(\alpha,\alpha) = q(\alpha) \]
\end{theorem}

于是设域 $F$ 上 $n$ 维线性空间 $V$ 上的对称双线性函数 $f$ 和其对应的二次函数 $q$。设 $f$ 在 $V$ 的一个基 $\ji{\alpha}{n}$ 下的度量矩阵 $A = (a_{ij})$,则对于 $\alpha = (\ji{\alpha}{n})X,\beta = (\ji{\alpha}{n})Y$,有
\[ f(\alpha,\beta) = \transpose{X}AY \]
从而有
\[ q(\alpha) = f(\alpha,\alpha) = \transpose{X}AX \]
即 $q$ 在基 $\ji{\alpha}{n}$ 下的表达式是 $n$ 元二次型 $\transpose{X}AX$,称其中的对称矩阵 $A$ 为二次函数 $q$ 在基 $\ji{\alpha}{n}$ 下的矩阵。于是可以用二次型的理论研究双线性函数,也可以用对称双线性函数来研究二次型。

\begin{theorem}[惯性定理]
    实数域上任意一个 $n$ 元二次型都可以经过非退化线性替换化成规范形,并且规范形是唯一的 。
\end{theorem}

\begin{theorem}[Witt 消去律的推广]
    设特征不为 $2$ 的域 $F$ 上的 $n$ 级对称矩阵 $A_1,A_2$,$m$ 级对称矩阵 $B_1,B_2$。如果
    \[ \left(\begin{matrix}
                A_1 & 0   \\
                0   & B_1 \\
            \end{matrix}\right) \simeq \left(\begin{matrix}
                A_2 & 0   \\
                0   & B_2 \\
            \end{matrix}\right) \]
    且 $A_1 \simeq A_2$,那么 $B_1 \simeq B_2$。
\end{theorem}

\paragraph{双线性函数空间}

\begin{definition}
    设域 $F$ 上的线性空间 $V$,我们把 $V$ 上所有双线性函数组成的集合记作 $T_2(V)$,容易验证 $T_2(V)$ 对域函数的加法和纯量乘法成为域 $F$ 上的一个线性空间,称为 $V$ 上的双线性函数空间。
\end{definition}

\begin{theorem}
    设特征不为 $2$ 的域 $F$ 上的线性空间,则
    \[ T_2(V) = S_2(V) \oplus A_2(V) \]
\end{theorem}



\section{欧几里得空间}

\begin{definition}[正定的]\index{zhengdingde@正定的}
    设实线性空间 $V$ 上的对称双线性函数,如果对任意 $\alpha \in V$ 有 $f(\alpha,\alpha) \geqslant 0$,等号成立当且仅当 $\alpha = 0$,那么称 $f$ 是正定的。
\end{definition}

\begin{definition}[内积] \index{neiji@内积}
    设实数域 $\RR$ 上的一个线性空间 $V$,$V$ 上的一个正定的对称双线性函数称为 $V$ 上的一个内积。
\end{definition}

习惯上把内积 $f(\alpha,\beta)$ 记作 $(\alpha,\beta)$。

\begin{definition}
    设实数域 $\RR$ 上的一个线性空间 $V$,若给定了 $V$ 上的一个内积,那么称 $V$ 是一个实内积空间。有限维的实内积空间称为欧几里得 Euclid 空间,并且把线性空间 $V$ 的维数称为 Euclid 空间 $V$ 的维数。
\end{definition}

\paragraph{实内积空间中的度量概念}

\begin{definition}
    非负实数 $\sqrt{(\alpha,\alpha)}$ 称为向量 $\alpha$ 的长度,记作 $|\alpha|$。
\end{definition}

长度为 $1$ 的向量称为单位向量。如果 $\alpha \ne 0$,那么 $\mfrac{\alpha}{|\alpha|}$ 是一个单位向量。把 $\alpha$ 变成 $\mfrac{\alpha}{|\alpha|}$ 称为把 $\alpha$ 单位化。

\begin{theorem}[Cauchy - Schwarz 不等式]\index{cauchyschwarzbudengshi@Cauchy - Schwarz 不等式}
    在实内积空间 $V$ 对任意向量 $\alpha,\beta$ 有
    \[ |(\alpha,\beta)| \leqslant |\alpha| |\beta| \]
    等号成立当且仅当 $\alpha,\beta$ 线性相关。
\end{theorem}

\begin{proposition}[Cauchy 不等式]
    对于任意的两组实数 $\ji{\alpha}{n}$ 与  $\ji{\beta}{n}$ 有
    \[ |a_1b_1 + \cdots + a_nb_n| \leqslant \sqrt{a_1^2 + \cdots + a_n^2}\sqrt{b_1^2 + \cdots + b_n^2} \]
    等号成立当且仅当 $(\ji{\alpha}{n})$ 与 $(\ji{\beta}{n})$ 线性相关。
\end{proposition}

\begin{proposition}[Schwarz 不等式]
    对于任意的 $f,g \in C[a,b]$ 有
    \[ \left| \int_a^b f(x)g(x) \dd x \right|^2 \leqslant \left| \int_a^b f^2(x) \dd x \right| \left| \int_a^b g^2(x) \dd x \right| \]
    等号成立当且仅当 $(\ji{\alpha}{n})$ 与 $(\ji{\beta}{n})$ 线性相关。
\end{proposition}

\begin{definition}
    实内积空间 $V$ 中,两个非零向量 $\alpha$ 与 $\beta$ 的夹角 $\langle\alpha,\beta\rangle$ 规定为
    \[ \langle\alpha,\beta\rangle \coloneqq \arccos \frac{(\alpha,\beta)}{|\alpha||\beta|} \]
\end{definition}

\begin{definition}
    在实内积空间 $V$ 中,若 $(\alpha,\beta) = 0$,那么称 $\alpha$ 与 $\beta$ 正交,记作 $\alpha \bot \beta$。
\end{definition}

\begin{proposition}
    在实内积空间 $V$ 中,三角形不等式成立,即对于任意的 $\alpha,\beta \in V$ 有
    \[ |\alpha+\beta| \leqslant |\alpha| + |\beta| \]
\end{proposition}

\begin{proposition}
    在实内积空间 $V$ 中,勾股定理成立,即如果 $\alpha$ 与 $\beta$ 正交,则
    \[ |\alpha + \beta|^2 = |\alpha|^2 + |\beta|^2 \]
\end{proposition}

\begin{proposition}
    在实内积空间 $V$ 中,余弦定理成立,即对于三个非零向量 $\alpha,\beta,\gamma$ 满足 $\gamma = \beta - \alpha$,则
    \[ |\gamma|^2  = |\alpha|^2 + |\beta|^2 - 2 |\alpha| |\beta| \cos\langle\alpha,\beta\rangle \]
\end{proposition}

\begin{definition}
    设 $E$ 是一个非空集合,若其上存在映射 $d : E \times E \to \RR$,如果对任意 $x,y,z\in E$ 都有
    (1) 对称性:$d(x,y) = d(y,x)$。
    (2) 正定性:$d(x,y) \geqslant $0,等号成立当且仅当 $x = y$。
    (3) 三角形不等式:$d(x,z) \leqslant d(x,y) + d(y,z)$。
    那么称 $d$ 是一个距离,称集合 $E$ 是一个度量空间,把 $d(x,y)$ 称为 $x$ 与 $y$ 之间的距离。
\end{definition}

\begin{definition}
    在 $n$ 维 Euclid 空间 $V$ 中,由 $n$ 个两两正交的非零向量组成的基称为 $V$ 的一个正交基,若该基皆为单位向量,称为 $V$ 的一个标准正交基。
\end{definition}

由于内积是正定的对称双线性函数,因此 $V$ 中存在一个基 $\ji{\eta}{n}$ 使得内积在此基下的度量矩阵为单位矩阵 $I$。从而
\[ (\eta_i,\eta_j) = \delta_{ij} \]
因此 $\ji{\eta}{n}$ 是 $V$ 的一个标准正交基。

更具体的,取 $V$ 的一个基 $\ji{\alpha}{n}$ 令 $\beta_1 = \alpha$,且
\[ \beta_n = \alpha_n - \sum_{j=1}^{n-1} \frac{(\alpha_n,\beta_j)}{(\beta_j,\beta_j)} \beta \]
则 $\ji{\beta}{n}$ 是 $V$ 的一个正交基。令
\[ \eta_i = \frac{\beta}{|\beta|} \]
则 $\ji{\eta}{n}$ 是 $V$ 的一个标准正交基。

第一步称为 Schmitdt 正交化,第二步称为单位化。\index{schmitdtzhengjiaohua@Schmitdt 正交化}

\begin{theorem}
    设 $\ji{\eta}{n}$ 是 $n$ 维 Euclid 空间 $V$ 的一个标准正交基,则对于任意的 $\alpha \in V$ 有
    \[ \alpha = \sum_{i=1}^n (\alpha,\eta_i)\eta_i \]
\end{theorem}

即 $\alpha$ 在标准正交基 $\ji{\eta}{n}$ 下的坐标的第 $i$ 个分量等于 $(\alpha,\eta_i)$。此式称为 $\alpha$ 的 Fourier 展开,其中每个系数 $(\alpha,\eta_i)$ 都称为 $\alpha$ 的 Fourier 系数。

\begin{definition}
    设实内积空间 $V,V'$,若存在双射 $\sigma : V \to V'$ 使得对于任意的 $\alpha,\beta \in V$ 和 $k \in \RR$ 有
    \begin{equation*}
        \begin{aligned}
            \sigma(\alpha + \beta) & = \sigma(\alpha) + \sigma(\beta)  \\
            \sigma(k\alpha)        & = k\sigma(\alpha)                 \\
            (\alpha,\beta)         & = (\sigma(\alpha),\sigma(\beta))
        \end{aligned}
    \end{equation*}
    那么称 $\sigma$ 是 $V \to V'$ 的一个同构映射,此时称 $V$ 与 $V'$ 同构,记作 $V \cong V'$。
\end{definition}

若 $\sigma$ 保持加法和数量乘法,那么称为线性同构;若线性同构还保持内积,则称它为一个保距同构。

\begin{theorem}
    两个 Euclid 空间同构的充要条件是它们的维数相同。
\end{theorem}

\section{正交补,正交投影}

\begin{definition}
    设实内积空间 $V$ 的一个非空子集 $S$。把 $V$ 中与 $S$ 中每一个向量都正交的所有向量的全体称作 $S$ 的正交补,记作 $S^\bot$。即
    \[ S^\bot \coloneqq \{ \alpha \in V \mid (\alpha,\beta) = 0 \} \]
\end{definition}

\begin{theorem}
    设实内积空间 $V$ 的一个有限维子空间 $U$,则
    \[ V = U \oplus U^\bot \]
\end{theorem}

设实内积空间 $V$ 的一个子空间,若 $V = U \oplus U^\bot$,那么有平行于 $U^\bot$ 在 $U$ 上的投影 $\bfP_U$。我们把这个投影 $\bfP_U$ 称作 $V$ 在 $U$ 上的正交投影;把  $\alpha$ 在 $\bfP_U$ 下的像 $\alpha_1$ 称为 $\alpha$ 在 $U$ 上的正交投影。此时 $\alpha = \alpha_1 + \alpha_2$。

即
\[ \bfP_U(\alpha) = \alpha_1 \Leftrightarrow \alpha - \alpha_1 \in U^\bot \]

\begin{theorem}
    设实内积空间 $V$ 的一个子空间 $U$,且 $V = U \oplus U^\bot$,则对于 $\alpha \in V$,则 $\alpha_1 = \bfP_U(\alpha) \in U$ 的充要条件是
    \[ d(\alpha,\alpha_1) \leqslant d(\alpha,\gamma) \]
\end{theorem}

\begin{definition}\index{zuijiabijinyuan@最佳逼近元}
    设实内积空间 $V$ 的子空间 $U$,若对于 $\alpha \in V$ 存在 $\delta \in U$ 有
    \[ d(\alpha,\delta) \leqslant d(\alpha,\gamma) \]
    那么称 $\delta$ 是 $\alpha$ 在 $U$ 上的最佳逼近元。
\end{definition}

设实内积空间 $V$ 的一个无限维子空间 $U$,如果 $\alpha \in V$ 在 $U$ 上的最佳逼近元存在(此时必唯一),那么称 $\delta$ 为 $\alpha$ 在 $U$ 上的正交投影。如果 $V$ 中每个向量 $\alpha$ 都有在 $U$ 上的正交投影 $\delta$,那么把 $\alpha$ 对应到 $\delta$ 的映射称为 $V$ 在 $U$ 上的正交投影。

\section{正交变换与对称变换}

\begin{definition}
    设实内积空间 $V$ 到自身的满射 $\bfA$,如果保持向量的内积不变,即
    \[ (\bfA\alpha,\bfA\beta) = (\alpha,\beta) \]
    那么称 $\bfA$ 是 $V$ 上的一个正交变换。
\end{definition}

\begin{proposition}
    正交变换 $\bfA$ 具有特性:
    
    (1) 保持向量长度。
    
    (2) 保持两个非零向量的夹角不变。
    
    (3) 保持正交性。
    
    (4) 一定是线性变换。
    
    (5) 保持向量间的距离不变。
    
    (6) 一定是单射,一定可逆。
\end{proposition}

\begin{theorem}
    设 $n$ 为 Euclid 空间 $V$ 上的线性变换 $\bfA$,$\bfA$ 在 $V$ 的标准正交基下的矩阵为 $A$,则下列描述等价:
    
    (1) $\bfA$ 是正交变换。
    
    (2) $\bfA$ 把 $V$ 的标准正交基映成标准正交基。
    
    (3) $A$ 是正交矩阵。
\end{theorem}

由于正交矩阵 $A$ 的行列式为 $\pm 1$,则把行列式为 $1$ 的矩阵称为第一类(或旋转),行列式等于 $-1$ 的称为第二类的。

$n$ 维线性空间的任意一个 $n-1$ 维子空间称为一个超平面。

\begin{definition}
    设 $n$ 维 Euclid 空间 $V$ 中的一个单位向量,$\bfP$ 是 $V$ 在 $\langle \eta \rangle$ 上的正交投影,令
    \[ \bfA = \bfI - 2 \bfP \]
    则 $\bfA$ 称为关于超平面 $\langle \eta \rangle^\bot$ 的镜面反射。
\end{definition}

\begin{definition}
    实内积空间 $V$ 上的变换 $\bfA$ 如果满足
    \[ (\bfA\alpha,\beta) = (\alpha,\bfA\beta) \]
    那么称 $\bfA$ 是 $V$ 上的对称变换。
\end{definition}

\section{*酉空间}

在复数域中引入度量概念。若复线性空间 $V$ 上的双线性函数 $f$ 具有性质
\[ f(\alpha,\beta) = \overline{f(\beta,\alpha)} \]
这个性质称为 Hermite 性。

\section{*正交空间与辛空间}

\begin{definition}
    设域 $F$ 上的线性空间 $V$ 上的一个对称双线性函数 $f$,那么称 $f$ 是 $V$ 上的一个内积(或度量),称 $V$ 是一个正交空间。用 $(V,f)$ 表示。
\end{definition}

如果 $f$ 是非退化的,则称 $(V,f)$ 是正则的,否则称为非正则的。

\begin{definition}
    在正交空间 $(V,f)$ 中,如果 $f(\alpha,\beta) = 0$,那么称 $\alpha$ 与 $\beta$ 正交,记作 $\alpha \bot \beta$。
\end{definition}

在正交空间 $(V,f)$ 中,一个非零向量 $\alpha$ 称为迷向的,如果 $f(\alpha,\alpha) = 0$,否则称为非迷向的。

正交空间 $(V,f)$ 包含一个迷向向量,则 $(V,f)$ 称为迷向的,否则为非迷向的。若 $V$ 中所有非零向量都是迷向的,则称为全迷向的。

\section{*正交群,酉群,辛群}

\begin{definition}
    设 $G$ 是一个非空集合,若 $G$ 上的乘法运算满足
    
    (1) 结合律:$a(bc) = (ab)c$。
    
    (2) 单位元:存在 $e \in G$ 使得 $ea = ae = a$。
    
    (3) 逆元:任取 $a \in G$ 总存在 $b \in G$ 使得 $ab = ba = 1$。
    
    那么称 $G$ 是一个群。
\end{definition}

如果群 $G$ 的运算还满足交换律,那么称 $G$ 为交换群,或 Abel 群。











 % 具有度量的线性空间

\backmatter

\printindex


\end{document}