\chapter{线性方程组的解系}

\section{\texorpdfstring{$n$ 维向量空间 $K^n$}{n 维向量空间 Kn}}

取定一个数域 $K$,设 $n$ 是任意给定的一个正整数。令
\[K^n=\{(\seq{a}{n}) \mid a_i\in K,i=1,\cdots,n\}\]
如果 $a_1=b_1,\cdots,a_n=b_n$,则称 $K^n$ 中的两个元素:$(\seq{a}{n}),(\seq{b}{n})$ 相等。

在 $K^n$ 中规定加法运算:
\[(\seq{a}{n})+(\seq{b}{n})\coloneqq (a_1+b_1,\cdots,a_n+b_n)\]
在 $K$ 的元素与 $K^n$ 的元素之间规定数量乘法运算:
\[k(\seq{a}{n}) \coloneqq  (\seq{ka}{n})\]
不难验证加法和数量乘法运算满足下述八条运算法则:对于 $\vbf{\alpha},\vbf{\beta},\vbf{\gamma}\in K^n,k,l\in K$ 有

\begin{enumerate}
	\item $\vbf{\alpha}+\vbf{\beta}=\vbf{\beta}+\vbf{\alpha}$
	\item $(\vbf{\alpha}+\vbf{\beta})+\vbf{\gamma}=\vbf{\alpha}+(\vbf{\beta}+\vbf{\gamma})$
	\item 把元素 $(0,\cdots,0)$ 记作零元素 $\vbf{0}$,使得
	      \[\vbf{0} + \vbf{\alpha} = \vbf{\alpha} + \vbf{0} = \vbf{\alpha}\]
	\item 对于 $\vbf{\alpha} = (\seq{a}{n})\in K^n$,定义其负元素
	      \[-\vbf{\alpha} \coloneqq  (\seq{-a}{n})\]
	      于是有
	      \[\vbf{\alpha} + (-\vbf{\alpha}) = (-\vbf{\alpha})+\vbf{\alpha} = 0\]
	\item $1\vbf{\alpha} = \vbf{\alpha}$
	\item $(kl)\vbf{\alpha} = k(l\vbf{\alpha})$
	\item $k (\vbf{\alpha}+\vbf{\beta}) = k\vbf{\alpha} + k\vbf{\beta}$
\end{enumerate}

\begin{definition}[$n$ 维向量空间]
	数域 $K$ 上所有 $n$ 元有序数组组成的集合 $K^n$,连同定义在它上面的加法运算和数量乘法运算,及其满足的 8 条运算法则一起,称为数域 $K$ 上的一个 $n$ 维向量空间。$K^n$ 的元素称为 $n$ 维向量;设向量 $\vbf{\alpha}  = (\seq{a}{n})$,称 $a_i$ 是 $\vbf{\alpha}$ 的第 $i$ 个分量。
\end{definition}

在 $n$ 维向量空间 $K^n$ 中,可以定义减法运算
\[\vbf{\alpha} - \vbf{\beta} \coloneqq  \vbf{\alpha} + (-\vbf{\beta})\]

$n$ 元有序数组写成一行,称为行向量;写成一列,称为列向量,也可以看作行向量的转置。

$K^n$ 可以看成是 $n$ 维行向量组成的向量空间,也可以看作是列向量组成的向量空间。

\begin{definition}[线性组合]
	给定向量组 $\seq{\vbf{\alpha}}{s}$,再任给 $K$ 中的一组数 $\seq{k}{s}$,那么向量
	\[k_1 \vbf{\alpha}_1+\cdots+k_s \vbf{\alpha}_s\]
	称为向量组 $\seq{k}{s}$ 的一个线性组合,其中 $\seq{k}{s}$ 称为系数。
\end{definition}

\begin{definition}[线性表出]
	给定向量组 $\seq{\vbf{\alpha}}{s}$,对于 $\vbf{\beta} \in K^n$,若存在 $K$ 中的一组数 $\seq{k}{s}$ 满足
	\[\vbf{\beta} = k_1\vbf{\alpha}_1+\cdots+k_s\vbf{\alpha}_s\]
	那么称 $\vbf{\beta}$ 可以由向量组 $\seq{\vbf{\alpha}}{s}$ 线性表出。
\end{definition}

于是可以把数域 $K$ 上的 $n$ 元线性方程组
\[
	\left\{
	\begin{matrix}
		a_{11}x_1+a_{12}x_2+\cdots+a_{1n}x_n=b_1 \\
		a_{21}x_1+a_{22}x_2+\cdots+a_{2n}x_n=b_2 \\
		\cdots \qquad \cdots \qquad \cdots       \\
		a_{n1}x_1+a_{n2}x_2+\cdots +a_{nn}x_n=b_n
	\end{matrix}
	\right.
\]
写成
\[x_1\left(\begin{matrix}
			a_{11} \\ a_{21} \\ \cdots \\ a_{n1}
		\end{matrix}\right)+x_2\left(\begin{matrix}
			a_{12} \\ a_{22} \\ \cdots \\ a_{n2}
		\end{matrix}\right)+\cdots+x_n\left(\begin{matrix}
			a_{1n} \\ a_{2n} \\ \cdots \\ a_{nn}
		\end{matrix}\right)=\left(\begin{matrix}
			b_{1} \\ b_{2} \\ \cdots \\ b_{n}
		\end{matrix}\right)\]
或者
\[ x_1\vbf{\alpha}_1+\cdots+x_n\vbf{\alpha}_n=\vbf{\beta} \]
其中 $\seq{\vbf{\alpha}}{n}$ 是线性方程组的列向量组,$\vbf{\beta}$ 是由常数项组成的列向量。因此方程组有解等价于 $\vbf{\beta}$ 可以被 $\seq{\vbf{\alpha}}{n}$ 线性表出。

\begin{definition}[线性子空间]
	$K^n$ 的一个非空子集 $U$ 是 $K^n$ 的一个线性子空间,那么满足

	(1) $U$ 对于 $K^n$ 的加法封闭:$\vbf{\alpha},\vbf{\gamma}\in U \Rightarrow \vbf{\alpha}+\vbf{\gamma} \in U$

	(2) $U$ 对于 $K^n$ 的乘法封闭:$\vbf{\alpha} \in U,k\in K \Rightarrow k\vbf{\alpha} \in U$
\end{definition}

特殊的,$\{0\}$ 也是 $K^n$ 的一个子空间,称为零子空间。$K^n$ 本身也是 $K^n$ 的一个子空间。

$\seq{\vbf{\alpha}}{n}$ 的所有线性组合也是 $K^n$ 的一个子空间,称为 $\seq{\vbf{\alpha}}{n}$ 生成(张成)的子空间,记作
\[\langle \seq{\vbf{\alpha}}{n}\rangle\coloneqq \{k_1\vbf{\alpha}_1+\cdots+k_s\vbf{\alpha}_s \mid k_i\in K,i=1,\cdots,s\}\]

于是线性方程组有解,等价与 $\vbf{\beta}$ 可以由 $\seq{\vbf{\alpha}}{n}$ 线性表出,即 $\vbf{\beta} \in \langle\seq{\vbf{\alpha}}{n}\rangle$。

\section{线性相关与无关}

\begin{definition}
	$K^n$ 中向量组 $\seq{\vbf{\alpha}}{s}$ 称为是线性相关的,如果有 $K$ 中不全为 $0$ 的数 $\seq{k}{s}$,使得
	\[k_1\vbf{\alpha}_1+\cdots+k_s\vbf{\alpha}_s=\vbf{0}\]
	否则称为线性无关。
\end{definition}

即线性无关意味着所有的系数只能都为 $0$。注意线性相关不意味着每个向量都可以由其他向量线性表出,该向量前的系数 $k$ 可以为 $0$。

\section{向量组的秩}

\begin{definition}[极大线性无关组]\index{jidaxianxingwuguanzu@极大线性无关组}
	向量组的一个部分组称为一个极大线性无关组,如果这个部分组本身是线性无关的,但是从这个向量组的其余向量(如果还有的话)中任取一个添进去,得到的新的部分组都线性相关。
\end{definition}

如果向量组 $\seq{\vbf{\alpha}}{s}$ 的每一个向量都可以由向量组 $\seq{\vbf{\beta}}{r}$ 线性表出,那么称向量组 $\seq{\vbf{\alpha}}{s}$ 可以由向量组线性表出。

\begin{definition}
	如果向量组 $\seq{\vbf{\alpha}}{s}$ 与向量组 $\seq{\vbf{\beta}}{r}$ 可以互相线性表出,那么称两个向量组等价,记作
	\[\{\seq{\vbf{\alpha}}{s}\} \cong \{\seq{\vbf{\beta}}{r}\}\]
\end{definition}

可以证明,这种关系具有三条性质(反身性,对称性,传递性),即是等价关系。

对矩阵作初等行变换,变换前后的行向量组等价,不保证列向量组等价。

那么向量组与它的极大线性无关组等价。

\begin{definition}
	向量组 $\seq{\vbf{\alpha}}{r}$ 的极大线性无关组所含向量的个数称为这个向量组的秩,记作
	\[\rank\{\seq{\vbf{\alpha}}{r}\}\]
\end{definition}

\section{子空间的基与维数}

\begin{definition}[子空间]
	设 $U$ 是 $K^n$ 的一个子空间,如果 $\seq{\vbf{\alpha}}{r}\in U$ 是 $U$ 的一个基,那么

	(1) $\seq{\vbf{\alpha}}{r}$ 线性无关。

	(2) $U$ 中每一个向量都可以由 $\seq{\vbf{\alpha}}{r}$ 线性表出。
\end{definition}

显然,单位向量组 $\seq{\eps}{n}$ 是 $K^n$ 的一个基,称作标准基。

\begin{theorem}
	$K^n$ 的任一非零子空间 $U$ 都有一个基。
\end{theorem}

\begin{theorem}
	$K^n$ 的任一非零子空间 $U$ 的任一两个基所含向量的个数相等,称为 $U$ 的维数,记作 $\dim_KU$ 或 $\dim U$。
\end{theorem}

\begin{theorem}
	向量组 $\seq{\vbf{\alpha}}{s}$ 的一个极大线性无关组是这个向量组生成的子空间的 $\langle \seq{\vbf{\alpha}}{s} \rangle$,从而
	\[\dim\langle \seq{\vbf{\alpha}}{s} \rangle = \rank\{\seq{\vbf{\alpha}}{s}\}\]
\end{theorem}

\section{矩阵的秩}

\begin{theorem}
	阶梯形矩阵 $J$ 的行秩与列秩相等,它们都等于 $J$ 的非零行的个数;并且 $J$ 的主元所在的列构成列向量的一个极大线性无关组。
\end{theorem}

\begin{theorem}
	矩阵的初等行变换不改变矩阵的行秩和列秩。
\end{theorem}

\begin{theorem}
	矩阵的行秩和列秩相等,统称为矩阵的秩。矩阵 $A$ 的秩记作 $\rank(A)$。
\end{theorem}

\begin{theorem}
	非零矩阵的秩等于它的不为零的子式的阶数。
\end{theorem}

若一个 $n$ 级矩阵的秩如果等于它的级数,那么称为满秩矩阵。

\begin{theorem}
	设 $A$ 是 $s \times n$ 矩阵,$B$ 是 $l \times m$ 矩阵,则
	\[ \rank\left(\begin{matrix}
				A & 0 \\ 0  & B
			\end{matrix}\right) = \rank(A) + \rank(B) \]
\end{theorem}

\begin{theorem}
	设 $A$ 是 $s \times n$ 矩阵,$B$ 是 $l \times m$ 矩阵,$C$ 是 $s \times m$ 矩阵,则
	\[ \rank\left(\begin{matrix}
				A & C \\ 0  & B
			\end{matrix}\right) \geqslant \rank(A) + \rank(B) \]
\end{theorem}

\section{线性方程组有解的充分必要条件}

\begin{theorem}
	数域 $K$ 上有线性方程组
	\[x_1\vbf{\alpha}_1 + \cdots + x_n\vbf{\alpha}_n = \vbf{\beta}\]
	有解的充分必要条件是:它的系数矩阵与增广矩阵的秩相等。
\end{theorem}

\begin{theorem}
	数域 $K$ 上 $n$ 元线性方程组有解时,如果它的系数矩阵满秩,那么方程组有唯一解;否则方程组有无穷多个解。
\end{theorem}

\section{齐次线性方程组解集的结构}

数域 $K$ 上 $n$ 元齐次线性方程组
\[x_1\vbf{\alpha}_1 + \cdots + x_n\vbf{\alpha}_n = 0\]
的一个解是 $K^n$ 中的一个向量,称它为齐次线性方程组的一个解向量。

可知齐次线性方程组的解集 $W$ 是 $K^n$ 的一个子空间,称为方程组的一个解空间。

\begin{definition}
	齐次线性方程组有非零解时,如果它的有限多个解 $\seq{\eta}{t}$ 是其基础解系

	(1) $\seq{\eta}{t}$ 线性无关。

	(2) 齐次线性方程组的每一个解都可以由 $\seq{\eta}{t}$ 线性表出。
\end{definition}

于是解空间为
\[W = \langle \seq{\eta}{t} \rangle\]

\begin{theorem}
	数域 $K$ 上 $n$ 元齐次线性方程组的解空间 $W$ 的维数为
	\[\dim W = n-\rank(A)\]
\end{theorem}

\begin{proof}
	倘若 $\rank(A) = n$,即只有零解,那么 $W = \{\vbf{0}\}$,此时显然成立。

	下面设 $\rank(A) = r < n$。第一步,将 $A$ 变化为简化行阶梯形矩阵,不妨设主元在前 $r$ 列,即得 $x_1, \cdots, x_r$ 关于自由变量 $x_{r+1}, \cdots, x_n$ 的线性关系。

	第二步,取 $n-r$ 个线性无关向量(最简单就是逐个取 $1$),带入 $x_{r+1}, \cdots, x_n$,得到方程组的 $n-r$ 个解 $\seq{\vbf{\eta}}{{n-r}}$,显然这些解线性无关。

	下证这些解是解空间的基。对于任何一个解 $\vbf{\eta}$,显然可以关于 $x_{r+1}, \cdots, x_n$ 的 $n-r$ 个线性无关向量分解,从而总是可以由 $\seq{\vbf{\eta}}{r}$ 线性表示。故
	\[ \dim W = n - \rank(A) \]
\end{proof}

\begin{example}
	求解
	\[ \left\{
		\begin{aligned}
			x_1    & + 3x_2 & - 5x_3 & - 2x_4 & = 0 \\
			-3x_1  & - 2x_2 & + x_3  & + x_4  & = 0 \\
			-11x_1 & - 5x_2 & - x_3  & + 2x_4 & = 0 \\
			5x_1   & + x_2  & +x_3   &        & = 0 \\
		\end{aligned} \right. \]
\end{example}

第一步,化成简化阶梯型行列式,求出一般解
\[ \left\{
	\begin{aligned}
		x_1 & = -x_3 - \frac{1}{7}x_4 \\
		x_2 & = 2x_3 + \frac{5}{7}x_4 \\
	\end{aligned} \right. \]
第二步,其中 $x_3, x_4$ 是自由变量,得到基础解系
\[ \vbf{\eta}_1 = \left(\begin{matrix}
			-1 \\ 2 \\ 1 \\ 0
		\end{matrix}\right), \quad \vbf{\eta}_2 = \left(\begin{matrix}
			-1 \\ 5 \\ 0 \\ 7
		\end{matrix}\right) \]
因此方程组的解集为
\[ W = \{ k_1 \vbf{\eta}_1 + k_2 \vbf{\eta}_2 \} \]

\section{非齐次线性方程组的结构}

称数域 $K$ 上 $n$ 元齐次线性方程组
\[x_1\vbf{\alpha}_1 + \cdots + x_n\vbf{\alpha}_n = \vbf{\beta}\]
的导出组为
\[x_1\vbf{\alpha}_1 + \cdots + x_n\vbf{\alpha}_n = 0\]
其的解空间用 $W$ 表示。

\begin{theorem}
	如果数域 $K$ 上 $n$ 元齐次线性方程组有解,那么它的解集 $U$ 为
	\[U = \{\vbf{\gamma}_0 + \vbf{\eta} \mid \vbf{\eta} \in W\}\]
	其中 $\vbf{\gamma}_0$ 是非其次线性方程组的一个特解,$W$ 是导出组的解空间用。
\end{theorem}

\begin{example}
	求解
	\[ \left\{
		\begin{aligned}
			x_1   & + 2x_2 & - 3x_3 & - 4x_4 & = -5 \\
			3x_1  & - x_2  & + 5x_3 & + 6x_4 & = -1 \\
			-5x_1 & - 3x_2 & + x_3  & + 2x_4 & = 11 \\
			-9x_1 & - 4x_2 & - x_3  &        & = 17 \\
		\end{aligned} \right. \]
\end{example}

第一步,化成简化阶梯型行列式,求出一般解
\[ \left\{
	\begin{aligned}
		x_1 & = -x_3 - \frac{8}{7}x_4 - 1  \\
		x_2 & = 2x_3 + \frac{18}{7}x_4 - 2 \\
	\end{aligned} \right. \]
第二步,其中 $x_3, x_4$ 是自由变量,得到特解和基础解系
\[ \vbf{\gamma}_0 = \left(\begin{matrix}
			-1 \\ -2 \\ 0 \\ 0
		\end{matrix}\right), \quad \vbf{\eta}_1 = \left(\begin{matrix}
			-1 \\ 2 \\ 1 \\ 0
		\end{matrix}\right), \quad \vbf{\eta}_2 = \left(\begin{matrix}
			-1 \\ 5 \\ 0 \\ 7
		\end{matrix}\right) \]
因此方程组的解集为
\[ W = \{ \vbf{\gamma}_0 + k_1 \vbf{\eta}_1 + k_2 \vbf{\eta}_2 \} \]

