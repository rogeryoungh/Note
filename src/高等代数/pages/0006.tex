\chapter{二次型 · 矩阵的合同}

\newcommand{\XAX}{\transpose{\vbf{X}}A\vbf{X}}

\section{二次型及其标准型}

\begin{definition}[二次型]
	数域 $K$ 上的一个 $n$ 元二次型是系数在 $K$ 中的 $n$ 个变量的齐次多项式,它的一般形式是
	\[f(\seq{x}{n}) = \sum_{i=1}^n\sum_{j=1}^na_{ij}x_ix_j\]
	其中 $a_{ij} = a_{ji}$。
\end{definition}

把二次型的系数按原来顺序排成一个 $n$ 级矩阵 $A$,则称 $A$ 是二次型 $f(\seq{x}{n})$ 的矩阵,它是对称矩阵。

再令 $\vbf{X} = \transpose{(\seq{x}{n})}$,则二次型可以写作
\[f(\seq{x}{n}) = \XAX\]

令 $\vbf{Y} = \transpose{(\seq{y}{n})}$,设 $C$ 是数域 $K$ 上的 $n$ 级可逆矩阵,则关系式
\[\vbf{X} = C\vbf{Y}\]
称为变量 $\seq{x}{n}$ 到变量 $\seq{y}{n}$ 的一个非退化线性替换。如果 $C$ 是正交矩阵,那么变量的替换 $\vbf{X} = C\vbf{Y}$ 称为正交替换。

\begin{definition}
	数域上两个 $n$ 元二次型 $\XAX$ 与 $\transpose{\vbf{Y}}A\vbf{Y}$,如果存在一个非退化线性替换 $\vbf{X} = C\vbf{Y}$,把 $\XAX$ 变成 $\transpose{\vbf{Y}}B\vbf{Y}$ 那么称二次型 $\XAX$ 与 $\transpose{\vbf{Y}}B\vbf{Y}$ 等价,记作 $\XAX \cong \transpose{\vbf{Y}}B\vbf{Y}$。
\end{definition}

\begin{definition}
	数域 $K$ 上两个 $n$ 级矩阵 $A$ 与 $B$,如果存在 $K$ 上的一个 $n$ 级可逆矩阵 $C$,使得
	\[\transpose{C}AC = B\]
	那么称 $A$ 与 $B$ 合同,记作 $A\simeq B$。
\end{definition}

带入 $\XAX$ 可以发现
\[ \XAX = \transpose{(C\vbf{Y})} A (C\vbf{Y}) = \transpose{\vbf{Y}} (\transpose{C} A C) \vbf{Y} \]
即二次型 $\XAX$ 和 $\transpose{\vbf{Y}}A\vbf{Y}$ 等价当且仅当 $A, B$ 合同。

如果二次型 $\transpose{\vbf{X}} A \vbf{X}$ 等价于一个只含平方项的二次型,那么这个只含平方项的二次型称为 $\XAX$ 的一个标准形。如果 $A$ 合同于一个对角矩阵,则这个对角矩阵称为 $A$ 的一个合同标准型。

对于 $n$ 级实对称矩阵 $A$,存在 $n$ 级的正交矩阵 $T$ 使得
\[ T^{-1}AT = \diag\{\seq{\lambda}{n}\} \]
其中 $\seq{\lambda}{n}$ 是 $A$ 的全部特征值,因此 $A$ 合同于 $T^{-1}AT$。从而在 $\vbf{X} = T\vbf{Y}$ 的替换下,得到一个标准型
\[ \lambda_1y_1^2 + \lambda_2y_2^2 + \cdots + \lambda_ny_n^2 \]

\begin{theorem}
	数域 $K$ 上的任一对称矩阵都合同于一个对角矩阵。
\end{theorem}

这意味着数域 $K$ 上任一 $n$ 元二次型都等价于一个只含平方项的二次型。有一个新方法来求得基本型:对于 $\XAX$,对 $A$ 做初等行列变换,仅对 $E$ 做其中的初等列变换
\[ \left(\begin{matrix}
			A \\ E
		\end{matrix}\right) \to \left(\begin{matrix}
			D \\ C
		\end{matrix}\right) \]
得到对角矩阵 $D = \diag{\seq{d}{n}}$,则
\[ \transpose{C}AC = D \]
令 $\vbf{X} = C \vbf{Y}$,则得到 $\XAX$ 的一个标准型。

二次型 $\XAX$ 的矩阵 $A$ 的秩就称为二次型 $\XAX$ 的秩。

\begin{example}
	用正交替换把下述实二次型化为标准型
	\[ f(x, y, z) = x^2 + 2y^2 + 3z^3 - 4xy - 4yz \]
\end{example}

\begin{solution}
	首先得到矩阵
	\[ A = \left(\begin{matrix}
				1 & -2 & 0 \\ -2 & 2 & -2 \\ 0 & -2 & 3
			\end{matrix}\right) \]
	计算其特征值
	\[ |\lambda E -A| = (\lambda - 2)(\lambda - 5)(\lambda + 1) \]
	即 $A$ 的全部特征值为 $\lambda = 2, 5, -1$,对应的基础解系单位化得到
	\[ \vbf{\eta}_1 = \transpose{\left(-\frac{2}{3}, \frac{1}{3}, \frac{2}{3}\right)}, \vbf{\eta}_2 = \transpose{\left(\frac{1}{3}, -\frac{2}{3}, \frac{2}{3}\right)}, \vbf{\eta}_3 = \transpose{\left(\frac{2}{3}, \frac{2}{3}, \frac{1}{3}\right)}  \]
	令
	\[ T = \left(\begin{matrix}
				-\frac{2}{3} & \frac{1}{3}  & \frac{2}{3} \\
				\frac{1}{3}  & -\frac{2}{3} & \frac{2}{3}        \\
				\frac{2}{3}  & \frac{2}{3}  & \frac{1}{3}
			\end{matrix}\right) \]
	此处 $T$ 即是正交矩阵,且 $T^{-1}AT = \diag\{2, 5, -1\}$。令
	\[ \transpose{\left(x, y, z\right)} = T \transpose{\left(a, b, c\right)} \]
	则
	\[ f(x, y, z) = 2a^2 + 5y^2 - z^2 \]
\end{solution}

\section{实二次型的规范形}

实数域上的二次型简称为实二次型,$n$ 元实二次型 $\XAX$ 经过一个适当的非退化线性替换 $\vbf{X} = C\vbf{Y}$ 可以化成下述形式的标准形
\[d_1y_1^2+\cdots+d_py_p^2-d_{p+1}y_{p+1}^2-\cdots-d_ry_r^2\]
其中 $d_i>0,i=1,\cdots,r$。再做一次非退化线性替换可以变成
\[z_1^2+\cdots+z_p^2-z_{p+1}^2-\cdots-z_r^2\]

\begin{theorem}
	$n$ 元实二次型 $\XAX$ 的规范形是唯一的。
\end{theorem}

\begin{definition}
	在实二次型 $\XAX$ 的规范形中,系数为 $+1$ 的平方项个数 $p$ 称为 $\XAX$ 的正惯性指数,系数为 $-1$ 的平方项个数 $r-p$ 称为 $\XAX$ 的负惯性指数;正惯性指数减去负惯性指数所得的差 $2p-r$ 称为 $\XAX$ 的符号差。
\end{definition}

任一 $n$ 级实对称矩阵合同于对角矩阵 $\diag\{1,\cdots,1,-1,\cdots,-1,0,\cdots,0\}$,其中 $0$ 的个数等于 $\XAX$ 的正惯性指数,$-1$ 的个数等于 $\XAX$ 的负惯性指数(也分别称作 $A$ 的惯性指数),这个对角举着称为 $A$ 的合同规范形。

现讨论复数域上的二次型,简称为复二次型。设 $n$ 元复二次型 $\XAX$ 经过一个适当的非退化线性替换 $\vbf{X} = C\vbf{Y}$ 变成下述形式的标准形
\[d_1y_1^2+\cdots+d_ry_r^2\]
其中 $d_i\ne 0,i=1,\cdots,n$,$r$ 是这个二次型的秩。再做一个非退化线性替换可得
\[z_1^2+\cdots+z_r^2\]
把这个标准形叫做复二次型 $\XAX$ 的规范形,显然其完全由其秩决定,故只有一种形式。

\section{正定二次型与正定矩阵}

\begin{definition}
	$n$ 元实二次型 $\XAX$ 称为正定的,如果对于 $\mathbb{R}^n$ 中任一非零列向量 $\alpha$,都由 $\transpose{\alpha}A\alpha>0$。
\end{definition}

\begin{theorem}
	$n$ 元实二次型 $\XAX$ 是正定的当且仅当它的正惯性系数等于 $n$。
\end{theorem}

\begin{definition}
	实对称矩阵 $A$ 称为正定的,如果实二次型 $\XAX$ 是正定的 。
\end{definition}

正定的实对称矩阵简称为正定矩阵。

\begin{theorem}
	实对称矩阵 $A$ 是正定的充分必要条件是 $A$ 的说有顺序主子式全大于 $0$。
\end{theorem}

\begin{definition}
	实对称矩阵 $A$ 称为半正定(负定,半负定,不定)的,如果实二次型对于 $\mathbb{R}^n$ 中任一非零列向量 $\alpha$,都有
	\[\transpose{\alpha}A\alpha \geqslant 0 \quad (\transpose{\alpha}A\alpha<0,\transpose{\alpha}A\alpha\leqslant 0)\]
	如果 $\XAX$ 既不是半正定的,又不是半负定的,那么称它是不定的。
\end{definition}

\begin{definition}
	实对称矩阵 $A$ 称为半正定(负定,半负定,不定)的,如果实二次型 $\XAX$ 是半正定(负定,半负定,不定)的。
\end{definition}

\begin{theorem}
	$n$ 级实对称矩阵 $A$ 是半正定的,当且仅当 $A$ 的所有主子式全非负。
\end{theorem}

\begin{theorem}
	实对称矩阵 $A$ 负定的充分必要条件是:它的奇数阶顺序主子式全小于 $0$,偶数阶顺序主子式全大于 $0$。
\end{theorem}

\begin{theorem}
	设二元实值函数 $F(x,y)$ 有一个稳定点 $\alpha=(x_0,y_0)$ (即 $F(x,y)$ 在 $(x_0,y_0)$ 处的一阶偏导数全为 $0$)。设 $F(x,y)$ 在 $(x_0,y_0)$ 的一个邻域内有 3 阶连续偏导数。令
	\[H = \left(\begin{matrix}
				F_{xx}''(x_0,y_0) & F_{xy}''(x_0,y_0)  \\
				F_{yx}''(x_0,y_0) & F_{yy}''(x_0,y_0)
			\end{matrix}\right)\]
	称 $H$ 是 $F(x,y)$ 在 $(x_0,y_0)$ 处的黑塞(Hesse)矩阵。如果 $H$ 是正定的,那么 $F(x,y)$ 在 $(x_0,y_0)$ 处达到极小值。如果 $H$ 是负定的,那么 $F(x,y)$ 在 $(x_0,y_0)$ 处达到极大值。
\end{theorem}

其可推广到 $n$ 元函数的情形:设 $F(\seq{x}{n})$ 有一个稳定点 $\alpha = (\seq{a}{n})$,设 $F(\seq{x}{n})$ 在 $\alpha$ 的一个邻域内有 3 阶连续偏导数,令
\[H = (F_{x_ix_j}''(\alpha))\]
称 $H$ 是 $F(\seq{x}{n})$ 在 $\alpha$ 处的黑塞矩阵。如果 $H$ 是正定的,那么 $F$ 在 $(x_0,y_0)$ 处达到极小值。如果 $H$ 是负定的,那么 $F$ 在 $(x_0,y_0)$ 处达到极大值。

\let\XAX\relax
