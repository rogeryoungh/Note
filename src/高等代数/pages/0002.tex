\chapter{行列式}

\section{排列}

\begin{definition}
	由 $1,2,\cdots,n$ 组成的一个有序数组称为一个 $n$ 阶排列。
\end{definition}

特殊的,排列 $12\cdots n$ 也是一个 $n$ 阶排列,称为自然排列。

\begin{definition}
	在一个排列中,如果一对数前面的数大于后面的数,那么它们就称为一个逆序,反之称为正序。
	
	排列中逆序的对数称为这个排列的逆序数。
\end{definition}

逆序数为偶数的排列称为偶排列,逆序数的排列称为奇排列。

排列 $j_1j_2\cdots j_n$ 的逆序数记为
\[\tau(j_1j_2\cdots j_n)\]

\begin{definition}
	把排列中某两个数位置互换,得到一个新排列。称这样的一个变换称为一个对换。
\end{definition}

\begin{theorem}
	对换改变排列的奇偶性。
\end{theorem}

\begin{proof}
	设对换为 $(i,j)$,分类讨论:

	\num{1} 若所换两数相邻,则不影响后面数字的逆序数。那么若 $ij$ 为一个逆序,则逆序数减 $1$,否则逆序数加 $1$,总之逆序数奇偶性改变。

	\num{2} 若所换两数不相邻,不妨设 $i$ 在 $j$ 之前,排列为
	\[\cdots i \quad k_1 \cdots k_s \quad j\cdots\]
	那么有
	\[(i,j)=(i,k_1)\cdots(i,k_s)(i,j)(k_s,j)\cdots(k_1,j)\]
	即任意对换即总可以分解为奇数个相邻对换的积。
\end{proof}

任何一个 $n$ 阶排列都可以与自然排列由一系列对换互变,即置换。奇置换可以分解为奇数个对换的积,偶置换可以分解为偶数个对换的积。

\section{\texorpdfstring{$n$ 阶行列式}{n 阶行列式}}

记 $\displaystyle\sum_{j_1j_2\cdots j_n}$ 表示对所有 $n$ 元排列求和。

\begin{definition}
	$n$ 阶行列式
	\begin{equation*}
		\left|\begin{matrix}
			a_{11}&a_{12}&\ldots&a_{1n}\\
			a_{21}&a_{22}&\ldots&a_{2n}\\
			\vdots&\vdots&&\vdots\\
			a_{n1}&a_{n2}&\ldots&a_{nn}
		\end{matrix}\right|
		 = \sum_{j_1j_2\cdots j_n}(-1)^{\tau(j_1j_2\cdots j_n)}a_{1j_1}a_{2j_2}\cdots a_{nj_n}
	\end{equation*}
	该式称为 $n$ 阶行列式的完全展开式。
\end{definition}

若对角线下方的元素全为 $0$,则称为上三角行列式,即对于所有的 $1\leqslant i < j\leqslant n$,有 $a_{ji}=0$。

\begin{theorem}
	上三角行列式的值为
	\[|A| = a_{11}a_{22}\cdots a_{nn}\]
\end{theorem}

\begin{proof}
	考虑其完整展开式的任意一项
	\[(-1)^{\tau(j_1j_2\cdots j_n)}a_{1j_1}a_{2j_2}\cdots a_{nj_n}\]
	若该项不为 $0$,则对 $1 \leqslant k \leqslant n$,皆有 $j_k\leqslant k$。
	
	因此只有 $j_k=k$,只有这一项不为 $0$。
\end{proof}

\begin{theorem}
	给定行指标的一个排列 $i_1i_2\cdots i_n$,则 $n$ 级矩阵的行列式为
	\[|A| = \sum_{k_1k_2\cdots k_n}(-1)^{\tau(i_1i_2\cdots i_n)+\tau(k_1k_2\cdots k_n)}a_{i_1k_1}a_{i_2k_2}\cdots a_{i_nk_n}\]
\end{theorem}

\begin{proof}
	设 $a_{1j_1}a_{2j_2}\cdots a_{nj_n}$ 经过 $s$ 次互换相邻元素变为 $a_{i_1k_1}a_{i_2k_2}\cdots a_{i_nk_n}$,则有
	\begin{equation*}
		\begin{aligned}
			(-1)^{\tau(i_1i_2\cdots i_n)}&=(-1)^s\\
			(-1)^{\tau(j_1j_2\cdots j_n)}(-1)^s&=(-1)^{\tau(k_1k_2\cdots k_n)}
		\end{aligned}
	\end{equation*}
	从而
	\begin{equation*}
		\begin{aligned}
			(-1)^{\tau(i_1i_2\cdots i_n)+\tau(k_1k_2\cdots k_n)}&=(-1)^s(-1)^{\tau(j_1j_2\cdots j_n)}(-1)^s\\
			&=(-1)^{\tau(j_1j_2\cdots j_n)}
		\end{aligned}
	\end{equation*}
\end{proof}

同理,给定列指标的一个排列 $k_1k_2\cdots k_n$,则 $n$ 级矩阵的行列式为
\[|A| = \sum_{i_1i_2\cdots i_n}(-1)^{\tau(i_1i_2\cdots i_n)+\tau(k_1k_2\cdots k_n)}a_{i_1k_1}a_{i_2k_2}\cdots a_{i_nk_n}\]

\section{行列式的性质}

\begin{theorem}
	转置后行列式值不变。
\end{theorem}

记矩阵 $A$ 转置后的矩阵 $A'$ 或 $\transpose{A}$。

\begin{proof}
	设行列式 $B=\transpose{A}$,即 $a_{ij}=b_{ji}$。由前文知,按列指标展开 $B$ 有(注意第 $1$ 个下标是列指标,第 $2$ 个下标是行指标)
	\[|B|=\sum_{i_1i_2\cdots i_n}(-1)^{\tau(i_1i_2\cdots i_n)}a_{1i_1}a_{2i_2}\cdots a_{ni_n}\]
	按列指标展开 $A$ 有
	\[|A|=\sum_{i_1i_2\cdots i_n}(-1)^{\tau(i_1i_2\cdots i_n)}a_{1i_1}a_{2i_2}\cdots a_{ni_n}\]
	因此 $|A|=|B|$。
\end{proof}

\begin{theorem}
	行列式的一行乘以一个数,等于行列式乘以这个数。
\end{theorem}

\begin{proof}
	即行列式 $B$ 除了第 $i_1$ 行有 $b_{i_1j}=ka_{i_1j}$,其他行都有 $b_{ij}=a_{ij}$。
	\begin{equation*}
		\begin{aligned}
			|B| &= \sum_{j_1j_2\cdots j_n}(-1)^{\tau(j_1j_2\cdots j_n)}a_{1j_1}\cdots (ka_{i_1j}) \cdots a_{nj_n}\\
			&= k\sum_{j_1j_2\cdots j_n}(-1)^{\tau(j_1j_2\cdots j_n)}a_{1j_1}\cdots a_{i_1j} \cdots a_{nj_n}\\
			&=k|A|
		\end{aligned}
	\end{equation*}
\end{proof}

\begin{theorem}
	除了同一行以外全部相等的两个行列式,与此行替换为这两行的和的行列式相等。
\end{theorem}

\begin{proof}
	即行列式 $A$ 除了第 $i_1$ 行有 $a_{i_1j}=b_{i_1j}+c_{i_1j}$,其他行都有 $a_{ij}=b_{ij}=c_{ij}$。
	\begin{equation*}
		\begin{aligned}
			|A| &= \sum_{j_1j_2\cdots j_n}(-1)^{\tau(j_1j_2\cdots j_n)}a_{1j_1}\cdots (b_{i_1j}+c_{i_1j}) \cdots a_{nj_n}\\
			&= \sum_{j_1j_2\cdots j_n}(-1)^{\tau(j_1j_2\cdots j_n)}a_{1j_1}\cdots b_{i_1j} \cdots a_{nj_n}+\sum_{j_1j_2\cdots j_n}(-1)^{\tau(j_1j_2\cdots j_n)}a_{1j_1}\cdots c_{i_1j} \cdots a_{nj_n}\\
			&=|B|+|C|
		\end{aligned}
	\end{equation*}
\end{proof}

\begin{theorem}
	行列式中有两行互换,行列式反号。
\end{theorem}

\begin{proof}
	即设行列式 $B$ 为行列式第 $k_1,k_2$ 两行交换的结果,又
	\[(-1)^{\tau(j_1\cdots j_{k_2} \cdots j_{k_1} \cdots j_n)} =- (-1)^{\tau(j_1\cdots j_{k_1} \cdots j_{k_2} \cdots j_n)}\]
	那么有
	\begin{equation*}
		\begin{aligned}
			|B| &= \sum_{j_1\cdots j_{k_2} \cdots j_{k_1} \cdots j_n}(-1)^{\tau(j_1\cdots j_{k_2} \cdots j_{k_1} \cdots j_n)}
			a_{1j_1}\cdots a_{k_1j_{k_2}}\cdots a_{k_2j_{k_1}}\cdots a_{nj_n}\\
			&=-\sum_{j_1\cdots j_{k_1} \cdots j_{k_2} \cdots j_n}(-1)^{\tau(j_1\cdots j_{k_1} \cdots j_{k_2} \cdots j_n)}
			a_{1j_1}\cdots a_{k_1j_{k_1}}\cdots a_{k_2j_{k_2}}\cdots a_{nj_n}\\
			&=-|A|
		\end{aligned}
	\end{equation*}
\end{proof}

\begin{theorem}
	行列式中有两行相等,行列式为零。
\end{theorem}

\begin{proof}
	交换这相同的两行,行列式变号,其仍与原来相等,只能为 $0$。
\end{proof}

\begin{theorem}
	行列式中两行成比例,行列式为零。
\end{theorem}

\begin{proof}
	提出公因子使两行相等,即为 $0$。
\end{proof}

\begin{theorem}
	把一行的倍数加到另一行,行列式不变。
\end{theorem}

\begin{proof}
	一行是另一行的倍数的行列式为 $0$,合并后自然不变。
\end{proof}

\section{行列式按一行展开}

\begin{definition}[代数余子式]\index{daishuyuzishi@代数余子式}
	$n$ 阶行列式 $|A|$ 中,划去第 $i$ 行和第 $j$ 列,剩下的元素按原来次序组成的 $n-1$ 阶行列式称为矩阵 $A$ 的 $(i,j)$ 元的余子式。记作 
	\[M_{ij} = \sum_{k_1\cdots k_{i-1}k_{i+1}\cdots k_n}(-1)^{\tau(k_1\cdots k_{i-1}k_{i+1}\cdots k_n)}a_{1k_1}\cdots a_{i-1,k_{i-1}}a_{i+1,k_{i+1}}\cdots a_{nk_n}\]
	其中 $j=k_i$,令 $A_{ij}=(-1)^{i+j}M_{ij}$,称 $A_{ij}$ 是 $A$ 的 $(i,j)$ 元的代数余子式。
\end{definition}

\begin{theorem}
	对于 $n$ 阶行列式 $|A|$ 有
	\[|A| = \sum_{j=1}^na_{ij}A_{ij} = \sum_{i=1}^na_{ij}A_{ij}\]
	前者称为 $n$ 阶行列式按第 $i$ 行的展开式,后者称为按第 $j$ 列的展开式。
\end{theorem}

\begin{proof}
	首先列出 $|A|$ 的行完全展开式,其中 $j=k_i$
	\[|A|=\sum_{k_1\cdots k_{i-1}jk_{i+1}\cdots k_n}(-1)^{\tau(k_1\cdots k_{i-1}jk_{i+1}\cdots k_n)}a_{1k_1}\cdots a_{i-1,k_{i-1}}a_{ij}a_{i+1,k_{i+1}}\cdots a_{nk_n}\]
	把第 $i$ 行换到第 $1$ 行,第 $j$ 列换到第 $1$ 列,由对换的性质有
	\[(-1)^{\tau(i1\cdots (i-1)(i+1)\cdots n)+\tau(jk_1\cdots k_{i-1}k_{i+1}\cdots k_n)}=(-1)^{i-1}(-1)^{j-1}(-1)^{\tau(k_1\cdots k_{i-1}k_{i+1}\cdots k_n)}\]
	因此
	\begin{equation*}
		\begin{aligned}
			|A|&=\sum_{j=1}^na_{ij}(-1)^{i+j}\sum_{k_1\cdots k_{i-1}k_{i+1}\cdots k_n}(-1)^{\tau(k_1\cdots k_{i-1}k_{i+1}\cdots k_n)}a_{1k_1}\cdots a_{i-1,k_{i-1}}a_{i+1,k_{i+1}}\cdots a_{nk_n}\\
			&=\sum_{j=1}^na_{ij}(-1)^{i+j}M_{ij}=\sum_{j=1}^na_{ij}A_{ij}
		\end{aligned}
	\end{equation*}
	对于列展开式,转置即可。
\end{proof}

\begin{theorem}
	对于 $n$ 阶行列式 $|A|$ 有
	\[\sum_{j=1}^na_{ij}A_{kj} = 0(k\ne i)\]
\end{theorem}

\begin{proof}
	设矩阵 $B$ 第 $k$ 行与第 $i$ 行相等,因此按第 $k$ 行展开有
	\[|B|=\sum_{j=1}^na_{kj}A_{kj}=\sum_{j=1}^na_{ij}A_{kj}=0\]
\end{proof}

\begin{definition}[Vandermonde 行列式]
	 若行列式满足 $a_{ij} = a_j^i$,则称为 Vandermonde 行列式。其值为(证略)
	 \[\prod_{1 \leqslant j < i \leqslant n}(a_i-a_j)\]
\end{definition}

\section{克莱姆 Cramer 法则}

对于数域 $K$ 上 $n$ 个方程的 $n$ 元线性方程组,

\begin{equation*}
	\left\{
		\begin{matrix}
			a_{11}x_1+a_{12}x_2+\cdots+a_{1n}x_n=b_1\\
			a_{21}x_1+a_{22}x_2+\cdots+a_{2n}x_n=b_2\\
			\cdots\qquad\cdots\qquad\cdots\\
			a_{n1}x_1+a_{n2}x_2+\cdot +a_{nn}x_n=b_n
		\end{matrix}
	\right.
\end{equation*}

其系数矩阵记作 $A$,增广矩阵记作 $\widetilde{A}$ 